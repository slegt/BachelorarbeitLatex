%! Author = slegt
%! Date = 20.08.2024

\documentclass[12pt,titlepage,ngerman]{article}

\usepackage[a4paper, portrait, margin=1in]{geometry}
\usepackage{graphicx}
\usepackage[version=4]{mhchem}
\usepackage[style=numeric,maxcitenames=2,sorting=none,doi=false,url=false,isbn=false]{biblatex}
\usepackage[ngerman]{babel}
\usepackage{csquotes}
\usepackage{siunitx}
\usepackage[justification=centering]{caption}
\usepackage{subcaption}
\usepackage{booktabs}
\usepackage{amsmath}
\usepackage{amssymb}
\usepackage{mathtools}
\usepackage[hidelinks]{hyperref}
\usepackage[noabbrev,nameinlink]{cleveref}
\usepackage{currfile}
\usepackage{pgf}
\usepackage{lmodern}
\usepackage{import}
\usepackage{pgffor}
\usepackage{ifthen}
\usepackage{makecell}
\usepackage{parskip}

% Package options
% Das löscht aus dem Literaturverzeichnis alles was da nicht rein soll, falls es in der .bib datei ist
% also z.B. dass aus "Müller (März 2023), https://www.etc.de" ---> "Müller (2023)" wird
\AtEveryBibitem{%
    \clearfield{month}%
    \clearfield{day}%
    \clearfield{urlyear}%
    \clearfield{urlmonth}%
}

% das braucht man auch, damits nicht irgendwo angezeigt wird
\AtEveryCitekey{%
    \clearfield{month}%
    \clearfield{day}%
    \clearfield{urlyear}%
    \clearfield{urlmonth}%
}

% das ändert den \textcite{key} command zu "Name (year) [#]"
% dafür braucht man das xpatch package

\sisetup{locale=DE,separate-uncertainty=true,range-phrase = { bis }, list-final-separator = { und }}
\DeclareCiteCommand{\citeauthoryear}
{\boolfalse{citetracker}%
\boolfalse{pagetracker}%
\usebibmacro{prenote}}
{\ifciteindex
{\indexnames{labelname}\indexfield{year}}
{}%
\printtext[bibhyperref]{%
    \printnames{labelname}%
    \setunit{\addspace}%
    \printtext{(}%
    \printfield{year}%
    \printtext{)}}}
{\multicitedelim}
{\usebibmacro{postnote}}

\addbibresource{literature.bib}

\newcommand{\imcite}[2][]{\\ Aus \cite[#1]{#2}.}
\newcommand{\imcitetwo}[2][]{\\ Nach \cite[#1]{#2}.}

\newcommand{\integral}[4]{\int_{#1}^{#2} #3 \mathrm{d} #4}
\newcommand{\derivative}[2]{\frac{\mathrm{d}}{\mathrm{d} #1} #2}
\newcommand{\heo}{\ce{(MgCoNiCuZn)O}}
\newcommand{\h}{\mathrm{h}}
\renewcommand{\c}{\mathrm{c}}

\newcommand{\sampleone}{$\mathrm{P}_{\num{0.01}}$}
\newcommand{\sampletwo}{$\mathrm{P}_{\num{0.001}}$}
\newcommand{\samplethree}{$\mathrm{P}_{\num{0.1}}$}
\newcommand{\samplefour}{$\mathrm{P}_{\num{0.00005}}$}
\newcommand{\csampleone}{$\mathrm{P}_{\num{0.01}}^{\mathrm{c}}$}
\newcommand{\csampletwo}{$\mathrm{P}_{\num{0.001}}^{\mathrm{c}}$}
\newcommand{\csamplethree}{$\mathrm{P}_{\num{0.1}}^{\mathrm{c}}$}
\newcommand{\csamplefour}{$\mathrm{P}_{\num{0.00005}}^{\mathrm{c}}$}


\DeclareSIUnit\angstrom{\text {Å}}
\DeclareSIUnit\bar{bar}
\captionsetup{justification=centering}


% Document
\begin{document}
    \pagenumbering{gobble}
    \begin{titlepage}
    \begin{center}
        \vfill
        \Huge
        \textbf{Ausheizstudie von \\
        \heo\ Dünnfilmen} \\

        \vfill
        \Large
        An der Universität Leipzig, \\
        Fakultät für Physik und Erdsystemwissenschaften, \\
        Felix-Bloch-Institut für Festkörperphysik , \\
        im Bachelor Physik eingereichte \\

        \vfill
        \Huge
        Bachelorarbeit\\

        \vfill
        \Large
        Zur Erlangung des akademischen Grades eines \\
        Bachelor of Science

        \vfill
        Vorgelegt von \\
        Simon Legtenborg, 3773994 \\
        Geboren am 18.08.2002 in Nordhorn \\


        \vfill
        Betreuer: \\
        M. Sc. Jorrit Bredow

        \vfill
        Gutachter: \\
        Prof. Dr. Marius Grundmann \\
        PD Dr. habil. Holger von Wenckstern


        \vfill
        Eingereicht am 08.11.2024
        \vfill


    \end{center}
\end{titlepage}
    \tableofcontents
    \cleardoublepage
    \pagenumbering{arabic}
    \section{Einleitung}\label{sec:einleitung}
Die Suche nach neuen und funktionalen Materialien, welche die Bedürfnisse moderner Technologien erfüllen, ist ein
wichtiger Bestandteil der Materialwissenschaften.
Methoden zur Entdeckung und Vorhersage solcher Materialien sind vielfältig und konnten erfolgreich große Teile
neuer Materialbibliotheken erschließen.
Dennoch benutzen sie in vielen Fällen die Dichtefunktionaltheorie und führten Berechnungen bei \qty{0}{\kelvin} durch.
Damit können Vorhersagen getroffen werden, welche die Stabilität auf Grundlage der Enthalpie bestimmen
\autocite{Rost2015}.
Die zu minimierende Größe ist jedoch die Gibbs-Energie $G=H-TS$, welche nicht nur durch die Enthalpie $H$, sondern auch
durch das Produkt von Entropie $S$ und Temperatur $T$ bestimmt wird.
Ein vielversprechender alternativer Ansatz, der sich in den letzten Jahren etablierte, ist die Herstellung und
Untersuchung entropiestabilisierter Materialien.
Diese Materialien sind durch eine hohe Entropie gekennzeichnet, die die Bildung einer stabilen Phase ermöglicht, indem
das Produkt $TS$ im Vergleich zur Enthalpie $H$ dominiert.

\citeauthoryear{cantor} untersuchten die Eigenschaften von Mehrkomponentenlegierungen, bestehend aus 20, beziehungsweise 16
equimolar verteilten Elementen.
Beide Systeme zeigten mehrere Phasen, aber auch eine gemeinsame fcc Struktur.
Mithilfe dieser Beobachtung konnte das equimolare Fünf-Komponenten-System \ce{CrMnFeCoNi} entwickeln werden, welches in
einphasiger fcc-Struktur kristallisiert \autocite{cantor}.
Die Fähigkeit, ein stabiles, einphasiges fcc Kristallgitter zu erreichen, war bemerkenswert.
In der Regel neigen Legierungen mit vielen Komponenten dazu, verschiedene Phasen zu bilden, was die
Vorhersagbarkeit der Eigenschaften erschwert.
Cantors Entdeckung zeigte, dass eine geeignete Kombination von Elementen die Bildung einer stabilen Phase ermöglicht
und schuf damit die Grundlage für die Entwicklung mehrkomponentiger Legierungen.

\citeauthoryear{yeh} konnten das Phänomen mithilfe der Mischungsentropie erklären und führten den Begriff der
hochentropischen Legierungen (HEAs, engl. \textit{high entropy alloys}) ein.
Dabei klassifizierten sie Legierungen als HEAs, falls diese mindestens fünf oder mehr Komponenten besitzen
und das molare Verhältnis jedes Konstituenten zwischen \qty{5}{\percent} und \qty{35}{\percent} liegt \autocite{yeh}.
Das war ein Wendepunkt in der Materialwissenschaft, da es die Möglichkeit eröffnete, neue Materialien mithilfe
der Entropie zu stabilisieren, die aus enthalpischer Sicht instabil waren.


\citeauthoryear{Rost2015} erweiterten das Konzept der HEAs auf Metalloxide und führten den Begriff der
entropiestabilisierten Metalloxide ein.
Sie synthetisierten \heo\ und untersuchten dessen Struktur und Eigenschaften.
Dabei haben sie mehrere experimentelle Hinweise dafür geliefert, dass eine reine Mischphase entsteht und diese
tatsächlich durch die Entropie stabilisiert wird.
Sie schufen damit die Grundlage für ein neues Forschungsfeld in der Materialwissenschaft \autocite{Rost2015}.

Im Laufe der Entwicklung hat sich der Begriff der hochentropischen Metalloxide (HEOs, engl. \textit{high entropy oxides})
etabliert.
Es wurden nicht nur verschiedenste Oxide entdeckt, sondern auch Keramiken, Polymere und Verbundsstoffe.
Die Entdeckung der HEAs führte damit zu einer riesigen Klasse der hochentropischen Materialien (HEMs, engl.
\textit{high entropy materials}) mit vielversprechenden Eigenschaften \autocite{Yeh2018}.

Dünnfilme sind ein wichtiger Bestandteil der modernen Technologie und finden Anwendung in unzähligen Bereichen.
Gerade für die Halbleiterelektronik spielen Dünnfilm eine Schlüsselrolle und bilden die Grundlage für
nahezu alle mikroelektronischen Bauelemente.
Die Vorhersage und Optimierung der Eigenschaften von Dünnfilmen ist eine komplexe Herausforderung, da diese stark von
den Substraten und den Prozessparametern abhängen und häufig nicht mit den Eigenschaften des zugrunde liegenden
Massivmaterials übereinstimmen.
Aufgrund der geringen Dicke von Dünnfilmen sind die Oberflächeneigenschaften von besonderer Bedeutung, welche in
analytischen Betrachtungen oftmals vernachlässigt werden.

Die Herstellung und Untersuchung von hochentropischen Materialien in Form von Dünnfilmen ermöglicht eine Verbindung
zwischen der Dünnfilmforschung und der Forschung zu HEMs, wodurch neue Erkenntnisse und innovative
Ansätze in beiden Bereichen gefördert werden.
Aus diesem Grund ist das Ausheizen und Charakterisieren von \heo\ Dünnfilmen zentrales Thema der vorliegenden Arbeit.
Das Ziel ist es, Dünnfilme mithilfe der gepulsten Laserdeposition herzustellen, durch unterschiedliche
Ausheizprozess eine stabile Phase zu erzeugen und währenddessen die Struktur und Eigenschaften dieser Dünnfilme
zu untersuchen.
Konkret soll die Kristallinität mithilfe von Röntgendiffraktometrie und die Oberflächenmorphologie mithilfe von
Rasterkraftmikroskopie charakterisiert werden.
Dabei sollen die Proben einem Prozess aus wiederholtem Ausheizen und anschließendem Messen unterzogen werden.
Es erfolgen drei Ausheizprozesse, die bei Sauerstoff, Vakuum und Luft durchgeführt werden.

Durch gezielte Auswahl von \heo\ als Dünnfilm können direkte Vergleiche zu den Forschungsergebnissen von
\citeauthoryear{Rost2015} gezogen werden.
Die Arbeit bietet weiterhin einen Ausgangspunkt für die Forschung an Kompositionsgradierten Proben, welche weitere
Rückschlüsse auf den Einfluss der Entropie auf die Stabilität von Materialien ermöglichen.
    \include{chapters/3_theorie}
    \section{Probenherstellung und Messmethoden}\label{sec:messmethoden}

\subsection{Gepulste Laserabscheidung}\label{subsec:pld}
\begin{figure}
    \centering
    \includegraphics[width=0.6\textwidth]{../assets/messmethoden/pld/aufbau}
    \caption{Schematischer Aufbau eines PLD-Systems. \imcite{Lorenz2019}}
    \label{fig:pld}
\end{figure}
Die im Rahmen dieser Arbeit betrachteten Dünnfilme wurden mithilfe der gepulsten Laserabscheidung
(PLD, engl. \textit{pulsed laser deposition}) hergestellt.
Dafür wird ein hochenergetischer, gepulster Laserstrahl auf ein Target in einer Vakuumkammer abgebildet, welches bei
hinreichender, materialabhängiger Energie beginnt zu schmelzen und zu verdampfen.
Durch die Wechselwirkung der herausgelösten Atome mit den Laserpulsen entsteht eine Plasmawolke, die sich senkrecht
zum Target ausbreitet.
Die Plasmawolke expandiert in der Vakuumkammer und kann auf einem passend platzierten Substrat abgeschieden werden.
Der schematische Aufbau eines PLD-Systems ist in \cref{fig:pld} dargestellt.
Im Folgenden soll dieser Mechanismus genauer beschrieben werden.
Dabei orientiert sich dieser Abschnitt an \citeauthoryear{Lorenz2019} \autocite{Lorenz2019}.

Bevor ein PLD Prozess gestartet werden kann, muss ein geeignetes Target ausgewählt werden,
indem die einzelnen Binäroxidpulver im äquimolaren Verhältnis abgewogen, in einer Kugelmühle gemahlen,
in eine zylindrische Form gepresst und anschließend für \qty{12}{\hour} bei \qty{1000}{\celsius} gesintert werden.
Anschließend kann es im Targethalter montiert werden.

Der gesamte Prozess findet in einer Vakuumkammer statt, welche exemplarisch in \cref{fig:pld_kammer} dargestellt ist.
Durch Vor- und Turbomolekularpumpen kann ein Vakuum in der Größenordnung von \qty{e-5}{\milli\bar} erzeugt werden.
Zusätzlich können auch Hintergrundgase, unter anderem Sauerstoff und Stickstoff, in die Kammer eingelassen werden.

Um einen PLD Prozess zu starten, muss vorher das Hintergrundgas und dessen Druck in der Vakuumkammer eingestellt werden.
Außerhalb der Vakuumkammer generiert ein \ce{KrF}-Excimerlaser Laserpulse mit einer Wellenlänge von
\qty{248}{\nano\meter}.
Diese Laserpulse werden durch ein Fenster in die Vakuumkammer geleitet und auf das Target abgebildet.
Sie werden anschließend vom Target absorbiert und regen Elektronen der Targetkonstituenten an, die durch
Elektron-Phonon-Wechselwirkung in thermische, chemische und mechanische Energie umgewandelt werden.
Das führt zur Erhitzung des Targets, welches schmilzt und verdampft.
Da das Target durch den Laserstrahl nur an einer kleinen Stelle erhitzt wird, ist es notwendig,
dass sich das Target relativ zum Laserstrahl bewegt, um eine gleichmäßige Abtragung zu gewährleisten.
Dies erfolgt durch Rotations- und Translationsbewegungen.

\begin{figure}
    \centering
    \includegraphics[width=0.6\textwidth]{../assets/messmethoden/pld/kammer}
    \caption{Innenansicht der PLD-Kammer. \imcite{Lorenz2019}}
    \label{fig:pld_kammer}
\end{figure}

Die verdampften Konstituenten interagieren ebenfalls mit den Laserphotonen und bilden durch Photoionisation eine
Plasmawolke, die sich in der Vakuumkammer ausbreitet.
Diese expandiert in der Vakuumkammer senkrecht zum Target.
Das Hintergrundgas in der Vakuumkammer interagiert mit der Plasmawolke durch Stoßprozesse.
Da der Substrathalter gegenüber vom Target positioniert ist, scheiden sich Anteile der Plasmawolke auf dem
Substrat ab.
Es beginnt ein Adsorptionsprozess, sodass sich ein Dünnfilm auf dem Substrat bildet.

Für die Arbeit wurden $\qty{10}{\milli\meter} \times \qty{10}{\milli\meter}$ Corning Eagle-XG-Glassubstrate verwendet.
Dies ist ein amorphes Glas, welches über eine hohe thermische Stabilität verfügt.
Da Phasenübergänge und Kristallisationsprozesse beobachtet werden sollen, ist es notwendig, ein Substrat zu wählen,
welches möglichst wenig Struktur auf die Dünnfilme überträgt.
\newpage
\subsection{Ausheizmethoden}\label{subsec:ausheiz}

\paragraph{Vakuumkammer}
\begin{figure}
    \centering
    \begin{subfigure}{0.35\textwidth}
        \centering
        \includegraphics[width=\textwidth]{../assets/messmethoden/heiz/chamber}
        \caption{Innenansicht der Vakuumkammer.}
        \label{fig:vakuumkammer}
    \end{subfigure}
    \begin{subfigure}{0.35\textwidth}
        \centering
        \includegraphics[width=\textwidth]{../assets/messmethoden/heiz/sample_holder}
        \caption{Substrathalter mit montierter Probe.}
        \label{fig:sample_holder}
    \end{subfigure}
    \caption{Für die Ausheizprozesse verwendete Vakuumkammer und verwendeter Substrathalter.}
\end{figure}
Eine für diese Arbeit verwendete Anlage zum Ausheizen der Dünnfilme ist eine Vakuumkammer, die mit einer Möglichkeit
zur Sauerstoffzufuhr ausgestattet ist, siehe \cref{fig:vakuumkammer}.
Die Kammer verfügt über eine Turbomolekularpumpe und eine Vorvakuumpumpe, die zusammen ein Vakuum von bis zu
\qty{e-4}{\milli\bar} erzeugen.
Die Proben werden auf einem Substrathalter, siehe \cref{fig:sample_holder}, montiert und in die Kammer eingesetzt.
Ein Heizlaser ist auf die Rückseite des Substrathalters gerichtet, und die Temperatur der Rückseite des Substrathalters
wird mittels eines Pyrometers gemessen.
Ein PID-Regler steuert die Temperatur, indem er die Laserleistung auf Basis der Temperaturmessungen anpasst, um eine
gleichmäßige Temperatur während des Ausheizprozesses zu gewährleisten.

\paragraph{Muffelofen}
\begin{figure}
    \centering
    \begin{subfigure}[t]{0.35\textwidth}
        \centering
        \includegraphics[width=\textwidth]{../assets/messmethoden/heiz/ofen}
        \caption{Innenansicht des Muffelofens.}
        \label{fig:muffelofen}
    \end{subfigure}
    \begin{subfigure}[t]{0.35\textwidth}
        \centering
        \includegraphics[width=\textwidth]{../assets/messmethoden/heiz/schale}
        \caption{Keramikschalen, in welcher die Proben platziert werden.}
        \label{fig:schale}
    \end{subfigure}
    \caption{Für die Ausheizprozesse verwendeter Muffelofen und Keramikschalen.}
\end{figure}
Neben der Vakuumkammer wurde ein Muffelofen für das Ausheizen der Dünnfilme in einer Luftatmosphäre genutzt,
siehe \cref{fig:muffelofen}.
Der Ofen ist mit einem Temperatursensor ausgestattet, der die Temperatur im Inneren überwacht.
Die Proben werden in einer Keramikschale platziert, siehe \cref{fig:schale}, mit einer zweite
Keramikschale abgedeckt und in den Ofen eingesetzt.

Die Temperaturkontrolle erfolgt auch hier über einen PID-Regler, der die Heizelemente des Ofens reguliert, um
die eingestellte Temperatur konstant zu halten.
Die Proben werden für eine festgelegte Dauer im Ofen belassen, um die gewünschten thermischen Effekte zu erzielen.

\subsection{Röntgendiffraktometrie}\label{subsec:xrd}
Röntgendiffraktometrie (XRD, engl. \textit{X-Ray diffraction}) ist eine weit verbreitete Methode, um die
Kristallstruktur von Dünnfilmen zu bestimmen.
Dabei wird ein Röntgendiffraktometer verwendet.
Im Folgenden wird der Aufbau und die Funktionsweise dieses Gerätes erläutert.
Dabei wird sich an den Erkenntnissen aus \citeauthoryear{btb-xrd} \autocite{btb-xrd} orientiert.

\subsubsection{Röntgendiffraktometer}

\begin{figure}
    \centering
    \includegraphics[width=0.6\textwidth]{../assets/messmethoden/xrd/img}
    \caption{Aufbau eines Röntgendiffraktometers. \imcite{btb-xrd}}
    \label{fig:xrd}
\end{figure}

Das Röntgendiffraktometer besteht aus fünf Hauptkomponenten: Röntgenquelle und Detektor, Ein- und Ausfallsoptik,
sowie dem Goniometer.
Zusätzlich ist das Diffraktometer durch eine Strahlungsschutzverkleidung abgeschirmt und mit einer Steuerungssoftware
verbunden.
Ein schematischer Aufbau ist in \cref{fig:xrd} zu erkennen.
Im Folgenden werden die einzelnen Komponenten näher erläutert.

\paragraph{Röntgenquelle}
Die Röntgenstrahlen werden in einer Röntgenröhre erzeugt.
In dieser werden Elektronen aus einer Wolfram-Glühkathode emittiert und durch das elektrische Feld auf eine Anode
beschleunigt.
Die Anode besteht meist aus hochreinem Kupfer.
Stromstärke und Beschleunigungsspannung der Röntgenröhre müssen so gewählt werden, dass die Energie beim Auftreffen der
Elektronen auf die Anode ausreicht, um die gebundenen Elektronen der Atome auf das nächsthöhere Energieniveau anzuregen.
Aufgrund der daraus resultierende Wärme muss die Anode ständig wassergekühlt werden.
Röntgendiffraktometer wird eine Beschleunigungsspannung von \qty{40}{\kilo\volt} und eine Stromstärke
von \qty{40}{\milli\ampere} verwendet.

Nach der Kollision zwischen Kupferatom und Elektron relaxiert das Elektron unter Bildung eines Röntgenphotons.
Man erhält ein Spektrum, welches durch die charakteristische Strahlung der Anode sowie durch Bremsstrahlung
geprägt ist.
Die charakteristische Strahlung wird vorrangig durch die K-Linien, insbesondere $K_{\alpha_1}$, $K_{\alpha_2}$
und K$_{\beta}$, dominiert.
Da $K_{\alpha_1}$ und $K_{\alpha_2}$ energetisch sehr nahe beieinander liegen, können sie nicht immer einzeln
aufgelöst werden.
Die $K_{\beta}$ Strahlung ist größtenteils unerwünscht und kann durch geeignete Filter unterdrückt werden.

Die Wolfram-Glühkathode emittiert unerwünschterweise nicht nur Elektronen, sondern auch Wolfram-Atome in kleinen Mengen.
Über längere Zeiträume führt dies zu einer nicht mehr zu vernachlässigenden Kontamination der Anode.
Dadurch können bei Elektronenstößen auch Wolfram-Atome angeregt werden, was zu einer zusätzlichen Wellenlänge im
Spektrum führt.
In den späteren Messergebnissen sind diese Beiträge erkennbar.
Abschließend gelangen die Röntgenstrahlen durch ein Berylliumfenster in die Einfallsoptik.

\paragraph{Goniometer}
Das Goniometer ist die mechanische Komponente des Röntgendiffraktometers.
Es besteht aus mehreren Drehachsen, die es ermöglichen, die Probe in unterschiedlichsten Winkeln auszurichten.
Nach der Braggschen Beugungstheorie ergeben sich konstruktive Interferenzen an denjenigen Winkeln, die der
Bragg-Bedingung genügen.
Existieren Möglichkeit, die Winkel für Quelle und Detektor zu variieren, kann diese Interferenz beobachtet werden.
Im Allgemeinen ist die Röntgenquelle jedoch fest, eine äquivalente Drehung von Probe und Detektor ist deshalb gängig.
In der einfachsten Betrachtungsweise muss das Goniometer also den Winkel zwischen Probe und Quelle ($\omega$) und dem
Winkel zwischen Probe und Detektor ($2\theta$) einstellen können.
Diese Freiheit reicht zwar für Pulverproben, jedoch nicht für Dünnfilme.
Zwar kann man mit beiden Freiheitsgraden Messungen durchführen, welche die out-of-plane Orientierung charakterisieren,
jedoch ist es nicht möglich, die in-plane Orientierung zu bestimmen.
Dafür werden weitere Achsen, wie $\varphi$ und $\chi$, benötigt.
Eine Konstruktion mit den vier Achsen wird Euler-Wiege genannt.

\paragraph{Ein- und Ausfallsoptik}
Es existieren zwei primäre Möglichkeiten, um den Strahlengang zwischen Quelle, Probe und Detektor zu modifizieren:
Bragg-Brentano-Geometrie und parallele Geometrie.
In der Bragg-Brentano-Geometrie wird ein hochintensiver, divergierender Röntgenstrahl auf die Probe gerichtet,
welcher unter Ausnutzung bestimmter Geometrien zurück auf den Detektor fokussiert werden kann.
Aufgrund der Divergenz des Strahls ist diese Methode fehleranfällig und gerade für $\varphi$ und $\omega$ scans
ist eine Parallelstrahloptik von Vorteil.

Die aus der Quelle austretenden divergierenden Strahlen werden durch einen Göbel-Spiegel
parallelisiert und durchlaufen anschließend einen Eingangs- und Axialspalt, die den Winkelbereich
festlegen.
Dazwischen liegt ein Soller Spalt, der die Divergenz in axialer Richtung begrenzt, da diese
nicht durch den Göbel-Spiegel eliminiert werden kann.
Anschließend reflektiert der Strahl an der Probe und gelangt über weitere Spalte zum Detektor.

\paragraph{Detektor}
Der Röntgendetektor dient dazu, die Intensität der reflektierten Röntgenstrahlen zu messen und
in elektrische Signale umzuwandeln.
Im für die Arbeit verwendeten Röntgendiffraktometer wird ein Halbleiterdetektor verwendet.
Die Grundidee eines Halbleiterdetektors ist, das Material durch Photonen zu ionisieren und dabei freie Ladungsträger zu
erzeugen.
Diese werden über ein elektrisches Feld extrahiert und elektronisch gezählt.
Durch die matrixartige Anordnung einzelner Halbleiterzellen kann die Intensität in Abhängigkeit der Position bestimmt
werden.
Somit können unterschiedliche Detektormodi realisiert werden, wie 0D Punktdetektion,
1D Linien- oder 2D Flächendetektion.
Wichtig ist, dass die maximale Zählrate des Detektors nicht überschritten wird.
Das führt zu nichtlinearen Antworten und kann den Sensor beschädigen.
Um das zu vermeiden, können Filter und Attenuatoren verwendet werden.

\subsubsection{Scanmethoden}
Durch den hohen Freiheitsgrad des Goniometers können unterschiedliche Scanmethoden realisiert werden.
Diejenigen, die in dieser Arbeit verwendet wurden, sind im Folgenden beschrieben.

\paragraph{$2\theta/\omega$ scan}
Ein $2\theta/\omega$ scan ist eine einfache Methode, um die Kristallstruktur von Dünnfilmen zu bestimmen
und bildet die Grundlage für weitere Messungen.
Die Probe wird auf den Probenhalter montiert und in das Goniometer eingesetzt.
Der Winkel zwischen Probenoberfläche und Quelle wird als $\omega$ und der Winkel zwischen Probenoberfläche und Detektor
als $2\theta$ bezeichnet.
Als Anfangsbedingung legt man Messdauer und ein Intervall für $\omega$ fest und startet die Messung.
Nun fahren Goniometer und Detektor über den angegebenen Winkelbereich mit der Bedingung $\omega = \theta$.
Als Ausgabe bekommt man ein Intensitätsprofil in Abhängigkeit von $2\theta$, in denen im Idealfall charakteristische
Peaks erkennbar sind.
Diese Peaks entstammen der Bragg-Reflexion von Kristallebenen, die parallel zur Probenoberfläche liegen.
Kennt man die Wellenlänge der Röntgenstrahlung, so kann man mithilfe des Peaks den Gitterabstand bestimmen.
Somit ist es möglich, Rückschlüsse auf die vorhandenen Kristallstrukturen im Dünnfilm zu ziehen.

\subsection{Rasterkraftmikroskopie}\label{subsec:afm}
Mithilfe der Rasterkraftmikroskopie (AFM, engl. \textit{Atomic Force Microscopy}) kann die Oberflächenstruktur der
Dünnfilme charakterisiert werden.
Das Rasterkraftmikroskop ist ein hochpräzises Messinstrument zum Erfassen von Oberflächenstrukturen.
Anders als bei Licht- oder Elektronenmikroskopie wird hierbei eine mechanische Funktionsweise verwendet.
Dabei fährt der Cantilever, eine Messnadel, rasterweise über eine Oberfläche und tastet diese ab.
Die Spitze des Cantilevers ist im Nanometerbereich dimensioniert und kann so auch kleinste Strukturen erfassen.
Die auf den Cantilever wirkenden interatomaren oder magnetischen Kräfte führen zu messbaren Auslenkungen,
woraus eine Topographiekarte der Oberfläche erstellt wird.
Das in der Arbeit benutzte Rasterkraftmikroskop ist das \textit{XE-150} von Park Systems.
In den folgenden Abschnitten wird der Aufbau und die Funktionsweise des Rasterkraftmikroskops
erläutert, basierend auf den Erkenntnissen von \citeauthoryear{afm-buch} \autocite{afm-buch}.

\subsubsection{Schematischer Aufbau und Funktionsweise}
\begin{figure}
    \centering
    \includegraphics[width=0.6\textwidth]{../assets/messmethoden/afm/01_aufbau}
    \caption{Schematischer Aufbau eines Rasterkraftmikroskops. \imcite{afm-buch}}
    \label{fig:afm_aufbau}
\end{figure}
Die grundlegende Funktionsweise ist in \cref{fig:afm_aufbau} dargestellt.
Fährt der Cantilever über die Probe, so wirken interatomare Kräfte auf die Spitze, welche den Cantilever auslenken.
Diese Auslenkung wird mithilfe eines Laserstrahls und eines Photodetektors gemessen und an ein Feedback System
übergeben.
Basierend auf dem gewählten Betriebsmodus wird entweder versucht, die Auslenkung oder die Schwingungsamplitude
des Cantilevers konstant zu halten.
Mithilfe dieser Regulation wird ein Korrektursignal ausgegeben, welches die Position des Cantilevers anpasst.
Dies geschieht mithilfe von Piezoelementen, wodurch der Cantilever in $\mathrm{z}$-Richtung und die Probe in
$\mathrm{x}$- und
$\mathrm{y}$-Richtung bewegt werden kann.
Die z-Position des Cantilevers wird aufgezeichnet und als Topographiesignal am Computer ausgewertet.
Im Folgenden werden die einzelnen Komponenten näher erläutert.

\paragraph{Auslenkungserkennungssystem}
\begin{figure}
    \centering
    \includegraphics[width=0.6\textwidth]{../assets/messmethoden/afm/02_beam}
    \caption{Schematischer Aufbau eines Auslenkungserkennungssystems. \imcite{afm-handbuch}}
    \label{fig:afm-beam}
\end{figure}
Das System zur Auslenkungserkennung ist in \cref{fig:afm-beam} dargestellt.
Ein Laserstrahl wird aus einer Laserdiode emittiert und auf die Rückseite des Cantilevers gerichtet.
Dieser besitzt eine reflektierende Rückseite, sodass der Strahl am Cantilever zum beweglichen Spiegel reflektiert wird.
Über diesen lässt sich die Position des Strahls auf den Photodektor einstellen.
Der Strahl wird durch einen letzten Spiegel auf den Photodetektor reflektiert.


Der Photodetektor besteht aus vier Segmenten, welche in einem Quadrat angeordnet sind, wobei jedes Segment einen
Quadranten dieses Quadrats belegt.
Damit kann die Intensität des reflektierten Strahls ortsaufgelöst gemessen werden.
Im unausgelenkten Zustand muss der bewegliche Spiegel so eingestellt werden, dass der Strahl mittig auf die vier
Segmente trifft.
Kommt es zur Auslenkung, ändert sich der Winkel des Cantilevers und damit auch die Position des Strahls auf dem
Photodetektor.
Über die vier Segmente können Auslenkung und Torsion des Cantilevers bestimmt werden.

\paragraph{Feedback Controller}
Damit eine genaue Topografiekarte aufgezeichnet werden kann, ist ein schnelles und präzises Regelsystem nötig.
Durch die Regelung soll eine vorgegebene Kenngröße, beispielsweise die Auslenkung des Cantilevers, möglich konstant
gehalten werden.
Für diesen Zweck wird ein geschlossenes Regelkreissystem verwendet.
Der Regler besteht aus einem Proportional-Integral-Controller (PI-Controller), der das Fehlersignal zwischen dem
gemessenen Wert und dem Sollwert verarbeitet.
Der Proportionalanteil reagiert direkt auf den aktuellen Fehler, während der Integralanteil konstante Störungen über
die Zeit eliminiert.
Die Kombination beider Anteile ermöglicht es, sowohl kurzzeitige als auch langzeitige Abweichungen zu
korrigieren.

\paragraph{Positionierung}
Piezoelemente werden verwendet, um den Cantilever in $\mathrm{z}$-Richtung und die Probe in $\mathrm{xy}$-Richtung zu
bewegen, basierend auf dem Piezoeffekt.
Dieser Effekt tritt auf, wenn bestimmte Kristalle wie Quarz unter mechanischer Spannung elektrische Spannung erzeugen
können.
Durch Anlegen einer elektrischen Spannung an Piezokristalle können Deformationen im Bereich von \qty{0.1}{\nano\meter}
und damit in atomaren Dimensionen erreicht werden.
Spezielle Metallbiegeelemente mit eingebetteten Piezoelementen ermöglichen gezielte Verformungen, um Bewegungen
in verschiedenen Richtungen zu erzeugen.
Damit kann sowohl die Probe in $\mathrm{xy}$-Richtung, als auch der Cantilever in $\mathrm{z}$-Richtung bewegt werden.
Die Feinpositionierung und Scanbewegungen in der Rasterkraftmikroskopie werden durch Piezo-Biegeelemente gesteuert,
während eine Grobpositionsbühne für die grobe Positionierung verwendet wird.

\subsubsection{Interaktion zwischen Probe und Spitze}
Die auf die Spitze wirkende Gesamtkraft setzt sich aus verschiedenen Komponenten zusammen.
Der wichtigste und weitreichendste Beitrag entspringt der Van-der-Waals Wechselwirkung, einer attraktiven Kraft,
die durch die spontane Ausbildung fluktuierender Dipole entsteht.
Für kleinere Distanzen müssen zwei weitere Interaktionen berücksichtigt werden.
Überlappen die äußeren Elektronenhüllen von Proben- und Spitzenatomen, so können chemische Bindungen entstehen, welche
attraktive oder repulsive Kräfte hervorrufen.
Für noch kleinere Distanzen wird die Pauli-Abstoßung zwischen den Elektronen der Atome relevant.
Da die geschlossenen Elektronenschalen der Atome nicht überlappen können, müssen die zusätzlichen Elektronen auf ein
höheres Energieniveau gehoben werden.
Dadurch entsteht eine effektive repulsive Kraft.

Obwohl das System quantenmechanisch präzise beschrieben werden kann, ist die Lösung solch komplexer Systeme
nicht trivial.
Aus diesem Grund werden Modellpotentiale genutzt, die die Interaktionen zwischen Probe und Spitze approximieren.
Ein beliebtes ist das Lennard-Jones-Potential, welches sowohl die Van-der-Waals Wechselwirkung als auch die repulsiven
Anteile berücksichtigt:
\begin{equation}
    U_{\mathrm{LJ}}=4U_{0}\left[ \left( \frac{R_{\mathrm{a}}}{r} \right)^{12} -\left( \frac{R_{\mathrm{a}}}{r}
    \right)^{6}\right].
    \label{eq:lennard_jones}
\end{equation}
Hierbei ist $U_{0}$ die Tiefe des Potentials, $R_{\mathrm{a}}$ der Gleichgewichtsabstand und $r$ der Abstand zwischen
den Atomen.
Das Potential ist in \cref{fig:lennard_jones} dargestellt.
Auch wenn das Lennard-Jones-Potential für interatomare Wechselwirkungen entwickelt wurde, erklärt es die relevanten
Kräfte zwischen Probe und Spitze.

\begin{figure}
    \centering
    \import{../plots/messmethoden/}{lennard_jones_potential.pgf}
    \caption{Lennard-Jones-Potential.}
    \label{fig:lennard_jones}
\end{figure}

\subsubsection{Betriebsmodi}
Das Rasterkraftmikroskop verfügt über unterschiedliche Betriebsmodi.
In der vorliegenden Arbeit wurde der statische Kontaktmodus verwendet, um die Oberflächenstruktur zu erfassen.
Dabei wird die Probe in $\mathrm{xy}$-Richtung bewegt, während die Spitze des Cantilevers einen so kleinen Abstand zur
Probe hat, dass die repulsiven Kräfte dominieren.
Für den statischen Kontaktmodus liegt der Abstand im Definitionsbereich der orange dargestellten Kurve in
\cref{fig:lennard_jones}.
In diesem Bereich ist die Ableitung des Potentials negativ, was zu einer repulsiven Kraft führt.
Diese bewirkt eine Auslenkung $\Delta z$ des Cantilevers, welche durch das Erkennungssystem aufgezeichnet wird.
Mithilfe eines \textit{setpoint}-Parameters wird eine konstante Auslenkung festgelegt, die vom Feedback-System
eingehalten wird.
Scannt man über eine Erhöhung, so ändert sich die Auslenkung und das Feedback-System versucht, den Cantilever erneut
auszurichten, indem es die Höhe anpasst.
Dadurch entsteht eine Topografiekarte der Oberfläche.
Da sich die Spitze in stetigem Kontakt mit der Oberfläche befindet, müssen die Wechselwirkungskräfte
möglichst klein gehalten werden, da ansonsten die Spitze leicht kontaminiert oder beschädigt werden kann.

Nicht nur statische, sondern auch dynamische Betriebsmodi existieren.
Dabei wird der Cantilever mit einer bestimmten Frequenz und Amplitude angeregt, während er über die Probe fährt.
Ändert sich die Kraft, dann ändert sich die Resonanzfrequenz und damit auch die Amplitude.
Das Topografiesignal wird durch die Forderung einer konstanten Amplitude erzeugt.

Die in der Arbeit aufgenommenen AFM-Bilder wurden stets mit einer Scanrate von \qty{1}{\hertz}, einem
Setpoint von \num{0.4} und, bis auf eine Ausnahme, einem Scanbereich von $\qty{5}{\micro\meter} \times
\qty{5}{\micro\meter}$ aufgenommen.
Für alle Aufnahmen wurden NSC18-TI/PT Cantilever von MikroMasch verwendet.

\subsection{Weitere Messmethoden}\label{subsec:weitere-messmethoden}
Die in diesem Abschnitt vorgestellten Methoden wurden für die Arbeit verwendet, werden aber
nicht im Detail erläutert.

\paragraph{Profilometer}
Ein Profilometer ist ein Messinstrument, welches zur präzisen Messung von Oberflächenprofilen verwendet wird.
Sowohl die Topografie als auch die Rauheit der Oberfläche können hiermit bestimmt werden.
Im Gegensatz zum Rasterkraftmikroskop ist das Profilometer für gröbere Oberflächenstrukturen auf größeren Flächenskalen
ausgelegt.
Das verwendete Profilometer \textit{DektakXT} von Bruker fährt mit einer Diamantspitze über die Probe, um
die Oberflächenstruktur zu erfassen.

Die mithilfe von PLD hergestellten Dünnfilme sind durch Klemmen an den Substrathalter montiert, welche
üblicherweise kleine Regionen des Substrats, meist die Ecken, verdecken.
Da an diesen Stellen kein Dünnfilm abgeschieden wird, kann mithilfe dieser Regionen die Dicke des
Dünnfilms bestimmt werden, indem die Spitze des Profilometers sowohl über Dünnfilm als auch über eine
verdeckte Stelle fährt.
Der Höhenunterschied zwischen beiden Stellen entspricht der Dicke des Dünnfilms.

\paragraph{Energiedispersive Röntgenspektroskopie}
Eine weitere wichtige Methode zur Charakterisierung von Dünnfilmen ist die energiedispersive Röntgenspektroskopie
(EDX, engl. \textit{energy dispersive X-ray spectroscopy}), welche ortsaufgelöste chemische Kompositionsanalysen
ermöglicht.
Hierbei werden die Atome des Dünnfilms durch den Elektronenstrahl eines Rasterelektronenmikroskops (SEM, engl.
\textit{scanning electron microscope}) angeregt, welcher durch Stoßprozesse innere Elektronen aus den Atomen
herausschlägt.
Durch die anschließende Relaxation werden Röntgenstrahlen emittiert, deren Energie charakteristisch für das jeweilige
Element ist.
Die Energien der Röntgenstrahlen werden mithilfe eines Detektors aufgezeichnet und in ein Spektrum umgewandelt, welches
die Intensität in Abhängigkeit der Energie darstellt.
Da mehrere Elektronenübergänge bei der Relaxation möglich sind, entstehen in diesem Spektrum mehrere charakteristische
Peaks pro Element.
Durch die Position der Peaks können die enthaltenen Elemente ermittelt werden,
die korrespondierende Intensität gibt Aufschluss über die Konzentration der Elemente \autocite{edx}.




    \section{Auswertung}\label{sec:auswertung}
Von Jorrit Bredow wurden vier Proben \samplethree, \sampleone, \sampletwo, \samplefour\ in vier Prozessen einer
PLD-Anlage mit $\qty{10}{\milli\meter} \times \qty{10}{\milli\meter}$ Eagle XG Glassubstraten und dem in
\cref{subsec:pld} beschriebenen Target hergestellt.
Für jeden Prozess wurden jeweils \num{20000} Laserpulse mit einer Frequenz von \qty{20}{\hertz} und einer Energie von
\qty{650}{\milli\joule} abgegeben.
Dabei wurde der eingebaute Widerstandsheizer nicht verwendet.
Der variable Parameter war der Abscheidedruck der Sauerstoffatmosphäre innerhalb der Kammer.
Nach der Herstellung wurden die Proben im Profilometer und im Röntgendiffraktometer bezüglich ihrer Schichtdicke
vermessen.
Der jeweilige Druck und die Schichtdicken sind in \cref{tab:samples} aufgeführt.
\begin{table}[h]
    \centering
    \begin{tabular}{l l l l l}
        \toprule
        Probenname & \makecell[l]{Abscheidedruck \\ in \unit{\milli \bar}} & \makecell[l]{Dicke in \unit{\nano\meter} \\
        Profilometermessung} & \makecell[l]{Dicke in \unit{\nano\meter}     \\ XRR Messung}   \\
        \midrule
        \samplethree   & \num{0.1}   & \num{65(4)} & \num{109} \\
        \sampleone  & \num{0.01} & \num{95(7)} & - \\
        \sampletwo  & \num{0.001} & \num{160(12)} & \num{142} \\
        \samplefour  & \num{0.00005} & \num{135(16)} & \num{120} \\
        \bottomrule
    \end{tabular}
    \caption{Abscheidedruck und Schichtdicken der zu untersuchenden Proben}
    \label{tab:samples}
\end{table}

In jedem Prozess ist neben dem Eagle XG Glassubstrat ein c-Saphir Substrat eingebaut worden.
Damit konnten vier weitere Proben \csamplethree, \csampleone, \csampletwo, \csamplefour\ hergestellt werden, welche
jedoch ausschließlich für die Konzentrationsanalyse von \cref{subsec:edx-analyse} Verwendung finden.
Der untere Index der Bezeichnung steht erneut für den Abscheidedruck in \unit{\milli \bar}.

Im Anschluss an die Probenherstellung wurden \samplethree, \sampleone, \sampletwo, \samplefour\ in
$\qty{5}{\milli\meter}\times \qty{5}{\milli\meter}$ gevierteilt, damit Ausheizstudien mit unterschiedlichen Atmosphären
vergleichbar durchgeführt werden konnten.
Dazu wurde der Dünnfilm der jeweiligen Probe mit Photolack beschichtet, um die Oberfläche vor
mechanischen Beschädigungen beim Zersägen zu schützen.
Das Substrat wurde mit der Dünnfilmseite nach oben auf eine Glasplatte mit Wachs geklebt und mit einer
Diamantsäge zerteilt.
Anschließend wurden Wachs und Photolack mit Wachsentferner, NMP und Ethanol entfernt.
Es ist zu beachten, dass diese Prozedur durch die Lösungsmittel die Zusammensetzung und Oberfläche des Dünnfilms
verändern kann.
Optisch wurden keine Veränderungen festgestellt.

Nach der Präparation wurden die Eagle XG Proben auf drei verschiedene Arten ausgeheizt:
In der im \cref{subsec:ausheiz} vorgestellten Vakuumkammer unter Sauerstoffatmosphäre, in der Vakuumkammer unter
Vakuumbedingungen und im Muffelofen aus \cref{subsec:ausheiz} unter Luftatmosphäre.

Um einen Ausgangspunkt für die Temperatur festzulegen, wurde sich an den Erkenntnissen von Rost (2015) orientiert.
In diesem Paper wurden, wie in \cref{subsubsec:heo} beschrieben, Pellets angefangen bei \qty{750}{\degreeCelsius}
auszuheizen.
Bei dieser Temperatur wurden mehrere Phasen beobachtet, ausgeprägte Peaks der Natriumchloridstruktur waren jedoch
bereits erkennbar \autocite{Rost2015}.
Da sich Dünnfilme nicht wie massive Proben verhalten, wurde die Temperatur auf \qty{600}{\degreeCelsius} festgelegt,
um sicherzustellen, dass die Phasenübergangstemperatur nicht zu Beginn bereits überschritten wird.
Um dies zu überprüfen, wurde eine der vier geviertelten Proben von \sampletwo\ in Luft auf \qty{600}{\degreeCelsius}
für eine Stunde ausgeheizt und anschließend im Röntgendiffraktometer vermessen.
Da keine Peaks des Dünnfilms zu erkennen waren, wurde diese Probe anschließend für drei Stunden und sonst identischen
Bedingungen auf \qty{600}{\degreeCelsius} ausgeheizt und erneut im Röntgendiffraktometer vermessen.
Auch hier waren keine Peaks des Dünnfilms zu erkennen, weshalb die Temperatur als Ausgangspunkt für den
Sauerstoff-Ausheizvorgang festgelegt wurde.

Während des Sauerstoffausheizvorgangs wurde ein nichtsignifikanter Peak bei ${2\theta}{\approx}\qty{30}{\degree}$
ab einer Temperatur von \qty{600}{\degreeCelsius} beobachtet, welcher ein Hinweis auf eine Phasenbildung sein könnte.
% TODO nachgucken
Aus diesem Grund wurde beim Vakuumausheizvorgang die Starttemperatur auf \qty{500}{\degreeCelsius} festgelegt.
Da keine signifikanten Peaks in dieser Serie zu erkennen waren, wurde die Starttemperatur erneut
auf \qty{600}{\degreeCelsius} für die Ausheizung im Muffelofen gesetzt.

Da die Dünnfilme auf Eagle XG Glassubstraten gewachsen sind, muss die Viskosität des Glases in Abhängigkeit der
Temperatur beachtet werden.
Der obere Kühlpunkt des Glases liegt bei \qty{722}{\degreeCelsius}.
Bis zu diesem Punkt treten zwar Entspannungen auf, Formänderungen sind jedoch nicht zu erwarten.
Der Erweichungspunkt des Glases liegt bei \qty{971}{\degreeCelsius}.
Ab dieser Temperatur beginnt das Glas merklich zu fließen und sich unter Einfluss des Eigengewichts zu verformen.
Da die Dünnfilme im Substrathalter der Vakuumkammer unter mechanischer Spannung stehen, ist bereits bei
\qty{875}{\degreeCelsius} eine Deformierung erkennbar, sodass diese Temperatur als obere Grenze für die Ausheizstudien
in der Vakuumkammer festgelegt wurde.
Diese Temperatur ist außerdem die von Rost verwendete Phasenübergangstemperatur für massive equimolare Proben.
\autocite{Rost2015}
Für den Ausheizvorgang im Muffelofen wurde die Temperatur auf \qty{950}{\degreeCelsius} festgelegt, da diese noch
unterhalb des Erweichungspunkts des Glases liegt.

Für den Sauerstoffausheizvorgang wurde die Vakuumkammer mit einer Sauerstoffatmosphäre von circa \qty{800}{\milli\bar}
verwendet.
Die Proben \samplethree, \sampleone, \sampletwo, \samplefour\ wurden einem Prozess aus wiederholtem Ausheizen und
anschließendem Messen unterzogen.
Als Ausheiztemperaturen sind \qtylist{600;700;750;800;875}{\degreeCelsius} gewählt worden.
Dabei dienen \qty{750}{\degreeCelsius} und \qty{875}{\degreeCelsius} als Referenztemperaturen zur Forschung
von Rost \autocite{Rost2015}.
Die Proben wurden für eine Stunde bei der jeweiligen Temperatur ausgeheizt, vom Substrathalter entfernt und auf eine
Glasplatte gelegt, um ein schnelles Abkühlen zu erreichen.
Während des Ausheizens rotierte der Substrathalter mit einer Winkelgeschwindigkeit von
\qty{1080}{\degree\per\minute}, um eine
gleichmäßige Temperaturverteilung zu gewährleisten.

Für den Ausheizvorgang unter Vakuumbedingungen wurde die gleiche Vakuumkammer wie für den Sauerstoffausheizvorgang
verwendet.
Auch hier wurden die Proben \samplethree, \sampleone, \sampletwo, \samplefour\ einem Prozess von wiederholendem
Ausheizen und anschließendem Messen unterzogen.
Die Kammer wird auf einen Druck von \qty{4e-4}{\milli\bar} evakuiert, um Vakuumbedingungen zu erreichen.
Dies ist die geringste Druckstufe, die der PID-Regler der Vakuumkammer zuverlässig erreicht.
Alle anderen Parameter sind identisch zum Sauerstoffausheizvorgang.

Für den Ausheizvorgang in Luftatmosphäre wurde der Muffelofen verwendet.
Die Proben \samplethree, \sampleone, \sampletwo, \samplefour\ wurden einem Zyklus von Ausheizen und anschließendem
Messen unterzogen.
Für das Ausheizen wurden die Temperaturen \qtylist{600;700;750;800;875;950}{\degreeCelsius} gewählt.
Die Proben wurden in einer Keramikschale für eine Stunde bei der jeweiligen Temperatur ausgeheizt, anschließend
aus dem Ofen und der Schale genommen und auf einer Glasplatte abgekühlt.
Anders als bei den vorherigen Ausheizvorgängen wurde die Probe mit einer langsamen Ausheizrate von
\qty{300}{\kelvin\per\hour} auf die Endtemperatur gebracht.

\subsection{Temperaturkalibrierung der A-Kammer}\label{subsec:temperaturkalibrierung}
\begin{figure}
    \centering
    \foreach \i/\desc in {
        furnace_calibration_1.pgf/{Kalibrierung des Lithografiesensors, Bild 1},
        furnace_calibration_2.pgf/{Kalibrierung des Lithografiesensors, Bild 2},
        a_chamber_calibration.pgf/{Kalibrierung der Vakuumkammer},
        final_calibration.pgf/{Abhängigkeit zwischen $T_{\mathrm{Pyro}}$ und $T_{\mathrm{Lit}}$},
        quenching_time.pgf/{Abkühlzeit der Vakuumkammer}
    }{
        \begin{subfigure}[t]{0.49\textwidth}
            \import{../plots/calibration}{\i}
            \caption{\desc}
            \label{fig:\i}
        \end{subfigure}
    }
    \caption{Abhängigkeiten zur Temperaturbestimmung der Vakuumkammer}
    \label{fig:temperature_calibration_1}
\end{figure}
Zentrales Thema der vorliegenden Arbeit ist das Ausheizen und anschließende Charakterisieren der \heo\ Dünnfilme.
Dafür wird unter anderem die Vakuumkammer aus \cref{subsec:ausheiz} verwendet.
In dieser ist ein Heizlaser verbaut, welcher auf die Rückseite des Substrathalters
gerichtet ist.
Durch ein Pyrometer wird die Temperatur auf der Rückseite des Substrathalters gemessen.
Die relevante Temperatur ist jedoch die des Dünnfilms auf der Vorderseite des Substrathalters, welche durch die
Atmosphäre, die Geometrie und Wärmeleitfähigkeit des Substrathalters und des Substrats beeinflusst wird.

Um diese abzuschätzen, wurde von Tim Düvel ein spezieller Temperatursensor hergestellt.
Dieser besteht aus einem c-Saphir Substrat, auf dem eine Maske für die Platinbahnen infolge eines
Lithografieprozesses aufgebracht wurde.
Das Substrat wurde zunächst für \qty{10}{\second} mit Platinoxid für und anschließend für \qty{120}{\second} mit Platin
im Sputterverfahren beschichtet.
Nach dem Ablösen der Maske wurde das Substrat durch das gleiche Verfahren an beiden Enden der Platinbahnen mit Gold beschichtet.
Zum Schutz der Platinbahnen vor mechanischer Beschädigung wurde eine Schicht Aluminiumoxid aufgebracht.
Dieser Sensor wird im Folgenden als Lithografiesensor bezeichnet.

In Zusammenarbeit mit Tim Düvel wurde eine Kalibrierung des Lithografiesensors durchgeführt.
Dazu wurde ein PT1000-Temperatursensor genutzt, für welchen eine bekannte quadratische Abhängigkeit zwischen Widerstand
$R_{\mathrm{Pt}}$ und Temperatur $T_{\mathrm{Pt}}$ existiert\autocite{din_pt}:
\begin{equation}
    R_{\mathrm{Pt}}(T_{\mathrm{Pt}})
    =R_0 \cdot (1 + A \cdot T_{\mathrm{Pt}} + B \cdot T_{\mathrm{Pt}}^2).
    \label{eq:pt1000_calibration}
\end{equation}
Hierbei ist $R_0 = \qty{1000}{\ohm}$ der Widerstand bei \qty{0}{\degreeCelsius},
$A = \qty{3.9083e-3}{\degreeCelsius^{-1}}$ und $B = \qty{-5.775e-7}{\degreeCelsius^{-2}}$.
Der PT1000-Temperatursensor wurde mithilfe von Wärmeleitpaste thermisch an den Lithografiesensor gekoppelt und
in einem Muffelofen ausgeheizt.
Der Widerstand des Lithografiesensors $R_\mathrm{Lit}$ und der des PT1000-Temperatursensors $R_\mathrm{Pt}$
wurden in Abhängigkeit der Zeit gemessen und sind in Abbildung \cref{fig:furnace_calibration_1.pgf} dargestellt.
Aus der Parametrisierung $t \to (R_\mathrm{Lit}(t), R_\mathrm{Pt}(t))$ kann die Abhängigkeit beider Widerstände
voneinander bestimmt werden.
Aufgrund der gleichen Materialbeschaffenheit ist die Abhängigkeit $R_\mathrm{Lit}(R_\mathrm{Pt})$ linear und kann durch
einen linearen Fit der Form $f(x)=mx+n$ beschrieben werden, siehe \cref{fig:furnace_calibration_2.pgf}.
Für die Fitparameter ergibt sich:
\begin{equation*}
    m = \num{0.501(0.0002)} \quad n = \qty{-0.176(0.001)}{\kilo\ohm}.
\end{equation*}

Im nächsten Schritt wurde der Lithografiesensor in den Probenhalter der Vakuumkammer eingebaut und eine
zweite Messung durchgeführt.
Dabei wird die Temperatur des Heizlasers eingestellt und nach einer festen Zeitspanne die Pyrometertemperatur
$T_\mathrm{Pyro}$ und der Lithografiewiderstand $R_\mathrm{Lit}$ gemessen.
Es wurden zwei separate Messreihen durchgeführt: Eine bei steigender Temperatur und eine bei abfallender Temperatur.
Auch hier zeigt sich eine lineare Abhängigkeit zwischen beiden Größen, dessen Parameter durch einen linearen Fit
ermittelt werden können, siehe \cref{fig:a_chamber_calibration.pgf}.
Für die Fitparameter ergibt sich:
\begin{equation*}
    m = \qty{0.0067(0.0002)}{\kilo\ohm\per\degreeCelsius} \quad n = \qty{2.6(0.1)}{\kilo\ohm}.
\end{equation*}
Als Referenztemperatur, und damit als Temperatur des Dünnfilms, wird die Temperatur des PT1000-Sensors angesehen.
Diese kann durch folgende Gesamtfunktion bestimmt werden:
\begin{equation}
    T_{\mathrm{Pt}}=\underbrace{ T_{\mathrm{Pt}}(R_{\mathrm{Pt}}) }_{
        \substack{\text{quadratische} \\ \text{Abhängigkeit}}}
    =T_{\mathrm{Pt}}(\underbrace{ R_{\mathrm{Pt}}(R_{\mathrm{Lit}}) }_{
        \substack{\text{lineare} \\ \text{Abhängigkeit}}  })
    =T_{\mathrm{Pt}}(R_{\mathrm{Pt}}(\underbrace{ R_{\mathrm{Lit}}(T_{\mathrm{Pyro}}) }_{
        \substack{\text{lineare} \\ \text{Abhängigkeit}}  }))
    \label{eq:temperature_calibration}
\end{equation}
Da alle Funktionen bekannt sind, zeigt sich folgender Zusammenhang zwischen $T_{\mathrm{Pyro}}$ und $T_{\mathrm{Pt}}$
in \cref{fig:final_calibration.pgf}.
Der Graph zeigt, dass die Temperatur des Dünnfilms mit einer Unsicherheit von circa \qtyrange{10}{15}{\degreeCelsius}
der Temperatur des Pyrometers entspricht.

Wichtig für das Ausheizen ist außerdem die Abkühlzeit der Vakuumkammer.
Um diese zu untersuchen, wurde der Lithografiesensor eingebaut und das der Heizlaser auf \qty{350}{\degreeCelsius}
eingestellt.
Nachdem sich ein konstanter Lithografiewiderstand eingestellt hatte, wurde der Heizlaser abgeschaltet und der
Widerstand in Abhängigkeit der Zeit gemessen, siehe \cref{fig:quenching_time.pgf}.
Für einen Temperaturabfall von circa \qty{350}{\degreeCelsius} auf circa \qty{25}{\degreeCelsius}
benötigt die Vakuumkammer etwa \qty{4}{\minute}.
Damit zeigt der Temperaturbereich, die Präzision des Pyrometers und die Abkühlzeit, dass
die Vakuumkammer für die Ausheizstudien geeignet ist.

Es ist zu beachten, dass dieser Sensor bestmöglich die Thermodynamik von c-Saphir Substraten erfasst.
Die in dieser Arbeit betrachteten Dünnfilme wurden auf Eagle XG Glassubstraten abgeschieden.
Eagle XG hat bei Raumtemperatur eine Wärmeleitfähigkeit von \qty{1.09}{\watt\per\meter\per\kelvin}, wohingegen
c-Saphir eine Wärmeleitfähigkeit von \qty{42}{\watt\per\meter\per\kelvin} hat.
Auch bei einer Temperatur von \qty{300}{\degreeCelsius} hat Eagle XG eine Wärmeleitfähigkeit von
\qty{1.34}{\watt\per\meter\per\kelvin}, wohingegen c-Saphir eine Wärmeleitfähigkeit von
\qty{20}{\watt\per\meter\per\kelvin} hat.
Da die Wärmeleitfähigkeit von Eagle XG Glassubstraten um mehr als eine Größenordnung geringer ist als die von
c-Saphir Substraten, ist die Abschätzung der Temperatur des Dünnfilms durch den Lithografiesensor nur als
Approximation zu betrachten.
Würde man das c-Saphir Substrat des Sensors durch ein Eagle XG Glassubstrat ersetzen, würde die gemessene Temperatur
niedriger ausfallen.

\subsection{Konzentrationsanalyse}\label{subsec:edx-analyse}
Die Abscheidung mithilfe von PLD ist ein hochkomplexer Prozess, welcher nicht im thermodynamischen Gleichgewicht
stattfindet und damit schwer analytisch zu beschreiben ist.
Durch die komplexen Wechselwirkungen der Laserpulse mit den sechs verschiedenen Konstituenten des Targets
ist es schwer, Vorhersagen über die Stöchiometrie des Dünnfilms zu treffen.
Die Komposition des Dünnfilms muss nicht der des Targets entsprechen.
Zu diesem Anlass wurden von Jorrit Bredow ortsaufgelöste Konzentrationsaufnahmen der Proben \csamplethree, \csampleone,
\csampletwo, \csamplefour\ mithilfe von SEM-EDX durchgeführt.

Das Dünnfilmwachstum ist nicht nur abhängig von den Prozessparametern, sondern auch vom gewählten Substrat.
Idealerweise sollte die Konzentrationsanalyse demnach auf demselben Substrat durchgeführt werden, welches auch im
Ausheizprozess verwendet wurde.
Problematisch ist, dass die genaue Zusammensetzung des Eagle XG Glassubstrats nicht bekannt ist.
Es ist jedoch bekannt, dass Magnesium ein Konstituent des Substrats ist.
Da auch der Dünnfilm Magnesium enthält, ist es dadurch nicht möglich, das Magnesium des Dünnfilms vom Magnesium des
Substrats zu unterscheiden.
Daher wurde ein c-Saphir Substrat verwendet, welches keine Metallkationen des Dünnfilms enthält.

Da SEM-EDX Messungen nur auf leitenden Proben durchgeführt werden können und die Dünnfilmee isolierende Eigenschaften
aufweisen, wurden die c-Saphir Proben vor der Messung mit einer dünnen Schicht Kohlenstoff beschichtet.
\cref{fig:edx_map} zeigt die Oberfläche der Probe \csamplethree\ unter der Kohlenstoffschicht.
Erkennbar ist eine glatte Morphologie ohne sichtbare Verunreinigungen.
Die \cref{fig:edx_Mg,fig:edx_Co,fig:edx_Ni,fig:edx_Cu,fig:edx_Zn} zeigen die Konzentrationen der
Elemente Magnesium, Cobalt, Nickel, Kupfer und Zink in Abhängigkeit der Position.
Über den gesamten Dünnfilm ist eine gleichmäßige Verteilung alle Elemente erkennbar.
Es breiten sich keine erkennbaren Cluster aus, welche auf eine Phasentrennung hindeuten würden.
Der Dünnfilm wurde erfolgreich und homogen abgeschieden.
Hierbei ist wichtig zu betonen, dass die Konzentrationsanalyse in einer Skala von $\qty{25.6}{\micro\meter} \times
\qty{25.6}{\micro\meter}$ durchgeführt wurde.
Da keine atomare Auflösung erreicht wurde, können Cluster auf kleineren Skalen nicht ausgeschlossen werden.

Die ortsabhängige Konzentrationsanalyse von \csamplethree\ wurde repräsentativ für alle Proben gezeigt.
Die Probe \csampleone, \csampletwo, \csamplefour\ zeigen sehr ähnliche Ergebnisse und sind im Anhang zu finden.
Auch die Konzentrationsverhältnisse der c-Saphir Proben lassen sich mithilfe dieser Aufnahmen bestimmen.
Die Ergebnisse sind in \cref{tab:concentration} aufgeführt.
\begin{table}[h]
    \centering
    \begin{tabular}{l l l l l l}
        \toprule
        Probe & \ce{Mg} in \unit{\percent} & \ce{Co} in \unit{\percent} & \ce{Ni} in \unit{\percent}&
        \ce{Cu} in \unit{\percent}& \ce{Zn} in \unit{\percent}\\
        \midrule
        \csamplethree & \num{17.62} & \num{19.08} & \num{20.06} & \num{22.52} & \num{20.72} \\
        \csampleone   & \num{16.50} & \num{19.01} & \num{20.04} & \num{23.29} & \num{21.16} \\
        \csampletwo   & \num{16.13} & \num{19.56} & \num{20.79} & \num{22.66} & \num{20.86} \\
        \csamplefour  & \num{15.03} & \num{20.93} & \num{21.32} & \num{21.61} & \num{21.12} \\
        \bottomrule
    \end{tabular}
    \caption{Konzentrationen der Elemente \ce{Mg}, \ce{Co}, \ce{Ni}, \ce{Cu} und \ce{Zn} in den Proben \csamplethree,
        \csampleone, \csampletwo, \csamplefour}
    \label{tab:concentration}
\end{table}

Erkennbar ist ist eine leichte Abweichung der Konzentrationen der Elemente in den Proben.
Vor allem die Konzentration von Magnesium ist geringer als die der anderen Elemente und nimmt mit fallendem Druck
ab.
Durch die Abweichung der Konzentration von der equimolaren Zusammensetzung ist die Mischentropie des Dünnfilms
nicht maximal.
Dadurch ist eine höhere Temperatur nötig, um die entropiestabilisierte Phase zu erreichen.

Auch wenn Sauerstoff nicht direkt mithilfe von EDX gemessen werden kann, so liegt die Vermutung nahe, dass es
trotzdem in den Dünnfilmen vorhanden ist.
Wäre dies nicht der Fall, so entstände durch die Metalle ein leitender Dünnfilm.
Da dieser jedoch isolierende Eigenschaften hat, ist davon auszugehen, dass Sauerstoff in den Dünnfilmen vorhanden ist.


\begin{figure}
    \centering
    \foreach \i/\desc in {map/Oberfläche, Mg/Magnesium, Co/Kobalt, Ni/Nickel, Cu/Kupfer, Zn/Zink}{
        \begin{subfigure}[t]{0.40\textwidth}
            \includegraphics[width=\textwidth]{../plots/EDX/W6823-3D/\i}
            \caption{\desc}
            \label{fig:edx_\i}
        \end{subfigure}
    }
    \caption{EDX Aufnahmen der Probe \csamplethree}
    \label{fig:edx1}
\end{figure}

% AUSHEIZVORGÄNGE %

\subsection{Glas}\label{subsec:glas}


\newcommand{\temperaturesS}{pre,600,700,750,800,875}
\newcommand{\temperaturesV}{pre,500,600,700,750, 800, 875}
\newcommand{\temperaturesVthree}{pre,500,600,700}
\newcommand{\temperatureVfour}{pre, 500, 600, 700, 750, 800}
\newcommand{\temperaturesL}{pre,600, 700, 750, 800, 875}

% PROBE W6823-1 %

\newpage

\subsection{Probe \samplethree}\label{subsec:probe-W6823-1}

\subsubsection{Sauerstoff Ausheizvorgang}\label{subsubsec:W6823-1B_Sauerstoff}
\begin{figure}
    \centering
    \import{../plots/XRD}{W6823-1B_Sauerstoff.pgf}
    \caption{$2\theta/\omega$ Diffraktogramme der Sauerstoff-Ausheizserie von Probe \samplethree.
    Grau hinterlegt sind Peaks des Probenhalters.}
    \label{fig:W6823-1B_Sauerstoff_XRD}
\end{figure}
\cref{fig:W6823-1B_Sauerstoff_XRD} zeigt $2\theta/\omega$-Diffraktogramme der Probe \samplethree\ zu ausgewählten
Temperaturen während des Sauerstoff-Ausheizvorgangs.
Bei allen Temperaturen ist ein ausgeweitetes Maximum bei circa \qty{23}{\degree} zu erkennen.
Dieses ist auf das Eagle XG Glassubstrat zurückzuführen.
Weiterhin ist ein scharf definierter Peak bei \qty{44.32}{\degree} zu erkennen, welcher durch die
Kristallstruktur des Probenhalters verursacht wird.
Der Probenhalter sorgt für weitere, kaum sichtbare Peaks bei \qtylist{64.66; 81.96; 98.61; 115.89}{\degree}.
Diese sind in den Diffraktogrammen grau hinterlegt.
Somit sind trotz der bis zu einer Temperatur von \qty{875}{\degreeCelsius} eingestellten Ausheizprozesse keine Peaks
des \heo\ Dünnfilms zu erkennen.
Dieser ist somit röntgenamorph.

Die AFM-Topografieaufnahmen zu verschiedenen Temperaturen der Probe \samplethree\ während des Sauerstoff Ausheizvorgangs
sind in \cref{fig:W6823-1B_Sauerstoff_AFM} dargestellt.
Die Aufnahme des Initialzustands, \cref{W6823-1B_Sauerstoff_AFM_pre}, zeigt einen ebenen Untergrund mit zahlreichen
zufällig orientierten Erhebungen.
Aus \cref{subsec:glas} ist bekannt, dass sich die Rauheit des Eagle XG Glassubstrats nach der Zersägung von
\qty{383}{\pico\meter} auf \qty{470}{\pico\meter} erhöht.
Bei stichprobenartigen Maskierungen und separaten Auswertungen des ebenen Untergrunds zeigt sich eine durchschnittliche
Rauheit von \qty{2.3}{\nano\meter}.
Die durchschnittliche Höhe der großen Kristallite beträgt \qtyrange{50}{60}{\nano\meter}.
Nach Angaben der XRR Messungen beträgt die Dicke des Films cira \qty{109}{\nano\meter}, woraus sich schließen lässt,
dass der ebene Untergrund ein flächig gewachsener Dünnfilm ist.
Die Erhebungen sind somit einzelne Kristallite des Dünnfilms.
Das Stranski-Krastanov-Wachstum erklärt die beobachtete Kristallmorphologie.
Bis zu einer kritischen Dicke bildet sich eine kontinuierliche Schicht, nach welcher dreidimensionale Inseln wachsen.

Nach dem ersten Ausheizschritt bei \qty{600}{\degreeCelsius}, siehe \cref{W6823-1B_Sauerstoff_AFM_600}, ist ein
deutlicher Unterschied in der Morphologie des Dünnfilms zu erkennen.
Die Kristallite sind in ihrer Fläche gleich geblieben, jedoch sind sie deutlich niedriger als im Initialzustand.
Das Maximum der Höhenskala hat sich von \qty{68}{\nano\meter} auf \qty{30.7}{\nano\meter} verringert.
Ein Erklärversuch dieses Phänomens ist die Evaporation von Atomen, welche die Kristallite schrumpfen lässt.

Des Weiteren sind Risse in der Oberfläche zu erkennen.
Diese Risse werden nicht auf reinen Eagle XG Glassubstraten beobachtet und sind damit auf den Dünnfilm zurückzuführen.
Gründe dafür können Spannungen im Dünnfilm und im Substrat sein, welche durch die Temperatur und die Substrathalterung
verursacht werden.
Da \samplethree\ die geringste Schichtdicke der Serie aufweist, ist sie am anfälligsten für Spannungen.

Die Topografieaufnahme nach Ausheizen auf \qty{700}{\degreeCelsius} in \cref{W6823-1B_Sauerstoff_AFM_700} ähnelt
der Aufnahme bei \qty{600}{\degreeCelsius}.
Erkennbar ist eine Verringerung der durchschnittlichen Fläche der kaum noch sichtbaren Kristallite.
Die Risse in der Oberfläche sind weiterhin vorhanden und haben sich vergrößert.
Auch ein großer Riss ist zu erkennen, welcher sich über die gesamte Aufnahme erstreckt.
Nach dem Ausheizen bei \qty{750}{\degreeCelsius} in \cref{W6823-1B_Sauerstoff_AFM_750} sind die Kristallite
vollständig verschwunden.
Erkennbar sind deutliche Risse in der Oberfläche, welche sich weiter vergrößert haben.

Selbst bei einer Temperatur von \qty{800}{\degreeCelsius} setzen die Risse in Abbildung 15e ihre Ausdehnung fort.
Ebenso ist eine Zunahme der maximalen Skalenhöhe zu beobachten.
Bei dem finalen Ausheizschritt bei \qty{875}{\degreeCelsius} in \cref{W6823-1B_Sauerstoff_AFM_875} sind die Risse
weiterhin vorhanden, haben sich jedoch verkleinert.
Es prägt sich eine feingliedrige Rissstruktur aus.

\begin{figure}
    \centering
    \foreach \i in \temperaturesS{
        \begin{subfigure}[t]{0.40\textwidth}
            \includegraphics[width=\textwidth]
            {../plots/AFM/XG-Sauerstoff/XG-\i/W6823-1B/W6823-1B_XG_Sauerstoff_\i_Topography_1}
            \caption{\ifthenelse{\equal{\i}{pre}}{Initialzustand}{\qty{\i}{\degreeCelsius}}}
            \label{W6823-1B_Sauerstoff_AFM_\i}
        \end{subfigure}
    }
    \caption{W6823-1B, Sauerstoff, AFM}
    \caption{Topografieaufnahmen der Sauerstoff-Ausheizserie von Probe \samplethree.}
    \label{fig:W6823-1B_Sauerstoff_AFM}
\end{figure}
\newpage

\subsubsection{Vakuum Ausheizvorgang}\label{subsubsec:W6823-1C_Vakuum}
\begin{figure}
    \centering
    \import{../plots/XRD}{W6823-1C_Vakuum.pgf}
    \caption{$2\theta/\omega$ Diffraktogramme der Vakuum-Ausheizserie von Probe \samplethree.
    Grau hinterlegt sind Peaks des Probenhalters.}
    \label{fig:W6823-1C_Vakuum_XRD}
\end{figure}
\cref{fig:W6823-1C_Vakuum_XRD} zeigt die Diffraktogramme der Probe \samplethree\ bei
ausgewählten Temperaturen des Vakuum-Ausheizvorgangs.
Anstelle der festgelegten sieben Temperaturen wurden nur vier Temperaturen untersucht.
Grund dafür war ein Fehler in der Substratmontage.
Eine vorangegangene Messung der Vakuumkammer nutzte Wärmeleitpaste mit Silbernanopartikeln.
Die trotz Reinigung verbliebenen Rückstände und die umgekehrte Einbaurichtung des Substrats
führten zu einer Kontamination des Dünnfilms.
Da sich die Paste auf der Oberfläche verfestigt, kann die Probe nicht mehr verwendet werden, da jegliche
Vergleichbarkeit verloren geht.

Die übrigen Temperaturen weisen fast identische Diffraktogramme auf, wie diejenigen aus
\cref{subsubsec:W6823-1B_Sauerstoff}.
Weiterhin sind die einzigen Peaks auf das Eagle XG Glassubstrat und den Probenhalter zurückzuführen.
Auch dieser \heo\ Dünnfilm ist weiterhin röntgenamorph.

Die AFM-Topografieaufnahmen bei verschiedenen Temperaturen der Probe \samplethree\ während des Vakuum-Ausheizvorgangs
sind in \cref{fig:W6823-1C_Vakuum_AFM} dargestellt.
Die Aufnahme des Initialzustands in \cref{fig:W6823-1C_Vakuum_AFM_pre} zeigt einen deutlichen Unterschied zu dem
Initialzustand aus \cref{subsubsec:W6823-1B_Sauerstoff}.
Anstelle der flächig verteilten Kristallite auf einem ebenen Untergrund sind hier große Unebenheiten
über das gesamte Bild verteilt.
Auf den Bergen des Untergrundes sind einzelne Kristallite zu erkennen.
Es zeigt sich möglicherweise ein Inselwachstum in einer größeren Skala.

Nach erstem Ausheizen auf \qty{500}{\degreeCelsius} in \cref{fig:W6823-1C_Vakuum_AFM_500} ändert sich die Morphologie
erneut.
Es ist kein unebener Untergrund mehr zu erkennen.
Stattdessen sind einzelne Kristallite auf einem ebenen Untergrund erkennbar.
Für die Rauheit des ebenen Untergrunds findet man\qty{350}{\pico\meter}.
Damit ist diese Rauheit vergleichbar mit der von Eagle XG Glassubstraten.
Es liegt nahe, dass der Dünnfilm nicht flächig gewachsen ist, sondern aus einzelnen Kristalliten besteht.
Nach dem Ausheizen auf \qty{600}{\degreeCelsius} in \cref{fig:W6823-1C_Vakuum_AFM_600} sind die Kristallite in ihrer
Fläche deutlich kleiner.
Auch bei \qty{700}{\degreeCelsius} in \cref{fig:W6823-1C_Vakuum_AFM_700} verkleinerten sich die Kristallite
in ihrer Fläche und Höhe.
Unter den Bedingungen des Ausheizvorgangs im Vakuum ist eine Verkleinerung der Kristallite aufgrund der Evaporation zu
erwarten.
Das sorgt dafür, dass kaum noch Dünnfilmatome auf der Oberfläche verbleiben.

\begin{figure}
    \centering
    ,\foreach \i in \temperaturesVthree{
        \begin{subfigure}[t]{0.40\textwidth}
            \includegraphics[width=\textwidth]
            {../plots/AFM/XG-Vakuum/XG-\i/W6823-1C/W6823-1C_XG_Vakuum_\i_Topography_1}
            \caption{\ifthenelse{\equal{\i}{pre}}{Initialzustand}{\qty{\i}{\degreeCelsius}}}
            \label{fig:W6823-1C_Vakuum_AFM_\i}
        \end{subfigure}
    }
    \caption{Topografieaufnahmen der Vakuum-Ausheizserie von Probe \samplethree.}
    \label{fig:W6823-1C_Vakuum_AFM}
\end{figure}
\newpage

\subsubsection{Luft Ausheizvorgang}\label{subsubsec:W6823-1D_Luft}
\begin{figure}
    \centering
    \import{../plots/XRD}{W6823-1D_Luft.pgf}
    \caption{$2\theta/\omega$ Diffraktogramme der Luft-Ausheizserie von Probe \samplethree.
    Grau hinterlegt sind Peaks des Probenhalters.}
    \label{fig:W6823-1D_Luft_XRD}
\end{figure}
In \cref{fig:W6823-1D_Luft_XRD} sind die $2\theta/\omega$-Scans der Probe \samplethree\ zu den festgelegten Temperaturen
während des Luft-Ausheizvorgangs dargestellt.
Wie in den vorherigen Abschnitten sind nur die Peaks des Eagle XG Glassubstrats und des Probenhalters zu erkennen.
Damit ist der \heo\ Dünnfilm weiterhin röntgenamorph.

Die AFM-Topografiekarten dieses Ausheizvorgangs sind in \cref{fig:W6823-1D_Luft_AFM} dargestellt.
Der Initialzustand in \cref{fig:W6823-1D_Luft_AFM_pre} zeigt rundliche Kristallite, welche zufällig angeordnet über eine
ebene Oberfläche verteilt sind.
Die Morphologie ähnelt dem Initialzustand des Sauerstoff Ausheizvorgangs in \cref{W6823-1B_Sauerstoff_AFM_pre},
die Kristallite sind jedoch bezüglich ihrer Fläche und Höhe kleiner.
Die Rauheit des ebenen Untergrund beträgt \qty{2.3}{\nano\meter}.
Die größeren Kristallite haben eine durchschnittliche Höhe von \qtyrange{40}{50}{\nano\meter}.
Da mithilfe des Profilometers eine Schichtdicke von \qty{109}{\nano\meter} bestimmt wurde, bestätigt auch diese
Messung des Initialzustands das bereits beschriebene Stranski-Krastanov-Wachstum.

Nach dem Ausheizen auf \qty{600}{\degreeCelsius} in \cref{fig:W6823-1D_Luft_AFM_600} sind die Kristallite vollständig
verschwunden.
Stattdessen liegt eine homogene Oberfläche vor.
Die Rauheit dieser Oberfläche beträgt \qty{2.1}{\nano\meter} und ist damit
vergleichbar mit der Rauheit des ebenen Untergrunds des Initialzustands, was die Annahme der Evaporation der Kristallite
bestätigt.
Die Aufnahme bei \qty{700}{\degreeCelsius} in \cref{fig:W6823-1D_Luft_AFM_700} zeichnet sich durch eine feingliedrige
Rissstruktur aus, die sich über die gesamte Aufnahme erstreckt.
Dadurch nimmt auch die maximale Höhe der Skala zu.
In der Aufnahme von \qty{750}{\degreeCelsius} in \cref{fig:W6823-1D_Luft_AFM_750} vergrößern sich die Risse weiter.
Bei \qty{800}{\degreeCelsius} und \qty{875}{\degreeCelsius} in
\cref{fig:W6823-1D_Luft_AFM_800,fig:W6823-1D_Luft_AFM_875} sind die Risse weiterhin vorhanden, haben sich jedoch
verkleinert.
Die Aufnahme nach \qty{875}{\degreeCelsius} zeigt eine noch feingliedrigere Rissstruktur als die Aufnahme
bei \qty{700}{\degreeCelsius}.
Erneut sind die Risse auf Volumenänderungen durch thermisch-induzierte Spannungen und möglicherweise auch auf
Phasenübergänge zurückzuführen.
Da der Effekt unabhängig vom Substrathalter auftritt, kann auf eine materialspezifische Reaktion geschlussfolgert werden.
Die Änderung der Morphologie des Dünnfilms während des Luft Ausheizvorgangs ist Vergleichbar mit der des Sauerstoffs
in \cref{subsubsec:W6823-1B_Sauerstoff}.

\begin{figure}[h]
    \centering
    ,\foreach \i in \temperaturesL{
        \begin{subfigure}[t]{0.40\textwidth}
            \includegraphics[width=\textwidth]
            {../plots/AFM/XG-Luft/XG-\i/W6823-1D/W6823-1D_XG_Luft_\i_Topography_1}
            \caption{\ifthenelse{\equal{\i}{pre}}{Initialzustand}{\qty{\i}{\degreeCelsius}}}
            \label{fig:W6823-1D_Luft_AFM_\i}
        \end{subfigure}
    }
    \caption{Topografieaufnahmen der Luft-Ausheizserie von Probe \samplethree.}
    \label{fig:W6823-1D_Luft_AFM}
\end{figure}
\newpage


% PROBE W6821-1 %

\newpage

\subsection{Probe \sampleone}\label{subsec:probe-W6821-1}

\subsubsection{Sauerstoff Ausheizvorgang}\label{subsubsec:W6821-1B_Sauerstoff}
\begin{figure}
    \centering
    \import{../plots/XRD}{W6821-1B_Sauerstoff.pgf}
    \caption{$2\theta/\omega$ Diffraktogramme der Sauerstoff-Ausheizserie von Probe \sampleone.
    Grau hinterlegt sind Peaks des Probenhalters.}
    \label{fig:W6821-1B_Sauerstoff_XRD}
\end{figure}
In \cref{fig:W6821-1B_Sauerstoff_XRD} sind die Diffraktogramme der Probe \sampleone\ zu ausgewählten
Temperaturen der Ausheizserie in Sauerstoff dargestellt.
Wie in den vorherigen Abschnitten sind die Peaks auf das Eagle XG Glassubstrat und den Probenhalter zurückzuführen.
Es sind keine Peaks des \heo\ Dünnfilms zu erkennen.
Der Dünnfilm ist röntgenamorph.

\cref{fig:W6821-1B_Sauerstoff_AFM} zeigt die AFM-Topografieaufnahmen der Probe \sampleone\ zu den verschiedenen
Temperaturen.
Die Aufnahme des Initialzustands in \cref{W6821-1B_Sauerstoff_AFM_pre} zeigt viele zufällig angeordnete Kristallite,
welche dicht beieinander liegen.
Die Höhe der großen Kristallite beträgt circa \qtyrange{45}{55}{\nano\meter}.
Da mithilfe des Profilometers eine Filmdicke von circa \qty{95}{\nano\meter} bestimmt wurde, ist es naheliegend,
dass unterhalb der Kristallite ein flächig gewachsener Dünnfilm liegt.
Auch hier weist die Morphologie auf ein Stranski-Krastanov-Wachstum hin.

Nach dem ersten Ausheizschritt auf \qty{600}{\degreeCelsius} in \cref{W6821-1B_Sauerstoff_AFM_600} hat sich die
Oberflächenmorphologie deutlich verändert.
Die Kristallitgröße hat sich signifikant verringert, während sich deren Dichte vergrößert hat.
Auch die maximale Höhe der Skala hat sich von \qty{62}{\nano\meter} auf \qty{42.7}{\nano\meter} verringert.
Die Topografie bei \qty{700}{\degreeCelsius} in \cref{W6821-1B_Sauerstoff_AFM_700} weist im Vergleich zu
\qty{600}{\degreeCelsius} eine geringere Dichte an Kristalliten auf.
Zudem sind erste Löcher in der Oberfläche sichtbar.

Die Oberflächenmorphologien von \qty{750}{\degreeCelsius} in \cref{W6821-1B_Sauerstoff_AFM_750} und
\qty{800}{\degreeCelsius} in \cref{W6821-1B_Sauerstoff_AFM_800} ähneln sich.
Die Kristallite sind im Gegensatz zu der Aufnahme bei \qty{700}{\degreeCelsius} deutlich niedriger.
Zusätzlich ist die Anzahl der Löcher in der Oberfläche gestiegen.
Diese Löcher sind auf den Dünnfilm zurückzuführen und sind nicht auf reinen Eagle XG Glassubstraten zu beobachten.
Da sie bezüglich ihrer Fläche in der Größenordnung der Kristallite liegen, ist es naheliegend, dass die Kristallite
vollständig evaporieren.

Die Aufnahme nach dem Ausheizen bei \qty{875}{\degreeCelsius} in \cref{W6821-1B_Sauerstoff_AFM_875} zeigt zusätzliche
Deformationen auf einer größeren Skala.
\begin{figure}[h]
    \centering
    \foreach \i in \temperaturesS{
        \begin{subfigure}[t]{0.40\textwidth}
            \includegraphics[width=\textwidth]
            {../plots/AFM/XG-Sauerstoff/XG-\i/W6821-1B/W6821-1B_XG_Sauerstoff_\i_Topography_1}
            \caption{\ifthenelse{\equal{\i}{pre}}{Initialzustand}{\qty{\i}{\degreeCelsius}}}
            \label{W6821-1B_Sauerstoff_AFM_\i}
        \end{subfigure}
    }
    \caption{Topografieaufnahmen der Sauerstoff-Ausheizserie von Probe \sampleone.}
    \label{fig:W6821-1B_Sauerstoff_AFM}
\end{figure}
\newpage

\subsubsection{Vakuum Ausheizvorgang}\label{subsubsec:W6821-1C_Vakuum}
\begin{figure}
    \centering
    \import{../plots/XRD}{W6821-1C_Vakuum.pgf}

    \caption{$2\theta/\omega$ Diffraktogramme der Vakuum-Ausheizserie von Probe \sampleone.
    Grau hinterlegt sind Peaks des Probenhalters.}
    \label{fig:W6821-1C_Vakuum_XRD}
\end{figure}
In \cref{fig:W6821-1C_Vakuum_XRD} sind die $2\theta/\omega$-Diffraktogramme der Probe \sampleone\ für
verschiedene Temperaturen der Ausheizserie in Sauerstoff dargestellt.
Wie in den vorherigen Abschnitten beschrieben, resultieren die Peaks aus dem Eagle XG Glassubstrat sowie dem
Probenhalter.
Es sind keine Peaks des \heo\ Dünnfilms erkennbar.
Der Dünnfilm ist röntgenamorph.

Mithilfe des Rasterkraftmikroskops wurden Topografiekarten der Probe \sampleone\ während des Vakuum Ausheizvorgangs
aufgenommen, siehe \cref{fig:W6821-1C_Vakuum_AFM}.
Die Aufnahme des Initialzustands in \cref{W6821-1C_Vakuum_AFM_pre} zeigt einen ebenen Untergrund der mit vielen
zufällig orientierten Kristalliten bedeckt ist.
Die größeren Kristallite haben eine Höhe von circa \qtyrange{60}{75}{\nano\meter}.
Zwischen diesen großen Kristalliten sind kleinere Kristallite zu, die den Untergrund bedecken.

Die \qty{500}{\degreeCelsius} Aufnahme in \cref{W6821-1C_Vakuum_AFM_500} weißt einen ebenen Untergrund auf,
der nicht flächig mit Erhebungen bedeckt ist.
Diese Erhebungen liegen bezüglich ihrer Fläche in der Größenordnung der Kristallite des Initialzustands.
Die Höhe der Erhebungen ist jedoch deutlich geringer, sodass erneut die Evaporation von Atomen als Ursache
angenommen werden kann.
Die Rauheit des ebenen Untergrunds beträgt circa \qty{1.5}{\nano\meter}, sodass davon ausgegangen werden kann,
dass der Film flächig gewachsen ist.

Die Aufnahme bei \qty{600}{\degreeCelsius} in \cref{W6821-1C_Vakuum_AFM_600} zeigt eine homogene Oberfläche, ohne
Erhebungen.
Auch die Aufnahme bei \qty{700}{\degreeCelsius} in \cref{W6821-1C_Vakuum_AFM_700} ist weitesgehend homogen,
einzelne Kristallite scheinen sich jedoch gebildet zu haben.
Diese sind auch bei \qty{750}{\degreeCelsius} in \cref{W6821-1C_Vakuum_AFM_750} zu erkennen.
Die Kristallite sind in ihrer Fläche und Höhe deutlich größer als bei \qty{700}{\degreeCelsius},
jedoch kleiner als im Initialzustand.
Auch in dieser Aufnahme sind Löcher in der Oberfläche zu erkennen.
Die Löcher weisen darauf hin, dass der Untergrund der Dünnfilm ist und nicht das Eagle XG Glassubstrat.
Die Aufnahme bei \qty{800}{\degreeCelsius} in \cref{W6821-1C_Vakuum_AFM_800} zeigt
deutlich weniger Kristallite und Löcher.
Einen Sonderfall stellt die Aufnahme bei \qty{875}{\degreeCelsius} in \cref{W6821-1C_Vakuum_AFM_875} dar.
Diese enthält viele Krisallite mit dreieckiger Form und einheitlicher Orientierung.
Dies deutet auf eine abgebrochene oder beschädigte Cantilever-Spitze hin.
Trotz mehrmaliger Messung und Wechsel der Spitze, einem geringeren Setpoint und einer geringeren Scanrate, blieb die
Struktur erhalten.
Eine Verfälschung dieser Aufnahme kann jedoch nicht ausgeschlossen werden.
\begin{figure}
    \centering
    ,\foreach \i in \temperaturesV{
        \begin{subfigure}[t]{0.40\textwidth}
            \includegraphics[width=\textwidth]
            {../plots/AFM/XG-Vakuum/XG-\i/W6821-1C/W6821-1C_XG_Vakuum_\i_Topography_1}
            \caption{\ifthenelse{\equal{\i}{pre}}{Initialzustand}{\qty{\i}{\degreeCelsius}}}
            \label{W6821-1C_Vakuum_AFM_\i}
        \end{subfigure}
    }
    \caption{Topografieaufnahmen der Vakuum-Ausheizserie von Probe \sampleone.}
    \label{fig:W6821-1C_Vakuum_AFM}
\end{figure}
\newpage

\subsubsection{Luft Ausheizvorgang}\label{subsubsec:W6821-1D_Luft}
\begin{figure}
    \centering
    \import{../plots/XRD}{W6821-1D_Luft.pgf}
    \caption{$2\theta/\omega$ Diffraktogramme der Luft-Ausheizserie von Probe \sampleone.
    Grau hinterlegt sind Peaks des Probenhalters.}
    \label{fig:W6821-1D_Luft_XRD}
\end{figure}
In \cref{fig:W6821-1D_Luft_XRD} sind die $2\theta/\omega$-Diffraktogramme der Probe \sampleone\ dargestellt,
wobei die Messungen für festgelegte Temperaturen während der Ausheizphase in Luft durchgeführt wurden.
Wie bereits in den vorherigen Abschnitten erläutert, resultieren die ermittelten Peaks aus dem Eagle XG Glassubstrat
sowie dem Probenhalter.
Es sind keine Peaks des \heo\ Dünnfilms erkennbar sind, was darauf hindeutet, dass der Dünnfilm eine röntgenamorphe
Struktur aufweist.

Die AFM-Topografieaufnahmen der Probe \sampleone\ während des Luft Ausheizvorgangs sind in \cref{fig:W6821-1D_Luft_AFM}
abgebildet.
Die Aufnahme des Initialzustands in \cref{fig:W6821-1D_Luft_AFM_pre}
unterscheidet sich deutlich von den anderen Aufnahmen.
Anstelle der üblichen $\qty{5}{\micro\meter} \times \qty{5}{\micro\meter}$ Aufnahmen wurde hier eine
$\qty{15}{\micro\meter} \times \qty{15}{\micro\meter}$ Aufnahme erstellt, um die Morphologie des Dünnfilms besser zu
erfassen.
Es sind viele großflächige und hohe Kristallite zu erkennen, die nicht in vorherigen Aufnahmen zu sehen waren.
Dabei nimmt ein Kristallit bereits eine Fläche von fast $\qty{5}{\micro\meter} \times \qty{5}{\micro\meter}$ ein.
Der Dünnfilm könnte nicht flächig gewachsen sein, sondern aus einzelnen Kristalliten bestehen.

Nach dem Ausheizen auf \qty{600}{\degreeCelsius} in \cref{fig:W6821-1D_Luft_AFM_600} sind die großen Kristallite
verschwunden, stattdessen sind viele bezüglich ihrer Fläche kleine, aber dennoch sehr hohe
Kristallite zu erkennen.
Nach \qty{700}{\degreeCelsius} in \cref{fig:W6821-1D_Luft_AFM_700} sind die Kristallite in ihrer Fläche gleich
geblieben, jedoch deutlich höher.
Bei \qty{750}{\degreeCelsius} in \cref{fig:W6821-1D_Luft_AFM_750} sind keine Kristallite nicht mehr zu sehen.
Stattdessen sind gleichmäßig verteilte Löcher zu erkennen, die bezüglich ihrer Fläche der Kristallitgröße entsprechen.
Das deutet darauf hin, dass die Kristallite evaporiert sind.
Die Morphologie bei \qty{800}{\degreeCelsius} und \qty{875}{\degreeCelsius} in
\cref{fig:W6821-1D_Luft_AFM_800,fig:W6821-1D_Luft_AFM_875} ähneln der aufnahme \cref{fig:W6821-1D_Luft_AFM_750}.
Bei \qty{875}{\degreeCelsius} sind zusätzlich leichte Verformungen des Untergrunds erkennbar,
was auf Verformungen des Substrats durch thermische Spannungen zurückzuführen ist.

\begin{figure}[h]
    \centering
    ,\foreach \i in \temperaturesL{
        \begin{subfigure}[t]{0.40\textwidth}
            \includegraphics[width=\textwidth]
            {../plots/AFM/XG-Luft/XG-\i/W6821-1D/W6821-1D_XG_Luft_\i_Topography_1}
            \caption{\ifthenelse{\equal{\i}{pre}}{Initialzustand}{\qty{\i}{\degreeCelsius}}}
            \label{fig:W6821-1D_Luft_AFM_\i}
        \end{subfigure}
    }
    \caption{Topografieaufnahmen der Vakuum-Ausheizserie von Probe \sampleone.}
    \label{fig:W6821-1D_Luft_AFM}
\end{figure}
\newpage

% PROBE W6822-1 %

\newpage

\subsection{Probe \sampletwo}\label{subsec:probe-W6822-1}

\subsubsection{Sauerstoff Ausheizvorgang}\label{subsubsec:W6822-1B_Sauerstoff}
\begin{figure}
    \centering
    \import{../plots/XRD}{W6822-1B_Sauerstoff.pgf}
    \caption{$2\theta/\omega$ Diffraktogramme der Sauerstoff-Ausheizserie von Probe \sampletwo.
    Grau hinterlegt sind Peaks des Probenhalters.}
    \label{fig:W6822-1B_Sauerstoff_XRD}
\end{figure}

Die Diffraktogramme der Probe \sampletwo\ nach den jeweiligen Ausheizschritten in Sauerstoff sind in
\cref{fig:W6822-1B_Sauerstoff_XRD} dargestellt.
Auch hier sind jegliche Peaks auf das Substrat und den Probenhalter zurückzuführen.
Der \heo\ Dünnfilm ist röntgenamorph.

\cref{fig:W6822-1B_Sauerstoff_AFM} zeigt die AFM-Topografieaufnahmen der Probe \sampletwo\ während des Sauerstoff
Ausheizvorgangs.
Diese zeigen, wie die Initialzustände der bisherigen Sauerstoff Ausheizvorgänge, verteilte Kristallite, die einen
ebenen Untergrund flächig bedecken.
Anders als die bisherigen Kristallite der Initialzustände sind diese weniger rund, sondern zerfaserter.
Die Rauheit des ebenen Untergrunds liegt bei circa \qty{1}{\nano\meter}, die Höhe der Kristallite zwischen
\qtyrange{20}{30}{\nano\meter}.
Da mithilfe von XRR eine Dünnfilmdicke von  \qty{140}{\nano\meter} bestimmt wurde, kann auf einen flächig
gewachsenen Dünnfilm geschlossen werden, analog zu \samplethree\ und \sampleone.
Nach dem Ausheizen auf \qty{600}{\degreeCelsius} in \cref{W6822-1B_Sauerstoff_AFM_600} sind alle Kristallite
verschwunden und eine homogene raue Oberfläche ist zu erkennen.
Mit \qty{700}{\degreeCelsius} beginnt die Bildung kleiner Kristallite in \cref{W6822-1B_Sauerstoff_AFM_700},
die bei \qty{750}{\degreeCelsius} bezüglich ihrer Fläche und Höhe wachsen, siehe \cref{W6822-1B_Sauerstoff_AFM_750}.
Auch bei \qty{800}{\degreeCelsius} und \qty{875}{\degreeCelsius} in
\cref{W6822-1B_Sauerstoff_AFM_800,W6822-1B_Sauerstoff_AFM_875} sind Kristallite erkennbar, die die Oberfläche
mittlerweile flächig überdecken und die Intervallskala deutlich erhöhen.

\begin{figure}
    \centering
    \foreach \i in \temperaturesS{
        \begin{subfigure}[t]{0.40\textwidth}
            \includegraphics[width=\textwidth]
            {../plots/AFM/XG-Sauerstoff/XG-\i/W6822-1B/W6822-1B_XG_Sauerstoff_\i_Topography_1}
            \caption{\ifthenelse{\equal{\i}{pre}}{Initialzustand}{\qty{\i}{\degreeCelsius}}}
            \label{W6822-1B_Sauerstoff_AFM_\i}
        \end{subfigure}
    }
    \caption{Topografieaufnahmen der Sauerstoff-Ausheizserie von Probe \sampletwo.}
    \label{fig:W6822-1B_Sauerstoff_AFM}
\end{figure}
\newpage

\subsubsection{Vakuum Ausheizvorgang}\label{subsec:vakuum-ausheizvorgang-1}
\begin{figure}
    \centering
    \import{../plots/XRD}{W6822-1C_Vakuum.pgf}
    \caption{$2\theta/\omega$ Diffraktogramme der Vakuum-Ausheizserie von Probe \sampletwo.
    Grau hinterlegt sind Peaks des Probenhalters.}
    \label{fig:W6822-1C_Vakuum_XRD}
\end{figure}

Die Diffraktogramme der Probe \sampletwo\ sind in \cref{fig:W6822-1C_Vakuum_XRD} dargestellt.
Auch hier ist der Dünnfilm röntgenamorph.

In \cref{fig:W6822-1C_Vakuum_AFM} sind die AFM-Topografieaufnahmen der Probe \sampletwo\ während des Vakuum
Ausheizvorgangs dargestellt.
Der in \cref{W6822-1C_Vakuum_AFM_pre} gezeigte Initialzustand ist vergleichbar mit dem Initialzustand des Sauerstoff
Ausheizvorgangs, jedoch sind die Kristallite bezüglich ihrer Fläche etwas größer.
Bei \qty{500}{\degreeCelsius} in \cref{W6822-1C_Vakuum_AFM_500} sind große Löcher zu erkennen, welche eine
ähnliche Flächenverteilung haben wie die Kristallite des Initialzustands.
Das deutet darauf hin, dass die Kristallite des Dünnfilms evaporiert sind.
Bei \qty{600}{\degreeCelsius} in \cref{W6822-1C_Vakuum_AFM_600} ist ein homogener rauer Film zu erkennen, auf dem
einige kleine Kristallite zu sehen sind.
Bei \qty{700}{\degreeCelsius} in \cref{W6822-1C_Vakuum_AFM_700} sind viele kleine Kristallite zu erkennen, die sowohl
in ihrer Fläche als auch in ihrer Höhe größer sind als bei \qty{600}{\degreeCelsius}.
Diese Entwicklung ist auch bei \qty{750}{\degreeCelsius} in \cref{W6822-1C_Vakuum_AFM_750} zu beobachten.
Nach dem Ausheizen auf \qty{800}{\degreeCelsius} in \cref{W6822-1C_Vakuum_AFM_800} ist die Anzahl der beobachteten
Kristallite gesunken.
Außerdem werden Löcher im Film sichtbar, deren Fläche der der Kristallite entspricht.
Bei \qty{875}{\degreeCelsius} in \cref{W6822-1C_Vakuum_AFM_875} sind einzelne Kristallite zu erkennen.
Der Untergrund ist nicht mehr eben, sondern weist eine bergige Struktur auf, die sich über das gesamte Bild zieht.

\begin{figure}
    \centering
    ,\foreach \i in \temperaturesV{
        \begin{subfigure}[t]{0.40\textwidth}
            \includegraphics[width=\textwidth]
            {../plots/AFM/XG-Vakuum/XG-\i/W6822-1C/W6822-1C_XG_Vakuum_\i_Topography_1}
            \caption{\ifthenelse{\equal{\i}{pre}}{Initialzustand}{\qty{\i}{\degreeCelsius}}}
            \label{W6822-1C_Vakuum_AFM_\i}
        \end{subfigure}
    }
    \caption{Topografieaufnahmen der Vakuum-Ausheizserie von Probe \sampletwo.}
    \label{fig:W6822-1C_Vakuum_AFM}
\end{figure}
\newpage

\subsubsection{Luft Ausheizvorgang}\label{subsec:luft-ausheizvorgang-1}
\begin{figure}
    \centering
    \import{../plots/XRD}{W6822-1D_Luft.pgf}
    \caption{$2\theta/\omega$ Diffraktogramme der Luft-Ausheizserie von Probe \sampletwo.
    Grau hinterlegt sind Peaks des Probenhalters.}
    \label{fig:W6822-1D_Luft_XRD}
\end{figure}

In \cref{fig:W6822-1D_Luft_XRD} sind die $2\theta/\omega$-Scans der Probe \sampletwo\ zu den festgelegten Temperaturen
während des Luft-Ausheizvorgangs dargestellt.
Auch diese Ausheizserie zeigt keine Peaks des \heo\ Dünnfilms, sodass dieser röntgenamorph ist.

Die AFM-Topografieaufnahmen der Probe \sampletwo\ während des Luft Ausheizvorgangs sind in \cref{fig:W6822-1D_Luft_AFM}
dargestellt.
Der Initialzustand in \cref{W6822-1D_Luft_AFM_pre} zeigt eine bis jetzt nicht beobachtete Morphologie.
Es sind sowohl kleine Kristallite, die bisher erst nach dem Ausheizen auftraten, als auch größere Erhebungen die
Dimensionen der Kristallite der anderen Initialzustände aufweisen, zu erkennen.
Der Grund dafür liegt darin, dass die Probe \sampletwo\ vorher bereits als Testprobe für \qty{1}{\hour} und
\qty{3}{\hour} in Sauerstoff ausgeheizt wurde.
Aus diesem Grund sind beide Phänomene erkennbar.
Nach dem Ausheizen auf \qty{600}{\degreeCelsius} in \cref{W6822-1D_Luft_AFM_600} sind die großen Kristallite
verschwunden und es ist eine homogene raue Oberfläche zu erkennen.
Diese zeigen eine ähnliche Morphologie wie die Aufnahmen bei \qty{700}{\degreeCelsius} der Sauerstoff- und
Vakuum-Ausheizserie.
Dies ist möglicherweise ein Indiz dafür, dass die Temperatur des Dünnfilms in der Vakuumkammer deutlich
unterhalb der Temperatur des Muffelofens liegt.
Bei \qty{700}{\degreeCelsius} in \cref{W6822-1D_Luft_AFM_700} sind vereinzelt kleine Löcher und Kristallite zu
erkennen.
Die Reihe von \qty{750}{\degreeCelsius} bis \qty{875}{\degreeCelsius} in
\cref{W6822-1D_Luft_AFM_750,W6822-1D_Luft_AFM_800,W6822-1D_Luft_AFM_875} ist vergleichbar mit der
Luft-Ausheizaufnahme der Probe \sampleone.
Vor allem kleine Löcher sind zu erkennen, die auf eine Evaporation der Kristallite hindeuten.
Die Aufnahme bei \qty{875}{\degreeCelsius} zeigt zusätzlich eine Deformierung des Untergrunds.


\begin{figure}[h]
    \centering
    ,\foreach \i in \temperaturesL{
        \begin{subfigure}[t]{0.40\textwidth}
            \includegraphics[width=\textwidth]
            {../plots/AFM/XG-Luft/XG-\i/W6822-1D/W6822-1D_XG_Luft_\i_Topography_1}
            \caption{\ifthenelse{\equal{\i}{pre}}{Initialzustand}{\qty{\i}{\degreeCelsius}}}
            \label{W6822-1D_Luft_AFM_\i}
        \end{subfigure}
    }
    \caption{Topografieaufnahmen der Luft-Ausheizserie von Probe \sampletwo.}
    \label{fig:W6822-1D_Luft_AFM}
\end{figure}
\newpage


% PROBE W6824-1 %

\newpage

\subsection{Probe W6824-1}\label{subsec:probe-W6824-1}

\subsubsection{Sauerstoff Ausheizvorgang}\label{subsec:sauerstoff-ausheizvorgang-1}

\begin{figure}
    \centering
    \import{../plots/XRD}{W6824-1B_Sauerstoff.pgf}
    \caption{W6824-1B, Sauerstoff, XRD}
    \label{fig:W6824-1B_Sauerstoff_XRD}
\end{figure}
\begin{figure}
    \centering
    \foreach \i in \temperaturesS{
        \begin{subfigure}[t]{0.40\textwidth}
            \includegraphics[width=\textwidth]
            {../plots/AFM/XG-Sauerstoff/XG-\i/W6824-1B/W6824-1B_XG_Sauerstoff_\i_Topography_1}
            \caption{\ifthenelse{\equal{\i}{pre}}{Initialzustand}{\qty{\i}{\degreeCelsius}}}
            \label{W6824-1B_Sauerstoff_AFM_\i}
        \end{subfigure}
    }
    \caption{W6824-1B, Sauerstoff, AFM}
    \label{fig:W6824-1B_Sauerstoff_AFM}
\end{figure}
\newpage

\subsubsection{Vakuum Ausheizvorgang}\label{subsec:vakuum-ausheizvorgang-1}
\begin{figure}
    \centering
    \import{../plots/XRD}{W6824-1C_Vakuum.pgf}
    \caption{W6824-1C, Vakuum, XRD}
    \label{fig:W6824-1C, Vakuum, XRD}
\end{figure}
\begin{figure}
    \centering
    ,\foreach \i in \temperatureVfour{
        \begin{subfigure}[t]{0.40\textwidth}
            \includegraphics[width=\textwidth]
            {../plots/AFM/XG-Vakuum/XG-\i/W6824-1C/W6824-1C_XG_Vakuum_\i_Topography_1}
            \caption{\ifthenelse{\equal{\i}{pre}}{Initialzustand}{\qty{\i}{\degreeCelsius}}}
            \label{W6824-1C, Vakuum, AFM, \i}
        \end{subfigure}
    }
    \caption{AFM, Vakuum, W6824-1C}
    \label{fig: AFM, Vakuum, W6824-1C}
\end{figure}
\newpage

\subsubsection{Luft Ausheizvorgang}\label{subsec:luft-ausheizvorgang-1}
\begin{figure}
    \centering
    \import{../plots/XRD}{W6824-1D_Luft.pgf}
    \caption{W6824-1D, Luft, XRD}
    \label{fig:W6824-1D, Luft, XRD}
\end{figure}
\begin{figure}
    \centering
    ,\foreach \i in \temperaturesL{
        \begin{subfigure}[t]{0.40\textwidth}
            \includegraphics[width=\textwidth]
            {../plots/AFM/XG-Luft/XG-\i/W6824-1D/W6824-1D_XG_Luft_\i_Topography_1}
            \caption{\ifthenelse{\equal{\i}{pre}}{Initialzustand}{\qty{\i}{\degreeCelsius}}}
            \label{W6824-1D, Luft, AFM, \i}
        \end{subfigure}
    }
    \caption{AFM, Luft, W6824-1D}
    \label{fig: AFM, Luft, W6824-1D}
\end{figure}
\newpage

\subsection{Rauheiten}\label{subsec:Rauheit}
\begin{figure}
    \centering
    \import{../plots/AFM}{sauerstoff.pgf}
    \caption{AFM, Sauerstoff}
    \label{fig: AFM, Sauerstoff}
\end{figure}

\begin{figure}
    \centering
    \import{../plots/AFM}{vakuum.pgf}
    \caption{AFM, Vakuum}
    \label{fig: AFM, Vakuum}
\end{figure}

    \section{Fazit und Ausblick}\label{sec:fazit-und-ausblick}
Diese Studie zielte darauf ab, \heo-Dünnfilme zu synthetisieren und durch Ausheizprozesse in eine reine
Natriumchloridstruktur zu überführen.
Die Ergebnisse deuten zwar auf eine erfolgreiche Abscheidung der Dünnfilme hin, jedoch konnte der erwartete
Phasenübergang in die Natriumchloridstruktur bei keinem der drei Ausheizserien beobachtet werden.
Eine mögliche Ursache hierfür könnte in der Abweichung der Dünnfilmzusammensetzung von der equimolaren Zusammensetzung
des Targets liegen, was zu erhöhten Übergangstemperaturen führte und die Wahl des Temperaturbereichs sowie des Substrats
als ungeeignet erscheinen ließ.

Die Abwesenheit charakteristischer Peaks in den Röntgendiffraktogrammen zeigt, dass die Dünnfilme keine kristalline
Natriumchloridstruktur aufweisen.
Die Proben \samplethree, \sampleone, \sampletwo\ und \samplefour\ zeigen bei allen drei Ausheizserien eine röntgenamorphe Struktur.
Ein Grund für dieses Verhalten könnte eine erhöhte Übergangstemperatur infolge der modifizierten Stöchiometrie der
Dünnfilme sein.
Die in \cref{tab:concentration} dargestellten Konzentrationen zeigen, dass die Komposition der \heo-Dünnfilme nicht der
equimolaren Zusammensetzung des Targets entsprechen.
Die Filme weisen eine geringere Magnesiumkonzentration auf, welche mit sinkendem Druck abnimmt.
Um für zukünftige Abscheideprozesse eine equimolare Zusammensetzung zu erreichen, kann das Target in seiner
Komposition angepasst werden.
Durch einen erhöhten Magnesiumanteil im Target könnte auch die Magnesiumkonzentration in den Filmen erhöht werden.


Aus der abweichenden Stöchiometrie der Dünnfilme ergibt sich eine niedrigere Mischungsentropie, welche für die
Phasenstabilität von \heo\ entscheidend ist.
Um die Phasenübergangstemperatur in die reine Natriumchloridstruktur besser abschätzen zu können, kann die
Mischungsentropie mit den gemessenen Konzentrationen $\{ x_i \}$ aus \cref{tab:concentration} berechnet, und mit
derjenigen Mischungsentropie gleichgesetzt werden, die sich bei vier equimolaren Konstituenten und einem Konstituent mit
variabler Konzentration $x$ ergibt.
Aus \cref{eq:Mischungsentropie} und \cref{eq:Mischungsentropie2} folgt:
\begin{equation}
    -\mathrm{R}\sum_{i=1}^{5}x_{i}\ln(x_{i}) \stackrel{!}{=}-\mathrm{R}\left( x\log(x)+(1-x)\ln
    \left( \frac{1-x}{4} \right) \right).
    \label{eq:fazit}
\end{equation}
Löst man diese Gleichung numerisch nach $x$, findet man für die Proben \csamplethree, \csampleone, \csampletwo\ und \csamplefour,
folgende Konzentrationen:
\begin{equation*}
    x(\mathrm{P}_{\num{0.1}}^{\mathrm{c}}) = \qty{16.8}{\percent}, \quad x(\mathrm{P}_{\num{0.01}}^{\mathrm{c}})
    = \qty{15.6}{\percent}, \quad
    x(\mathrm{P}_{\num{0.001}}^{\mathrm{c}}) = \qty{15.7}{\percent}, \quad x(\mathrm{P}_{\num{0.00005}}^{\mathrm{c}})
    = \qty{15.0}{\percent}.
\end{equation*}
Die von \citeauthoryear{Rost2015} angegebenen Phasenübergangstemperaturen in eine reine Natriumchloridstruktur
bei vier equimolaren Konstituenten und einer variablen Konzentration von $x \simeq \qty{15}{\percent}$ liegen, je nach
Material, zwischen \qty{925}{\degreeCelsius} und \qty{1000}{\degreeCelsius}.
Auch wenn in dieser Arbeit Dünnfilme und keine Massivproben untersucht wurden, kann die Phasenübergangstemperatur
als erste Abschätzung für die Wahl des Temperaturbereichs herangezogen werden und liegt deutlich oberhalb der
gewählten Temperaturbereiche.


Damit einher geht auch die ungeeignete Wahl des Eagle XG Substrats, dessen Erweichungspunkt bei
\qty{971}{\degreeCelsius} liegt und sich schon bei \qty{875}{\degreeCelsius} merklich verformt.
Für nachfolgende Abscheidungen von \heo-Dünnfilmen sollten Substrate gewählt werden, deren Erweichungspunkte weit
oberhalb der Übergangstemperatur liegen.
Ein geeigneter Kandidat wäre das Substrat \ce{SiO2}, dessen Erweichungspunkt bei \qty{1308}{\degreeCelsius} liegt.
Dieses Substrat hätte außerdem den Vorteil, dass es keine relevanten Metallkationen enthält, und damit
EDX Messungen ermöglicht, die nicht durch das Substrat verfälscht werden.

Weiterhin sollte versucht werden, die mechanischen Einwirkungen auf den Dünnfilm und das Substrat, vor allem
während des Ausheizprozesses, zu minimieren.
Der Druck der Klemmen des Substrathalters in der Vakuumkammer sorgte bei einer Temperatur von \qty{875}{\degreeCelsius}
für merkliche Verformungen des Substrats.
Wie Tim Düvel in seiner aktuellen Arbeit festgestellt hat, sorgt der Druck der Klemmen nicht nur für
mechanischen Spannungen, sondern auch für eine ungleichmäßige Temperaturverteilung über das Substrat hinweg.
Auch durch das punktuelle Auftreffen des Laserstrahls auf den Halter entsteht ein radialer Temperaturgradient.
Diese Effekte führen zu einer ungleichmäßigen Ausheizung des Dünnfilms.
Da auch die Verunreinigungen der reinen Eagle XG Glassubstrate im Muffelofen deutlich geringer ausfallen als in der
Vakuumkammer, sollte für zukünftige Ausheizprozesse auf den Muffelofen zurückgegriffen werden.

Die Topografieaufnahmen der unterschiedlichen Initialzustände der Proben, die sich durch eine örtliche und zeitliche
Versetzung der Messung auszeichnen, deuten auf örtliche oder zeitliche Veränderungen des Dünnfilms hin.
Für zukünftige Messungen kann die Oberfläche der Proben an mehreren Stellen über den gesamten Film hinweg
untersucht werden, um lokale Unterschiede in der Morphologie zu erfassen.
Durch die von Jorrit Bredow bereitgestellten AFM-Probenhalter kann die Abhängigkeit der Topografie von der Zeit
untersucht werden.
Die Proben können in den Probenhalter eingespannt und über einen längeren Zeitraum hinweg an der gleichen Stelle
untersucht werden.

Obwohl alle Dünnfilme, unabhängig vom Abscheidedruck, eine röntgenamorphe Struktur aufweisen, zeigen die
Topografieaufnahmen der Proben \samplethree, \sampleone, \sampletwo\ und \samplefour\ unterschiedliche
Oberflächenmorphologien und Rauheiten.
Die Initialmessung zeigt meist hohe Kristallite, die zufällig angeordnet einen ebenen Untergrund bedecken.
Die Ausheizprozesse führten zu einer Evaporation der Kristallite, welche die Oberfläche der Kristallite glätteten.
Bei allen Proben wurde eine thermisch induzierte Kristallitbildung beobachtet, die sich durch eine erhöhte Rauheit
der Proben auszeichnet.
Höhere Temperaturen gingen oftmals mit Löchern in den Dünnfilmen einher.

Diese in den Dünnfilmen auftretenden Löcher sind ein weiteres Indiz für die Evaporation der Kristallite.
Durch diese Evaporationsprozesse wurden einzelne Kristallite von der Oberfläche abgetragen, sodass sich die
Dünnfilmdicke verringert.
Durch zunehmende Transparenz der Proben bei höheren Temperaturen wird dieses Phänomen auch optisch sichtbar,
sodass für zukünftige Untersuchungen optische Messungen in Betracht gezogen werden sollten.
Dies stellt eine weitere Herausforderung für die Ausheizprozesse dar, da die Dicke der Dünnfilme nicht konstant bleibt.
Für weitere Untersuchungen sollte die Dicke der Dünnfilme nach dem Ausheizprozess mithilfe von XRR- oder
Profilometermessungen bestimmt werden, um Rückschlüsse auf die Evaporationsraten ziehen zu können.
Eine größere Dünnfilmdicke würde den Einfluss von Abtragungsvorgängen verringern.

Da die Dünnfilme teilweise nicht signifikante Peaks aufweisen, empfiehlt es sich, GIXRD-Aufnahmen
nach den jeweiligen Ausheizschritten durchzuführen.
Diese sind empfindlicher gegenüber Dünnfilmen und ermöglichen damit auch das Erfassen kleinerer Peaks, was gerade
bei sehr geringen Schichtdicken von Vorteil ist.

Aufgrund der Komplexität der Evaporationsprozesse ist davon auszugehen, dass nicht alle Konstituenten gleichermaßen
verdampfen.
Infolgedessen kann sich die Stöchiometrie der Dünnfilme weiter verändern, sodass die Mischungsentropie
weiter sinkt.
Um dieses Phänomen zu untersuchen, können mehrere Proben unter gleichen Bedingungen hergestellt und anschließend
bei verschiedenen Temperaturen ausgeheizt werden.
Nach der jeweiligen Temperatur kann die Komposition einer Probe durch EDX-Messungen bestimmt werden.
Zu beachten ist dabei, dass die Probe mit Kohlenstoff beschichtet werden muss, was sie für weitere
Untersuchungen unbrauchbar macht.

Die vorliegende Arbeit hat gezeigt, dass die Synthese von \heo-Dünnfilmen und deren thermische Behandlung ein komplexes
Zusammenspiel verschiedener Faktoren ist.
Obwohl der angestrebte Phasenübergang nicht erreicht werden konnte, haben die gewonnenen Erkenntnisse wertvolle Hinweise
für zukünftige Untersuchungen geliefert.
Durch eine systematische Variation der Prozessparameter, eine optimierte Wahl von Substrat, Target und Temperaturbereich
und eine detaillierte Charakterisierung der Dünnfilme mithilfe von Schichtdickenmessungen und optischen Untersuchungen
können die Voraussetzungen für eine erfolgreiche Überführung in die gewünschte Struktur geschaffen werden.
    \section{Appendix}\label{sec:appendix}

\subsection{Bildvergleich AFM}\label{subsec:bildvergleich-afm}

\newcommand{\plotpath}{../plots/AFM}
\newcommand{\samplesB}{W6821-1B, W6822-1B, W6823-1B, W6824-1B}
\newcommand{\samplesC}{W6821-1C, W6822-1C, W6823-1C, W6824-1C}
\newcommand{\samplesD}{W6821-1D, W6822-1D, W6823-1D, W6824-1D}

\subsection{Sauerstoff Ausheizvorgang}\label{subsec:sauerstoff-ausheizvorgang}

\subsubsection{Vor dem Ausheizen}
\begin{figure}[ht]
    \centering
    \foreach \sample in \samplesB {
        \begin{subfigure}[t]{0.40\textwidth}
            \centering
            \includegraphics[width=\textwidth]
            {\plotpath/XG-Sauerstoff/XG-pre/\sample/\sample_XG_Sauerstoff_pre_Topography_1}
            \caption{\sample, Bild 1}
        \end{subfigure}
        \begin{subfigure}[t]{0.40\textwidth}
            \centering
            \includegraphics[width=\textwidth]
            {\plotpath/XG-Sauerstoff/XG-pre/\sample/\sample_XG_Sauerstoff_pre_Topography_3}
            \caption{\sample, Bild 2}
        \end{subfigure}
    }
    \caption{AFM, Sauerstoff, pre}\label{fig: AFM, Sauerstoff, pre}
\end{figure}

\foreach \temp in {600, 700, 750, 800, 875} {
    \subsubsection{\qty{\temp}{\degreeCelsius}}
    \begin{figure}[ht]
        \centering
        \foreach \sample in \samplesB {
            \begin{subfigure}[t]{0.40\textwidth}
                \centering
                \includegraphics[width=\textwidth]
                {\plotpath/XG-Sauerstoff/XG-\temp/\sample/\sample_XG_Sauerstoff_\temp_Topography_1}
                \caption{\sample, Bild 1}
            \end{subfigure}
            \begin{subfigure}[t]{0.40\textwidth}
                \centering
                \includegraphics[width=\textwidth]
                {\plotpath/XG-Sauerstoff/XG-\temp/\sample/\sample_XG_Sauerstoff_\temp_Topography_3}
                \caption{\sample, Bild 2}
            \end{subfigure}
        }
        \caption{AFM, Sauerstoff, \qty{\temp}{\degreeCelsius}}\label{fig: AFM, Sauerstoff, \temp}
    \end{figure}
}

\subsection{Vakuum Ausheizvorgang}\label{subsec:vacuum-ausheizvorgang}

\subsubsection{Vor dem Ausheizen}
\begin{figure}[ht]
    \centering
    \foreach \sample in \samplesC {
        \begin{subfigure}[t]{0.40\textwidth}
            \centering
            \includegraphics[width=\textwidth]
            {\plotpath/XG-Vakuum/XG-pre/\sample/\sample_XG_Vakuum_pre_Topography_1}
            \caption{\sample, Bild 1}
        \end{subfigure}
        \begin{subfigure}[t]{0.40\textwidth}
            \centering
            \includegraphics[width=\textwidth]
            {\plotpath/XG-Vakuum/XG-pre/\sample/\sample_XG_Vakuum_pre_Topography_3}
            \caption{\sample, Bild 2}
        \end{subfigure}
    }
    \caption{AFM, Vakuum, pre}\label{fig: AFM, Vakuum, pre}
\end{figure}

\foreach \temp in {500, 600, 700, 750} {
    \ifthenelse{\equal{\temp}{750}}{\renewcommand{\samplesC}{W6821-1C, W6822-1C, W6824-1C}}{}

    \subsubsection{\qty{\temp}{\degreeCelsius}}
    \begin{figure}[ht]
        \centering
        \foreach \sample in \samplesC {
            \begin{subfigure}[t]{0.40\textwidth}
                \centering
                \includegraphics[width=\textwidth]
                {\plotpath/XG-Vakuum/XG-\temp/\sample/\sample_XG_Vakuum_\temp_Topography_1}
                \caption{\sample, Bild 1}
            \end{subfigure}
            \begin{subfigure}[t]{0.40\textwidth}
                \centering
                \includegraphics[width=\textwidth]
                {\plotpath/XG-Vakuum/XG-\temp/\sample/\sample_XG_Vakuum_\temp_Topography_3}
                \caption{\sample, Bild 2}
            \end{subfigure}
        }
        \caption{AFM, Vakuum, \qty{\temp}{\degreeCelsius}}\label{fig: AFM, Vakuum, \temp}
    \end{figure}
}

\subsection{Luft Ausheizvorgang}\label{subsec:luft-ausheizvorgang}

\subsubsection{Vor dem Ausheizen}
\begin{figure}[ht]
    \centering
    \foreach \sample in \samplesD {
        \begin{subfigure}[t]{0.40\textwidth}
            \centering
            \includegraphics[width=\textwidth]
            {\plotpath/XG-Luft/XG-pre/\sample/\sample_XG_Luft_pre_Topography_1}
            \caption{\sample, Bild 1}
        \end{subfigure}
        \begin{subfigure}[t]{0.40\textwidth}
            \centering
            \includegraphics[width=\textwidth]
            {\plotpath/XG-Luft/XG-pre/\sample/\sample_XG_Luft_pre_Topography_3}
            \caption{\sample, Bild 2}
        \end{subfigure}
    }
    \caption{AFM, Luft, pre}\label{fig: AFM, Luft, pre}
\end{figure}


\foreach \temp in {600} {
    \subsubsection{\qty{\temp}{\degreeCelsius}}
    \begin{figure}[ht]
        \centering
        \foreach \sample in \samplesD {
            \begin{subfigure}[t]{0.40\textwidth}
                \centering
                \includegraphics[width=\textwidth]
                {\plotpath/XG-Luft/XG-\temp/\sample/\sample_XG_Luft_\temp_Topography_1}
                \caption{\sample, Bild 1}
            \end{subfigure}
            \begin{subfigure}[t]{0.40\textwidth}
                \centering
                \includegraphics[width=\textwidth]
                {\plotpath/XG-Luft/XG-\temp/\sample/\sample_XG_Luft_\temp_Topography_3}
                \caption{\sample, Bild 2}
            \end{subfigure}
        }
        \caption{AFM, Luft, \qty{\temp}{\degreeCelsius}}\label{fig: AFM, Luft, \temp}
    \end{figure}
}

\subsection{EDX Bilder}\label{subsec:edx-bilder}
\newcommand{\samplesEDX}{W6822-3D, W6823-3D, W6824-3D}

\foreach \sample in \samplesEDX {
    \subsubsection{\sample}
    \begin{figure}
        \centering
        \foreach \i/\desc in {map/Oberfläche, Mg/Magnesium, Co/Kobalt, Ni/Nickel), Cu/Kupfer, Zn/Zink}{
            \begin{subfigure}[t]{0.40\textwidth}
                \includegraphics[width=\textwidth]{../plots/EDX/\sample/\i}
                \caption{\desc}
            \end{subfigure}
        }
        \caption{EDX Aufnahmen der Probe \sample}
        \label{fig:edx_\sample}
    \end{figure}
}

    \printbibliography
\end{document}
