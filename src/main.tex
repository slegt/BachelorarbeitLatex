%! Author = slegt
%! Date = 20.08.2024

\documentclass[12pt,titlepage,ngerman]{article}

\usepackage[a4paper, portrait, margin=1in]{geometry}

\usepackage{graphicx}
\usepackage[version=4]{mhchem}
\usepackage[style=authortitle, autocite=footnote]{biblatex}
\usepackage[ngerman]{babel}
\usepackage[justification=centering]{caption}
\usepackage{subcaption}
\usepackage{booktabs}
\usepackage{siunitx}
\usepackage{amsmath}
\usepackage{amssymb}
\usepackage{mathtools}
\usepackage[hidelinks]{hyperref}
\usepackage{cleveref}
\usepackage{currfile}
\usepackage{pgf}
\usepackage{lmodern}
\usepackage{import}
\usepackage{pgffor}
\usepackage{ifthen}


\sisetup{separate-uncertainty=true}
\setlength{\parindent}{0pt}
\addbibresource{literature.bib}

\newcommand{\imcite}[2][]{\\ Aus: \cite[#1]{#2}}
\newcommand{\imcitetwo}[2][]{\\ Nach: \cite[#1]{#2}}

\newcommand{\integral}[4]{\int_{#1}^{#2} #3 \mathrm{d} #4}
\newcommand{\derivative}[2]{\frac{\mathrm{d}}{\mathrm{d} #1} #2}
\newcommand{\heo}{\ce{(MgCoNiCuZn)O}}
\newcommand{\h}{\mathrm{h}}
\renewcommand{\c}{\mathrm{c}}

\DeclareSIUnit\angstrom{\text {Å}}
\DeclareSIUnit\bar{bar}


% Document
\begin{document}
    \begin{titlepage}
    \begin{center}
        \vfill
        \Huge
        \textbf{Ausheizstudie von \\
        \heo\ Dünnfilmen} \\

        \vfill
        \Large
        An der Universität Leipzig, \\
        Fakultät für Physik und Erdsystemwissenschaften, \\
        Felix-Bloch-Institut für Festkörperphysik , \\
        im Bachelor Physik eingereichte \\

        \vfill
        \Huge
        Bachelorarbeit\\

        \vfill
        \Large
        Zur Erlangung des akademischen Grades eines \\
        Bachelor of Science

        \vfill
        Vorgelegt von \\
        Simon Legtenborg, 3773994 \\
        Geboren am 18.08.2002 in Nordhorn \\


        \vfill
        Betreuer: \\
        M. Sc. Jorrit Bredow

        \vfill
        Gutachter: \\
        Prof. Dr. Marius Grundmann \\
        PD Dr. habil. Holger von Wenckstern


        \vfill
        Eingereicht am 08.11.2024
        \vfill


    \end{center}
\end{titlepage}
    \tableofcontents
    \section{Theorie}\label{sec:theorie}

\subsection{Kristallgitter}\label{subsec:kristallgitter}
Um das Material \heo, sowie die Röntgendiffraktometrie verstehen zu können, ist ein grundlegendes Verständnis
über die Struktur kristalliner Festkörpern unerlässlich.
Die nachfolgenden Seiten dienen deshalb als Zusammenfassung der für die Arbeit relevanten Konzepte der Festkörperphysik.

\subsubsection{Bravaisgitter, Elementarzelle und Basis}
Als Idealisierung vieler Festkörper wird das Modell des idealen Kristalls herangezogen.
Ein idealer Kristall ist eine dreidimensionale, unendlich ausgedehnte Anordnung, die sich aus identischen, periodisch
wiederkehrenden Baueinheiten zusammensetzt.
Diese Baueinheiten werden als Basis bezeichnet.
Sie können einzelne Atome, aber auch komplexe Atomstrukturen repräsentieren.
Reduziert man jede Baueinheit auf einen einzigen Punkt, so entsteht ein einfach zu beschreibendes Punktgitter.
\autocite[49]{Hunklinger}
Dieses unterliegt verschiedenen Symmetrien, sodass das Gitter in unterschiedliche Kristallsysteme eingeteilt werden
kann.
Eine einfache Einteilung kann mithilfe von Drehachsen erfolgen.
Hierbei betrachtet man diejenigen Rotationsoperatoren $R_{\hat{e}}(2\pi / n)$ für eine beliebige Achse $\hat{e}$ um
einen Winkel $2 \pi /n$, die das Punktgitter auf sich selbst abbilden.
Der Parameter $n \in \mathbb{N}$ wird als Zähligkeit bezeichnet und teilt die Punktgitter in sieben verschiedene
Kristallklassen ein, die in \cref{tab:krystallsysteme} aufgeführt sind.
\autocite[53]{Hunklinger}
\begin{table}[h]
    \centering
    \begin{tabular}{c c c c}
        \toprule
        Kristallsystem           & Gitterkonstanten  & Winkel                                       & Zähligkeit \\ \midrule
        triklin                  & $a \neq b \neq c$ & $\alpha \neq\beta \neq\gamma$                & 1          \\
        monoklin                 & $a \neq b \neq c$ & $\alpha=\gamma=\ang{90},\beta \neq \ang{90}$ & 2          \\
        orthorhombisch           & $a \neq b \neq c$ & $\alpha=\beta=\gamma=\ang{90}$               & 2 (zwei)   \\
        tetragonal               & $a = b \neq c$    & $\alpha=\beta=\gamma=\ang{90}$               & 4          \\
        hexagonal                & $a = b \neq c$    & $\alpha=\beta=\ang{90}, \gamma=\ang{120}$    & 6          \\
        trigonal (rhomboedrisch) & $a=b=c$           & $\alpha=\beta=\gamma \neq \ang{90}$          & 3          \\
        kubisch                  & $a=b=c$           & $\alpha=\beta=\gamma=\ang{90}$               & 3 (vier)   \\ \bottomrule
    \end{tabular}
    \caption{Klassifikation der verschiedenen Kristallsysteme. \imcite[65]{Hunklinger} }
    \label{tab:krystallsysteme}
\end{table}

Eine weitere wichtige Symmetrie ist die Translationssymmetrie.
Betrachtet man diejenigen Translationsoperatoren $T(\mathbf{O})$, die das Gitter auf sich selbst abbilden, dann erkennt
man aufgrund der Periodizität des Gitters den Zusammenhang
$\mathbf{O} = n_{1}\mathbf{a}_{1}+n_{2}\mathbf{a}_{2}+n_{3}\mathbf{a}_{3}$, wobei
$\mathbf{a}_{i}\in\mathbb{R}^{3}, n_{i}\in\mathbb{Z}, i=\{1,2,3\}$. \autocite[50]{Hunklinger}
Die Vektoren $\mathbf{a}_{i}$
definieren ein schiefwinkliges Koordinatensystem und werden als primitive Gittervektoren bezeichnet.
Sie spannen ein dreidimensionales Bravaisgitter auf.
Die Abstände zwischen zwei benachbarten Gitterpunkten, also die Größen
$\lvert \mathbf{a}_{i} \rvert$, werden Gitterkonstanten genannt. \autocite[82]{Ashcroft}
Je nachdem, wie sich das Kristallsystem unter Symmetrieoperationen verhält, ergeben sich unterschiedliche Bedingungen für
Gitterkonstanten und die Winkel zwischen den Gittervektoren.
Mithilfe der Definition einer Basis und eines Bravaisgitters lässt sich jeder ideale Kristall beschreiben.
Eine Kristallstruktur wird durch identische Kopien der Basis an jedem Punkt des Bravaisgitters aufgebaut.
\autocite[94-95]{Ashcroft}

Durch geschickte Wahl von Teilmengen des Ortsraumes kann der gesamte Ortsraum durch überlappungsfreie Aneinanderreihung
der Teilmengen lückenlos gefüllt werden.
Solche Mengen nennt man Elementarzellen.
Wählt man die Elementarzelle so, dass sie nur einen Gitterpunkt enthält, spricht man von einer primitiven
Elementarzelle.
Mit einer primitiven Elementarzelle lässt sich der Raum lückenlos und überlappungsfrei überdecken, indem man die Zelle
entlang jedes Bravaisgittervektors verschiebt.
Eine einfache Konstruktion liefert ein Parallelepiped, welches von den drei Basisvektoren aufgespannt wird.
Das Volumen $V_\mathrm{EZ}= \lvert \mathbf{a}_1 \cdot (\mathbf{a}_2 \times  \mathbf{a}_3) \rvert$ dieses
Parallelepipeds gibt das effektive Volumen pro Bravais-Gitterpunkt an.
\autocite[90-91]{Ashcroft}

Eine weitere Unterteilung der Kristallsysteme kann mithilfe der dreidimensionalen Bravaisgitter erfolgen.
Diese lassen sich durch Operationen der Punktgruppe erzeugen und kategorisieren periodische Strukturen
in 14 verschiedene Bravaisgitter. \autocite[37]{Grundmann}
Die für die Arbeit relevanten Gitter werden im Folgenden vorgestellt.

\subsubsection{Ausgewählte Kristallgitter}
\begin{figure}
    \centering
    \begin{subfigure}[t]{0.3\textwidth}
        \centering
        \includegraphics[width=\textwidth]{../assets/theorie/fcc}
        \caption{fcc-Gitter.} \label{fcc}
    \end{subfigure}
    \begin{subfigure}[t]{0.3\textwidth}
        \centering
        \includegraphics[width=\textwidth]{../assets/theorie/rocksalt}
        \caption{NaCl-Gitter} \label{nacl}
    \end{subfigure}
    \begin{subfigure}[t]{0.3\textwidth}
        \centering
        \includegraphics[width=\textwidth]{../assets/theorie/CuO}
        \caption{\ce{CuO}-Gitter} \label{cuo}
    \end{subfigure}
    \begin{subfigure}[t]{0.3\textwidth}
        \centering
        \includegraphics[width=\textwidth]{../assets/theorie/hex}
        \caption{hexagonales Gitter} \label{hex}
    \end{subfigure}
    \begin{subfigure}[t]{0.3\textwidth}
        \centering
        \includegraphics[width=\textwidth]{../assets/theorie/hcp}
        \caption{hcp-Gitter} \label{hcp}
    \end{subfigure}
    \begin{subfigure}[t]{0.3\textwidth}
        \centering
        \includegraphics[width=\textwidth]{../assets/theorie/wurzite}
        \caption{Wurtzit-Gitter} \label{wurtzit}
    \end{subfigure}
    \caption{Ausgewählte Kristallgitter für die Untersuchung von \heo.} \label{fig:gitterstrukturen}
\end{figure}

\paragraph{Natriumchloridstruktur}
Die erste und wichtigste Kristallstruktur ist die Natriumchloridstruktur.
Um diese zu verstehen, betrachtet man zuerst ein kubisch-flächen\-zen\-trier\-tes Gitter (fcc, engl.
\textit{face-centered cubic}) welches durch die primitiven Vektoren
\begin{equation}
    \mathbf{a}_1 = a / 2 (\hat{x} + \hat{y}), \quad
    \mathbf{a}_2 = a / 2 (\hat{y} + \hat{z}), \quad
    \mathbf{a}_3 = a / 2 (\hat{x} + \hat{z})
    \label{eq:fcc}
\end{equation}
definiert ist.
Die Gitterpunkte liegen an Würfelecken und den dazugehörigen Flächenmittelpunkten, wie in \cref{fcc} dargestellt.
\autocite[37-38]{Grundmann}
Die Natriumchloridstruktur entsteht aus einem fcc-Gitter mit zweiatomiger Basis.
Dabei ist das erste Basisatom an der Gitterposition $(0,0,0)$ und das zweite an der Position $(a/2,a/2,a/2)$, wie in
\cref{nacl} dargestellt.\autocite[45]{Grundmann}
Sowohl Magnesiumoxid (\ce{MgO}), Kobaltoxid (\ce{CoO}) und Nickeloxid (\ce{NiO}) kristallisieren in dieser Struktur.
Auch \heo\ kristallisiert in der Natriumchloridstruktur, wenn
Sauerstoff als erstes Basisatom und ein zufällig ausgewähltes Metall als zweites Basisatom betrachtet wird.
\autocite[5]{Rost2015}

\paragraph{Tenoritstruktur}
Kupferoxid kristallisiert in einer monoklinen Kristall\-struk\-tur mit achtatomiger Basis.
Dabei sind vier dieser Basisatome Kupferatome und vier Sauerstoffatome.
Die Struktur ist in \cref{cuo} dargestellt.\autocite[7]{kupferoxid}

\paragraph{Wurtzitstruktur}
Um die Wurtzitstruktur zu verstehen, betrachtet man zuerst die einfach-hexagonale Struktur, die durch die primitiven
Vektoren
\begin{equation}
    \mathbf{a}_1 = a\hat{x}, \quad
    \mathbf{a}_2 = a/2 (\hat{x} + \sqrt{3} \hat{y}), \quad
    \mathbf{a}_3 = c \hat{z}
\end{equation}
definiert ist.
Es entsteht ein Gitter, welches an den Ecken eines Sechsecks und den dazugehörigen Flächenmittelpunkten Gitterpunkte
besitzt, wie in \cref{hex} dargestellt.
Aus der einfach-hexagonalen Struktur ergibt sich die hexagonal dichtest gepackte Struktur
(hcp, engl. \textit{hexagonal-closed packet}).
Dieser liegt ein einfach-hexagonales Gitter mit zweiatomiger Basis bei den Gitterpositionen $(0,0,0)$ und
$(\mathbf{a}_1/3 + \mathbf{a}_2/3 + \mathbf{a}_3/2)$ zugrunde, wie in \cref{hcp} dargestellt.\autocite[97-98]{Ashcroft}
Die Wurtzitstruktur besteht aus einem einfach-hexagonalen Gitter mit vieratomiger Basis.
Eine bessere Vorstellung ergibt sich jedoch aus der Betrachtung zweier übereinander liegender hcp-Gitter, die um
die Höhe $\sqrt {3 / 8} a$ gegeneinander verschoben sind, wie in \cref{wurtzit} dargestellt.
Zinkoxid (\ce{ZnO}) kristallisiert in dieser Struktur.\autocite[47-48]{Grundmann}

\subsubsection{Reziprokes Gitter}
Das reziproke Gitter spielt für die weitere Betrachtung periodischer Strukturen eine fundamentale Rolle.
Ziel ist es, eine Funktion zu konstruieren, die gitterperiodisch im Bravaisgitter ist.
Für diese Funktion soll also gelten
$f(\mathbf{x})=f(\mathbf{x}+\mathbf{O})$, falls $\mathbf{O}=\sum_{i=1}^{3} \alpha_{i}\mathbf{a}_{i}$.
Mithilfe einer Reihenentwicklung ergibt sich die folgende, allgemeine Form:
\begin{align}
    f(\mathbf{x})&=\sum_{\mathbf{R}}a_{\mathbf{R}}\cdot \exp(\mathrm{i}\mathbf{R}\cdot\mathbf{x}),\\
    f(\mathbf{x}+\mathbf{O})&=\sum_{\mathbf{R}}a_{\mathbf{R}}\cdot \exp(\mathrm{i}\mathbf{R}\cdot \mathbf{x})\cdot
    \underbrace{ \exp(\mathrm{i}\mathbf{R}\cdot \mathbf{O}) }_{ \stackrel{!}{=}1 }  \stackrel{!}{=} f(\mathbf{x}).
\end{align}
Erkennbar ist die notwendige Bedingung $\exp(\mathrm{i}\mathbf{R}\cdot \mathbf{O})=1$.
Dies ist äquivalent zur Aussage $\mathbf{R}\cdot \mathbf{O}=2\pi z$ mit $z \in \mathbb{Z}$.
Damit lässt sich das reziproke Gitter durch die Menge
$\{ \mathbf{R} \,\vert\, \exp(\mathrm{i}\mathbf{R}\cdot \mathbf{O})=1 \quad
\forall \mathbf{O} \in \text{span}(\mathbf{a}_{i}) \}$ definieren. \autocite[108]{Ashcroft}
Analog zum Ortsraum lassen sich auch hier primitiven Vektoren mit folgender Vorschrift bilden:
\begin{align*}
    \mathbf{b}_{1} = 2\pi \cdot \frac{\mathbf{a}_{2} \times \mathbf{a}_{3}}{V_{\mathrm{EZ}}} \quad
    \mathbf{b}_{2} = 2\pi \cdot \frac{\mathbf{a}_{3} \times \mathbf{a}_{1}}{V_{\mathrm{EZ}}} \quad
    \mathbf{b}_{3} = 2\pi \cdot \frac{\mathbf{a}_{1} \times \mathbf{a}_{2}}{V_{\mathrm{EZ}}}.
\end{align*}
Jeder Punkt im reziproken Gitter kann durch $\sum_{i=1}^{3} \beta_{i}\mathbf{b}_{i}$ mit $\beta_i \in \mathbb{Z}, i=\{1,2,3\}$
beschrieben werden.
Es gilt die Relation $\mathbf{b}_{i}\cdot \mathbf{a}_{j}=2 \pi \delta_{ij}$.
Hierbei ist $\delta_{ij}$ das Kronecker-Delta.
\autocite[109]{Ashcroft}

\subsubsection{Indizierung von Gitterebenen und Gitterrichtungen}\label{subsubsec:indizierung}
Die erste wichtige Anwendung des reziproken Gitters ist die Charakterisierung von Gitterebenen.
Eine Gitterebene ist eine beliebige, im Bravaisgitter liegende Ebene, die mindestens drei nicht kollineare Gitterpunkte
enthält.
Aufgrund der Kristallsymmetrie liegen damit unendlich viele weitere Gitterpunkte innerhalb dieser Ebene.
Mithilfe der Translationssymmetrie findet man parallele Gitterebenen im Abstand $d$.
Diese fasst man als Gitterebenenscharen zusammen. \autocite[113]{Ashcroft}

Gitterebenenscharen kann man mithilfe des reziproken Gitters charakterisieren, denn für jede
Gitterebenenschar im Abstand $d$ existieren Vektoren des reziproken Gitters, welche senkrecht auf den Ebenen stehen.
Für die eindeutige Beschreibung wählt man den kleinsten dieser Vektoren $\mathbf{r}$, welche stets die Länge $2 \pi / d$ besitzt.
Auch die Umkehrung gilt: Für jeden Vektor $\mathbf{R}$ aus dem reziproken Gitter, existiert eine Schar von senkrechten
Gitterebenen.
Der Abstand $d$ dieser Ebenen ist an den Betrag des kleinsten parallelen Vektors $\mathbf{r}$ durch $\lvert \mathbf{r}
\rvert=2\pi  /d$ gekoppelt.
Es existiert also eine einfache Möglichkeit, Gitterebenen mithilfe von reziproken Gittervektoren eindeutig zu
identifizieren. \autocite[113]{Ashcroft}

Um Gitterebenenscharen zu kennzeichnen, verwendet man die Millerschen Indizes.
Sei dazu $\mathbf{r}$ der kürzeste reziproke Gittervektor, welcher senkrecht auf der zu charakterisierenden Ebene steht.
Dieser lässt sich darstellen durch $ \mathbf{r} = h \mathbf{b_1} + k \mathbf{b_2} + l \mathbf{b_3}$.
Das Tupel $(h\, k\,l)$ sind die Millerschen Indizes, welche definitionsgemäß aus ganzen Zahlen bestehen müssen.


Für die Millerschen Indizes existiert eine geometrische Interpretation, die es erlaubt, die Indizes im Ortsraum zu
visualisieren.
Für jede Gitterebene findet man ein entsprechendes $A$, sodass die Ebenengleichung $\mathbf{r} \cdot \mathbf{x} = A$
erfüllt ist.
Nun definiert man die Durchstoßpunkte zwischen den durch die primitiven Vektoren $\mathbf{a}_i$ aufgespannten
Koordinatenachsen und der Ebene durch die Zahlen $x_{1}\mathbf{a}_{1}, x_{2}\mathbf{a}_{2}, x_{3}\mathbf{a}_{3}$.
Da die Durchstoßpunkte in der Ebene liegen, ist die Ebenengleichung $\mathbf{r}\cdot(x_{i}\mathbf{a}_{i})=A$ erfüllt und
man findet mit $\mathbf{r}\cdot\mathbf{a}_{1}=2\pi h$, $\mathbf{r}\cdot \mathbf{a}_{2}=2\pi k$,
$ \mathbf{r}\cdot \mathbf{a}_{3}=2\pi l$ folgenden Zusammenhang:
\begin{equation*}
    x_{1}=\frac{A}{2\pi h}, \quad x_{2}=\frac{A}{2\pi k}, \quad x_{3} =\frac{A}{2\pi l}.
\end{equation*}
Kennt man die Achsendurchstoßpunkte $x_i$, kann man die Millerschen Indizes finden, indem man den Parameter $A$
kleinstmöglich wählt, sodass $h$, $k$ und $l$ ganzzahlig sind. \autocite[115]{Ashcroft}

Nicht nur Gitterebenen, sondern auch Gitterrichtungen lassen sich in ähnlicher Weise indizieren.
Das Tupel $[h\,k\,l]$ beschreibt diejenige Richtung, die durch den Vektor $\mathbf{O} = h\mathbf{a}_{1}+k\mathbf{a}_{2}+
l\mathbf{a}_{3}$ vorgegeben wird.
Zu beachten ist, dass Richtungsvektoren in unserem Kontext im Ortsraum leben,
währenddessen Ebenennormalenvektoren durch Vektoren im reziproken Raum dargestellt werden.
Um dies zu verdeutlichen werden eckige anstelle der runden Klammern verwendet.
Es existiert weiterhin eine besondere Notation zur Kennzeichnung äquivalenter Gitterebenenscharen und Raumrichtungen.
In diesem Kontext bedeutet das, dass die Möglichkeit besteht, äquivalente Gitterebenenscharen und Raumrichtung durch
Symmetrieoperationen ineinander zu überführen.
Äquivalente Ebenen notiert man mit $\{h \,k\, l \}$, für Richtungen gilt entsprechend $\langle h\, k \, l \rangle$.
\autocite[116]{Ashcroft}

\subsubsection{Röntgenbeugung und die Laue Bedingung}
Ziel ist es, mithilfe elektromagnetischer Strahlung und den bisherigen Überlegungen, Aussagen über die
Kristallstruktur eines Festkörpers zu gewinnen.
Genauer gesagt sucht man einen Formalismus, um die elastische Streuung von Licht am Kristallgitter zu beschreiben.
Für eine geeignete Wellenlänge der Photonen betrachtet man die typische zwischenatomare Entfernung von circa
\qty{1}{\angstrom}.
Aus der Optik ist bekannt, dass mindestens eine Wellenlänge gleicher Größenordnung genutzt werden
muss, um beide Punkte hinreichend genau aufzulösen.
Entsprechend muss die Photonenenergie in der Ordnung von
$\h f =\h \c / \lambda \simeq \qty{12.3}{\electronvolt}$ liegen.
Hierbei ist $f$ die Frequenz, $\lambda$ die Wellenlänge, $\h$ das Plancksche Wirkungsquantum und $\c$ die
Lichtgeschwindigkeit.
Solche Energien sind charakteristisch für Röntgenstrahlung. \autocite[120]{Ashcroft}

Laue und Bragg entwickelten zwei Formalismen, um elastischen Streuung elektromagnetischer Strahlung am Kristallgitter
zu beschreiben.
Im Folgenden wird der Laue-Formalismus erklärt und die Äquivalenz zur Bragg-Bedingung gezeigt.

\paragraph{Streuung an zwei Gitterpunkten}
\begin{figure}
    \centering
    \includegraphics[width=0.5\textwidth]{../assets/theorie/lauebeugung}
    \caption{Skizze zur Erklärung der Laue-Bedingung. \imcitetwo[123]{Ashcroft}} \label{fig:laue}
\end{figure}
Zur Herleitung der Beugungsbedingung nach Laue betrachtet man die Bestrahlung zweier Gitterpunkte mit Photonen unter
der Annahme der elastischen Streuung.
Der Abstand der Gitterpunkte ist durch den Vektor $\mathbf{O}$ gegeben.
Dabei ist die einfallende Strahlung durch den Wellenzahlvektor  $\mathbf{k}$ charakterisiert.
Es gilt der Zusammenhang $\lvert \mathbf{k} \rvert = 2 \pi / \lambda$ und die Dispersionsrelation
$\omega = \c \lvert \mathbf{k} \rvert$.
Die gestreute Strahlung wird durch den Wellenzahlvektor $\mathbf{k}'$ beschrieben.
Da nur elastische Streuung betrachtet wird, gilt $\lvert \mathbf{k} \rvert=\lvert \mathbf{k}' \rvert$.
Somit sind die Frequenzen ($\omega$ und $\omega'$) beider Wellen gleich.
Für den Wegunterschied $\Delta s$ findet man anhand \cref{fig:laue} den Zusammenhang:
\begin{equation}
    \Delta s = \Delta s_{1} + \Delta s_{2} = \underbrace{ \mathbf{O} \cdot \frac{\mathbf{k}}{\lvert \mathbf{k} \rvert }
    -\mathbf{O}\cdot \frac{\mathbf{k}'}{\lvert \mathbf{k}' \rvert}  }_{ \text{Projektion von } \mathbf{O} \text{ auf }
    \mathbf{k} \text{ bzw. }\mathbf{k'} } =  \frac{\lambda}{2\pi} \mathbf{O}\cdot\Delta \mathbf{k}.
    \label{eq:laue}
\end{equation}
Hierbei ist $\Delta \mathbf{k} = \mathbf{k}-\mathbf{k}'$.
Für konstruktive Interferenz muss die Bedingung $\Delta s = n \lambda, n \in \mathbb{N}$ erfüllt sein, sodass durch Gleichsetzen
die Beziehung $\mathbf{O}\cdot\Delta \mathbf{k} =2\pi n$ folgt.
Aus der geometrischen Anordnung der Gitterpunkte ergibt sich ein Wegunterschied, der äquivalent
zu einer Phasendifferenz von $\Delta\varphi=(2\pi / \lambda) \cdot \Delta s = \Delta \mathbf{k}\cdot \mathbf{O}$ ist.
\autocite[122-123]{Ashcroft}

Unter der Annahme, dass beide Gitterpunkte Kugelwellen mit Amplituden $u_{0}(\mathbf{r}, t)$ und $u_{1}(\mathbf{r}, t)$
ausstrahlen, ergibt sich für die Überlagerung beider Wellen die folgende Form:
\begin{align}
    u(\mathbf{r},t)=u_{0}(\mathbf{r})\cdot \exp(\mathrm{i}\omega t+\mathrm{i}\lvert \mathbf{k}
    \rvert r) + u_{1}(\mathbf{r}) \cdot \exp(\mathrm{i}\omega t + \mathrm{i}\lvert \mathbf{k}
    \rvert r+\mathrm{i}\Delta\varphi)
\end{align}
Die Überlagerung der beiden Wellen führt zu einer Interferenz, die durch die Phasendifferenz $\Delta\varphi$ bestimmt
wird.

\paragraph{Streuung an allen Gitterpunkten}
Will man die Streuung im gesamten Kristall betrachten, muss über alle Streuzentren, das heißt über alle Gittervektoren,
im Kristall summiert werden.
Diese sind gegeben durch $\mathbf{O}_{pqr}=p\mathbf{a}_{1}+q\mathbf{a}_{2}+r\mathbf{a}_{3}$.
Für die Superposition aller Wellen mit Amplitude $u_{pqr}(\mathbf{r})$ und Phasendifferenz $\Delta\varphi_{pqr}$
ergibt sich:
\begin{align}
    u(\mathbf{r}, t)
    &=\sum_{pqr} u_{pqr}(\mathbf{r})\cdot \exp(\mathrm{i}\omega t+\mathrm{i}
    \lvert \mathbf{k} \rvert r+\mathrm{i}\underbrace{ \Delta\varphi_{pqr} }_{ = \Delta
    \mathbf{k}\cdot \mathbf{O}_{pqr}})) \\
    &=\exp(\mathrm{i}\omega t+\mathrm{i}\lvert \mathbf{k} \rvert r)\cdot
    \underbrace{ \sum_{pqr}u_{pqr}(\mathbf{r})\cdot \exp(\mathrm{i}\Delta \mathbf{k}
    \cdot \mathbf{O}_{pqr}) }_{ \coloneqq A }.
    \label{eq:amplitude}
\end{align}
Die Größe $A$ wird als Streuamplitude bezeichnet.
Die tatsächliche Streuung erfolgt an der Elektronenverteilung.
Das führt zu zusätzlichen Effekten wie dem Atomformfaktor und dem Strukturfaktor,
welche für weitere Betrachtungen jedoch vernachlässigt werden können.\autocite[66-69]{Spiess}
\begin{equation*}
    A \propto \int n(\mathbf{r}) \exp(\mathrm{i} \Delta \mathbf{k}\cdot
    \mathbf{r}) \, \mathrm{d}V(\mathbf{r})
\end{equation*}

\paragraph{Beugungsbedingung}
Die Bedingung für konstruktive Interferenz ist äquivalent zur Maximierung der Streuamplitude.
Damit diese maximal wird, muss die Interferenzbedingung $\mathbf{O}_{pqr}\cdot\Delta \mathbf{k} =2\pi z$
für alle $p$, $q$, $r$ gelten.
Zerlegt man $\mathbf{O}_{pqr}$ in seine Komponenten, erhält man folgende Gleichungen:
\begin{align}
    \mathbf{a}_{1}\cdot\Delta \mathbf{k} &= 2\pi h \\
    \mathbf{a}_{2}\cdot\Delta \mathbf{k} &= 2\pi k \\
    \mathbf{a}_{3}\cdot\Delta \mathbf{k} &= 2\pi l \\
\end{align}
Dies sind die Laue Gleichungen für Beugungsmaxima.
Sie sind erfüllt, falls $\Delta \mathbf{k}$ ein reziproker Gittervektor ist.
Dies stellt die Beugungsbedingung nach Laue dar, die besagt, dass konstruktive Interferenz genau dann auftritt,
wenn $\Delta \mathbf{r}$ einem Vektor des reziproken Gitters entspricht.\autocite[125]{Ashcroft}

Für die Strukturamplitude ergibt sich anschließend:
\begin{equation}
    A = \sum_{pqr} u_{pqr}(\mathbf{r}) \cdot\underbrace{ \exp(2\pi \mathrm{i}
    \cdot(\underbrace{ mh+nk+pl }_{ \in\mathbb{Z} })) }_{ =1 }
    = \sum_{pqr} u_{pqr}(\mathbf{r}).
    \label{eq:strukturamplitude}
\end{equation}


\paragraph{Bragg-Bedingung}
Die Bragg Bedingung kann sowohl mithilfe der Laue-Bedingung als auch geometrisch hergeleitet werden.
Um die Äquivalenz zur Laue-Bedingung zu zeigen, wird die Bragg-Bedingung im Folgenden aus der Laue-Bedingung
hergeleitet.
Betrachtet man die einen Differenzvektor $\Delta \mathbf{k}=\mathbf{k}'-\mathbf{k}$, wobei $\mathbf{k}$ und
$\mathbf{k'}$ den Winkel $\alpha$ einschließen,so ist sein Betrag gegeben durch:
\begin{align}
    \lvert \Delta \mathbf{k} \rvert ^{2}&=\langle \Delta \mathbf{k} ,\Delta \mathbf{k}\rangle =\langle \mathbf{k}-\mathbf{k}', \mathbf{k}-\mathbf{k}' \rangle = k^{2}+k'^{2}-2kk'\cos(\alpha)  \\
    &=2{k}^{2}(1-\cos(\alpha))=4k^{2}\sin ^{2}\left( \frac{\alpha}{2} \right)
\end{align}
Hierbei ist $\nu = \alpha / 2$ der Bragg-Winkel und $\langle \cdot , \cdot \rangle$ das Skalarprodukt.
Da $\Delta \mathbf{k}$ ein Vektor aus dem reziproken Raum ist und senkrecht auf der Gitterebene
steht, an welcher er gestreut wird, existiert ein kürzester Vektor $\mathbf{g}$, sodass
$n\cdot \mathbf{g} =\Delta \mathbf{k}$.
Aus \cref{subsection:indizierung} ist bekannt, dass $\mathbf{g}$ den Abstand der Netzebenen definiert
mit $d = 2 \pi \cdot\lvert \mathbf{g} \rvert ^{-1}$.
Daraus folgt $\lvert \Delta \mathbf{k} \rvert=n \lvert \mathbf{g} \rvert = 2\pi n/d$.
Kombiniert man beide Gleichungen miteinander, ergibt sich die Bragg-Bedingung:\autocite[125-126]{Ashcroft}
\begin{align}
    2k\sin(\nu)&=\frac{2\pi n}{d} \\
    2d\sin(\nu)&=n \lambda
\end{align}

\subsection{Entropiestabilisierte Metalloxide}\label{subsec:hochentropische-metalloxide}
Entropiestabilisierte Metalloxide bilden eine neuere Klasse von Materialien, die durch die Kombination unterschiedlicher
zufällig ausgewählter Metallkationen in einem Sauerstoff Anionengitter entstehen.
Wie der Name bereits andeutet, ist die Entropie ein entscheidender Faktor für die Stabilität dieser Materialien.
    \section{Messmethoden}\label{sec:messmethoden}
\subsection{PLD}
%TODO
\subsection{XRD}\label{subsec:xrd}
Nachdem die Dünnfilme hergestellt wurden, ist der nächste Schritt, ihre Struktur zu charakterisieren.
Zwischen Dünnfilmen und ihren korrespondierenden Massivkörpern existieren signifikante Unterschiede.
Diese resultieren vorrangig aus dem Verhältnis von Oberfläche zu Volumen, sowie den jeweiligen
Wachstumsbedingungen, wie Temperatur, Druck und Substrat.
Sie zeigen sich beispielsweise in der Qualität der Kristallinität, sowie in Kompositionsgradienten.
Da Proben mit unterschiedlichen Wachstumsbedingungen hergestellt wurden und deren Eigenschaften dadurch
maßgeblich beeinflusst werden, ist es notwendig, die Kristallinität der Dünnfilme zu charakterisieren.
Röntgendiffraktometrie (XRD, engl. \textit{X-Ray diffraction}) ist eine weit verbreitete Methode, um die
Kristallstruktur von Dünnfilmen zu bestimmen.
Dabei wird ein Röntgendiffraktometer verwendet.

\subsubsection{Röntgendiffraktometer}
Das Röntgendiffraktometer besteht aus fünf Hauptkomponenten: Röntgenquelle und Detektor, Ein- und Ausfallsoptik,
sowie dem Goniometer.
Zusätzlich ist das Diffraktometer durch eine Strahlungsschutzverkleidung abgeschirmt und mit einer Steuerungssoftware
verbunden.
Im Folgenden werden die einzelnen Komponenten näher erläutert.

\paragraph{Röntgenquelle}
Die Röntgenstrahlen werden in einer Röntgenröhre erzeugt.
In dieser werden Elektronen aus einer Wolfram-Glühkathode emittiert und durch das elektrische Feld auf eine Anode
beschleunigt.
Die Anode besteht meist aus hochreinem Kupfer.
Stromstärke und Beschleunigungsspannung der Röntgenröhre müssen so gewählt werden, dass die Energie beim Auftreffen der
Elektronen auf die Anode ausreicht, um die gebundenen Elektronen der Atome auf das nächsthöhere Energieniveau anzuregen.
Aufgrund der daraus resultierende Wärme muss die Anode ständig wassergekühlt werden.
Im hauseigenen Röntgendiffraktometer wird eine Beschleunigungsspannung von \qty{40}{\kilo\volt} und eine Stromstärke
von \qty{40}{\milli\ampere} verwendet.

Nach der Kollision zwischen Kupferatom und Elektron relaxiert das Elektron unter Bildung eines Röntgenphotons.
Man erhält ein Spektrum, welches durch die charakteristische Strahlung der Anode sowie durch Bremsstrahlung
geprägt ist.
Die charakteristische Strahlung wird vorrangig durch die K-Linien, insbesondere $K_{\alpha_1}$, $K_{\alpha_2}$
und K$_{\beta}$, dominiert.
Da $K_{\alpha_1}$ und $K_{\alpha_2}$ energetisch sehr nahe beieinander liegen, können sie nicht immer einzeln
aufgelöst werden.
Die $K_{\beta}$ Strahlung ist größtenteils unerwünscht und kann durch geeignete Filter unterdrückt werden.


Die Wolfram-Glühkathode emittiert unerwüschterweise nicht nur Elektronen, sondern auch Wolfram-Atome in kleinen Mengen.
Über längere Zeiträume führt dies zu einer nicht mehr zu vernachlässigenden Kontamination der Anode.
Dadurch können bei Elektronenstößen auch Wolfram-Atome angeregt werden, was zu einer zusätzlichen Wellenlänge im
Spektrum führt
In den späteren Messergebnissen sind diese Beiträge erkennbar.
Abschließend gelangen die Röntgenstrahlen durch ein Berylliumfenster in die Einfallsoptik.

\paragraph{Röntgendetektor}
Die durch die Röntgenquelle erzeugten Strahlen gelangen nach Reflektion an der Probe in den Detektor.
Dieser dient dazu, den reflektierten Strahl in ein elektrisches Signal umzuwandeln.
Kategorisieren kann man Röntgendektektoren nach ihrer Funktionsweise.
Eine weitere Unterteilung erfolgt nach der Dimensionalität des Detektors.
Es können Punktdetektoren (0D), Linien- (1D) oder Flächendetektoren (2D) verwendet werden.
Im hauseigenen Röntgendiffraktometer ist ein Halbleiterdetektor verbaut, der in verschiedenen Dimensionalitäten
arbeiten kann.
Wichtig ist, dass die maximale Zählrate des Detektors nicht überschritten wird.
Das führt zu nichtlinearen Antworten und kann den Sensor beschädigen.
Um das zu vermeiden, können Filter und Attenuatoren verwendet werden.

\paragraph{Goniometer}
Das Goniometer ist die mechanische Komponente des Röntgendiffraktometers.
Es besteht aus mehreren Drehachsen, die es ermöglichen, die Probe in unterschiedlichsten Winkeln auszurichten.
Nach der Braggschen Beugungstheorie ergeben sich konstruktive Interferenzen an denjenigen Winkeln, die der
Bragg-Bedingung genügen.
Existieren Möglichkeit, die Winkel für Quelle und Detektor zu variieren, kann diese Interferenz beobachtet werden.
Im Allgemeinen ist die Röntgenquelle jedoch fest, eine äquivalente Drehung von Probe und Detektor ist deshalb gängig.
In der einfachsten Betrachtungsweise muss das Goniometer also den Winkel zwischen Probe und Quelle ($\omega$) und dem
Winkel zwischen Probe und Detektor ($2\theta$) einstellen können.
Diese Freiheit reicht zwar für Pulverproben, jedoch nicht für Dünnfilme.
Zwar kann man mit beiden Freiheitsgraden Messungen durchführen, welche die out-of-plane Orientierung charakterisieren,
jedoch ist es nicht möglich, die in-plane Orientierung zu bestimmen.
Dafür werden weitere Achsen, wie $\varphi$ und $\chi$, benötigt.
Eine Konstruktion mit den vier Achsen wird Euler-Wiege genannt.
%%TODO



\subsection{AFM}
Das Raster-Kraft-Mikroskop ist ein hochpräzises Messinstrument zum Erfassen von Oberflächenstrukturen.
Anders als bei Licht- oder Elektronenmikroskopie wird hierbei eine mechanische Funktionsweise umgesetzt.
Dabei fährt eine Messapparatur, der Cantilever, rasterweise über eine Oberfläche und tastet diese ab.
Die auf den Cantilever wirkenden atomaren oder magnetischen Kräfte werden gemessen, woraus eine Topographiekarte der
Oberfläche erstellt wird.

\subsubsection{Schematischer Aufbau und Funktionsweise}
Die grundlegende Funktionsweise ist in Abbildung 1 dargestellt.
Markierung 1 zeigt den Cantilever, der mit einer Messspitze mit Dimensionen im Nanometerbereich ausgestattet ist.
Fährt diese über die Probe, siehe Markierung 2, so wirken Kräfte auf die Spitze, welche den Cantilever auslenken.
Diese Auslenkung wird mittels eines Ablenkungserkennungssystems, Markierung 3, ausgewertet.
Hierbei wird ein Laserstrahl an der Rückseite des Cantilevers reflektiert, welcher anschließend auf einen Photodetektor
trifft.
Dieser Detektor kann nun anhand der Intensitätsverteilung auf den einzelnen Sektoren die Auslenkung und Torsion des
Cantilevers messen.
Die gemessene Auslenkung wird an das Feedback System übergeben, was Markierung 4 zeigt.
Basierend auf dem gewählten Betriebsmodus wird versucht, die gemessene Kraft oder Amplitude konstant zu halten.
Mithilfe dieser Regulation wird ein Korrektursignal ausgegeben, welches die Position des Cantilevers anpasst.
Dies geschieht mithilfe von Piezoelementen, wodurch der Cantilever in x, y oder z-Richtung bewegt werden kann.
Die z-Position des Cantilevers wird aufgezeichnet und als Topographiesignal am Computer ausgewertet, siehe Markierung 5.

\subsubsection{PID-System}

    \section{Auswertung}\label{sec:auswertung}

\subsection{Sauerstoff Ausheizvorgang}\label{subsec:sauerstoff-ausheizvorgang2}

\subsubsection{Initialzustand}
\begin{figure}[ht]
    \foreach \i in {W6821-1B,W6822-1B,W6823-1B,W6824-1B}{
        \begin{subfigure}[t]{0.40\textwidth}
            \includegraphics[width=\textwidth]
            {../plots/AFM/XG-Sauerstoff/XG-pre/\i/\i_XG_Sauerstoff_pre_Topography_1}
            \caption{\i}
        \end{subfigure}
    }
    \label{fig: AFM, Sauerstoff, Initialzustand}
\end{figure}

\subsubsection{Probe W6821-1B}
\begin{figure}
    \centering
    \foreach \i in {pre,600,700,750,800,875}{
        \begin{subfigure}[t]{0.40\textwidth}
            \centering
            \includegraphics[width=\textwidth]
            {../plots/AFM/XG-Sauerstoff/XG-\i/W6821-1B/W6821-1B_XG_Sauerstoff_\i_Topography_1}
            \caption{W6821-1B, \i}
        \end{subfigure}
    }
    \label{fig: AFM, Sauerstoff, W6821-1B}
\end{figure}

\subsubsection{Probe W6822-1B}
\begin{figure}
    \centering
    \foreach \i in {pre,600,700,750,800,875}{
        \begin{subfigure}[t]{0.40\textwidth}
            \centering
            \includegraphics[width=\textwidth]
            {../plots/AFM/XG-Sauerstoff/XG-\i/W6822-1B/W6822-1B_XG_Sauerstoff_\i_Topography_1}
            \caption{W6822-1B, \i}
        \end{subfigure}
    }
    \label{fig: AFM, Sauerstoff, W6822-1B}
\end{figure}

\subsection{Rauigkeit}\label{subsec:rauigkeit}
\begin{figure}
    \centering
    \import{../plots/AFM}{sauerstoff.pgf}
    \caption{AFM, Sauerstoff}
    \label{fig: AFM, Sauerstoff}
\end{figure}

\begin{figure}
    \centering
    \import{../plots/AFM}{vakuum.pgf}
    \caption{AFM, Vakuum}
    \label{fig: AFM, Vakuum}
\end{figure}

\subsection{XRD}\label{subsec:xrd2}

\subsubsection{Sauerstoff}
\begin{figure}
    \centering
    \import{../plots/XRD}{W6821-1B_Sauerstoff.pgf}
    \caption{W6821-1B, Sauerstoff}
    \label{fig: XRD, W6821-1B, Sauerstoff}
\end{figure}

\begin{figure}
    \centering
    \import{../plots/XRD}{W6822-1B_Sauerstoff.pgf}
    \caption{W6822-1B, Sauerstoff}
    \label{fig: XRD, W6822-1B, Sauerstoff}
\end{figure}

\begin{figure}
    \centering
    \import{../plots/XRD}{W6823-1B_Sauerstoff.pgf}
    \caption{W6823-1B, Sauerstoff}
    \label{fig: XRD, W6823-1B, Sauerstoff}
\end{figure}

\begin{figure}
    \centering
    \import{../plots/XRD}{W6824-1B_Sauerstoff.pgf}
    \caption{W6824-1B, Sauerstoff}
    \label{fig: XRD, W6824-1B, Sauerstoff}
\end{figure}

\subsubsection{Vakuum}
\begin{figure}
    \centering
    \import{../plots/XRD}{W6821-1C_Vakuum.pgf}
    \caption{W6821-1C, Vakuum}
    \label{fig: XRD, W6821-1C, Vakuum}
\end{figure}

\begin{figure}
    \centering
    \import{../plots/XRD}{W6822-1C_Vakuum.pgf}
    \caption{W6822-1C, Vakuum}
    \label{fig: XRD, W6822-1C, Vakuum}
\end{figure}

\begin{figure}
    \centering
    \import{../plots/XRD}{W6823-1C_Vakuum.pgf}
    \caption{W6823-1C, Vakuum}
    \label{fig: XRD, W6823-1C, Vakuum}
\end{figure}

\begin{figure}
    \centering
    \import{../plots/XRD}{W6824-1C_Vakuum.pgf}
    \caption{W6824-1C, Vakuum}
    \label{fig: XRD, W6824-1C, Vakuum}
\end{figure}


    \section{Appendix}\label{sec:appendix}

\subsection{Bildvergleich AFM}\label{subsec:bildvergleich-afm}

\subsection{Sauerstoff Ausheizvorgang}\label{subsec:sauerstoff-ausheizvorgang}

\subsubsection{Vor dem Ausheizen}
\begin{figure}[ht]
\centering
% W6821-1B
\begin{subfigure}[t]{0.40\textwidth}
\centering
\includegraphics[width=\textwidth]
{../plots/AFM/XG-Sauerstoff/XG-pre/W6821-1B/W6821-1B_XG_Sauerstoff_pre_Topography_1}
\caption{W6821-1B, Bild 1}
\end{subfigure}
\begin{subfigure}[t]{0.40\textwidth}
\centering
\includegraphics[width=\textwidth]
{../plots/AFM/XG-Sauerstoff/XG-pre/W6821-1B/W6821-1B_XG_Sauerstoff_pre_Topography_3}
\caption{W6821-1B, Bild 2}
\end{subfigure}
% W6822-1B
\begin{subfigure}[t]{0.40\textwidth}
\centering
\includegraphics[width=\textwidth]
{../plots/AFM/XG-Sauerstoff/XG-pre/W6822-1B/W6822-1B_XG_Sauerstoff_pre_Topography_1}
\caption{W6822-1B, Bild 1}
\end{subfigure}
\begin{subfigure}[t]{0.40\textwidth}
\centering
\includegraphics[width=\textwidth]
{../plots/AFM/XG-Sauerstoff/XG-pre/W6822-1B/W6822-1B_XG_Sauerstoff_pre_Topography_3}
\caption{W6822-1B, Bild 2}
\end{subfigure}
% W6823-1B
\begin{subfigure}[t]{0.40\textwidth}
\centering
\includegraphics[width=\textwidth]
{../plots/AFM/XG-Sauerstoff/XG-pre/W6823-1B/W6823-1B_XG_Sauerstoff_pre_Topography_1}
\caption{W6823-1B, Bild 1}
\end{subfigure}
\begin{subfigure}[t]{0.40\textwidth}
\centering
\includegraphics[width=\textwidth]
{../plots/AFM/XG-Sauerstoff/XG-pre/W6823-1B/W6823-1B_XG_Sauerstoff_pre_Topography_3}
\caption{W6823-1B, Bild 2}
\end{subfigure}
% W6824-1B
\begin{subfigure}[t]{0.40\textwidth}
\centering
\includegraphics[width=\textwidth]
{../plots/AFM/XG-Sauerstoff/XG-pre/W6824-1B/W6824-1B_XG_Sauerstoff_pre_Topography_1}
\caption{W6824-1B, Bild 1}
\end{subfigure}
\begin{subfigure}[t]{0.40\textwidth}
\centering
\includegraphics[width=\textwidth]
{../plots/AFM/XG-Sauerstoff/XG-pre/W6824-1B/W6824-1B_XG_Sauerstoff_pre_Topography_3}
\caption{W6824-1B, Bild 2}
\end{subfigure}
\caption{AFM, Sauerstoff, pre}\label{fig: AFM, Sauerstoff, pre}
\end{figure}

\subsubsection{\qty{600}{\degreeCelsius}}
\begin{figure}[ht]
\centering
% W6821-1B
\begin{subfigure}[t]{0.40\textwidth}
\centering
\includegraphics[width=\textwidth]
{../plots/AFM/XG-Sauerstoff/XG-600/W6821-1B/W6821-1B_XG_Sauerstoff_600_Topography_1}
\caption{W6821-1B, Bild 1}
\end{subfigure}
\begin{subfigure}[t]{0.40\textwidth}
\centering
\includegraphics[width=\textwidth]
{../plots/AFM/XG-Sauerstoff/XG-600/W6821-1B/W6821-1B_XG_Sauerstoff_600_Topography_3}
\caption{W6821-1B, Bild 2}
\end{subfigure}
% W6822-1B
\begin{subfigure}[t]{0.40\textwidth}
\centering
\includegraphics[width=\textwidth]
{../plots/AFM/XG-Sauerstoff/XG-600/W6822-1B/W6822-1B_XG_Sauerstoff_600_Topography_1}
\caption{W6822-1B, Bild 1}
\end{subfigure}
\begin{subfigure}[t]{0.40\textwidth}
\centering
\includegraphics[width=\textwidth]
{../plots/AFM/XG-Sauerstoff/XG-600/W6822-1B/W6822-1B_XG_Sauerstoff_600_Topography_3}
\caption{W6822-1B, Bild 2}
\end{subfigure}
% W6823-1B
\begin{subfigure}[t]{0.40\textwidth}
\centering
\includegraphics[width=\textwidth]
{../plots/AFM/XG-Sauerstoff/XG-600/W6823-1B/W6823-1B_XG_Sauerstoff_600_Topography_1}
\caption{W6823-1B, Bild 1}
\end{subfigure}
\begin{subfigure}[t]{0.40\textwidth}
\centering
\includegraphics[width=\textwidth]
{../plots/AFM/XG-Sauerstoff/XG-600/W6823-1B/W6823-1B_XG_Sauerstoff_600_Topography_3}
\caption{W6823-1B, Bild 2}
\end{subfigure}
% W6824-1B
\begin{subfigure}[t]{0.40\textwidth}
\centering
\includegraphics[width=\textwidth]
{../plots/AFM/XG-Sauerstoff/XG-600/W6824-1B/W6824-1B_XG_Sauerstoff_600_Topography_1}
\caption{W6824-1B, Bild 1}
\end{subfigure}
\begin{subfigure}[t]{0.40\textwidth}
\centering
\includegraphics[width=\textwidth]
{../plots/AFM/XG-Sauerstoff/XG-600/W6824-1B/W6824-1B_XG_Sauerstoff_600_Topography_3}
\caption{W6824-1B, Bild 2}
\end{subfigure}
\caption{AFM, Sauerstoff, \qty{600}{\degreeCelsius}}\label{fig: AFM, Sauerstoff, 600}
\end{figure}

\subsubsection{\qty{700}{\degreeCelsius}}
\begin{figure}[ht]
\centering
% W6821-1B
\begin{subfigure}[t]{0.40\textwidth}
\centering
\includegraphics[width=\textwidth]
{../plots/AFM/XG-Sauerstoff/XG-700/W6821-1B/W6821-1B_XG_Sauerstoff_700_Topography_1}
\caption{W6821-1B, Bild 1}
\end{subfigure}
\begin{subfigure}[t]{0.40\textwidth}
\centering
\includegraphics[width=\textwidth]
{../plots/AFM/XG-Sauerstoff/XG-700/W6821-1B/W6821-1B_XG_Sauerstoff_700_Topography_3}
\caption{W6821-1B, Bild 2}
\end{subfigure}
% W6822-1B
\begin{subfigure}[t]{0.40\textwidth}
\centering
\includegraphics[width=\textwidth]
{../plots/AFM/XG-Sauerstoff/XG-700/W6822-1B/W6822-1B_XG_Sauerstoff_700_Topography_1}
\caption{W6822-1B, Bild 1}
\end{subfigure}
\begin{subfigure}[t]{0.40\textwidth}
\centering
\includegraphics[width=\textwidth]
{../plots/AFM/XG-Sauerstoff/XG-700/W6822-1B/W6822-1B_XG_Sauerstoff_700_Topography_3}
\caption{W6822-1B, Bild 2}
\end{subfigure}
% W6823-1B
\begin{subfigure}[t]{0.40\textwidth}
\centering
\includegraphics[width=\textwidth]
{../plots/AFM/XG-Sauerstoff/XG-700/W6823-1B/W6823-1B_XG_Sauerstoff_700_Topography_1}
\caption{W6823-1B, Bild 1}
\end{subfigure}
\begin{subfigure}[t]{0.40\textwidth}
\centering
\includegraphics[width=\textwidth]
{../plots/AFM/XG-Sauerstoff/XG-700/W6823-1B/W6823-1B_XG_Sauerstoff_700_Topography_3}
\caption{W6823-1B, Bild 2}
\end{subfigure}
% W6824-1B
\begin{subfigure}[t]{0.40\textwidth}
\centering
\includegraphics[width=\textwidth]
{../plots/AFM/XG-Sauerstoff/XG-700/W6824-1B/W6824-1B_XG_Sauerstoff_700_Topography_1}
\caption{W6824-1B, Bild 1}
\end{subfigure}
\begin{subfigure}[t]{0.40\textwidth}
\centering
\includegraphics[width=\textwidth]
{../plots/AFM/XG-Sauerstoff/XG-700/W6824-1B/W6824-1B_XG_Sauerstoff_700_Topography_3}
\caption{W6824-1B, Bild 2}
\end{subfigure}
\caption{AFM, Sauerstoff, \qty{700}{\degreeCelsius}}\label{fig: AFM, Sauerstoff, 700}
\end{figure}

\subsubsection{\qty{750}{\degreeCelsius}}
\begin{figure}[ht]
\centering
% W6821-1B
\begin{subfigure}[t]{0.40\textwidth}
\centering
\includegraphics[width=\textwidth]
{../plots/AFM/XG-Sauerstoff/XG-750/W6821-1B/W6821-1B_XG_Sauerstoff_750_Topography_1}
\caption{W6821-1B, Bild 1}
\end{subfigure}
\begin{subfigure}[t]{0.40\textwidth}
\centering
\includegraphics[width=\textwidth]
{../plots/AFM/XG-Sauerstoff/XG-750/W6821-1B/W6821-1B_XG_Sauerstoff_750_Topography_3}
\caption{W6821-1B, Bild 2}
\end{subfigure}
% W6822-1B
\begin{subfigure}[t]{0.40\textwidth}
\centering
\includegraphics[width=\textwidth]
{../plots/AFM/XG-Sauerstoff/XG-750/W6822-1B/W6822-1B_XG_Sauerstoff_750_Topography_1}
\caption{W6822-1B, Bild 1}
\end{subfigure}
\begin{subfigure}[t]{0.40\textwidth}
\centering
\includegraphics[width=\textwidth]
{../plots/AFM/XG-Sauerstoff/XG-750/W6822-1B/W6822-1B_XG_Sauerstoff_750_Topography_3}
\caption{W6822-1B, Bild 2}
\end{subfigure}
% W6823-1B
\begin{subfigure}[t]{0.40\textwidth}
\centering
\includegraphics[width=\textwidth]
{../plots/AFM/XG-Sauerstoff/XG-750/W6823-1B/W6823-1B_XG_Sauerstoff_750_Topography_1}
\caption{W6823-1B, Bild 1}
\end{subfigure}
\begin{subfigure}[t]{0.40\textwidth}
\centering
\includegraphics[width=\textwidth]
{../plots/AFM/XG-Sauerstoff/XG-750/W6823-1B/W6823-1B_XG_Sauerstoff_750_Topography_3}
\caption{W6823-1B, Bild 2}
\end{subfigure}
% W6824-1B
\begin{subfigure}[t]{0.40\textwidth}
\centering
\includegraphics[width=\textwidth]
{../plots/AFM/XG-Sauerstoff/XG-750/W6824-1B/W6824-1B_XG_Sauerstoff_750_Topography_1}
\caption{W6824-1B, Bild 1}
\end{subfigure}
\begin{subfigure}[t]{0.40\textwidth}
\centering
\includegraphics[width=\textwidth]
{../plots/AFM/XG-Sauerstoff/XG-750/W6824-1B/W6824-1B_XG_Sauerstoff_750_Topography_3}
\caption{W6824-1B, Bild 2}
\end{subfigure}
\caption{AFM, Sauerstoff, \qty{750}{\degreeCelsius}}\label{fig: AFM, Sauerstoff, 750}
\end{figure}

\subsubsection{\qty{800}{\degreeCelsius}}
\begin{figure}[ht]
\centering
% W6821-1B
\begin{subfigure}[t]{0.40\textwidth}
\centering
\includegraphics[width=\textwidth]
{../plots/AFM/XG-Sauerstoff/XG-800/W6821-1B/W6821-1B_XG_Sauerstoff_800_Topography_1}
\caption{W6821-1B, Bild 1}
\end{subfigure}
\begin{subfigure}[t]{0.40\textwidth}
\centering
\includegraphics[width=\textwidth]
{../plots/AFM/XG-Sauerstoff/XG-800/W6821-1B/W6821-1B_XG_Sauerstoff_800_Topography_3}
\caption{W6821-1B, Bild 2}
\end{subfigure}
% W6822-1B
\begin{subfigure}[t]{0.40\textwidth}
\centering
\includegraphics[width=\textwidth]
{../plots/AFM/XG-Sauerstoff/XG-800/W6822-1B/W6822-1B_XG_Sauerstoff_800_Topography_1}
\caption{W6822-1B, Bild 1}
\end{subfigure}
\begin{subfigure}[t]{0.40\textwidth}
\centering
\includegraphics[width=\textwidth]
{../plots/AFM/XG-Sauerstoff/XG-800/W6822-1B/W6822-1B_XG_Sauerstoff_800_Topography_3}
\caption{W6822-1B, Bild 2}
\end{subfigure}
% W6823-1B
\begin{subfigure}[t]{0.40\textwidth}
\centering
\includegraphics[width=\textwidth]
{../plots/AFM/XG-Sauerstoff/XG-800/W6823-1B/W6823-1B_XG_Sauerstoff_800_Topography_1}
\caption{W6823-1B, Bild 1}
\end{subfigure}
\begin{subfigure}[t]{0.40\textwidth}
\centering
\includegraphics[width=\textwidth]
{../plots/AFM/XG-Sauerstoff/XG-800/W6823-1B/W6823-1B_XG_Sauerstoff_800_Topography_3}
\caption{W6823-1B, Bild 2}
\end{subfigure}
% W6824-1B
\begin{subfigure}[t]{0.40\textwidth}
\centering
\includegraphics[width=\textwidth]
{../plots/AFM/XG-Sauerstoff/XG-800/W6824-1B/W6824-1B_XG_Sauerstoff_800_Topography_1}
\caption{W6824-1B, Bild 1}
\end{subfigure}
\begin{subfigure}[t]{0.40\textwidth}
\centering
\includegraphics[width=\textwidth]
{../plots/AFM/XG-Sauerstoff/XG-800/W6824-1B/W6824-1B_XG_Sauerstoff_800_Topography_3}
\caption{W6824-1B, Bild 2}
\end{subfigure}
\caption{AFM, Sauerstoff, \qty{800}{\degreeCelsius}}\label{fig: AFM, Sauerstoff, 800}
\end{figure}

\subsubsection{\qty{875}{\degreeCelsius}}
\begin{figure}[ht]
\centering
% W6821-1B
\begin{subfigure}[t]{0.40\textwidth}
\centering
\includegraphics[width=\textwidth]
{../plots/AFM/XG-Sauerstoff/XG-875/W6821-1B/W6821-1B_XG_Sauerstoff_875_Topography_1}
\caption{W6821-1B, Bild 1}
\end{subfigure}
\begin{subfigure}[t]{0.40\textwidth}
\centering
\includegraphics[width=\textwidth]
{../plots/AFM/XG-Sauerstoff/XG-875/W6821-1B/W6821-1B_XG_Sauerstoff_875_Topography_3}
\caption{W6821-1B, Bild 2}
\end{subfigure}
% W6822-1B
\begin{subfigure}[t]{0.40\textwidth}
\centering
\includegraphics[width=\textwidth]
{../plots/AFM/XG-Sauerstoff/XG-875/W6822-1B/W6822-1B_XG_Sauerstoff_875_Topography_1}
\caption{W6822-1B, Bild 1}
\end{subfigure}
\begin{subfigure}[t]{0.40\textwidth}
\centering
\includegraphics[width=\textwidth]
{../plots/AFM/XG-Sauerstoff/XG-875/W6822-1B/W6822-1B_XG_Sauerstoff_875_Topography_3}
\caption{W6822-1B, Bild 2}
\end{subfigure}
% W6823-1B
\begin{subfigure}[t]{0.40\textwidth}
\centering
\includegraphics[width=\textwidth]
{../plots/AFM/XG-Sauerstoff/XG-875/W6823-1B/W6823-1B_XG_Sauerstoff_875_Topography_1}
\caption{W6823-1B, Bild 1}
\end{subfigure}
\begin{subfigure}[t]{0.40\textwidth}
\centering
\includegraphics[width=\textwidth]
{../plots/AFM/XG-Sauerstoff/XG-875/W6823-1B/W6823-1B_XG_Sauerstoff_875_Topography_3}
\caption{W6823-1B, Bild 2}
\end{subfigure}
% W6824-1B
\begin{subfigure}[t]{0.40\textwidth}
\centering
\includegraphics[width=\textwidth]
{../plots/AFM/XG-Sauerstoff/XG-875/W6824-1B/W6824-1B_XG_Sauerstoff_875_Topography_1}
\caption{W6824-1B, Bild 1}
\end{subfigure}
\begin{subfigure}[t]{0.40\textwidth}
\centering
\includegraphics[width=\textwidth]
{../plots/AFM/XG-Sauerstoff/XG-875/W6824-1B/W6824-1B_XG_Sauerstoff_875_Topography_3}
\caption{W6824-1B, Bild 2}
\end{subfigure}
\caption{AFM, Sauerstoff, \qty{875}{\degreeCelsius}}\label{fig: AFM, Sauerstoff, 875}
\end{figure}

\subsection{Vakuum Ausheizvorgang}\label{subsec:vacuum-ausheizvorgang}

\subsubsection{Vor dem Ausheizen}
\begin{figure}[ht]
\centering
% W6821-1B
\begin{subfigure}[t]{0.40\textwidth}
\centering
\includegraphics[width=\textwidth]
{../plots/AFM/XG-Vakuum/XG-pre/W6821-1C/W6821-1C_XG_Vakuum_pre_Topography_1}
\caption{W6821-1C, Bild 1}
\end{subfigure}
\begin{subfigure}[t]{0.40\textwidth}
\centering
\includegraphics[width=\textwidth]
{../plots/AFM/XG-Vakuum/XG-pre/W6821-1C/W6821-1C_XG_Vakuum_pre_Topography_3}
\caption{W6821-1C, Bild 2}
\end{subfigure}
% W6822-1C
\begin{subfigure}[t]{0.40\textwidth}
\centering
\includegraphics[width=\textwidth]
{../plots/AFM/XG-Vakuum/XG-pre/W6822-1C/W6822-1C_XG_Vakuum_pre_Topography_1}
\caption{W6822-1C, Bild 1}
\end{subfigure}
\begin{subfigure}[t]{0.40\textwidth}
\centering
\includegraphics[width=\textwidth]
{../plots/AFM/XG-Vakuum/XG-pre/W6822-1C/W6822-1C_XG_Vakuum_pre_Topography_3}
\caption{W6822-1C, Bild 2}
\end{subfigure}
% W6823-1C
\begin{subfigure}[t]{0.40\textwidth}
\centering
\includegraphics[width=\textwidth]
{../plots/AFM/XG-Vakuum/XG-pre/W6823-1C/W6823-1C_XG_Vakuum_pre_Topography_1}
\caption{W6823-1C, Bild 1}
\end{subfigure}
\begin{subfigure}[t]{0.40\textwidth}
\centering
\includegraphics[width=\textwidth]
{../plots/AFM/XG-Vakuum/XG-pre/W6823-1C/W6823-1C_XG_Vakuum_pre_Topography_3}
\caption{W6823-1C, Bild 2}
\end{subfigure}
% W6824-1C
\begin{subfigure}[t]{0.40\textwidth}
\centering
\includegraphics[width=\textwidth]
{../plots/AFM/XG-Vakuum/XG-pre/W6824-1C/W6824-1C_XG_Vakuum_pre_Topography_1}
\caption{W6824-1C, Bild 1}
\end{subfigure}
\begin{subfigure}[t]{0.40\textwidth}
\centering
\includegraphics[width=\textwidth]
{../plots/AFM/XG-Vakuum/XG-pre/W6824-1C/W6824-1C_XG_Vakuum_pre_Topography_3}
\caption{W6824-1C, Bild 2}
\end{subfigure}
\caption{AFM, Vakuum, pre}\label{fig: AFM, Vakuum, pre}
\end{figure}

\subsubsection{\qty{500}{\degreeCelsius}}
\begin{figure}[ht]
\centering
% W6821-1C
\begin{subfigure}[t]{0.40\textwidth}
\centering
\includegraphics[width=\textwidth]
{../plots/AFM/XG-Vakuum/XG-500/W6821-1C/W6821-1C_XG_Vakuum_500_Topography_1}
\caption{W6821-1C, Bild 1}
\end{subfigure}
\begin{subfigure}[t]{0.40\textwidth}
\centering
\includegraphics[width=\textwidth]
{../plots/AFM/XG-Vakuum/XG-500/W6821-1C/W6821-1C_XG_Vakuum_500_Topography_3}
\caption{W6821-1C, Bild 2}
\end{subfigure}
% W6822-1C
\begin{subfigure}[t]{0.40\textwidth}
\centering
\includegraphics[width=\textwidth]
{../plots/AFM/XG-Vakuum/XG-500/W6822-1C/W6822-1C_XG_Vakuum_500_Topography_1}
\caption{W6822-1C, Bild 1}
\end{subfigure}
\begin{subfigure}[t]{0.40\textwidth}
\centering
\includegraphics[width=\textwidth]
{../plots/AFM/XG-Vakuum/XG-500/W6822-1C/W6822-1C_XG_Vakuum_500_Topography_3}
\caption{W6822-1C, Bild 2}
\end{subfigure}
% W6823-1C
\begin{subfigure}[t]{0.40\textwidth}
\centering
\includegraphics[width=\textwidth]
{../plots/AFM/XG-Vakuum/XG-500/W6823-1C/W6823-1C_XG_Vakuum_500_Topography_1}
\caption{W6823-1C, Bild 1}
\end{subfigure}
\begin{subfigure}[t]{0.40\textwidth}
\centering
\includegraphics[width=\textwidth]
{../plots/AFM/XG-Vakuum/XG-500/W6823-1C/W6823-1C_XG_Vakuum_500_Topography_3}
\caption{W6823-1C, Bild 2}
\end{subfigure}
% W6824-1C
\begin{subfigure}[t]{0.40\textwidth}
\centering
\includegraphics[width=\textwidth]
{../plots/AFM/XG-Vakuum/XG-500/W6824-1C/W6824-1C_XG_Vakuum_500_Topography_1}
\caption{W6824-1C, Bild 1}
\end{subfigure}
\begin{subfigure}[t]{0.40\textwidth}
\centering
\includegraphics[width=\textwidth]
{../plots/AFM/XG-Vakuum/XG-500/W6824-1C/W6824-1C_XG_Vakuum_500_Topography_3}
\caption{W6824-1C, Bild 2}
\end{subfigure}
\caption{AFM, Vakuum, \qty{500}{\degreeCelsius}}\label{fig: AFM, Vakuum, 500}
\end{figure}

\subsubsection{\qty{600}{\degreeCelsius}}
\begin{figure}[ht]
\centering
% W6821-1C
\begin{subfigure}[t]{0.40\textwidth}
\centering
\includegraphics[width=\textwidth]
{../plots/AFM/XG-Vakuum/XG-600/W6821-1C/W6821-1C_XG_Vakuum_600_Topography_1}
\caption{W6821-1C, Bild 1}
\end{subfigure}
\begin{subfigure}[t]{0.40\textwidth}
\centering
\includegraphics[width=\textwidth]
{../plots/AFM/XG-Vakuum/XG-600/W6821-1C/W6821-1C_XG_Vakuum_600_Topography_3}
\caption{W6821-1C, Bild 2}
\end{subfigure}
% W6822-1C
\begin{subfigure}[t]{0.40\textwidth}
\centering
\includegraphics[width=\textwidth]
{../plots/AFM/XG-Vakuum/XG-600/W6822-1C/W6822-1C_XG_Vakuum_600_Topography_1}
\caption{W6822-1C, Bild 1}
\end{subfigure}
\begin{subfigure}[t]{0.40\textwidth}
\centering
\includegraphics[width=\textwidth]
{../plots/AFM/XG-Vakuum/XG-600/W6822-1C/W6822-1C_XG_Vakuum_600_Topography_3}
\caption{W6822-1C, Bild 2}
\end{subfigure}
% W6823-1C
\begin{subfigure}[t]{0.40\textwidth}
\centering
\includegraphics[width=\textwidth]
{../plots/AFM/XG-Vakuum/XG-600/W6823-1C/W6823-1C_XG_Vakuum_600_Topography_1}
\caption{W6823-1C, Bild 1}
\end{subfigure}
\begin{subfigure}[t]{0.40\textwidth}
\centering
\includegraphics[width=\textwidth]
{../plots/AFM/XG-Vakuum/XG-600/W6823-1C/W6823-1C_XG_Vakuum_600_Topography_3}
\caption{W6823-1C, Bild 2}
\end{subfigure}
% W6824-1C
\begin{subfigure}[t]{0.40\textwidth}
\centering
\includegraphics[width=\textwidth]
{../plots/AFM/XG-Vakuum/XG-600/W6824-1C/W6824-1C_XG_Vakuum_600_Topography_1}
\caption{W6824-1C, Bild 1}
\end{subfigure}
\begin{subfigure}[t]{0.40\textwidth}
\centering
\includegraphics[width=\textwidth]
{../plots/AFM/XG-Vakuum/XG-600/W6824-1C/W6824-1C_XG_Vakuum_600_Topography_3}
\caption{W6824-1C, Bild 2}
\end{subfigure}
\caption{AFM, Vakuum, \qty{600}{\degreeCelsius}}\label{fig: AFM, Vakuum, 600}
\end{figure}

\subsubsection{\qty{700}{\degreeCelsius}}
\begin{figure}[ht]
\centering
% W6821-1C
\begin{subfigure}[t]{0.40\textwidth}
\centering
\includegraphics[width=\textwidth]
{../plots/AFM/XG-Vakuum/XG-700/W6821-1C/W6821-1C_XG_Vakuum_700_Topography_1}
\caption{W6821-1C, Bild 1}
\end{subfigure}
\begin{subfigure}[t]{0.40\textwidth}
\centering
\includegraphics[width=\textwidth]
{../plots/AFM/XG-Vakuum/XG-700/W6821-1C/W6821-1C_XG_Vakuum_700_Topography_3}
\caption{W6821-1C, Bild 2}
\end{subfigure}
% W6822-1C
\begin{subfigure}[t]{0.40\textwidth}
\centering
\includegraphics[width=\textwidth]
{../plots/AFM/XG-Vakuum/XG-700/W6822-1C/W6822-1C_XG_Vakuum_700_Topography_1}
\caption{W6822-1C, Bild 1}
\end{subfigure}
\begin{subfigure}[t]{0.40\textwidth}
\centering
\includegraphics[width=\textwidth]
{../plots/AFM/XG-Vakuum/XG-700/W6822-1C/W6822-1C_XG_Vakuum_700_Topography_3}
\caption{W6822-1C, Bild 2}
\end{subfigure}
% W6823-1C
\begin{subfigure}[t]{0.40\textwidth}
\centering
\includegraphics[width=\textwidth]
{../plots/AFM/XG-Vakuum/XG-700/W6823-1C/W6823-1C_XG_Vakuum_700_Topography_1}
\caption{W6823-1C, Bild 1}
\end{subfigure}
\begin{subfigure}[t]{0.40\textwidth}
\centering
\includegraphics[width=\textwidth]
{../plots/AFM/XG-Vakuum/XG-700/W6823-1C/W6823-1C_XG_Vakuum_700_Topography_3}
\caption{W6823-1C, Bild 2}
\end{subfigure}
% W6824-1C
\begin{subfigure}[t]{0.40\textwidth}
\centering
\includegraphics[width=\textwidth]
{../plots/AFM/XG-Vakuum/XG-700/W6824-1C/W6824-1C_XG_Vakuum_700_Topography_1}
\caption{W6824-1C, Bild 1}
\end{subfigure}
\begin{subfigure}[t]{0.40\textwidth}
\centering
\includegraphics[width=\textwidth]
{../plots/AFM/XG-Vakuum/XG-700/W6824-1C/W6824-1C_XG_Vakuum_700_Topography_3}
\caption{W6824-1C, Bild 2}
\end{subfigure}
\caption{AFM, Vakuum, \qty{700}{\degreeCelsius}}\label{fig: AFM, Vakuum, 700}
\end{figure}

\subsubsection{\qty{750}{\degreeCelsius}}
\begin{figure}[ht]
\centering
% W6821-1C
\begin{subfigure}[t]{0.40\textwidth}
\centering
\includegraphics[width=\textwidth]
{../plots/AFM/XG-Vakuum/XG-750/W6821-1C/W6821-1C_XG_Vakuum_750_Topography_1}
\caption{W6821-1C, Bild 1}
\end{subfigure}
\begin{subfigure}[t]{0.40\textwidth}
\centering
\includegraphics[width=\textwidth]
{../plots/AFM/XG-Vakuum/XG-750/W6821-1C/W6821-1C_XG_Vakuum_750_Topography_3}
\caption{W6821-1C, Bild 2}
\end{subfigure}
% W6822-1C
\begin{subfigure}[t]{0.40\textwidth}
\centering
\includegraphics[width=\textwidth]
{../plots/AFM/XG-Vakuum/XG-750/W6822-1C/W6822-1C_XG_Vakuum_750_Topography_1}
\caption{W6822-1C, Bild 1}
\end{subfigure}
\begin{subfigure}[t]{0.40\textwidth}
\centering
\includegraphics[width=\textwidth]
{../plots/AFM/XG-Vakuum/XG-750/W6822-1C/W6822-1C_XG_Vakuum_750_Topography_3}
\caption{W6822-1C, Bild 2}
\end{subfigure}
% W6824-1C
\begin{subfigure}[t]{0.40\textwidth}
\centering
\includegraphics[width=\textwidth]
{../plots/AFM/XG-Vakuum/XG-750/W6824-1C/W6824-1C_XG_Vakuum_750_Topography_1}
\caption{W6824-1C, Bild 1}
\end{subfigure}
\begin{subfigure}[t]{0.40\textwidth}
\centering
\includegraphics[width=\textwidth]
{../plots/AFM/XG-Vakuum/XG-750/W6824-1C/W6824-1C_XG_Vakuum_750_Topography_3}
\caption{W6824-1C, Bild 2}
\end{subfigure}
\caption{AFM, Vakuum, \qty{750}{\degreeCelsius}}\label{fig: AFM, Vakuum, 750}
\end{figure}


    \printbibliography
\end{document}
