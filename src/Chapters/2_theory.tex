\section{Theorie}
\subsection{Kristallgitter}
Um das Material \ce{(MgCoNiCuZn)O}, sowie die Messtechnik XRD verstehen zu können, ist ein grundlegendes Verständnis über die Struktur von kristallinen Festkörpern unerlässlich. Die nachfolgenden Seiten dienen also als Zusammenfassung für die relevanten Konzepte der Festkörperphysik. 

\subsubsection{Bravaisgitter, Elementarzelle und Basis}
Als Idealisierung vieler Festkörper wird als Modell des idealen Kristalls herangezogen. Ein idealer Kristall ist eine dreidimensionale, unendlich ausgedehnte Anordnung, die sich aus identischen, periodischen wiederkehrenden Baueinheiten zusammensetzt. Diese Baueinheiten werden als Basis bezeichnet. Sie können einzelne Atome, aber auch komplexe Atomstrukturen repräsentieren. Reduziert man jede Baueinheit auf einen einzigen Punkt, so entsteht ein einfach zu beschreibendes Punktgitter. \autocite[49]{Hunklinger} Dieses unterliegt verschiedenen Symmetrien, sodass das Gitter in unterschiedliche Kristallsysteme eingeteilt werden kann. Eine einfache Einteilung kann mithilfe von Drehachsen erfolgen. Hierbei betrachtet man diejenigen Rotationsoperatoren $R_{\hat{e}}(2\pi / n)$ für eine beliebige Achse $\hat{e}$ um einen Winkel $2 \pi /n$, die das Punktgitter auf sich selbst abbilden. Der Parameter $n$ wird als Zähligkeit bezeichnet und teilt die Punktgitter in sieben verschiedene Kristallklassen ein: \autocite[53]{Hunklinger}
\begin{table}[h]
	\centering
	\begin{tabular}{c c c c}
		\toprule
		     Kristallsystem      & Gitterkonstanten  & Winkel                                       & Zähligkeit \\ \midrule
		        triklin          & $a \neq b \neq c$ & $\alpha \neq\beta \neq\gamma$                & 1          \\
		        monoklin         & $a \neq b \neq c$ & $\alpha=\gamma=\ang{90},\beta \neq \ang{90}$ & 2          \\
		     orthorhombisch      & $a \neq b \neq c$ & $\alpha=\beta=\gamma=\ang{90}$               & 2 (zwei)   \\
		       tetragonal        & $a = b \neq c$    & $\alpha=\beta=\gamma=\ang{90}$               & 4          \\
		       hexagonal         & $a = b \neq c$    & $\alpha=\beta=\ang{90}, \gamma=\ang{120}$    & 6          \\
		trigonal (rhomboedrisch) & $a=b=c$           & $\alpha=\beta=\gamma \neq \ang{90}$          & 3          \\
		        kubisch          & $a=b=c$           & $\alpha=\beta=\gamma=\ang{90}$               & 3 (vier)   \\ \bottomrule
	\end{tabular}
	\caption{Klassifikation der verschiedenen Kristallsysteme. \imcite[65]{Hunklinger} }
\end{table}
Eine weitere wichtige Symmetrie ist die Translationssymmetrie. Betrachtet man diejenigen Translationsoperatoren 
$T(\mathbf{R})$, die das Gitter auf sich selbst abbilden, dann erkennt man aufgrund der Periodizität des Gitters 
den Zusammenhang $\mathbf{R} = n_{1}\mathbf{a}_{1}+n_{2}\mathbf{a}_{2}+n_{3}\mathbf{a}_{3}$, wobei 
$\mathbf{a}_{i}\in\mathbb{R}^{3}, n_{i}\in\mathbb{Z}$. \autocite[50]{Hunklinger} Die Vektoren $\mathbf{a}_{i}$ 
definieren ein schiefwinkliges Koordinatensystem und werden als primitive Gittervektoren bezeichnet. Sie spannen 
ein dreidimensionales Bravaisgitter auf. Die Abstände zwischen zwei benachbarten Gitterpunkten, also die Größen  
$\lvert \mathbf{a}_{i} \rvert$, werden Gitterkonstanten genannt. \autocite[82]{Ashcroft}
Mithilfe der Definition einer Basis und eines Bravaisgitters lässt sich jeder ideale Kristall  beschreiben. 
Eine Kristallstruktur wird also durch identische Kopien der Basis an jedem Punkt des Bravaisgitters aufgebaut. 
\autocite[94-95]{Ashcroft} 
Man kann Teilmengen des Ortsraumes geschickt wählen, die durch Aneinanderreihung den gesamten Raum lückenlos und 
überlappungsfrei überdecken. Solche Mengen nennt man Elementarzellen. Wählt man die Elementarzelle so, dass sie 
nur einen 
Gitterpunkt enthält, tritt der Spezialfall einer primitiven Elementarzelle ein. Mit einer primitiven Elementarzelle 
lässt sich der Raum lückenlos und überlappungsfrei überdecken, indem man die Zelle entlang der primitiven Gittervektoren 
verschiebt. Ein einfache Konstruktion liefert ein Parallelepiped, welches von den drei Basisvektoren aufgespannt wird. 
\autocite[90-91]{Ashcroft}  Da man die Symmetriebeziehungen voll ausschöpfen möchte, benutzt man meist keine primitiven Elementarzellen
, sondern wählt geschickt nicht-primitive Elementarzellen, die möglichst viele Punktsymmetrieelemente beinhalten. 
Das führt zu einer Einteilung in 14 Bravaisgitter. \autocite[51]{Hunklinger}
\subsubsection{Relevante Gittertypen}
\subsubsection{Reziprokes Gitter}
Das reziproke Gitter spielt für die weitere Betrachtung von periodischen Strukturen eine fundamentale Rolle. Ziel 
ist es, eine Funktion zu konstruieren, die gitterperiodisch im Bravaisgitter $\mathbf{a}_{i}$ ist. Für diese Funktion
soll gelten $f(\mathbf{r})=f(\mathbf{r}+\mathbf{R})$ falls $\mathbf{R}=\sum_{i=1}^{3} \alpha_{i}\mathbf{a}_{i}$. 
Mithilfe einer Reihenentwicklung erhalten wir die folgende, allgemeine Form:

\begin{align*}
	f(\mathbf{r})&=\sum_{\mathbf{G}}a_{\mathbf{G}}\cdot \exp(\mathrm{i}\mathbf{G}\cdot\mathbf{r}) \\
	f(\mathbf{R}+\mathbf{r})&=\sum_{\mathbf{G}}a_{\mathbf{G}}\cdot \exp(\mathrm{i}\mathbf{G}\cdot \mathbf{r})\cdot
	\underbrace{ \exp(\mathrm{i}\mathbf{G}\cdot \mathbf{R}) }_{ \stackrel{!}{=}1 }  \stackrel{!}{=} f(\mathbf{r})
\end{align*}

Erkennbar ist die notwendige Bedingung $\exp(\mathrm{i}\mathbf{G}\cdot \mathbf{R})=1$. Dies ist äquivalent zur 
Aussage $\mathbf{G}\cdot \mathbf{R}=2\pi n$ mit $n \in \mathbb{N}_{0}$. Mithilfe dieser Überlegung lässt sich nun 
das reziproke Gitter durch die Menge $\{ \mathbf{G} \,\vert\, \exp(\mathrm{i}\mathbf{G}\cdot \mathbf{R}) \stackrel{!}{=}1 \quad \forall \mathbf{R} \in \text{span}(\mathbf{a}_{i}) \}$ definieren. \autocite[108]{Ashcroft} Analog zum Ortsraum lassen sich auch hier primitiven Vektoren bilden mit der Vorschrift: 
\begin{align*}
	\mathbf{b}_{1} = 2\pi \cdot \frac{\mathbf{a}_{2} \times \mathbf{a}_{3}}{V_{\mathrm{EZ}}} \quad
	\mathbf{b}_{2} = 2\pi \cdot \frac{\mathbf{a}_{3} \times \mathbf{a}_{1}}{V_{\mathrm{EZ}}} \quad
	\mathbf{b}_{3} = 2\pi \cdot \frac{\mathbf{a}_{1} \times \mathbf{a}_{2}}{V_{\mathrm{EZ}}}
\end{align*}
Jeder Punkt im reziproken Gitter kann also durch $\sum_{i=1}^{3} \beta_{i}\mathbf{b}_{i}$ mit $\beta_i \in \mathbb{Z}
$ beschrieben werden. Es gelten die Relation $\mathbf{b}_{i}\cdot \mathbf{a}_{j}=2 \pi \delta_{ij}$. 
\autocite[109]{Ashcroft}

\paragraph{Indizierung von Gitterebenen und Gitterrichtungen}
Die erste wichtige Anwendung des reziproken Gitters ist die Charakterisierung von Gitterebenen. Eine Gitterebene 
ist eine beliebige, im Bravaisgitter liegende Ebene, die mindestens drei nicht kollineare Gitterpunkte enthält.
Aufgrund der Kristallsymmetrie liegen damit unendlich viele weitere Gitterpunkte innerhalb der Ebene. Mithilfe 
der Translationssymmetrie findet man parallele Gitterebenen im Abstand $d$. Diese fasst man als Gitterebenenscharen 
zusammen.

Nun kann man diese Gitterebenenscharen mithilfe des reziproken Gitters charakterisieren, denn für jede 
Gitterebenenschar im Abstand  existieren Vektoren des reziproken Gitters, welche senkrecht auf den Ebenen stehen. 
Für die eindeutige Beschreibung wählt man den kleinsten dieser Vektoren, welche die Länge $2 \pi / d$ besitzt. Auch
die Umkehrung gilt: Für jeden Vektor $\mathbf{K}$ existiert eine Schar von senkrechten Gitterebenen. 
Der Abstand $d$ dieser Ebenen ist an den Betrag den kleinsten parallelen Vektor $\mathbf{k}$ durch $\lvert \mathbf{k} \rvert=2\pi  /d$ gekoppelt. Es existiert also eine einfache Möglichkeit, Gitterebenen mithilfe von reziproken 
Gittervektoren eindeutig zu identifizieren. \autocite[113]{Ashcroft}

Nun können wir die sogenannten Millerschen Indizes verwenden, um Gitterebenenscharen eindeutig zu bestimmen. Sei 
dazu $\mathbf{k}$  der kürzeste reziproke Gittervektor, welcher senkrecht auf der zu charakterisierenden Ebene steht. Dieser lässt sich darstellen durch $ \mathbf{K} = h \mathbf{b_1} + k \mathbf{b_2} + l \mathbf{b_3}$. Das Tupel (h\, k\,l) sind die Millerschen Indizes, welche per Definition aus ganzen Zahlen 
bestehen müssen. Für die Millerschen Indizes existiert eine geometrische Interpretation, die es erlaubt, die 
Indizes im Ortsraum zu visualisieren. Für jede Gitterebene findet man ein entsprechendes $A$, sodass  die 
Ebenengleichung $\mathbf{K} \cdot \mathbf{r} = A$ erfüllt ist. Nun definieren wir die Durchstoßpunkte zwischen den 
durch die primitiven Vektoren $\mathbf{a_i}$ aufgespannten Koordinatenachsen und der Ebene durch die Zahlen 
$x_{1}\mathbf{a}_{1}, x_{2}\mathbf{a}_{2}, x_{3}\mathbf{a}_{3}$. Da die Durchstoßpunkte in der Ebene liegen, 
ist die Ebenengleichung $\mathbf{K}\cdot(x_{i}\mathbf{a}_{i})=A$ erfüllt und man findet mit $\mathbf{K}\cdot
\mathbf{a}_{1}=2\pi h$, $\mathbf{K}\cdot \mathbf{a}_{2}=2\pi k$,  $ \mathbf{K}\cdot \mathbf{a}_{3}=2\pi l$  
folgenden Zusammenhang:
\begin{equation*}
	x_{1}=\frac{A}{2\pi h}, \quad x_{2}=\frac{A}{2\pi k}, \quad x_{3} =\frac{A}{2\pi l}
\end{equation*}
Kennt man die Achsendurchstoßpunkte, kann man die Millerschen Indizes finden, indem man den Parameter $A$ 
kleinstmöglich wählt, sodass $h$, $k$ und $l$ ganzzahlig sind. \autocite[115]{Ashcroft}

Nicht nur Gitterebenen, sondern auch Gitterrichtungen lassen sich in ähnlicher Weise indizieren. Das Tupel
$[h\,k\,l]$ beschreibt diejenige Richtung, die durch den Vektor $\mathbf{R} = h\mathbf{a}_{1}+k\mathbf{a}_{2}+
l\mathbf{a}_{3}$ 
