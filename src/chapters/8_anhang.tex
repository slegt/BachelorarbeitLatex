\section*{Anhang}\label{sec:appendix}
\addcontentsline{toc}{section}{Anhang}

\newcommand{\plotpath}{../plots/AFM}
\newcommand{\samplesB}{W6823-1B/\samplethree, W6821-1B/\sampleone, W6822-1B/\sampletwo,  W6824-1B/\samplefour}
\newcommand{\samplesC}{W6823-1C/\samplethree, W6821-1C/\sampleone, W6822-1C/\sampletwo,  W6824-1C/\samplefour}
\newcommand{\samplesCC}{W6821-1C/\sampleone, W6822-1C/\sampletwo,  W6824-1C/\samplefour}
\newcommand{\samplesCCC}{W6821-1C/\sampleone, W6822-1C/\sampletwo}


\newcommand{\samplesD}{W6823-1D/\samplethree, W6821-1D/\sampleone, W6822-1D/\sampletwo,  W6824-1D/\samplefour}


\subsection*{Bildvergleich der AFM-Topografieaufnahmen}\label{subsec:bildvergleich-der-topografieaufnahmen}
\begin{figure}[ht]
    \centering
    \foreach \sample/\name in \samplesB {
        \begin{subfigure}[t]{0.40\textwidth}
            \centering
            \includegraphics[width=\textwidth]
            {\plotpath/XG-Sauerstoff/XG-pre/\sample/\sample_XG_Sauerstoff_pre_Topography_1}
            \caption{\name, Bild 1}
        \end{subfigure}
        \begin{subfigure}[t]{0.40\textwidth}
            \centering
            \includegraphics[width=\textwidth]
            {\plotpath/XG-Sauerstoff/XG-pre/\sample/\sample_XG_Sauerstoff_pre_Topography_3}
            \caption{\name, Bild 2}
        \end{subfigure}
    }
    \caption{Topografieaufnahmen des Sauerstoff-Ausheizvorgangs der Initialzustände.}
    \label{fig: AFM, Sauerstoff, pre}
\end{figure}
\foreach \temp in {600, 700, 750, 800, 875} {
    \begin{figure}[ht]
        \centering
        \foreach \sample/\name in \samplesB {
            \begin{subfigure}[t]{0.4\textwidth}
                \centering
                \includegraphics[width=\textwidth]
                {\plotpath/XG-Sauerstoff/XG-\temp/\sample/\sample_XG_Sauerstoff_\temp_Topography_1}
                \caption{\name, Bild 1}
            \end{subfigure}
            \begin{subfigure}[t]{0.4\textwidth}
                \centering
                \includegraphics[width=\textwidth]
                {\plotpath/XG-Sauerstoff/XG-\temp/\sample/\sample_XG_Sauerstoff_\temp_Topography_3}
                \caption{\name, Bild 2}
            \end{subfigure}
        }
        \caption{Topografieaufnahmen des Sauerstoff-Ausheizvorgangs bei \qty{\temp}{\degreeCelsius}.}
        \label{fig: AFM, Sauerstoff, \temp}
    \end{figure}
}
\begin{figure}[ht]
    \centering
    \foreach \sample/\name in \samplesC {
        \begin{subfigure}[t]{0.40\textwidth}
            \centering
            \includegraphics[width=\textwidth]
            {\plotpath/XG-Vakuum/XG-pre/\sample/\sample_XG_Vakuum_pre_Topography_1}
            \caption{\name, Bild 1}
        \end{subfigure}
        \begin{subfigure}[t]{0.40\textwidth}
            \centering
            \includegraphics[width=\textwidth]
            {\plotpath/XG-Vakuum/XG-pre/\sample/\sample_XG_Vakuum_pre_Topography_3}
            \caption{\name, Bild 2}
        \end{subfigure}
    }
    \caption{Topografieaufnahmen des Vakuum-Ausheizvorgangs der Initialzustände.}
    \label{fig: AFM, Vakuum, pre}
\end{figure}
\foreach \temp in {600, 700} {
    \begin{figure}[ht]
        \centering
        \foreach \sample/\name in \samplesC {
            \begin{subfigure}[t]{0.4\textwidth}
                \centering
                \includegraphics[width=\textwidth]
                {\plotpath/XG-Vakuum/XG-\temp/\sample/\sample_XG_Vakuum_\temp_Topography_1}
                \caption{\name, Bild 1}
            \end{subfigure}
            \begin{subfigure}[t]{0.4\textwidth}
                \centering
                \includegraphics[width=\textwidth]
                {\plotpath/XG-Vakuum/XG-\temp/\sample/\sample_XG_Vakuum_\temp_Topography_3}
                \caption{\name, Bild 2}
            \end{subfigure}
        }
        \caption{Topografieaufnahmen des Vakuum-Ausheizvorgangs bei \qty{\temp}{\degreeCelsius}.}
        \label{fig: AFM, Vakuum, \temp}
    \end{figure}
}
\foreach \temp in {750, 800} {
    \begin{figure}[ht]
        \centering
        \foreach \sample/\name in \samplesCC {
            \begin{subfigure}[t]{0.4\textwidth}
                \centering
                \includegraphics[width=\textwidth]
                {\plotpath/XG-Vakuum/XG-\temp/\sample/\sample_XG_Vakuum_\temp_Topography_1}
                \caption{\name, Bild 1}
            \end{subfigure}
            \begin{subfigure}[t]{0.4\textwidth}
                \centering
                \includegraphics[width=\textwidth]
                {\plotpath/XG-Vakuum/XG-\temp/\sample/\sample_XG_Vakuum_\temp_Topography_3}
                \caption{\name, Bild 2}
            \end{subfigure}
        }
        \caption{Topografieaufnahmen des Vakuum-Ausheizvorgangs bei \qty{\temp}{\degreeCelsius}.}
        \label{fig: AFM, Vakuum, \temp}
    \end{figure}
}

\foreach \temp in {875} {
    \begin{figure}[ht]
        \centering
        \foreach \sample/\name in \samplesCCC {
            \begin{subfigure}[t]{0.4\textwidth}
                \centering
                \includegraphics[width=\textwidth]
                {\plotpath/XG-Vakuum/XG-\temp/\sample/\sample_XG_Vakuum_\temp_Topography_1}
                \caption{\name, Bild 1}
            \end{subfigure}
            \begin{subfigure}[t]{0.4\textwidth}
                \centering
                \includegraphics[width=\textwidth]
                {\plotpath/XG-Vakuum/XG-\temp/\sample/\sample_XG_Vakuum_\temp_Topography_3}
                \caption{\name, Bild 2}
            \end{subfigure}
        }
        \caption{Topografieaufnahmen des Vakuum-Ausheizvorgangs bei \qty{\temp}{\degreeCelsius}.}
        \label{fig: AFM, Vakuum, \temp}
    \end{figure}
}
\begin{figure}[ht]
    \centering
    \foreach \sample/\name in \samplesD {
        \begin{subfigure}[t]{0.40\textwidth}
            \centering
            \includegraphics[width=\textwidth]
            {\plotpath/XG-Luft/XG-pre/\sample/\sample_XG_Luft_pre_Topography_1}
            \caption{\name, Bild 1}
        \end{subfigure}
        \begin{subfigure}[t]{0.40\textwidth}
            \centering
            \includegraphics[width=\textwidth]
            {\plotpath/XG-Luft/XG-pre/\sample/\sample_XG_Luft_pre_Topography_3}
            \caption{\name, Bild 2}
        \end{subfigure}
    }
    \caption{Topografieaufnahmen des Luft-Ausheizvorgangs der Initialzustände.}
    \label{fig: AFM, Luft, pre}
\end{figure}
\foreach \temp in {600, 700, 750, 800, 875} {
    \begin{figure}[ht]
        \centering
        \foreach \sample/\name in \samplesD {
            \begin{subfigure}[t]{0.4\textwidth}
                \centering
                \includegraphics[width=\textwidth]
                {\plotpath/XG-Luft/XG-\temp/\sample/\sample_XG_Luft_\temp_Topography_1}
                \caption{\name, Bild 1}
            \end{subfigure}
            \begin{subfigure}[t]{0.4\textwidth}
                \centering
                \includegraphics[width=\textwidth]
                {\plotpath/XG-Luft/XG-\temp/\sample/\sample_XG_Luft_\temp_Topography_3}
                \caption{\name, Bild 2}
            \end{subfigure}
        }
        \caption{Topografieaufnahmen des Luft-Ausheizvorgangs bei \qty{\temp}{\degreeCelsius}.}
        \label{fig: AFM, Luft, \temp}
    \end{figure}
}

\clearpage
\subsection*{EDX-Aufnahmen}\label{subsec:edx-bilder}
\newcommand{\samplesEDX}{W6821-3D/\csampleone ,W6822-3D/\csampletwo, W6824-3D/\csamplefour}

\foreach \sample/\name in \samplesEDX {
    \begin{figure}
        \centering
        \foreach \i/\desc in {map/Oberfläche, Mg/Magnesium, Co/Kobalt, Ni/Nickel, Cu/Kupfer, Zn/Zink}{
            \begin{subfigure}[t]{0.40\textwidth}
                \includegraphics[width=\textwidth]{../plots/EDX/\sample/\i}
                \caption{\desc}
            \end{subfigure}
        }
        \caption{Ortsaufgelöste EDX-Aufnahmen der Probe \name.}
        \label{fig:edx_\sample}
    \end{figure}
}

\clearpage
\subsection*{GIXRD-Aufnahmen}\label{subsec:gixrd-bilder}
\begin{figure}[h!]
    \centering
    \import{../plots/XRD/}{gixrd_W6822.pgf}
    \caption{GIXRD-Aufnahme bei einem festen Winkel $\omega=\qty{2}{\degree}$ von Probe \sampletwo\ bei
    \qty{875}{\degreeCelsius}.}
    \label{fig:GIXRD_W6821}
\end{figure}
\begin{figure}[h!]
    \centering
    \import{../plots/XRD/}{gixrd_W6823.pgf}
    \caption{GIXRD-Aufnahme bei einem festen Winkel $\omega=\qty{2}{\degree}$ von Probe \samplethree\ bei
    \qty{875}{\degreeCelsius}.}
    \label{fig:GIXRD_W6821}
\end{figure}
\begin{figure}[h!]
    \centering
    \import{../plots/XRD/}{gixrd_W6824.pgf}
    \caption{GIXRD-Aufnahme bei einemfesten Winkel $\omega=\qty{2}{\degree}$ von Probe \samplefour\ bei
    \qty{875}{\degreeCelsius}.}
    \label{fig:GIXRD_W6821}
\end{figure}