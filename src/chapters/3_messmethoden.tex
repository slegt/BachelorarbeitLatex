\section{Messmethoden}\label{sec:messmethoden}
\subsection{PLD}
%TODO
\subsection{XRD}\label{subsec:xrd}
Nachdem die Dünnfilme hergestellt wurden, ist der nächste Schritt, ihre Struktur zu charakterisieren.
Zwischen Dünnfilmen und ihren korrespondierenden Massivkörpern existieren signifikante Unterschiede.
Diese resultieren vorrangig aus dem Verhältnis von Oberfläche zu Volumen, sowie den jeweiligen
Wachstumsbedingungen, wie Temperatur, Druck und Substrat.
Sie zeigen sich beispielsweise in der Qualität der Kristallinität, sowie in Kompositionsgradienten.
Da Proben mit unterschiedlichen Wachstumsbedingungen hergestellt wurden und deren Eigenschaften dadurch
maßgeblich beeinflusst werden, ist es notwendig, die Kristallinität der Dünnfilme zu charakterisieren.
Röntgendiffraktometrie (XRD, engl. \textit{X-Ray diffraction}) ist eine weit verbreitete Methode, um die
Kristallstruktur von Dünnfilmen zu bestimmen.
Dabei wird ein Röntgendiffraktometer verwendet.

\subsubsection{Röntgendiffraktometer}
Das Röntgendiffraktometer besteht aus fünf Hauptkomponenten: Röntgenquelle und Detektor, Ein- und Ausfallsoptik,
sowie dem Goniometer.
Zusätzlich ist das Diffraktometer durch eine Strahlungsschutzverkleidung abgeschirmt und mit einer Steuerungssoftware
verbunden.
Im Folgenden werden die einzelnen Komponenten näher erläutert.

\paragraph{Röntgenquelle}
Die Röntgenstrahlen werden in einer Röntgenröhre erzeugt.
In dieser werden Elektronen aus einer Wolfram-Glühkathode emittiert und durch das elektrische Feld auf eine Anode
beschleunigt.
Die Anode besteht meist aus hochreinem Kupfer.
Stromstärke und Beschleunigungsspannung der Röntgenröhre müssen so gewählt werden, dass die Energie beim Auftreffen der
Elektronen auf die Anode ausreicht, um die gebundenen Elektronen der Atome auf das nächsthöhere Energieniveau anzuregen.
Aufgrund der daraus resultierende Wärme muss die Anode ständig wassergekühlt werden.
Im hauseigenen Röntgendiffraktometer wird eine Beschleunigungsspannung von \qty{40}{\kilo\volt} und eine Stromstärke
von \qty{40}{\milli\ampere} verwendet.

Nach der Kollision zwischen Kupferatom und Elektron relaxiert das Elektron unter Bildung eines Röntgenphotons.
Man erhält ein Spektrum, welches durch die charakteristische Strahlung der Anode sowie durch Bremsstrahlung
geprägt ist.
Die charakteristische Strahlung wird vorrangig durch die K-Linien, insbesondere $K_{\alpha_1}$, $K_{\alpha_2}$
und K$_{\beta}$, dominiert.
Da $K_{\alpha_1}$ und $K_{\alpha_2}$ energetisch sehr nahe beieinander liegen, können sie nicht immer einzeln
aufgelöst werden.
Die $K_{\beta}$ Strahlung ist größtenteils unerwünscht und kann durch geeignete Filter unterdrückt werden.


Die Wolfram-Glühkathode emittiert unerwüschterweise nicht nur Elektronen, sondern auch Wolfram-Atome in kleinen Mengen.
Über längere Zeiträume führt dies zu einer nicht mehr zu vernachlässigenden Kontamination der Anode.
Dadurch können bei Elektronenstößen auch Wolfram-Atome angeregt werden, was zu einer zusätzlichen Wellenlänge im
Spektrum führt
In den späteren Messergebnissen sind diese Beiträge erkennbar.
Abschließend gelangen die Röntgenstrahlen durch ein Berylliumfenster in die Einfallsoptik.

\paragraph{Röntgendetektor}
Die durch die Röntgenquelle erzeugten Strahlen gelangen nach Reflektion an der Probe in den Detektor.
Dieser dient dazu, den reflektierten Strahl in ein elektrisches Signal umzuwandeln.
Kategorisieren kann man Röntgendektektoren nach ihrer Funktionsweise.
Eine weitere Unterteilung erfolgt nach der Dimensionalität des Detektors.
Es können Punktdetektoren (0D), Linien- (1D) oder Flächendetektoren (2D) verwendet werden.
Im hauseigenen Röntgendiffraktometer ist ein Halbleiterdetektor verbaut, der in verschiedenen Dimensionalitäten
arbeiten kann.
Wichtig ist, dass die maximale Zählrate des Detektors nicht überschritten wird.
Das führt zu nichtlinearen Antworten und kann den Sensor beschädigen.
Um das zu vermeiden, können Filter und Attenuatoren verwendet werden.

\paragraph{Goniometer}
Das Goniometer ist die mechanische Komponente des Röntgendiffraktometers.
Es besteht aus mehreren Drehachsen, die es ermöglichen, die Probe in unterschiedlichsten Winkeln auszurichten.
Nach der Braggschen Beugungstheorie ergeben sich konstruktive Interferenzen an denjenigen Winkeln, die der
Bragg-Bedingung genügen.
Existieren Möglichkeit, die Winkel für Quelle und Detektor zu variieren, kann diese Interferenz beobachtet werden.
Im Allgemeinen ist die Röntgenquelle jedoch fest, eine äquivalente Drehung von Probe und Detektor ist deshalb gängig.
In der einfachsten Betrachtungsweise muss das Goniometer also den Winkel zwischen Probe und Quelle ($\omega$) und dem
Winkel zwischen Probe und Detektor ($2\theta$) einstellen können.
Diese Freiheit reicht zwar für Pulverproben, jedoch nicht für Dünnfilme.
Zwar kann man mit beiden Freiheitsgraden Messungen durchführen, welche die out-of-plane Orientierung charakterisieren,
jedoch ist es nicht möglich, die in-plane Orientierung zu bestimmen.
Dafür werden weitere Achsen, wie $\varphi$ und $\chi$, benötigt.
Eine Konstruktion mit den vier Achsen wird Euler-Wiege genannt.
%%TODO



\subsection{AFM}
Das Raster-Kraft-Mikroskop ist ein hochpräzises Messinstrument zum Erfassen von Oberflächenstrukturen.
Anders als bei Licht- oder Elektronenmikroskopie wird hierbei eine mechanische Funktionsweise umgesetzt.
Dabei fährt eine Messapparatur, der Cantilever, rasterweise über eine Oberfläche und tastet diese ab.
Die auf den Cantilever wirkenden atomaren oder magnetischen Kräfte werden gemessen, woraus eine Topographiekarte der
Oberfläche erstellt wird.

\subsubsection{Schematischer Aufbau und Funktionsweise}
Die grundlegende Funktionsweise ist in Abbildung 1 dargestellt.
Markierung 1 zeigt den Cantilever, der mit einer Messspitze mit Dimensionen im Nanometerbereich ausgestattet ist.
Fährt diese über die Probe, siehe Markierung 2, so wirken Kräfte auf die Spitze, welche den Cantilever auslenken.
Diese Auslenkung wird mittels eines Ablenkungserkennungssystems, Markierung 3, ausgewertet.
Hierbei wird ein Laserstrahl an der Rückseite des Cantilevers reflektiert, welcher anschließend auf einen Photodetektor
trifft.
Dieser Detektor kann nun anhand der Intensitätsverteilung auf den einzelnen Sektoren die Auslenkung und Torsion des
Cantilevers messen.
Die gemessene Auslenkung wird an das Feedback System übergeben, was Markierung 4 zeigt.
Basierend auf dem gewählten Betriebsmodus wird versucht, die gemessene Kraft oder Amplitude konstant zu halten.
Mithilfe dieser Regulation wird ein Korrektursignal ausgegeben, welches die Position des Cantilevers anpasst.
Dies geschieht mithilfe von Piezoelementen, wodurch der Cantilever in x, y oder z-Richtung bewegt werden kann.
Die z-Position des Cantilevers wird aufgezeichnet und als Topographiesignal am Computer ausgewertet, siehe Markierung 5.

\subsubsection{PID-System}
