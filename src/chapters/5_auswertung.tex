\section{Auswertung und Diskussion}\label{sec:auswertung}
Für die in dieser Arbeit durchgeführte Ausheizstudie wurden vier Proben bei Raumtemperatur mittels gepulster
Laserabscheidung bei verschiedenen Drücken unter Sauerstoffatmosphäre hergestellt.
Dafür wurden vier PLD-Prozesse initiiert, bei denen die Proben \samplethree\ bei einem Druck von \qty{0.1}{\milli\bar},
\sampleone\ bei \qty{0.01}{\milli\bar}, \sampletwo\ bei \qty{0.001}{\milli\bar} und \samplefour\ bei
\qty{0.00005}{\milli\bar} hergestellt wurden.
Für alle Prozesse wurde ein $\qty{10}{\milli\meter} \times \qty{10}{\milli\meter}$ Corning Eagle XG Glassubstrat und
das in \cref{subsec:pld} beschriebene Target verwendet.
In allen Prozessen wurden jeweils \num{40000} Laserpulse mit einer Frequenz von \qty{20}{\hertz} und einer Energie von
\qty{650}{\milli\joule} abgegeben.
Nach der Herstellung wurden die Proben im Profilometer und im Röntgendiffraktometer bezüglich ihrer Schichtdicke
vermessen.
Der jeweilige Druck und die Schichtdicken sind in \cref{tab:samples} aufgeführt.
\begin{table}[h]
    \centering
    \begin{tabular}{l l l l l}
        \toprule
        Probenname & \makecell[l]{Abscheidedruck \\ in \unit{\milli \bar}} & \makecell[l]{Dicke in \unit{\nano\meter} \\
        Profilometermessung} & \makecell[l]{Dicke in \unit{\nano\meter}     \\ XRR Messung}   \\
        \midrule
        \samplethree   & \num{0.1}   & \num{65(4)} & \num{109} \\
        \sampleone  & \num{0.01} & \num{95(7)} & - \\
        \sampletwo  & \num{0.001} & \num{160(12)} & \num{142} \\
        \samplefour  & \num{0.00005} & \num{135(16)} & \num{120} \\
        \bottomrule
    \end{tabular}
    \caption{Abscheidedruck und Schichtdicken der zu untersuchenden Proben.}
    \label{tab:samples}
\end{table}
Die Schichtdicke von \sampleone\ konnte nicht mittels XRR bestimmt werden, da das Oszillationsmuster, welches aufgrund
der Interferenz der Röntgenstrahlen an der Oberfläche und der Grenzfläche entsteht, nicht zu erkennen war.
Die Schichtdickenbestimmung beider Methoden weisen nur begrenzte Übereinstimmung auf.
Vor allem die Dicken von \samplethree\ weichen stark voneinander ab.
Grund dafür liegt in der Messmethode des Profilometers.
Dessen Spitze startet von der Ecke einer Probe und fährt in Richtung Probenmittelpunkt, bis keine
Steigung mehr gemessen wird.
Da die Schichtdicke nur durch das Messen an den vier Ecken bestimmt wurde, werden lokale Änderungen der Schichtdicke
innerhalb der Probe nicht erfasst.
Die XRR-Messung hingegen misst die Schichtdicke über die gesamte Probe hinweg und ist daher genauer.

In jedem Prozess ist neben dem Eagle XG Glassubstrat ein c-Saphir Substrat eingebaut worden.
Damit konnten vier weitere Proben \csamplethree, \csampleone, \csampletwo, \csamplefour\ hergestellt werden, welche
jedoch ausschließlich für die Kompositionsanalyse in \cref{subsec:edx-analyse} Verwendung finden.
Der untere Index der Bezeichnung steht erneut für den Abscheidedruck in \unit{\milli \bar}.

Im Anschluss an die Probenherstellung wurden \samplethree, \sampleone, \sampletwo, \samplefour\ in
$\qty{5}{\milli\meter}\times \qty{5}{\milli\meter}$ gevierteilt, damit mehrere Ausheizstudien mit unterschiedlichen
Atmosphären vergleichbar durchgeführt werden konnten.
Dazu wurde der Dünnfilm der jeweiligen Probe mit Photolack beschichtet, um die Oberfläche vor
mechanischen Beschädigungen beim Zersägen zu schützen.
Das Substrat wurde mit der Dünnfilmseite nach oben auf eine Glasplatte mit Wachs geklebt und mit einer
Diamantsäge zerteilt.
Anschließend wurden Wachs und Photolack mit den entsprechenden Lösungsmitteln entfernt.
Es ist zu beachten, dass diese Prozedur durch die Lösungsmittel die Zusammensetzung und Oberfläche des Dünnfilms
verändern kann.
Optisch wurden keine Veränderungen festgestellt.

Nach der Präparation wurden die Eagle XG Proben auf drei verschiedene Arten ausgeheizt:
In der im \cref{subsec:ausheiz} vorgestellten Vakuumkammer unter Sauerstoffatmosphäre, in der gleichen Vakuumkammer
unter Vakuumbedingungen und im Muffelofen aus \cref{subsec:ausheiz} unter Luftatmosphäre.
Die Vakuumkammer wurde für die Ausheizvorgänge einer Temperaturkalibrierung unterzogen, die in
\cref{subsec:temperaturkalibrierung} erläutert wird.

Um einen Ausgangspunkt für die Temperatur festzulegen, wurde sich an den Erkenntnissen von \citeauthoryear{Rost2015}
orientiert.
In diesem Paper wurden, wie in \cref{subsubsec:heo} beschrieben, Pellets zunächst bei \qty{750}{\degreeCelsius}
ausgeheizt.
Bei dieser Temperatur wurden mehrere Phasen beobachtet, ausgeprägte Peaks der Natriumchloridstruktur waren jedoch
bereits erkennbar \autocite{Rost2015}.
Da sich Dünnfilme nicht wie massive Proben verhalten, wurde der Startwert der Temperatur auf \qty{600}{\degreeCelsius}
festgelegt, um sicherzustellen, dass die Phasenübergangstemperatur nicht zu Beginn bereits überschritten wird.
Um dies zu überprüfen, wurde eine der vier geviertelten Proben von \sampletwo\, im Röntgendiffraktometer vermessen,
in Luft auf \qty{600}{\degreeCelsius} für eine Stunde ausgeheizt und anschließend erneut im Röntgendiffraktometer vermessen.
Da der Dünnfilm vor und nach dem Ausheizen röntgenamorph ist, wurde diese Probe anschließend für drei Stunden und sonst
identischen Bedingungen bei \qty{600}{\degreeCelsius} ausgeheizt und erneut im Röntgendiffraktometer vermessen.
Auch hier waren keine Beugungsreflexe des Dünnfilms zu erkennen, weshalb die Temperatur als Ausgangspunkt für den
Sauerstoff-Aus\-heiz\-vor\-gang festgelegt wurde.

Während des Aus\-heiz\-vor\-gangs in Sauerstoff wurde ein nichtsignifikanter Peak bei ${2\theta}{\approx}\qty{37}{\degree}$
ab einer Temperatur von \qty{600}{\degreeCelsius} beobachtet, welcher ein Hinweis auf eine beginnende Phasenbildung
sein könnte.
Aus diesem Grund wurde beim Vakuum-Aus\-heiz\-vor\-gang die Starttemperatur auf \qty{500}{\degreeCelsius} festgelegt.
Da keine signifikanten Beugungsreflexe in dieser Serie zu erkennen waren, wurde die Starttemperatur erneut
auf \qty{600}{\degreeCelsius} für die Ausheizung im Muffelofen gesetzt.

Da die Dünnfilme auf Eagle XG Glassubstraten gewachsen sind, muss die Viskosität des Glases in Abhängigkeit der
Temperatur beachtet werden.
Der obere Kühlpunkt des Glases liegt bei \qty{722}{\degreeCelsius}.
Bis zu diesem Punkt werden zwar innere mechanische Spannungen schnell abgebaut, Formänderungen sind jedoch nicht zu
erwarten.
Der Erweichungspunkt des Glases liegt bei \qty{971}{\degreeCelsius}.
Ab dieser Temperatur beginnt das Glas merklich zu fließen und sich unter Einfluss des Eigengewichts zu verformen.
Da die Dünnfilme im Substrathalter der Vakuumkammer unter mechanischer Spannung stehen, ist bereits bei
\qty{875}{\degreeCelsius} eine Deformierung erkennbar, sodass diese Temperatur als obere Grenze für die Ausheizstudien
in der Vakuumkammer festgelegt wurde.
Diese Temperatur ist außerdem die von \citeauthoryear{Rost2015} verwendete Phasenübergangstemperatur für
massive äquimolare Proben \autocite{Rost2015}.

Für den Aus\-heiz\-vor\-gang im Muffelofen wurde die Temperatur zuerst auf \qty{950}{\degreeCelsius} festgelegt, da diese noch
unterhalb des Erweichungspunkts des Glases liegt.
Da sich die Glassubstrate jedoch bei dieser Temperatur bereits deformierten, sodass keine vergleichbaren Messungen
möglich waren, wurde die Temperatur auf \qty{875}{\degreeCelsius} reduziert.

In \cref{subsec:glas} wurden mit reinen Eagle-XG Glassubstraten ein Aus\-heiz\-vor\-gang in Luftatmosphäre im Muffelofen
und in der Vakuumkammer durchgeführt, um potenzielle Kontaminationen der Oberfläche durch Partikel oder Rückstände
aus den Anlagen zu untersuchen.
Dafür wurde ein reines Eagle XG Glassubstrat identisch zu den Dünnfilmen gevierteilt und in den in \cref{subsec:ausheiz}
beschriebenen Anlagen bei den Temperaturen ausgeheizt.
Nach jedem Ausheizschritt wurden $2 \theta/\omega$-Diffraktogramme und Topographien der Oberflächen aufgenommen.

Für den Sauerstoff-Aus\-heiz\-vor\-gang in wurde die Vakuumkammer mit einer Sauerstoffatmosphäre
von \qty{800}{\milli\bar} verwendet.
Die Proben \samplethree, \sampleone, \sampletwo, \samplefour\ wurden einem Prozess aus wiederholtem Ausheizen und
anschließendem Messen unterzogen.
Als Ausheiztemperaturen sind \qtylist{600;700;750;800;875}{\degreeCelsius} gewählt worden.
Dabei dienen \qty{750}{\degreeCelsius} und \qty{875}{\degreeCelsius} als Referenztemperaturen zur Forschung
von \citeauthoryear{Rost2015} \autocite{Rost2015}.
Die Proben wurden für eine Stunde bei der jeweiligen Temperatur ausgeheizt, vom Substrathalter entfernt und auf eine
Glasplatte gelegt, um ein schnelles Abkühlen zu erreichen.
Während des Ausheizens rotierte der Substrathalter mit einer Winkelgeschwindigkeit von
\qty{1080}{\degree\per\minute}, um eine
gleichmäßige Temperaturverteilung zu gewährleisten.

Für den Aus\-heiz\-vor\-gang unter Vakuumbedingungen wurde die gleiche Vakuumkammer wie für den Sauerstoff-Aus\-heiz\-vor\-gang
verwendet.
Auch hier wurden die Proben \samplethree, \sampleone, \sampletwo, \samplefour\ einem Prozess aus wiederholtem
Ausheizen und anschließendem Messen unterzogen.
Die Kammer wird auf einen Druck von \qty{4e-4}{\milli\bar} evakuiert, um Vakuumbedingungen zu erreichen.
Dies ist die geringste Druckstufe, die der PID-Regler der Vakuumkammer zuverlässig erreicht.
Alle anderen Parameter sind identisch zum Sauerstoff-Aus\-heiz\-vor\-gang.

Für den Aus\-heiz\-vor\-gang in Luftatmosphäre wurde der Muffelofen verwendet.
Die Proben \samplethree, \sampleone, \sampletwo, \samplefour\ wurden einem Zyklus von Ausheizen und anschließendem
Messen unterzogen.
Für das Ausheizen wurden die Temperaturen \qtylist{600;700;750;800;875;950}{\degreeCelsius} gewählt.
Die Proben wurden in einer Keramikschale für eine Stunde bei der jeweiligen Temperatur ausgeheizt, anschließend
aus dem Ofen und der Schale genommen und auf einer Glasplatte abgekühlt.
Anders als bei den vorherigen Ausheizvorgängen wurde die Probe mit einer langsamen Ausheizrate von
\qty{300}{\kelvin\per\hour} auf die Endtemperatur gebracht.

\subsection{Temperaturkalibrierung der Vakuumkammer}\label{subsec:temperaturkalibrierung}
\begin{figure}
    \centering
    \foreach \i/\desc in {
        furnace_calibration_1.pgf/{Kalibrierung des Lithografiesensors, Bild 1},
        furnace_calibration_2.pgf/{Kalibrierung des Lithografiesensors, Bild 2},
        a_chamber_calibration.pgf/{Kalibrierung der Vakuumkammer},
        final_calibration.pgf/{Abhängigkeit zwischen $T_{\mathrm{Pyro}}$ und $T_{\mathrm{Lit}}$},
        quenching_time.pgf/{Abkühlzeit der Vakuumkammer}
    }{
        \begin{subfigure}[t]{0.49\textwidth}
            \import{../plots/calibration}{\i}
            \caption{\desc}
            \label{fig:\i}
        \end{subfigure}
    }
    \caption{Abhängigkeiten zur Temperaturbestimmung der Vakuumkammer.}
    \label{fig:temperature_calibration_1}
\end{figure}
Zentrales Thema der vorliegenden Arbeit ist das Ausheizen und anschließende Charakterisieren der \heo\ Dünnfilme.
Dafür wird unter anderem die Vakuumkammer aus \cref{subsec:ausheiz} verwendet.
In dieser ist ein Heizlaser verbaut, welcher auf die Rückseite des Substrathalters
gerichtet ist.
Durch ein Pyrometer wird die Temperatur auf der Rückseite des Substrathalters gemessen.
Die relevante Temperatur ist jedoch die des Dünnfilms auf der Vorderseite des Substrathalters, welche durch die
Atmosphäre, die Geometrie und Wärmeleitfähigkeit des Substrathalters und des Substrats beeinflusst wird.

Um diese abzuschätzen, wurde von Tim Düvel ein Temperatursensor hergestellt.
Dieser besteht aus einem c-Saphir Substrat, auf dem eine Maske für die Platinbahnen mithilfe eines
Lithografieprozesses aufgebracht wurde.
Das Substrat wurde zunächst für \qty{10}{\second} mit Platinoxid und anschließend für \qty{120}{\second} mit Platin
im Sputterverfahren beschichtet.
Nach dem Ablösen der Maske wurde das Substrat durch das gleiche Verfahren an beiden Enden der Platinbahnen mit Gold beschichtet.
Zum Schutz der Platinbahnen vor mechanischer Beschädigung wurde eine Schicht Aluminiumoxid aufgebracht.
Dieser Sensor wird im Folgenden als Lithografiesensor bezeichnet.

In Zusammenarbeit mit Tim Düvel wurde eine Kalibrierung des Lithografiesensors durchgeführt.
Dazu wurde ein PT1000-Temperatursensor genutzt, für welchen eine bekannte quadratische Abhängigkeit zwischen Widerstand
$R_{\mathrm{Pt}}$ und Temperatur $T_{\mathrm{Pt}}$ existiert \autocite{din_pt}:
\begin{equation}
    R_{\mathrm{Pt}}(T_{\mathrm{Pt}})
    =R_0 \cdot (1 + A \cdot T_{\mathrm{Pt}} + B \cdot T_{\mathrm{Pt}}^2).
    \label{eq:pt1000_calibration}
\end{equation}
Hierbei ist $R_0 = \qty{1000}{\ohm}$ der Widerstand bei \qty{0}{\degreeCelsius},
$A = \qty{3.9083e-3}{\degreeCelsius^{-1}}$ und $B = \qty{-5.775e-7}{\degreeCelsius^{-2}}$.
Der PT1000-Temperatursensor wurde mithilfe von Wärmeleitpaste thermisch an den Lithografiesensor gekoppelt und
in einem Muffelofen ausgeheizt.
Der Widerstand des Lithografiesensors $R_\mathrm{Lit}$ und der des PT1000-Temperatursensors $R_\mathrm{Pt}$
wurden in Abhängigkeit der Zeit gemessen und sind in \cref{fig:furnace_calibration_1.pgf} dargestellt.
Aus der Parametrisierung $t \to (R_\mathrm{Lit}(t), R_\mathrm{Pt}(t))$ kann die Abhängigkeit beider Widerstände
voneinander bestimmt werden.
Aufgrund der gleichen Materialbeschaffenheit ist die Abhängigkeit $R_\mathrm{Lit}(R_\mathrm{Pt})$ linear und kann durch
einen linearen Fit der Form $f(x)=mx+n$ beschrieben werden, siehe \cref{fig:furnace_calibration_2.pgf}.
Für die Fitparameter ergibt sich:
\begin{equation*}
    m = \num{0.501(0.0002)} \quad n = \qty{-0.176(0.001)}{\kilo\ohm}.
\end{equation*}

Im nächsten Schritt wurde der Lithografiesensor in den Probenhalter der Vakuumkammer eingebaut und eine
zweite Messung durchgeführt.
Dabei wird die Temperatur des Heizlasers eingestellt und nach einer festen Zeitspanne die Pyrometertemperatur
$T_\mathrm{Pyro}$ und der Lithografiewiderstand $R_\mathrm{Lit}$ gemessen.
Es wurden zwei separate Messreihen durchgeführt: Eine bei steigender Temperatur und eine bei abfallender Temperatur.
Auch hier zeigt sich eine lineare Abhängigkeit zwischen beiden Größen, dessen Parameter durch einen linearen Fit
ermittelt werden können, siehe \cref{fig:a_chamber_calibration.pgf}.
Für die Fitparameter ergibt sich:
\begin{equation*}
    m = \qty{0.0067(0.0002)}{\kilo\ohm\per\degreeCelsius} \quad n = \qty{2.6(0.1)}{\kilo\ohm}.
\end{equation*}
Als Referenztemperatur, und damit als Temperatur des Dünnfilms, wird die Temperatur des PT1000-Sensors angesehen.
Diese kann durch folgende Gesamtfunktion bestimmt werden:
\begin{equation}
    T_{\mathrm{Pt}}=\underbrace{ T_{\mathrm{Pt}}(R_{\mathrm{Pt}}) }_{
        \substack{\text{quadratische} \\ \text{Abhängigkeit}}}
    =T_{\mathrm{Pt}}(\underbrace{ R_{\mathrm{Pt}}(R_{\mathrm{Lit}}) }_{
        \substack{\text{lineare} \\ \text{Abhängigkeit}}  })
    =T_{\mathrm{Pt}}(R_{\mathrm{Pt}}(\underbrace{ R_{\mathrm{Lit}}(T_{\mathrm{Pyro}}) }_{
        \substack{\text{lineare} \\ \text{Abhängigkeit}}  })).
    \label{eq:temperature_calibration}
\end{equation}
Da alle Funktionen bekannt sind, zeigt sich folgender Zusammenhang zwischen $T_{\mathrm{Pyro}}$ und $T_{\mathrm{Pt}}$
in \cref{fig:final_calibration.pgf}.
Der Graph zeigt, dass die Temperatur des Dünnfilms mit einer Unsicherheit von circa \qty{15}{\degreeCelsius}
der Temperatur des Pyrometers entspricht.

Wichtig für das Ausheizen ist außerdem die Abkühlzeit der Vakuumkammer.
Um diese zu untersuchen, wurde der Lithografiesensor eingebaut und der Heizlaser auf \qty{350}{\degreeCelsius}
eingestellt.
Nachdem sich ein konstanter Lithografiewiderstand eingestellt hatte, wurde der Heizlaser abgeschaltet und der
Widerstand in Abhängigkeit der Zeit gemessen, siehe \cref{fig:quenching_time.pgf}.
Für einen Temperaturabfall von circa \qty{350}{\degreeCelsius} auf circa \qty{25}{\degreeCelsius}
wird in der Vakuumkammer eine Zeit von etwa \qty{4}{\minute} benötigt.
Damit zeigt der mögliche Temperaturbereich, die Präzision des Pyrometers und die Abkühlzeit, dass
die Vakuumkammer für die Ausheizstudien geeignet ist.

Es ist zu beachten, dass dieser Sensor bestmöglich die Thermodynamik von c-Saphir Substraten erfasst.
Die in dieser Arbeit betrachteten Dünnfilme wurden auf Eagle XG Glassubstraten abgeschieden.
Eagle XG hat bei Raumtemperatur eine Wärmeleitfähigkeit von \qty{1.09}{\watt\per\meter\per\kelvin} \autocite{eaglexg},
wohingegen c-Saphir eine Wärmeleitfähigkeit von \qty{42}{\watt\per\meter\per\kelvin} \autocite{saphir} hat.
Auch bei einer Temperatur von \qty{300}{\degreeCelsius} hat Eagle XG eine Wärmeleitfähigkeit von
\qty{1.34}{\watt\per\meter\per\kelvin}, wohingegen c-Saphir eine Wärmeleitfähigkeit von
\qty{20}{\watt\per\meter\per\kelvin} hat.
Da die Wärmeleitfähigkeit von Eagle XG Glassubstraten um mehr als eine Größenordnung geringer ist als die von
c-Saphir Substraten, ist die Abschätzung der Temperatur des Dünnfilms durch den Lithografiesensor nur als
Approximation zu betrachten.

\subsection{Konzentrationsanalyse}\label{subsec:edx-analyse}
Die Abscheidung mithilfe von PLD ist ein hochkomplexer Prozess, welcher nicht im thermodynamischen Gleichgewicht
stattfindet und damit schwer analytisch zu beschreiben ist.
Durch die komplexen Wechselwirkungen der Laserpulse mit den sechs verschiedenen Konstituenten des Targets
ist es schwer, Vorhersagen über den stöchiometrischen Transfer zu treffen.
Die Komposition des Dünnfilms muss nicht der des Targets entsprechen.
Zu diesem Anlass wurden von Jorrit Bredow ortsaufgelöste Konzentrationsaufnahmen der Proben \csamplethree, \csampleone,
\csampletwo, \csamplefour\ mithilfe von SEM-EDX durchgeführt.

Das Dünnfilmwachstum ist nicht nur abhängig von den Prozessparametern, sondern auch vom gewählten Substrat.
Idealerweise sollte die Konzentrationsanalyse demnach auf demselben Substrat durchgeführt werden, welches auch im
Ausheizprozess verwendet wurde.
Problematisch ist, dass die genaue Zusammensetzung des Eagle XG Glassubstrats nicht bekannt ist.
Es ist jedoch bekannt, dass Magnesium ein Konstituent des Substrats ist.
Da auch der Dünnfilm Magnesium enthält, ist es dadurch nicht möglich, das Magnesium des Dünnfilms vom Magnesium des
Substrats zu unterscheiden.
Daher wurde ein c-Saphir Substrat verwendet, welches keine Metallkationen des Dünnfilms enthält.
Da die c-Saphir Proben aus demselben Prozess entstammen und Desorptionsprozesse bei Raumtemperatur eine zu
vernachlässigende Rolle einnehmen, kann davon ausgegangen werden, dass die Konzentrationen der Elemente in den
c-Saphir Proben denen der Eagle XG Proben entsprechen.

Da SEM-EDX Messungen nur auf leitenden Proben durchgeführt werden können und die Dünnfilme isolierende Eigenschaften
aufweisen, wurden die c-Saphir Proben vor der Messung mit einer dünnen Schicht Kohlenstoff beschichtet.
\cref{fig:edx_map} zeigt die Oberfläche der Probe \csamplethree\ unter der Kohlenstoffschicht.
Erkennbar ist eine glatte Morphologie ohne sichtbare Verunreinigungen.
Die \cref{fig:edx_Mg,fig:edx_Co,fig:edx_Ni,fig:edx_Cu,fig:edx_Zn} zeigen die Konzentrationen der
Elemente Magnesium, Cobalt, Nickel, Kupfer und Zink in Abhängigkeit der Position.
Über den gesamten Dünnfilm ist eine gleichmäßige Verteilung aller Elemente erkennbar.
Es breiten sich keine erkennbaren Cluster aus, welche auf eine Phasentrennung hindeuten würden.
Der Dünnfilm wurde erfolgreich und homogen abgeschieden.
Hierbei ist es wichtig zu betonen, dass die Konzentrationsanalyse in einer Skala von $\qty{25.6}{\micro\meter} \times
\qty{25.6}{\micro\meter}$ durchgeführt wurde.
Da keine atomare Auflösung erreicht wurde, können Cluster auf kleineren Skalen nicht ausgeschlossen werden.

Die ortsabhängige Konzentrationsanalyse von \csamplethree\ wurde repräsentativ für alle Proben gezeigt.
Die Probe \csampleone, \csampletwo, \csamplefour\ zeigen sehr ähnliche Ergebnisse, die im Anhang zu finden sind.
Die Konzentrationsverhältnisse der c-Saphir Proben lassen sich ebenfalls mithilfe dieser Aufnahmen bestimmen.
Die Ergebnisse sind in \cref{tab:concentration} aufgeführt.
\begin{table}[h]
    \centering
    \begin{tabular}{l l l l l l}
        \toprule
        Probe & \ce{Mg} in \unit{\percent} & \ce{Co} in \unit{\percent} & \ce{Ni} in \unit{\percent}&
        \ce{Cu} in \unit{\percent}& \ce{Zn} in \unit{\percent}\\
        \midrule
        \csamplethree & \num{17.62} & \num{19.08} & \num{20.06} & \num{22.52} & \num{20.72} \\
        \csampleone   & \num{16.50} & \num{19.01} & \num{20.04} & \num{23.29} & \num{21.16} \\
        \csampletwo   & \num{16.13} & \num{19.56} & \num{20.79} & \num{22.66} & \num{20.86} \\
        \csamplefour  & \num{15.03} & \num{20.93} & \num{21.32} & \num{21.61} & \num{21.12} \\
        \bottomrule
    \end{tabular}
    \caption{Konzentrationen der Elemente \ce{Mg}, \ce{Co}, \ce{Ni}, \ce{Cu} und \ce{Zn} in den Proben \csamplethree,
        \csampleone, \csampletwo, \csamplefour.}
    \label{tab:concentration}
\end{table}

Erkennbar ist eine leichte Abweichung der Konzentrationen der Elemente in den Proben.
Vor allem die Konzentration von Magnesium ist geringer als die der anderen Elemente und nimmt mit fallendem Druck
ab.

Aus der abweichenden Komposition der Dünnfilme ergibt sich eine niedrigere Mischungsentropie, welche für die
Phasenstabilität von \heo\ entscheidend ist.
Um die Phasenübergangstemperatur in die reine Natriumchloridstruktur besser abschätzen zu können, kann die
Mischungsentropie mit den gemessenen Konzentrationen $\{ x_i \}$ aus \cref{tab:concentration} berechnet, und mit
derjenigen Mischungsentropie gleichgesetzt werden, die sich bei vier äquimolaren Konstituenten und einem
Konstituent mit variabler Konzentration $x$ ergibt.
Aus \cref{eq:Mischungsentropie} und \cref{eq:Mischungsentropie2} folgt:
\begin{equation}
    -\mathrm{R}\sum_{i=1}^{5}x_{i}\ln(x_{i}) \stackrel{!}{=}-\mathrm{R}\left( x\log(x)+(1-x)\ln
    \left( \frac{1-x}{4} \right) \right).
    \label{eq:fazit}
\end{equation}
Löst man diese Gleichung numerisch nach $x$, findet man für die Proben \csamplethree, \csampleone, \csampletwo\ und \csamplefour,
folgende Konzentrationen:
\begin{equation*}
    x(\mathrm{P}_{\num{0.1}}^{\mathrm{c}}) = \qty{16.8}{\percent}, \quad x(\mathrm{P}_{\num{0.01}}^{\mathrm{c}})
    = \qty{15.6}{\percent}, \quad
    x(\mathrm{P}_{\num{0.001}}^{\mathrm{c}}) = \qty{15.7}{\percent}, \quad x(\mathrm{P}_{\num{0.00005}}^{\mathrm{c}})
    = \qty{15.0}{\percent}.
\end{equation*}
Die von \citeauthoryear{Rost2015} angegebenen Phasenübergangstemperaturen in eine reine Natriumchloridstruktur
bei vier äquimolaren Konstituenten und einer variablen Konzentration von $x \simeq \qty{15}{\percent}$ liegen, je nach
Material, zwischen \qty{925}{\degreeCelsius} und \qty{1000}{\degreeCelsius}.
Auch wenn in dieser Arbeit Dünnfilme und keine Massivproben untersucht wurden, kann die Phasenübergangstemperatur
als erste Abschätzung für die Wahl des Temperaturbereichs herangezogen werden und liegt deutlich oberhalb der
gewählten Temperaturbereiche.

\begin{figure}
    \centering
    \foreach \i/\desc in {map/Oberfläche, Mg/Magnesium, Co/Kobalt, Ni/Nickel, Cu/Kupfer, Zn/Zink}{
        \begin{subfigure}[t]{0.40\textwidth}
            \includegraphics[width=\textwidth]{../plots/EDX/W6823-3D/\i}
            \caption{\desc}
            \label{fig:edx_\i}
        \end{subfigure}
    }
    \caption{Ortsaufgelöste EDX Aufnahmen der Probe \csamplethree.}
    \label{fig:edx1}
\end{figure}
\newpage

% AUSHEIZVORGÄNGE %




\newcommand{\temperaturesS}{pre,600,700,750,800,875}
\newcommand{\temperaturesV}{pre,500,600,700,750, 800, 875}
\newcommand{\temperaturesVthree}{pre,500,600,700}
\newcommand{\temperatureVfour}{pre, 500, 600, 700, 750, 800}
\newcommand{\temperaturesL}{pre,600, 700, 750, 800, 875}
\newcommand{\temperaturesGlas}{pre, 700, 750, 800, 875}

\subsection{Analyse von Eagle XG Glassubstraten}\label{subsec:glas}
\begin{figure}
    \centering
    \import{../plots/XRD}{glass.pgf}
    \caption{$2\theta/\omega$-Diffraktogramm des Eagle XG Glassubstrats ohne Dünnfilm
    nach dem Ausheizen in der Vakuumkammer auf \qty{875}{\degreeCelsius} in Luft. Grau
    hinterlegt sind Beugungsreflexe des Probenhalters.}
    \label{fig:glass_XRD}
\end{figure}
Um den Einfluss der gewählten Ausheizmethode auf die Morphologie und Kristallinität des Substrats
zu untersuchen, wurden reine Eagle XG Glassubstrate in zwei Ausheizvorgängen für die Temperaturen
\qtylist{600; 700; 750; 800; 875}{\degreeCelsius} durchgeführt.
Ein Aus\-heiz\-vor\-gang wurde in der Vakuumkammer aus \cref{subsec:ausheiz} und ein weiterer
in dem Muffelofen aus \cref{subsec:ausheiz} durchgeführt.
In beiden Prozessen wurde Luft als Hintergrundgas verwendet.

\cref{fig:glass_XRD} zeigt exemplarisch ein $2\theta/\omega$-Diffraktogramm des Eagle XG Glassubstrats nach
dem Ausheizen in der Vakuumkammer auf \qty{875}{\degreeCelsius}.
Zu erkennen ist ein breites Maximum bei circa \qty{23}{\degree}, welches durch das amorphe Glas verursacht wird.
Weiterhin ist ein scharf definierter Peak bei \qty{44.32}{\degree} erkennbar, welcher durch die Kristallstruktur
des Probenhalters verursacht wird.
Der Probenhalter sorgt für weitere, kaum sichtbare Peaks bei \qtylist{64.66; 81.96; 98.61; 115.89}{\degree}.
Die $2\theta/\omega$-Diffraktogramme bei anderen Temperaturen zeigen sowohl beim Vakuum-Aus\-heiz\-vor\-gang als auch
beim Ausheizen im Muffelofen dieselbe Form.
Wie zu erwarten bleiben die Glassubstrate auch bei hohen Temperaturen röntgenamorph.

Die AFM-Topografieaufnahmen der Eagle XG Glassubstrate zu festgelegten Temperaturen während des Aus\-heiz\-vor\-gangs
in der Vakuumkammer sind in \cref{fig:glass_A} dargestellt.
Die Aufnahme des Initialzustands, siehe \cref{fig:glass_A_pre}, zeigt einen ebenen Untergrund mit kleinen Erhebungen,
die auf Rückstände der chemischen Behandlung infolge des Zersägens zurückzuführen sind.
Diese evaporieren bei den ersten Ausheizschritten.
In \cref{fig:glass_A_700,fig:glass_A_750,fig:glass_A_800,fig:glass_A_875} sind sowohl kleine Erhebungen als auch
größere Körner erkennbar.
Dies sind Kontaminationen, die während des Aus\-heiz\-vor\-gangs auf die Oberfläche gelangt sind.
Erkennbar ist, dass sich die Morphologie in den einzelnen Aufnahmen signifikant unterscheidet.
Grund dafür ist wahrscheinlich die ungleichmäßige Verteilung der Kontaminationen auf der Oberfläche.

Die Topografieaufnahmen der Eagle XG Glassubstrate zu ausgewählten Temperaturen während des Aus\-heiz\-vor\-gangs in
dem Muffelofen sind in \cref{fig:glass_B} dargestellt.
Die Aufnahme des Initialzustands, \cref{fig:glass_B_pre}, zeigt einen ebenen Untergrund ohne Erhebungen.
Rückstände aus der chemischen Behandlung sind nicht erkennbar.
Auch nach dem Ausheizen auf die ausgewählten Temperaturen in
\cref{fig:glass_B_700,fig:glass_B_750,fig:glass_B_800,fig:glass_B_875} sind im Gegensatz zur Vakuumkammer
nur sehr wenige Körner erkennbar, die die Oberfläche kontaminieren.

Die AFM-Topografieaufnahmen des Aus\-heiz\-vor\-gangs in Luftatmosphäre im Muffelofen und in der Vakuumkammer zeigen,
dass die Oberflächen der Eagle XG Glassubstrate bei hohen Temperaturen kontaminiert werden.
Dabei zeigt sich eine signifikant höhere Kontamination der Proben nach dem Ausheizprozess in der Vakuumkammer
im Vergleich zu den Proben, die im Muffelofen behandelt wurden.
Dieses Ergebnis ist plausibel, da die Vakuumkammer Teil einer PLD-Anlage ist und durch die Targetablation und das
daraus entstehende Plasma eine erhöhte Partikelkonzentration in der Kammeratmosphäre vorliegt.
Dies führt zu einer erhöhten Deposition von Fremdmaterial auf den Proben.

\begin{figure}[h]
    \centering
    ,\foreach \i in \temperaturesGlas{
        \begin{subfigure}[t]{0.40\textwidth}
            \includegraphics[width=\textwidth]
            {../plots/AFM/XG-Rein/XG-\i/A/EagleXG-A_XG_Rein_\i_Topography_1}
            \caption{\ifthenelse{\equal{\i}{pre}}{Initialzustand}{\qty{\i}{\degreeCelsius}}}
            \label{fig:glass_A_\i}
        \end{subfigure}
    }
    \caption{Topografieaufnahmen der Vakuumkammer-Ausheizserie des Eagle XG Glassubstrats ohne Dünnfilm.}
    \label{fig:glass_A}
\end{figure}

\begin{figure}
    \centering
    ,\foreach \i in \temperaturesGlas{
        \begin{subfigure}[t]{0.40\textwidth}
            \includegraphics[width=\textwidth]
            {../plots/AFM/XG-Rein/XG-\i/B/EagleXG-B_XG_Rein_\i_Topography_1}
            \caption{\ifthenelse{\equal{\i}{pre}}{Initialzustand}{\qty{\i}{\degreeCelsius}}}
            \label{fig:glass_B_\i}
        \end{subfigure}
    }
    \caption{Topografieaufnahmen der Muffelofen-Ausheizserie des Eagle XG Glassubstrats ohne Dünnfilm.}
    \label{fig:glass_B}
\end{figure}

% PROBE W6823-1 %

\newpage

\subsection{Analyse der Probe \samplethree}\label{subsec:probe-W6823-1}
Im Folgenden wird die bei dem höchsten Sauerstoffdruck von \qty{0.1}{\milli\bar} mittels PLD abgeschiedene Probe
\samplethree\ betrachtet.

\subsubsection{Sauerstoff-Aus\-heiz\-vor\-gang}\label{subsubsec:W6823-1B_Sauerstoff}
\begin{figure}
    \centering
    \includegraphics{../plots/XRD/W6823-1B_Sauerstoff}

    \caption{$2\theta/\omega$ Diffraktogramme der Sauerstoff-Ausheizserie von Probe \samplethree.
    Grau hinterlegt sind Beugungsreflexe des Probenhalters.}
    \label{fig:W6823-1B_Sauerstoff_XRD}
\end{figure}
\cref{fig:W6823-1B_Sauerstoff_XRD} zeigt $2\theta/\omega$-Diffraktogramme der Probe \samplethree\ zu den zuvor
definierten Temperaturen während des Sauerstoff-Aus\-heiz\-vor\-gangs.
Bei allen Temperaturen ist ein ausgeweitetes Maximum bei circa \qty{23}{\degree} zu erkennen, welches auf
das EagleXG Glassubstrat zurückzuführen ist.
Weiterhin ist ein scharf definierter Beugungsreflex bei \qty{44.32}{\degree} zu erkennen, welcher durch die
Kristallstruktur des Probenhalters verursacht wird.
Der Probenhalter sorgt für weitere, kaum sichtbare, Beugungsreflexe, welche in der Abbildung grau hinterlegt sind.
Somit sind trotz der bis zu einer Temperatur von \qty{875}{\degreeCelsius} eingestellten Ausheizprozesse keine Peaks
des \heo\ Dünnfilms zu erkennen.
Dieser ist somit unabhängig von der Ausheiztemperatur röntgenamoprh.

Die AFM-Topografieaufnahmen zu verschiedenen Temperaturen der Probe \samplethree\ während des Sauerstoff-Aus\-heiz\-vor\-gangs
sind in \cref{fig:W6823-1B_Sauerstoff_AFM} dargestellt.
Die Aufnahme des Initialzustands, \cref{W6823-1B_Sauerstoff_AFM_pre}, zeigt einen ebenen Untergrund mit zahlreichen
zufällig orientierten Erhebungen.
Aus \cref{subsec:glas} ist bekannt, dass sich die Rauheit des Eagle XG Glassubstrats nach der Zersägung von
\qty{383}{\pico\meter} auf \qty{470}{\pico\meter} erhöht.
Bei stichprobenartigen Maskierungen und separaten Auswertungen des ebenen Untergrunds zeigt sich eine durchschnittliche
Rauheit von \qty{2.3}{\nano\meter}.
Die durchschnittliche Höhe der großen Kristallite beträgt \qtyrange{50}{60}{\nano\meter}.

Mithilfe der zwei Dickenangaben und den Kristallitgrößen ergeben sich zwei Möglichkeiten, um das Kristallwachstum
der \heo\ Dünnfilme zu charakterisieren.
Nach Angaben der XRR Messungen beträgt die Dicke des Films cira \qty{109}{\nano\meter}.
Da diese deutlich über der Höhe der Kristallite liegt, ist es wahrscheinlich, dass der Dünnfilm bis zu einer
kritischen Dicke flächig gewachsen ist und sich erst danach dreidimensionale Inseln ausbilden.
Dieser Prozess wird als Stranski-Krastanov-Wachstum bezeichnet \autocite{Lorenz2019}.
Die zweite Möglichkeit ist, dass die Kristallite die gesamte Dicke des Dünnfilms ausmachen.
Auch dieses Modell erscheint plausibel, da die gemessene Dicke des Dünnfilms, die durch das Pyrometer bestimmt wurde,
bei \qty{65}{\nano\meter} liegt.
Es lässt sich auch ein nicht flächig gewachsener Dünnfilm mit einzelnen Inseln annehmen.

Nach dem ersten Ausheizschritt bei \qty{600}{\degreeCelsius}, siehe \cref{W6823-1B_Sauerstoff_AFM_600}, ist ein
deutlicher Unterschied in der Morphologie des Dünnfilms zu erkennen.
Die Kristallite sind in ihrer Fläche gleich geblieben, ihre Höhe ist jedoch deutlich niedriger als im Initialzustand.
Das Maximum der Höhenskala hat sich von \qty{68}{\nano\meter} auf \qty{30.7}{\nano\meter} verringert.
Eine mögliche Erklärung dieses Phänomens ist die Evaporation von Atomen, welche die Kristallite schrumpfen lässt.

Des Weiteren sind Risse in der Oberfläche zu erkennen.
Diese Risse werden nicht auf reinen Eagle XG Glassubstraten beobachtet und sind auf den Dünnfilm zurückzuführen.
Dieser flächig gewachsene Dünnfilm ist ein weiteres Argument, um das Kristallwachstum mit dem
Stranski-Krastanov-Wachstum zu erklären.
Gründe für die Risse können Spannungen im Dünnfilm und im Substrat sein, welche durch die Temperatur und die
Substrathalterung verursacht werden.
Da \samplethree\ die geringste Schichtdicke der Serie aufweist, ist sie am anfälligsten für Spannungen.

Die Topografieaufnahme nach Ausheizen auf \qty{700}{\degreeCelsius} in \cref{W6823-1B_Sauerstoff_AFM_700} ähnelt
der Aufnahme bei \qty{600}{\degreeCelsius}.
Erkennbar ist eine Verringerung der durchschnittlichen Fläche der kaum noch sichtbaren Kristallite.
Die Risse in der Oberfläche sind weiterhin vorhanden und haben sich vergrößert.
Auch ein großer Riss ist zu erkennen, welcher sich über die gesamte Aufnahme erstreckt.
Nach dem Ausheizen bei \qty{750}{\degreeCelsius} in \cref{W6823-1B_Sauerstoff_AFM_750} sind die Kristallite
vollständig verschwunden.
Erkennbar sind deutliche Risse in der Oberfläche, welche sich weiter vergrößert haben.

Selbst bei einer Temperatur von \qty{800}{\degreeCelsius} setzen die Risse in Abbildung 15e ihre Ausdehnung fort.
Ebenso ist eine Zunahme der maximalen Skalenhöhe zu beobachten.
Bei dem finalen Ausheizschritt bei \qty{875}{\degreeCelsius} in \cref{W6823-1B_Sauerstoff_AFM_875} sind die Risse
weiterhin vorhanden, haben sich jedoch verkleinert.
Es prägt sich eine feingliedrige Rissstruktur aus.

\begin{figure}
    \centering
    \foreach \i in \temperaturesS{
        \begin{subfigure}[t]{0.40\textwidth}
            \includegraphics[width=\textwidth]
            {../plots/AFM/XG-Sauerstoff/XG-\i/W6823-1B/W6823-1B_XG_Sauerstoff_\i_Topography_1}
            \caption{\ifthenelse{\equal{\i}{pre}}{Initialzustand}{\qty{\i}{\degreeCelsius}}}
            \label{W6823-1B_Sauerstoff_AFM_\i}
        \end{subfigure}
    }
    \caption{Topografieaufnahmen der Sauerstoff-Ausheizserie von Probe \samplethree.}
    \label{fig:W6823-1B_Sauerstoff_AFM}
\end{figure}
\newpage

\subsubsection{Vakuum-Aus\-heiz\-vor\-gang}\label{subsubsec:W6823-1C_Vakuum}
\begin{figure}
    \centering
    \includegraphics{../plots/XRD/W6823-1C_Vakuum}
    \caption{$2\theta/\omega$ Diffraktogramme der Vakuum-Ausheizserie von Probe \samplethree.
    Grau hinterlegt sind Beugungsreflexe des Probenhalters.}
    \label{fig:W6823-1C_Vakuum_XRD}
\end{figure}
\cref{fig:W6823-1C_Vakuum_XRD} zeigt die Diffraktogramme der Probe \samplethree\ bei
den Temperaturen des Vakuum-Aus\-heiz\-vor\-gangs.
Anstelle der festgelegten sieben Temperaturen wurden nur vier Temperaturen untersucht.
Grund dafür war ein Fehler in der Substratmontage.
Eine vorangegangene Messung der Vakuumkammer nutzte Wärmeleitpaste mit Silbernanopartikeln.
Die trotz Reinigung verbliebenen Rückstände und die umgekehrte Einbaurichtung des Substrats
führten zu einer Kontamination des Dünnfilms.
Da sich die Paste auf der Oberfläche verfestigt, kann die Probe nicht mehr verwendet werden, da jegliche
Vergleichbarkeit verloren geht.

Die übrigen Temperaturen weisen fast identische Diffraktogramme auf, wie diejenigen aus
\cref{subsubsec:W6823-1B_Sauerstoff}.
Weiterhin sind die einzigen Peaks auf das Eagle XG Glassubstrat und den Probenhalter zurückzuführen.
Dieser \heo\ Dünnfilm ist ebenfalls unabhängig von der Ausheiztemperatur röntgenamorph.

Die AFM-Topografieaufnahmen bei verschiedenen Temperaturen der Probe \samplethree\ während des Vakuum-Aus\-heiz\-vor\-gangs
sind in \cref{fig:W6823-1C_Vakuum_AFM} dargestellt.
Die Aufnahme des Initialzustands in \cref{fig:W6823-1C_Vakuum_AFM_pre} zeigt einen deutlichen Unterschied zu dem
Initialzustand aus \cref{subsubsec:W6823-1B_Sauerstoff}.
Anstelle der flächig verteilten Kristallite auf einem ebenen Untergrund sind hier große Unebenheiten
über das gesamte Bild verteilt.
Auf den Bergen des Untergrundes sind einzelne Kristallite zu erkennen.
Das beobachtete Ergebnis stellt eine unerwartete Entdeckung dar.
Die Topografieaufnahmen der Initialzustände, die bisher untersucht wurden, stammen alle aus derselben Probe, die
lediglich in vier Teile aufgeteilt wurde.
Der einzige Unterschied ist die Aufnahmeposition, sowie der Zeitpunkt der Aufnahme.
Diese Diskrepanz gibt einen Hinweis auf laterale Inhomogenitäten oder zeitliche Instabilitäten des Dünnfilms.


Nach erstem Ausheizen auf \qty{500}{\degreeCelsius} in \cref{fig:W6823-1C_Vakuum_AFM_500} ändert sich die Morphologie
erneut.
Es ist kein unebener Untergrund mehr zu erkennen.
Stattdessen sind einzelne Kristallite auf einem ebenen Untergrund erkennbar.
Für die Rauheit des ebenen Untergrunds findet man \qty{350}{\pico\meter}.
Damit ist diese Rauheit vergleichbar mit der von Eagle XG Glassubstraten.
Es liegt nahe, dass der Dünnfilm nicht flächig gewachsen ist, sondern aus einzelnen Kristalliten besteht.
Nach dem Ausheizen auf \qty{600}{\degreeCelsius} in \cref{fig:W6823-1C_Vakuum_AFM_600} sind die Kristallite in ihrer
Fläche deutlich kleiner.
Auch bei \qty{700}{\degreeCelsius} in \cref{fig:W6823-1C_Vakuum_AFM_700} verkleinern sich die Kristallite
in ihrer Fläche und Höhe.
Unter den Bedingungen des Aus\-heiz\-vor\-gangs im Vakuum ist eine Verkleinerung der Kristallite aufgrund der Evaporation zu
erwarten.
Das sorgt dafür, dass kaum noch Dünnfilmatome auf der Oberfläche verbleiben.

\begin{figure}
    \centering
    ,\foreach \i in \temperaturesVthree{
        \begin{subfigure}[t]{0.40\textwidth}
            \includegraphics[width=\textwidth]
            {../plots/AFM/XG-Vakuum/XG-\i/W6823-1C/W6823-1C_XG_Vakuum_\i_Topography_1}
            \caption{\ifthenelse{\equal{\i}{pre}}{Initialzustand}{\qty{\i}{\degreeCelsius}}}
            \label{fig:W6823-1C_Vakuum_AFM_\i}
        \end{subfigure}
    }
    \caption{Topografieaufnahmen der Vakuum-Ausheizserie von Probe \samplethree.}
    \label{fig:W6823-1C_Vakuum_AFM}
\end{figure}
\newpage

\subsubsection{Luft-Aus\-heiz\-vor\-gang}\label{subsubsec:W6823-1D_Luft}
\begin{figure}
    \centering
    \includegraphics{../plots/XRD/W6823-1D_Luft}
    \caption{$2\theta/\omega$ Diffraktogramme der Luft-Ausheizserie von Probe \samplethree.
    Grau hinterlegt sind Beugungsreflexe des Probenhalters.}
    \label{fig:W6823-1D_Luft_XRD}
\end{figure}
In \cref{fig:W6823-1D_Luft_XRD} sind die $2\theta/\omega$-Scans der Probe \samplethree\ zu den festgelegten Temperaturen
während des Luft-Aus\-heiz\-vor\-gangs dargestellt.
Wie in den vorherigen Abschnitten sind nur die Peaks des Eagle XG Glassubstrats und des Probenhalters zu erkennen.
Damit ist der \heo\ Dünnfilm unabhängig von der Ausheiztemperatur röntgenamorph.

Die AFM-Topografiekarten dieses Aus\-heiz\-vor\-gangs sind in \cref{fig:W6823-1D_Luft_AFM} dargestellt.
Der Initialzustand in \cref{fig:W6823-1D_Luft_AFM_pre} zeigt rundliche Kristallite, welche zufällig angeordnet über eine
ebene Oberfläche verteilt sind.
Die Morphologie ähnelt dem Initialzustand des Sauerstoff-Aus\-heiz\-vor\-gangs in \cref{W6823-1B_Sauerstoff_AFM_pre},
die Kristallite sind jedoch bezüglich ihrer Fläche und Höhe kleiner.
Das ist ein weiteres Indiz für örtliche und zeitliche Inhomogenitäten des Dünnfilms, da auch diese
Topografieaufnahme des Initialzustands der gleichen Probe entstammt, die gevierteilt und zu einem
anderen Zeitpunkt aufgenommen wurde.
Die Rauheit des ebenen Untergrund beträgt \qty{2.3}{\nano\meter}.
Die größeren Kristallite haben eine durchschnittliche Höhe von \qtyrange{40}{50}{\nano\meter}.
Auch diese Messung gibt, je nach präferierter Dicke, Hinweise auf Stranski-Krastanov-Wachstum oder
nichtflächiges Inselwachstum.

Nach dem Ausheizen auf \qty{600}{\degreeCelsius} in \cref{fig:W6823-1D_Luft_AFM_600} sind die Kristallite vollständig
verschwunden.
Stattdessen liegt eine homogene Oberfläche vor.
Die Rauheit dieser Oberfläche beträgt \qty{2.1}{\nano\meter} und ist damit
vergleichbar mit der Rauheit des ebenen Untergrunds des Initialzustands, was die Annahme der Evaporation der Kristallite
weiter verstärkt.
Die Aufnahme bei \qty{700}{\degreeCelsius} in \cref{fig:W6823-1D_Luft_AFM_700} zeichnet sich durch eine feingliedrige
Rissstruktur aus, die sich über die gesamte Aufnahme erstreckt.
Dadurch nimmt auch die maximale Höhe der Skala zu.
In der Aufnahme von \qty{750}{\degreeCelsius} in \cref{fig:W6823-1D_Luft_AFM_750} vergrößern sich die Risse weiter.
Bei \qty{800}{\degreeCelsius} und \qty{875}{\degreeCelsius} in
\cref{fig:W6823-1D_Luft_AFM_800,fig:W6823-1D_Luft_AFM_875} sind die Risse weiterhin vorhanden, haben sich jedoch
verkleinert.
Die Aufnahme nach \qty{875}{\degreeCelsius} zeigt eine noch feingliedrigere Rissstruktur als die Aufnahme
bei \qty{700}{\degreeCelsius}.
Die Risse können durch thermisch-induzierte Spannungen aufgrund der unterschiedlichen thermischen
Ausdehnungskoeffizienten von Dünnfilm und Substrat entstehen \autocite{cracks}.
Auch Phasenübergänge, die durch die Temperaturänderung hervorgerufen werden, können mit Volumenänderungen
einhergehen und so zu Rissen führen.
Da der Effekt unabhängig vom Substrathalter auftritt, können mechanische Einflüsse des Substrathalters als Grund
der Rissbildung ausgeschlossen werden.
Die Änderung der Morphologie des Dünnfilms während des Luft-Aus\-heiz\-vor\-gangs ist vergleichbar mit der des Sauerstoffs
in \cref{subsubsec:W6823-1B_Sauerstoff}.

\begin{figure}[h]
    \centering
    ,\foreach \i in \temperaturesL{
        \begin{subfigure}[t]{0.40\textwidth}
            \includegraphics[width=\textwidth]
            {../plots/AFM/XG-Luft/XG-\i/W6823-1D/W6823-1D_XG_Luft_\i_Topography_1}
            \caption{\ifthenelse{\equal{\i}{pre}}{Initialzustand}{\qty{\i}{\degreeCelsius}}}
            \label{fig:W6823-1D_Luft_AFM_\i}
        \end{subfigure}
    }
    \caption{Topografieaufnahmen der Luft-Ausheizserie von Probe \samplethree.}
    \label{fig:W6823-1D_Luft_AFM}
\end{figure}
\newpage


% PROBE W6821-1 %

\newpage

\subsection{Analyse der Probe \sampleone}\label{subsec:probe-W6821-1}
Im nächsten Schritt wird die Probe \sampleone\ betrachtet, die bei einem Sauerstoffdruck von \qty{0.01}{\milli\bar}
mittels PLD abgeschieden wurde.

\subsubsection{Sauerstoff-Aus\-heiz\-vor\-gang}\label{subsubsec:W6821-1B_Sauerstoff}
\begin{figure}
    \centering
    \includegraphics{../plots/XRD/W6821-1B_Sauerstoff}
    \caption{$2\theta/\omega$ Diffraktogramme der Sauerstoff-Ausheizserie von Probe \sampleone.
    Grau hinterlegt sind Beugungsreflexe des Probenhalters.}
    \label{fig:W6821-1B_Sauerstoff_XRD}
\end{figure}
In \cref{fig:W6821-1B_Sauerstoff_XRD} sind die Diffraktogramme der Probe \sampleone\ zu ausgewählten
Temperaturen der Ausheizserie in Sauerstoff dargestellt.
Wie in den vorherigen Abschnitten sind die Peaks auf das Eagle XG Glassubstrat und den Probenhalter zurückzuführen.
Es sind keine Peaks des \heo\ Dünnfilms zu erkennen.
Der Dünnfilm ist unabhängig von der Ausheiztemperatur röntgenamorph.

\cref{fig:W6821-1B_Sauerstoff_AFM} zeigt die AFM-Topografieaufnahmen der Probe \sampleone\ zu den verschiedenen
Temperaturen.
Die Aufnahme des Initialzustands in \cref{W6821-1B_Sauerstoff_AFM_pre} zeigt viele zufällig angeordnete Kristallite,
welche dicht beieinander liegen.
Die Höhe der großen Kristallite beträgt circa \qtyrange{45}{55}{\nano\meter}.
Die Topografie zeigt, dass das ganze Substrat mit Dünnfilm bedeckt ist.

Nach dem ersten Ausheizschritt auf \qty{600}{\degreeCelsius} in \cref{W6821-1B_Sauerstoff_AFM_600} hat sich die
Oberflächenmorphologie deutlich verändert.
Die Kristallitgröße hat sich signifikant verringert, während sich deren Dichte vergrößert hat.
Auch die maximale Höhe der Skala hat sich von \qty{62}{\nano\meter} auf \qty{42.7}{\nano\meter} verringert.
Die Topografie bei \qty{700}{\degreeCelsius} in \cref{W6821-1B_Sauerstoff_AFM_700} weist im Vergleich zu
\qty{600}{\degreeCelsius} eine geringere Dichte an Kristalliten auf.
Zudem sind erste Löcher in der Oberfläche sichtbar.

Die Oberflächenmorphologien von \qty{750}{\degreeCelsius} in \cref{W6821-1B_Sauerstoff_AFM_750} und
\qty{800}{\degreeCelsius} in \cref{W6821-1B_Sauerstoff_AFM_800} ähneln sich.
Die Kristallithöhe ist im Gegensatz zu der Aufnahme bei \qty{700}{\degreeCelsius} deutlich niedriger.
Zusätzlich ist die Anzahl der Löcher in der Oberfläche gestiegen.
Die Oberflächenmorphologien aus \cref{subsec:glas} weisen keine Löcher auf.
Dies ist ein Indiz dafür, dass das Material, in welchem die Löcher zu beobachten sind, der Dünnfilm ist
und nicht das Substrat.
Da sie bezüglich ihrer Fläche in der Größenordnung der Kristallite liegen, ist es naheliegend, dass die Kristallite
vollständig evaporieren.

Die Aufnahme nach dem Ausheizen bei \qty{875}{\degreeCelsius} in \cref{W6821-1B_Sauerstoff_AFM_875} zeigt zusätzlich
eine Vertiefung, welche sich über den gesamten oberen rechten Quadranten des Bildes erstreckt.
\begin{figure}[h]
    \centering
    \foreach \i in \temperaturesS{
        \begin{subfigure}[t]{0.40\textwidth}
            \includegraphics[width=\textwidth]
            {../plots/AFM/XG-Sauerstoff/XG-\i/W6821-1B/W6821-1B_XG_Sauerstoff_\i_Topography_1}
            \caption{\ifthenelse{\equal{\i}{pre}}{Initialzustand}{\qty{\i}{\degreeCelsius}}}
            \label{W6821-1B_Sauerstoff_AFM_\i}
        \end{subfigure}
    }
    \caption{Topografieaufnahmen der Sauerstoff-Ausheizserie von Probe \sampleone.}
    \label{fig:W6821-1B_Sauerstoff_AFM}
\end{figure}
\newpage

\subsubsection{Vakuum-Aus\-heiz\-vor\-gang}\label{subsubsec:W6821-1C_Vakuum}
\begin{figure}
    \centering
    \includegraphics{../plots/XRD/W6821-1C_Vakuum}
    \caption{$2\theta/\omega$ Diffraktogramme der Vakuum-Ausheizserie von Probe \sampleone.
    Grau hinterlegt sind Beugungsreflexe des Probenhalters.}
    \label{fig:W6821-1C_Vakuum_XRD}
\end{figure}
In \cref{fig:W6821-1C_Vakuum_XRD} sind die $2\theta/\omega$-Diffraktogramme der Probe \sampleone\ für
verschiedene Temperaturen der Ausheizserie in Sauerstoff dargestellt.
Wie in den vorherigen Abschnitten beschrieben, resultieren die Peaks aus dem Eagle XG Glassubstrat sowie dem
Probenhalter.
Es sind keine Peaks des \heo\ Dünnfilms erkennbar.
Der Dünnfilm ist unabhängig von der Ausheiztemperatur röntgenamorph.

Mithilfe des AFM wurden Topografiekarten der Probe \sampleone\ während des Vakuum-Aus\-heiz\-vor\-gangs
aufgenommen, siehe \cref{fig:W6821-1C_Vakuum_AFM}.
Die Aufnahme des Initialzustands in \cref{W6821-1C_Vakuum_AFM_pre} zeigt einen ebenen Untergrund, der mit vielen
zufällig orientierten Kristalliten bedeckt ist.
Die größeren Kristallite haben eine Höhe von circa \qtyrange{60}{75}{\nano\meter}.
Zwischen diesen großen Kristalliten sind kleinere Kristallite zu erkennen, die den Untergrund bedecken.
Auch diese Topografieaufnahme des Initialzustands weist Unterschiede zum Initialzustand der Sauerstoffserie auf
und liefert ein weiteres Indiz für die örtlichen oder zeitliche Änderungen des Dünnfilms.

Die \qty{500}{\degreeCelsius} Aufnahme in \cref{W6821-1C_Vakuum_AFM_500} weist einen ebenen Untergrund auf,
der nicht flächig mit Erhebungen bedeckt ist.
Diese Erhebungen liegen bezüglich ihrer Fläche in der Größenordnung der Kristallite des Initialzustands.
Die Höhe der Erhebungen ist jedoch deutlich geringer, dies ist ein weiteres Indiz für die Evaporation der Atome.
Die Rauheit des ebenen Untergrunds beträgt circa \qty{1.5}{\nano\meter}, sodass davon ausgegangen werden kann,
dass der Film flächig gewachsen ist.

Die Aufnahme bei \qty{600}{\degreeCelsius} in \cref{W6821-1C_Vakuum_AFM_600} zeigt eine homogene Oberfläche, ohne
Erhebungen.
Auch die Aufnahme bei \qty{700}{\degreeCelsius} in \cref{W6821-1C_Vakuum_AFM_700} ist weitesgehend homogen,
einzelne Kristallite scheinen sich jedoch gebildet zu haben.
Diese sind auch bei \qty{750}{\degreeCelsius} in \cref{W6821-1C_Vakuum_AFM_750} zu erkennen.
Die Kristallite sind in ihrer Fläche und Höhe deutlich größer als bei \qty{700}{\degreeCelsius},
jedoch kleiner als im Initialzustand.
Auch in dieser Aufnahme sind Löcher in der Oberfläche zu erkennen, analog zu \cref{subsubsec:W6821-1B_Sauerstoff}.
Die Löcher weisen darauf hin, dass der Untergrund der Dünnfilm ist und nicht das Eagle XG Glassubstrat.
Die Aufnahme bei \qty{800}{\degreeCelsius} in \cref{W6821-1C_Vakuum_AFM_800} zeigt
deutlich weniger Kristallite und Löcher.
Einen Sonderfall stellt die Aufnahme bei \qty{875}{\degreeCelsius} in \cref{W6821-1C_Vakuum_AFM_875} dar.
Diese enthält viele Krisallite mit dreieckiger Form und einheitlicher Orientierung.
Dies deutet auf eine abgebrochene oder beschädigte Cantilever-Spitze hin.
Trotz mehrmaliger Messung und Wechsel der Spitze, einem geringeren Setpoint und einer geringeren Scanrate, blieb die
Struktur erhalten.
Eine Verfälschung dieser Aufnahme kann jedoch nicht ausgeschlossen werden.
\begin{figure}
    \centering
    ,\foreach \i in \temperaturesV{
        \begin{subfigure}[t]{0.40\textwidth}
            \includegraphics[width=\textwidth]
            {../plots/AFM/XG-Vakuum/XG-\i/W6821-1C/W6821-1C_XG_Vakuum_\i_Topography_1}
            \caption{\ifthenelse{\equal{\i}{pre}}{Initialzustand}{\qty{\i}{\degreeCelsius}}}
            \label{W6821-1C_Vakuum_AFM_\i}
        \end{subfigure}
    }
    \caption{Topografieaufnahmen der Vakuum-Ausheizserie von Probe \sampleone.}
    \label{fig:W6821-1C_Vakuum_AFM}
\end{figure}
\newpage

\subsubsection{Luft-Aus\-heiz\-vor\-gang}\label{subsubsec:W6821-1D_Luft}
\begin{figure}
    \centering
    \includegraphics{../plots/XRD/W6821-1D_Luft}
    \caption{$2\theta/\omega$ Diffraktogramme der Luft-Ausheizserie von Probe \sampleone.
    Grau hinterlegt sind Beugungsreflexe des Probenhalters.}
    \label{fig:W6821-1D_Luft_XRD}
\end{figure}
In \cref{fig:W6821-1D_Luft_XRD} sind die $2\theta/\omega$-Diffraktogramme der Probe \sampleone\ dargestellt,
wobei die Messungen für festgelegte Temperaturen nach der Ausheizphase in Luft durchgeführt wurden.
Wie bereits in den vorherigen Abschnitten erläutert, resultieren die ermittelten Peaks aus dem Eagle XG Glassubstrat
sowie dem Probenhalter.
Es sind keine Peaks des \heo\ Dünnfilms erkennbar sind, was darauf hindeutet, dass der Dünnfilm unabhängig von der
Ausheiztemperatur eine röntgenamorphe Struktur aufweist.

Die AFM-Topografieaufnahmen der Probe \sampleone\ während des Luft-Aus\-heiz\-vor\-gangs sind in \cref{fig:W6821-1D_Luft_AFM}
abgebildet.
Die Aufnahme des Initialzustands in \cref{fig:W6821-1D_Luft_AFM_pre}
unterscheidet sich deutlich von den anderen Aufnahmen.
Anstelle der üblichen $\qty{5}{\micro\meter} \times \qty{5}{\micro\meter}$ Aufnahmen wurde hier eine
$\qty{15}{\micro\meter} \times \qty{15}{\micro\meter}$ Aufnahme erstellt, um die Morphologie des Dünnfilms besser zu
erfassen.
Es sind viele großflächige und hohe Kristallite zu erkennen, die nicht in vorherigen Aufnahmen zu sehen waren.
Dabei nimmt ein Kristallit bereits eine Fläche von fast $\qty{5}{\micro\meter} \times \qty{5}{\micro\meter}$ ein.
Der Dünnfilm könnte nicht flächig gewachsen sein, sondern aus einzelnen Kristalliten bestehen.
Auch diese Messung ist unvorhergesehen, da sich der Initialzustand im Vergleich zu den Initialzuständen aus
\cref{subsubsec:W6821-1B_Sauerstoff,subsubsec:W6821-1C_Vakuum} deutlich verändert, obwohl nur eine zeitliche
und örtliche Änderung des Dünnfilms vorliegt.
Es handelt sich um die gleiche Probe, die lediglich in vier Teile aufgeteilt wurde.
Diese Diskrepanz gibt einen weiteren Hinweis auf laterale oder zeitliche Inhomogenitäten.

Nach dem Ausheizen auf \qty{600}{\degreeCelsius} in \cref{fig:W6821-1D_Luft_AFM_600} sind die großen Kristallite
verschwunden, stattdessen sind viele bezüglich ihrer Fläche kleine, aber dennoch sehr hohe
Kristallite zu erkennen.
Nach \qty{700}{\degreeCelsius} in \cref{fig:W6821-1D_Luft_AFM_700} sind die Kristallite in ihrer Fläche gleich
geblieben, jedoch deutlich höher.
Bei \qty{750}{\degreeCelsius} in \cref{fig:W6821-1D_Luft_AFM_750} sind keine Kristallite nicht mehr zu sehen.
Stattdessen sind gleichmäßig verteilte Löcher zu erkennen, die bezüglich ihrer Fläche der Kristallitgröße entsprechen.
Das deutet darauf hin, dass die Kristallite evaporiert sind.
Die Morphologie bei \qty{800}{\degreeCelsius} und \qty{875}{\degreeCelsius} in
\cref{fig:W6821-1D_Luft_AFM_800,fig:W6821-1D_Luft_AFM_875} ähneln der Aufnahme \cref{fig:W6821-1D_Luft_AFM_750}.
Bei einer Temperatur von \qty{875}{\degreeCelsius} sind zusätzliche leichte Verformungen des Untergrunds zu beobachten,
die möglicherweise auf thermische Spannungen zurückzuführen sein könnten und auf potenzielle Verformungen des Substrats
hinweisen.

\begin{figure}[h]
    \centering
    ,\foreach \i in \temperaturesL{
        \begin{subfigure}[t]{0.40\textwidth}
            \includegraphics[width=\textwidth]
            {../plots/AFM/XG-Luft/XG-\i/W6821-1D/W6821-1D_XG_Luft_\i_Topography_1}
            \caption{\ifthenelse{\equal{\i}{pre}}{Initialzustand}{\qty{\i}{\degreeCelsius}}}
            \label{fig:W6821-1D_Luft_AFM_\i}
        \end{subfigure}
    }
    \caption{Topografieaufnahmen der Vakuum-Ausheizserie von Probe \sampleone.}
    \label{fig:W6821-1D_Luft_AFM}
\end{figure}
\newpage

% PROBE W6822-1 %

\newpage

\subsection{Analyse der Probe \sampletwo}\label{subsec:probe-W6822-1}
Im folgenden Kapitel wird die Probe \sampletwo\ charakterisiert, die bei einem Sauerstoffdruck von \qty{0.001}{\milli\bar}
mittels PLD abgeschieden wurde.

\subsubsection{Sauerstoff-Aus\-heiz\-vor\-gang}\label{subsubsec:W6822-1B_Sauerstoff}
\begin{figure}
    \centering
    \includegraphics{../plots/XRD/W6822-1B_Sauerstoff}
    \caption{$2\theta/\omega$ Diffraktogramme der Sauerstoff-Ausheizserie von Probe \sampletwo.
    Grau hinterlegt sind Beugungsreflexe des Probenhalters.}
    \label{fig:W6822-1B_Sauerstoff_XRD}
\end{figure}

Die Diffraktogramme der Probe \sampletwo\ nach den jeweiligen Ausheizschritten in Sauerstoff sind in
\cref{fig:W6822-1B_Sauerstoff_XRD} dargestellt.
Auch hier sind jegliche Peaks auf das Substrat und den Probenhalter zurückzuführen.
Der \heo\ Dünnfilm ist röntgenamorph.

\cref{fig:W6822-1B_Sauerstoff_AFM} zeigt die AFM-Topografieaufnahmen der Probe \sampletwo\ während des
Sauerstoff-Aus\-heiz\-vor\-gangs.
Diese zeigen, wie die Initialzustände der bisherigen Sauerstoff-Aus\-heiz\-vor\-gänge, verteilte Kristallite, die einen
ebenen Untergrund flächig bedecken.
Anders als die bisherigen Kristallite der Initialzustände sind diese weniger rund, sondern zerfasert.
Die Rauheit des ebenen Untergrunds liegt bei circa \qty{1}{\nano\meter}, die Höhe der Kristallite zwischen
\qtyrange{20}{30}{\nano\meter}.
Die Messergebnisse deuten darauf hin, dass es möglicherweise eine Tendenz zu einem flächig gewachsenen Dünnfilm gib,
da mithilfe von XRR eine Dünnfilmdicke von \qty{140}{\nano\meter} und mithilfe des Profilometers eine
Dünnfilmdicke von \qty{160}{\nano\meter} bestimmt wurde.

Nach dem Ausheizen auf \qty{600}{\degreeCelsius} in \cref{W6822-1B_Sauerstoff_AFM_600} sind alle Kristallite
verschwunden und eine homogene raue Oberfläche ist zu erkennen.
Mit \qty{700}{\degreeCelsius} beginnt die Bildung kleiner Kristallite in \cref{W6822-1B_Sauerstoff_AFM_700},
die bei \qty{750}{\degreeCelsius} bezüglich ihrer Fläche und Höhe wachsen, siehe \cref{W6822-1B_Sauerstoff_AFM_750}.
Auch bei \qty{800}{\degreeCelsius} und \qty{875}{\degreeCelsius} in
\cref{W6822-1B_Sauerstoff_AFM_800,W6822-1B_Sauerstoff_AFM_875} sind Kristallite erkennbar, die die Oberfläche
mittlerweile flächig überdecken und die Intervallskala deutlich erhöhen.
Hier ist eine thermisch-induzierte Rekristallisation erkennbar.

\begin{figure}
    \centering
    \foreach \i in \temperaturesS{
        \begin{subfigure}[t]{0.40\textwidth}
            \includegraphics[width=\textwidth]
            {../plots/AFM/XG-Sauerstoff/XG-\i/W6822-1B/W6822-1B_XG_Sauerstoff_\i_Topography_1}
            \caption{\ifthenelse{\equal{\i}{pre}}{Initialzustand}{\qty{\i}{\degreeCelsius}}}
            \label{W6822-1B_Sauerstoff_AFM_\i}
        \end{subfigure}
    }
    \caption{Topografieaufnahmen der Sauerstoff-Ausheizserie von Probe \sampletwo.}
    \label{fig:W6822-1B_Sauerstoff_AFM}
\end{figure}
\newpage

\subsubsection{Vakuum-Aus\-heiz\-vor\-gang}\label{subsubsec:W6822-1B_Vakuum}
\begin{figure}
    \centering
    \includegraphics{../plots/XRD/W6822-1C_Vakuum}
    \caption{$2\theta/\omega$ Diffraktogramme der Vakuum-Ausheizserie von Probe \sampletwo.
    Grau hinterlegt sind Beugungsreflexe des Probenhalters.}
    \label{fig:W6822-1C_Vakuum_XRD}
\end{figure}

Die Diffraktogramme der Probe \sampletwo\ sind in \cref{fig:W6822-1C_Vakuum_XRD} dargestellt.
Auch hier ist der Dünnfilm für alle untersuchten Ausheiztemperaturen röntgenamorph.

In \cref{fig:W6822-1C_Vakuum_AFM} sind die AFM-Topografieaufnahmen der Probe \sampletwo\ während des
Vakuum-Aus\-heiz\-vor\-gangs dargestellt.
Der in \cref{W6822-1C_Vakuum_AFM_pre} gezeigte Initialzustand ist vergleichbar mit dem Initialzustand des
Sauerstoff-Aus\-heiz\-vor\-gangs derselben Probe, jedoch sind die Kristallite bezüglich ihrer Fläche etwas größer.
Bei \qty{500}{\degreeCelsius} in \cref{W6822-1C_Vakuum_AFM_500} sind große Löcher zu erkennen, welche eine
ähnliche Flächenverteilung haben wie die Kristallite des Initialzustands.
Das deutet darauf hin, dass die Kristallite des Dünnfilms evaporiert sind.
Bei \qty{600}{\degreeCelsius} in \cref{W6822-1C_Vakuum_AFM_600} ist ein homogener rauer Film zu erkennen, auf dem
einige kleine Kristallite zu sehen sind.
Bei \qty{700}{\degreeCelsius} in \cref{W6822-1C_Vakuum_AFM_700} sind viele kleine Kristallite zu erkennen, die sowohl
in ihrer Fläche als auch in ihrer Höhe größer sind als bei \qty{600}{\degreeCelsius}.
Diese Entwicklung ist auch bei \qty{750}{\degreeCelsius} in \cref{W6822-1C_Vakuum_AFM_750} zu beobachten.
Nach dem Ausheizen auf \qty{800}{\degreeCelsius} in \cref{W6822-1C_Vakuum_AFM_800} ist die Anzahl der beobachteten
Kristallite gesunken.
Außerdem werden Löcher im Film sichtbar, deren Fläche der der vorher dagewesenen Kristallite entspricht.
Bei \qty{875}{\degreeCelsius} in \cref{W6822-1C_Vakuum_AFM_875} sind einzelne Kristallite zu erkennen.
Der Untergrund ist nicht mehr eben, sondern weist eine bergige Struktur auf, die sich über das gesamte Bild zieht.
Diese Struktur lässt sich möglicherweise mit der Verformung des Substrats erklären.

\begin{figure}
    \centering
    ,\foreach \i in \temperaturesV{
        \begin{subfigure}[t]{0.40\textwidth}
            \includegraphics[width=\textwidth]
            {../plots/AFM/XG-Vakuum/XG-\i/W6822-1C/W6822-1C_XG_Vakuum_\i_Topography_1}
            \caption{\ifthenelse{\equal{\i}{pre}}{Initialzustand}{\qty{\i}{\degreeCelsius}}}
            \label{W6822-1C_Vakuum_AFM_\i}
        \end{subfigure}
    }
    \caption{Topografieaufnahmen der Vakuum-Ausheizserie von Probe \sampletwo.}
    \label{fig:W6822-1C_Vakuum_AFM}
\end{figure}
\newpage

\subsubsection{Luft-Aus\-heiz\-vor\-gang}\label{subsubsec:W6822-1B_Luft}
\begin{figure}
    \centering
    \includegraphics{../plots/XRD/W6822-1D_Luft}
    \caption{$2\theta/\omega$ Diffraktogramme der Luft-Ausheizserie von Probe \sampletwo.
    Grau hinterlegt sind Beugungsreflexe des Probenhalters.}
    \label{fig:W6822-1D_Luft_XRD}
\end{figure}

In \cref{fig:W6822-1D_Luft_XRD} sind die $2\theta/\omega$-Scans der Probe \sampletwo\ zu den festgelegten Temperaturen
während des Luft-Aus\-heiz\-vor\-gangs dargestellt.
Es zeigen sich keine Reflexe für alle untersuchten Ausheiztemperaturen, sodass der Dünnfilm röntgenamorph
ist und bleibt.

Die AFM-Topografieaufnahmen der Probe \sampletwo\ während des Luft-Aus\-heiz\-vor\-gangs sind in \cref{fig:W6822-1D_Luft_AFM}
dargestellt.
Der Initialzustand in \cref{W6822-1D_Luft_AFM_pre} zeigt eine bis jetzt nicht beobachtete Morphologie.
Es sind sowohl kleine Kristallite, die bisher erst nach dem Ausheizen auftraten, als auch größere Erhebungen die
Dimensionen der Kristallite der anderen Initialzustände aufweisen, zu erkennen.
Ein möglicher Grund dafür könnte darin liegen, dass die Probe \sampletwo\ vorher bereits als Testprobe für \qty{1}{\hour} und
\qty{3}{\hour} in Sauerstoff ausgeheizt wurde, sodass beide Phänomene sichtbar werden.
Nach dem Ausheizen auf \qty{600}{\degreeCelsius} in \cref{W6822-1D_Luft_AFM_600} sind die großen Kristallite
verschwunden und es ist eine homogene raue Oberfläche zu erkennen.
Diese zeigt eine ähnliche Morphologie wie die Aufnahmen bei \qty{700}{\degreeCelsius} der Sauerstoff- und
Vakuum-Ausheizserie in \cref{subsubsec:W6822-1B_Sauerstoff,subsubsec:W6822-1B_Vakuum}.
Dies ist möglicherweise ein Indiz dafür, dass die Temperatur des Dünnfilms in der Vakuumkammer deutlich
unterhalb der Temperatur des Muffelofens liegt.
Da im Gegensatz zum Muffelofen nur die Rückseite des Substrathalters erhitzt wird, entsteht
ein Temperaturgradient zwischen der Vorder- und Rückseite des Substrats, sowie zwischen der Vorder- und
Rückseite des Dünnfilms, welcher für Eagle XG möglicherweise deutlich größer ist, als in
\cref{subsec:temperaturkalibrierung} abgeschätzt.
Bei \qty{700}{\degreeCelsius} in \cref{W6822-1D_Luft_AFM_700} sind vereinzelt kleine Löcher und Kristallite zu
erkennen.



Die AFM Aufnahmen nach den Ausheiztemperaturen von \qtylist{750;800;875}{\degreeCelsius}
in \cref{W6822-1D_Luft_AFM_750,W6822-1D_Luft_AFM_800,W6822-1D_Luft_AFM_875}
sind vergleichbar mit den AFM Aufnahmen nach dem Ausheizen mit den Temperaturen \qtylist{750;800;875}{\degreeCelsius}
an Luft der Probe \sampleone.



Die Reihe von \qty{750}{\degreeCelsius} bis \qty{875}{\degreeCelsius} in
\cref{W6822-1D_Luft_AFM_750,W6822-1D_Luft_AFM_800,W6822-1D_Luft_AFM_875} ist vergleichbar mit der
Luft-Ausheizaufnahme der Probe \sampleone.
Vor allem kleine Löcher sind zu erkennen, die auf eine Evaporation der Kristallite, die bis zum Ausheizen bei
\qty{700}{\degreeCelsius} in den AFM Aufnahmen erkennbar sind, hindeuten.
Die Aufnahme bei \qty{875}{\degreeCelsius} zeigt eine Deformierung des Untergrunds.
Dies könnte durch eine Deformierung des Substrates bedingt sein.

\begin{figure}[h]
    \centering
    ,\foreach \i in \temperaturesL{
        \begin{subfigure}[t]{0.40\textwidth}
            \includegraphics[width=\textwidth]
            {../plots/AFM/XG-Luft/XG-\i/W6822-1D/W6822-1D_XG_Luft_\i_Topography_1}
            \caption{\ifthenelse{\equal{\i}{pre}}{Initialzustand}{\qty{\i}{\degreeCelsius}}}
            \label{W6822-1D_Luft_AFM_\i}
        \end{subfigure}
    }
    \caption{Topografieaufnahmen der Luft-Ausheizserie von Probe \sampletwo.}
    \label{fig:W6822-1D_Luft_AFM}
\end{figure}
\newpage


% PROBE W6824-1 %
\newpage

\subsection{Analyse der Probe \samplefour}\label{subsec:probe-W6824-1}
Zum Schluss wird die Probe \samplefour\ charakterisiert, die bei einem Sauerstoffdruck von \qty{0.00005}{\milli\bar}
mittels PLD abgeschieden wurde.
\subsubsection{Sauerstoff-Aus\-heiz\-vor\-gang}\label{subsubsec:W6824-1B_Sauerstoff}
\begin{figure}
    \centering
    \includegraphics{../plots/XRD/W6824-1B_Sauerstoff}
    \caption{$2\theta/\omega$ Diffraktogramme der Sauerstoff-Ausheizserie von Probe \samplefour.
    Grau hinterlegt sind Beugungsreflexe des Probenhalters.}
    \label{fig:W6824-1B_Sauerstoff_XRD}
\end{figure}
\cref{fig:W6824-1B_Sauerstoff_XRD} zeigt die Diffraktogramme der Probe \samplefour\ zu den ausgewählten Temperaturen
der Sauerstoff-Ausheizserie von Probe \samplefour.
Da alle Peaks auf das Substrat und den Probenhalter zurückzuführen sind, ist der Dünnfilm unabhängig von der
Ausheiztemperatur röntgenamorph.


Die AFM-Topografieaufnahmen der Probe \samplefour\ während des Sauerstoff-Aus\-heiz\-vor\-gangs sind in
\cref{fig:W6824-1B_Sauerstoff_AFM} dargestellt.
Der in \cref{W6824-1B_Sauerstoff_AFM_pre} gezeigte Initialzustand ist vergleichbar mit dem Initialzustand der
Sauerstoff-Ausheizserie von Probe \sampletwo.
Die Kristallite, die den Film flächig bedecken, haben eine Höhe von circa \qtyrange{30}{40}{\nano\meter}.
Mithilfe der XRR-Messungen wurde eine Dicke des Dünnfilms von \qty{120}{\nano\meter} bestimmt, sodass davon
ausgegangen werden kann, dass der Film flächig gewachsen ist und einzelne Kristallite aufweist.
Nach dem Ausheizen auf \qty{600}{\degreeCelsius} in \cref{W6824-1B_Sauerstoff_AFM_600} sind die Kristallite
verschwunden und eine raue Oberfläche ist zu erkennen.
Die darin vorkommenden, vereinzelten Vertiefungen deuten erneut auf die Evaporation der Kristallite hin.

Bei \qty{700}{\degreeCelsius} in \cref{W6824-1B_Sauerstoff_AFM_750} sind erneut Kristallite zu erkennen, die
in ihrer Fläche und Höhe nach den Ausheizprozessen auf  \qty{750}{\degreeCelsius} und \qty{800}{\degreeCelsius} in
\cref{W6824-1B_Sauerstoff_AFM_750,W6824-1B_Sauerstoff_AFM_800} weiter wachsen.
Die Kristallite in den Aufnahmen bei \qty{750}{\degree} und \qty{800}{\degree} weisen eine leichte Orientierung auf,
die auf ein Artefakt des AFMs zurückzuführen sein könnte.
Da diese Artefakte erst bei der Auswertung der Aufnahmen auffielen, konnte keine erneute Messung durchgeführt werden.
Die Aufnahme bei \qty{875}{\degreeCelsius} in \cref{W6824-1B_Sauerstoff_AFM_875} zeigt erneut Löcher sowie
einen deformierten Untergrund, was auf eine Verformung des Substrats hindeuten könnte.

\begin{figure}
    \centering
    \foreach \i in \temperaturesS{
        \begin{subfigure}[t]{0.40\textwidth}
            \includegraphics[width=\textwidth]
            {../plots/AFM/XG-Sauerstoff/XG-\i/W6824-1B/W6824-1B_XG_Sauerstoff_\i_Topography_1}
            \caption{\ifthenelse{\equal{\i}{pre}}{Initialzustand}{\qty{\i}{\degreeCelsius}}}
            \label{W6824-1B_Sauerstoff_AFM_\i}
        \end{subfigure}
    }
    \caption{Topografieaufnahmen der Sauerstoff-Ausheizserie von Probe \samplefour.}
    \label{fig:W6824-1B_Sauerstoff_AFM}
\end{figure}
\newpage

\subsubsection{Vakuum-Aus\-heiz\-vor\-gang}\label{subsubsec:W6824-1B_Vakuum}
\begin{figure}
    \centering
    \includegraphics{../plots/XRD/W6824-1C_Vakuum}
    \caption{$2\theta/\omega$ Diffraktogramme der Vakuum-Ausheizserie von Probe \samplefour.
    Grau hinterlegt sind Beugungsreflexe des Probenhalters.}
    \label{fig:W6824-1C_Vakuum_XRD}
\end{figure}
Auch die in \cref{fig:W6824-1C_Vakuum_XRD} dargestellten Diffraktogramme der Probe \samplefour\ zeigen keine Peaks
des \heo\ Dünnfilms, sodass dieser für alle Temperaturen röntgenamorph ist und bleibt.

Die AFM-Topografieaufnahmen der Probe \samplefour\ während des Vakuum-Aus\-heiz\-vor\-gangs sind in
\cref{fig:W6824-1C_Vakuum_AFM} dargestellt.
Der in \cref{W6824-1C_Vakuum_AFM_pre} gezeigte Initialzustand ist vergleichbar mit dem Initialzustand der
Sauerstoff-Ausheizserie von Probe \samplefour, die einzelnen Kristallite sind jedoch bezüglich ihrer Fläche
kleiner.
Die Höhe der großen Kristallite beträgt circa \qtyrange{15}{30}{\nano\meter}, was erneut auf einen flächig gewachsenen
Dünnfilm hindeutet.
Nach dem Ausheizen auf \qty{500}{\degreeCelsius} in \cref{W6824-1C_Vakuum_AFM_500} ist eine homogene raue Oberfläche
erkennbar.
Bei \qty{600}{\degreeCelsius} in \cref{W6824-1C_Vakuum_AFM_600} bilden sich einzelne Kristallite aus dem ebenen
Untergrund, dessen Größe und Anzahl bei \qty{700}{\degreeCelsius} in \cref{W6824-1C_Vakuum_AFM_700} und
\qty{750}{\degreeCelsius} in \cref{W6824-1C_Vakuum_AFM_750} zunimmt.
Bei \qty{800}{\degreeCelsius} in \cref{W6824-1C_Vakuum_AFM_800} sind nur noch vereinzelte, sehr große und hohe
Kristallite zu erkennen.
Nach dem Ausheizschritt auf \qty{875}{\degreeCelsius} ist die Oberfläche von \samplefour\ bereits so stark verformt,
dass eine AFM-Aufnahme nicht möglich war.

\begin{figure}
    \centering
    ,\foreach \i in \temperatureVfour{
        \begin{subfigure}[t]{0.40\textwidth}
            \includegraphics[width=\textwidth]
            {../plots/AFM/XG-Vakuum/XG-\i/W6824-1C/W6824-1C_XG_Vakuum_\i_Topography_1}
            \caption{\ifthenelse{\equal{\i}{pre}}{Initialzustand}{\qty{\i}{\degreeCelsius}}}
            \label{W6824-1C_Vakuum_AFM_\i}
        \end{subfigure}
    }
    \caption{Topografieaufnahmen der Vakuum-Ausheizserie von Probe \samplefour.}
    \label{fig:W6824-1C_Vakuum_AFM}
\end{figure}
\newpage

\subsubsection{Luft-Aus\-heiz\-vor\-gang}\label{subsubsec:W6824-1B_Luft}
\begin{figure}
    \centering
    \includegraphics{../plots/XRD/W6824-1D_Luft}
    \caption{$2\theta/\omega$ Diffraktogramme der Luft-Ausheizserie von Probe \samplefour.
    Grau hinterlegt sind Beugungsreflexe des Probenhalters.}
    \label{fig:W6824-1D_Luft_XRD}
\end{figure}
Die Diffraktogramme der Probe \samplefour\ nach festgelegten Ausheizschritten in Luft sind in
\cref{fig:W6824-1D_Luft_XRD} dargestellt.
Auch hier sind alle Peaks auf das Substrat und den Probenhalter zurückzuführen.
Der \heo\ Dünnfilm ist röntgenamorph.

Die AFM-Topografieaufnahmen der Probe \samplefour\ während des Luft-Aus\-heiz\-vor\-gangs sind in
\cref{fig:W6824-1D_Luft_AFM} dargestellt.
Der Initialzustand in \cref{W6824-1D_Luft_AFM_pre} zeigt einen homogenen Untergrund, welcher flächig mit Kristalliten
bedeckt ist.
Die Topographie ist vergleichbar mit der des Initialzusatnds aus der Sauerstoff-Ausheizserie von Probe \samplefour.
Anders als die bisherigen Ausheizschritte auf \qty{600}{\degreeCelsius} in \cref{W6824-1D_Luft_AFM_600} sind die
Kristallite deutlich höher und größer als die des Initialzustands.
Eine vergleichbare Morphologie liefert der Luft-Ausheizschritt auf \qty{700}{\degreeCelsius} von \sampleone,
siehe \cref{fig:W6821-1D_Luft_AFM_700}.
Auch die Aufnahmen von \qtylist{750;800;875}{\degreeCelsius} in
\cref{W6824-1D_Luft_AFM_750,W6824-1D_Luft_AFM_800,W6824-1D_Luft_AFM_875} zeigen eine ähnliche Entwicklung wie
die Luft-Ausheizserie von Probe \sampleone.
Erneut sind viele Löcher zu erkennen, die auf eine Evaporation der Kristallite hindeuten.
Die Überreste der großen Kristallite können ebenfalls in den Aufnahmen erkannt werden.
Bei der Aufnahme von \qty{875}{\degreeCelsius} kann erneut eine Deformierung des Untergrunds beobachtet werden.
\begin{figure}
    \centering
    ,\foreach \i in \temperaturesL{
        \begin{subfigure}[t]{0.40\textwidth}
            \includegraphics[width=\textwidth]
            {../plots/AFM/XG-Luft/XG-\i/W6824-1D/W6824-1D_XG_Luft_\i_Topography_1}
            \caption{\ifthenelse{\equal{\i}{pre}}{Initialzustand}{\qty{\i}{\degreeCelsius}}}
            \label{W6824-1D_Luft_AFM_\i}
        \end{subfigure}
    }
    \caption{Topografieaufnahmen der Luft-Ausheizserie von Probe \samplefour.}
    \label{fig:W6824-1D_Luft_AFM}
\end{figure}
\newpage

\subsection{Analyse der Rauheiten}\label{subsec:Rauheit}
\begin{figure}
    \centering
    \begin{subfigure}{0.48\textwidth}
        \centering
        \import{../plots/AFM}{rauheit_W6823.pgf}
        \caption{\samplethree}
        \label{fig: AFM, Sauerstoff}
    \end{subfigure}
    \begin{subfigure}{0.48\textwidth}
        \centering
        \import{../plots/AFM}{rauheit_W6821.pgf}
        \caption{\sampleone}
        \label{fig: AFM, Sauerstoff}
    \end{subfigure}
    \begin{subfigure}{0.48\textwidth}
        \centering
        \import{../plots/AFM}{rauheit_W6822.pgf}
        \caption{\sampletwo}
        \label{fig: AFM, Sauerstoff}
    \end{subfigure}
    \begin{subfigure}{0.48\textwidth}
        \centering
        \import{../plots/AFM}{rauheit_W6824.pgf}
        \caption{\samplefour}
        \label{fig: AFM, Sauerstoff}
    \end{subfigure}
    \caption{Rauheit der jeweiligen Probe in Abhängigkeit der Ausheiztemperatur bei Sauerstoff- (blau), Vakuum-
        (orange) oder Luftatmosphäre (grün).}
    \label{fig:Rauheit}
\end{figure}
Die Rauheiten der Proben \samplethree, \sampleone, \sampletwo\ und \samplefour\ in Abhängigkeit der Ausheiztemperatur
und -atmosphäre sind in \cref{fig:Rauheit} dargestellt.
Anhand der Temperaturverläufe sind verschiedene Phasen des Aus\-heiz\-vor\-gangs erkennbar.
Der Initialzustand ist im Vergleich zu den nachfolgenden ausgeheizten Zuständen rauer,
da während der Probenherstellung Kristallinseln auf dem Substrat gewachsen sind.
In fast allen Serien sinkt die Rauheit nach dem Ausheizen auf \qty{500}{\degreeCelsius} beziehungsweise
\qty{600}{\degreeCelsius}.
Eine möglichen Ursache für diesen Rückgang könnte die Evaporation der Kristallite sein, die selbst von der
Rauigkeit abhängt \autocite{evaporation} und daher zu Beginn eine entscheidende Rolle spielt.
Bei höheren Temperaturen steigt die Rauheit an, was auf thermisch induzierte Rekristallisationen
\autocite{evaporation} zurückzuführen ist.
Die entstehenden Kristallite sind in den AFM-Aufnahmen zu erkennen.
Beide Mechanismen treten gleichzeitig auf, und da die Verdampfung der Kristallite schneller erfolgt, je höher die
Rauheit ist, könnte sich ein Gleichgewicht einstellen.
Für die meisten Proben liegt die Rauheit bei der Endtemperatur zwischen \qtyrange{7.5}{10}{\nano\meter}.
\newpage

\subsection{GIXRD-Analyse}\label{subsec:GIXRD}
Zur zusätzlichen Charakterisierung der Proben \samplethree, \sampleone\, \sampletwo\ und \samplefour\ wurden
GIXRD-Aufnahmen bei einem festen Winkel von $\omega=\qty{2}{\degree}$ nach dem Ausheizschritt auf
\qty{875}{\degreeCelsius} sowohl für die Sauerstoff-, Vakuum- und Luft-Ausheizserie durchgeführt.
Diese Aufnahmen dienen zur weiteren Charakterisierung der Dünnfilme und sind besonders sensitiv
gegenüber dünnen Schichten.

Exemplarisch werden in diesem Abschnitt die GIXRD-Aufnahmen der Probe \sampleone\ in \cref{fig:GIXRD_W6821} gezeigt.
Die restlichen Aufnahmen sind im Anhang zu finden, und weisen fast identische Ergebnisse auf.
Zusätzlich zu den anderen Aufnahmen zeigt die GIXRD-Aufnahme von Probe \sampleone\ in \cref{fig:GIXRD_W6821} den
Probenhalter.
Erkennbar ist ein breites Maximum bei einem Winkel von \qty{41.6}{\degree}, welches in allen Aufnahmen auftritt.
Dieses Maximum ist auf die Kristallstruktur des Probenhalters zurückzuführen.
Weiterhin ist ein schwacher Beugungsreflex bei einem Winkel von \qty{43}{\degree} zu erkennen.
Dieser ist jedoch nicht signifikant,
da keine weiteren Beugungsreflexe zu erkennen sind, ist der \heo\ Dünnfilm auch in den GIXRD-Aufnahmen röntgenamorph.
\begin{figure}
    \centering
    \import{../plots/XRD/}{gixrd_sampl.pgf}
    \caption{GIXRD-Aufnahme bei einem festen Winkel $\omega=\qty{2}{\degree}$ von Probe \sampleone\ bei
    \qty{875}{\degreeCelsius}.}
    \label{fig:GIXRD_W6821}
\end{figure}

