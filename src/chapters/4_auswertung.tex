\section{Auswertung}\label{sec:auswertung}

\subsection{Temperaturkalibrierung der A-Kammer}\label{subsec:temperaturkalibrierung}
\begin{figure}
    \centering
    \foreach \i/\desc in {
        furnace_calibration_1.pgf/{Kalibrierung des Lithografiesensors, Bild 1},
        furnace_calibration_2.pgf/{Kalibrierung des Lithografiesensors, Bild 2},
        a_chamber_calibration.pgf/{Kalibrierung der A-Kammer},
        final_calibration.pgf/{Abhängigkeit zwischen $T_{\mathrm{Pyro}}$ und $T_{\mathrm{Lit}}$, Bild 1},
        quenching_time.pgf/{Abschreckzeit der A-Kammer}
    }{
        \begin{subfigure}[t]{0.49\textwidth}
            \import{../plots/calibration}{\i}
            \caption{\desc}
            \label{fig:\i}
        \end{subfigure}
    }
    \caption{Abhängigkeiten zur Temperaturbestimmung der A-Kammer}
    \label{fig:temperature_calibration_1}
\end{figure}

Zentrales Thema der vorliegenden Arbeit ist das Ausheizen der \heo-Dünnfilme.
Dabei ist es wichtig, einen entsprechend großen Temperaturbereich von \qtyrange{300}{1000}{\degreeCelsius} zu erreichen
und die Temperatur des Dünnfilms möglichst genau zu bestimmen.
Ein geeigneter Kandidat für diesen Prozess ist die A-Kammer, die mithilfe eines Heizlasers die Rückseite des
Substrathalters erhitzt.
Um die Temperatur des Dünnfilms zu bestimmen, wird ein Pyrometer verwendet, welches die Rückseite des Substrathalters
misst.
Mithilfe eines PID-Reglers wird die Temperatur auf den eingestellten Wert geregelt.

Nun stellt sich die Frage, wie genau die Temperatur des Dünnfilms bestimmt werden kann.
Dazu wurde von Tim Düvel ein Temperatursensor durch Lithografie auf ein C-Saphir Substrat fabriziert.
Dieser wird im folgenden als Lithografiesensor bezeichnet.

In Zusammenarbeit mit Tim Düvel wurde eine Kalibrierung des Lithografiesensors durchgeführt.
Dazu wurde ein PT1000 Temperatursensor mithilfe von Wärmeleitpaste thermisch mit dem Lithografiesensor verbunden und
im Muffelofen ausgeheizt.
Beide Widerstände wurden in Abhängigkeit der Temperatur gemessen und sind in Abbildung \ref{fig:furnace_calibration_1}
dargestellt.
Da jedem Zeitpunkt zwei Widerstände zugeordnet werden können, kann durch diese Parametrisierung auch die Abhängigkeit
beider Widerstände voneinander bestimmt werden, siehe \ref{fig:furnace_calibration_2}.
Wie erwartet ist die Abhängigkeit $R_\mathrm{Lit}(R_\mathrm{Pt})$ linear und kann durch
einen linearen Fit beschrieben werden.

Im nächsten Schritt wird der Lithografiesensor in den Probenhalter der A-Kammer eingebaut und eine Temperaturserie
aufgenommen.
Dabei wird die Temperatur des Heizlasers eingestellt und nach einer festen Zeitspanne Pyrometertemperatur
$T_\mathrm{Pyro}$ und Lithografiewiderstand $R_\mathrm{Lit}$ gemessen.
Auch hier ist eine lineare Abhängigkeit zwischen beiden Größen erkennbar und kann durch einen linearen Fit
beschrieben werden.

Die Temperatur des Dünnfilms $T_\mathrm{Real}$ kann nun durch folgende Gesamtfunktion bestimmt werden:
\begin{equation}
    T_{\mathrm{Real}}=\underbrace{ T_{\mathrm{Real}}(R_{\mathrm{Pt}}) }_{ \substack{\text{quadratic} \\ \text{dependency}} }
    =T_{\mathrm{Real}}(\underbrace{ R_{\mathrm{Pt}}(R_{\mathrm{Lit}}) }_{  \substack{\text{linear} \\ \text{dependency}}  })
    =T_{\mathrm{Real}}(R_{\mathrm{Pt}}(\underbrace{ R_{\mathrm{Lit}}(T_{\mathrm{Pyro}}) }_{    \substack{\text{linear} \\
    \text{dependency}}  }))
    \label{eq:temperature_calibration}
\end{equation}
Da alle Funktionen bekannt sind, ergibt sich folgender Zusammenhang zwischen $T_{\mathrm{Pyro}}$ und $T_{\mathrm{Real}}$.
Dieser ist in Abbildung \ref{fig:final_calibration} dargestellt.

Wichtig für das Ausheizen ist außerdem die Abkühlzeit der A-Kammer.
Dazu wurde eine Zeitserie aufgenommen, die den Lithografiewiderstand in Abhängigkeit der Zeit zeigt, siehe
Abbildung \ref{fig:quenching_time}.
Für einen Temperaurabfall von circa \qty{350}{\degreeCelsius} auf circa \qty{25}{\degreeCelsius}
benötigt die A-Kammer etwa \qty{4}{\minute}.

\subsection{Probenherstellung}\label{subsec:probenherstellung}
- jeweils 20000 Laserpule
- mit Frequenz 20 Hz
- Energie von 650 mJ
- ohne Heating power
- QM 17
- Exzentrität 2

Unterschiede
\begin{table}[h]
    \centering
    \begin{tabular}{c c c c}
        \toprule
        Probenname & Abscheidedruck in \unit{\milli \bar} & Dicke des Dünnfilms in \unit{\nano\meter} \\
        \midrule
        W6821-1    & 0.01                                 & \num{95(7)}                               \\
        W6822-1    & 0.001                                & \num{160(12)}                             \\
        W6823-1    & 0.1                                  & \num{65(4)}                               \\
        W6824-1    & 0.0005                               & \num{135(16)}                             \\
        \bottomrule
    \end{tabular}
    \caption{Klassifikation der verschiedenen Kristallsysteme. \imcite[65]{Hunklinger} }
    \label{tab:samples}
\end{table}

\subsection{EDX Analyse}\label{subsec:edx-analyse}a
\begin{figure}
    \centering
    \foreach \i/\desc in {map/Oberfläche, Mg/Magnesium, Co/Kobalt, Ni/Nickel), Cu/Kupfer, Zn/Zink}{
        \begin{subfigure}[t]{0.40\textwidth}
            \includegraphics[width=\textwidth]{../plots/EDX/W6821-3D/\i}
            \caption{\desc}
        \end{subfigure}
    }
    \caption{EDX Aufnahmen der Probe W6821-3D}
    \label{fig:edx1}
\end{figure}

% AUSHEIZVORGÄNGE %
\newcommand{\temperaturesS}{pre,600,700,750,800,875}
\newcommand{\temperaturesV}{pre,500,600,700,750, 800}
\newcommand{\temperaturesVV}{pre,500,600,700}
\newcommand{\temperaturesL}{pre,600}

% PROBE W6821-1 %

\newpage

\subsection{Probe W6821-1}\label{subsec:probe-W6821-1}

\subsubsection{Sauerstoff Ausheizvorgang}\label{subsec:sauerstoff-ausheizvorgang-1}
\begin{figure}
    \centering
    \import{../plots/XRD}{W6821-1B_Sauerstoff.pgf}
    \caption{W6821-1B, Sauerstoff, XRD}
    \label{fig:W6821-1B, Sauerstoff, XRD}
\end{figure}
\begin{figure}
    \centering
    \foreach \i in \temperaturesS{
        \begin{subfigure}[t]{0.40\textwidth}
            \includegraphics[width=\textwidth]
            {../plots/AFM/XG-Sauerstoff/XG-\i/W6821-1B/W6821-1B_XG_Sauerstoff_\i_Topography_1}
            \caption{\ifthenelse{\equal{\i}{pre}}{Initialzustand}{\qty{\i}{\degreeCelsius}}}
            \label{W6821-1B, Sauerstoff, AFM, \i}
        \end{subfigure}
    }
    \caption{W6821-1B, Sauerstoff, AFM}
    \label{fig:W6821-1B, Sauerstoff, AFM}
\end{figure}
\newpage

\subsubsection{Vakuum Ausheizvorgang}\label{subsec:vakuum-ausheizvorgang-1}
\begin{figure}
    \centering
    \import{../plots/XRD}{W6821-1C_Vakuum.pgf}
    \caption{W6821-1C, Vakuum, XRD}
    \label{fig:W6821-1C, Vakuum, XRD}
\end{figure}
\begin{figure}
    \centering
    ,\foreach \i in \temperaturesV{
        \begin{subfigure}[t]{0.40\textwidth}
            \includegraphics[width=\textwidth]
            {../plots/AFM/XG-Vakuum/XG-\i/W6821-1C/W6821-1C_XG_Vakuum_\i_Topography_1}
            \caption{\ifthenelse{\equal{\i}{pre}}{Initialzustand}{\qty{\i}{\degreeCelsius}}}
            \label{W6821-1C, Vakuum, AFM, \i}
        \end{subfigure}
    }
    \caption{AFM, Vakuum, W6821-1C}
    \label{fig: AFM, Vakuum, W6821-1C}
\end{figure}
\newpage

\subsubsection{Luft Ausheizvorgang}\label{subsec:luft-ausheizvorgang-1}
\begin{figure}
    \centering
    \import{../plots/XRD}{W6821-1D_Luft.pgf}
    \caption{W6821-1D, Luft, XRD}
    \label{fig:W6821-1D, Luft, XRD}
\end{figure}
\begin{figure}
    \centering
    ,\foreach \i in \temperaturesL{
        \begin{subfigure}[t]{0.40\textwidth}
            \includegraphics[width=\textwidth]
            {../plots/AFM/XG-Luft/XG-\i/W6821-1D/W6821-1D_XG_Luft_\i_Topography_1}
            \caption{\ifthenelse{\equal{\i}{pre}}{Initialzustand}{\qty{\i}{\degreeCelsius}}}
            \label{W6821-1D, Luft, AFM, \i}
        \end{subfigure}
    }
    \caption{AFM, Luft, W6821-1D}
    \label{fig: AFM, Luft, W6821-1D}
\end{figure}
\newpage

% PROBE W6822-1 %

\newpage

\subsection{Probe W6822-1}\label{subsec:probe-W6822-1}

\subsubsection{Sauerstoff Ausheizvorgang}\label{subsec:sauerstoff-ausheizvorgang-1}
\begin{figure}
    \centering
    \import{../plots/XRD}{W6822-1B_Sauerstoff.pgf}
    \caption{W6822-1B, Sauerstoff, XRD}
    \label{fig:W6822-1B, Sauerstoff, XRD}
\end{figure}
\begin{figure}
    \centering
    \foreach \i in \temperaturesS{
        \begin{subfigure}[t]{0.40\textwidth}
            \includegraphics[width=\textwidth]
            {../plots/AFM/XG-Sauerstoff/XG-\i/W6822-1B/W6822-1B_XG_Sauerstoff_\i_Topography_1}
            \caption{\ifthenelse{\equal{\i}{pre}}{Initialzustand}{\qty{\i}{\degreeCelsius}}}
            \label{W6822-1B, Sauerstoff, AFM, \i}
        \end{subfigure}
    }
    \caption{W6822-1B, Sauerstoff, AFM}
    \label{fig:W6822-1B, Sauerstoff, AFM}
\end{figure}
\newpage

\subsubsection{Vakuum Ausheizvorgang}\label{subsec:vakuum-ausheizvorgang-1}
\begin{figure}
    \centering
    \import{../plots/XRD}{W6822-1C_Vakuum.pgf}
    \caption{W6822-1C, Vakuum, XRD}
    \label{fig:W6822-1C, Vakuum, XRD}
\end{figure}
\begin{figure}
    \centering
    ,\foreach \i in \temperaturesV{
        \begin{subfigure}[t]{0.40\textwidth}
            \includegraphics[width=\textwidth]
            {../plots/AFM/XG-Vakuum/XG-\i/W6822-1C/W6822-1C_XG_Vakuum_\i_Topography_1}
            \caption{\ifthenelse{\equal{\i}{pre}}{Initialzustand}{\qty{\i}{\degreeCelsius}}}
            \label{W6822-1C, Vakuum, AFM, \i}
        \end{subfigure}
    }
    \caption{AFM, Vakuum, W6822-1C}
    \label{fig: AFM, Vakuum, W6822-1C}
\end{figure}
\newpage

\subsubsection{Luft Ausheizvorgang}\label{subsec:luft-ausheizvorgang-1}
\begin{figure}
    \centering
    \import{../plots/XRD}{W6822-1D_Luft.pgf}
    \caption{W6822-1D, Luft, XRD}
    \label{fig:W6822-1D, Luft, XRD}
\end{figure}
\begin{figure}
    \centering
    ,\foreach \i in \temperaturesL{
        \begin{subfigure}[t]{0.40\textwidth}
            \includegraphics[width=\textwidth]
            {../plots/AFM/XG-Luft/XG-\i/W6822-1D/W6822-1D_XG_Luft_\i_Topography_1}
            \caption{\ifthenelse{\equal{\i}{pre}}{Initialzustand}{\qty{\i}{\degreeCelsius}}}
            \label{W6822-1D, Luft, AFM, \i}
        \end{subfigure}
    }
    \caption{AFM, Luft, W6822-1D}
    \label{fig: AFM, Luft, W6822-1D}
\end{figure}
\newpage

% PROBE W6823-1 %

\newpage

\subsection{Probe W6823-1}\label{subsec:probe-W6823-1}

\subsubsection{Sauerstoff Ausheizvorgang}\label{subsec:sauerstoff-ausheizvorgang-1}
\begin{figure}
    \centering
    \import{../plots/XRD}{W6823-1B_Sauerstoff.pgf}
    \caption{W6823-1B, Sauerstoff, XRD}
    \label{fig:W6823-1B, Sauerstoff, XRD}
\end{figure}
\begin{figure}
    \centering
    \foreach \i in \temperaturesS{
        \begin{subfigure}[t]{0.40\textwidth}
            \includegraphics[width=\textwidth]
            {../plots/AFM/XG-Sauerstoff/XG-\i/W6823-1B/W6823-1B_XG_Sauerstoff_\i_Topography_1}
            \caption{\ifthenelse{\equal{\i}{pre}}{Initialzustand}{\qty{\i}{\degreeCelsius}}}
            \label{W6823-1B, Sauerstoff, AFM, \i}
        \end{subfigure}
    }
    \caption{W6823-1B, Sauerstoff, AFM}
    \label{fig:W6823-1B, Sauerstoff, AFM}
\end{figure}
\newpage

\subsubsection{Vakuum Ausheizvorgang}\label{subsec:vakuum-ausheizvorgang-1}
\begin{figure}
    \centering
    \import{../plots/XRD}{W6823-1C_Vakuum.pgf}
    \caption{W6823-1C, Vakuum, XRD}
    \label{fig:W6823-1C, Vakuum, XRD}
\end{figure}
\begin{figure}
    \centering
    ,\foreach \i in \temperaturesVV{
        \begin{subfigure}[t]{0.40\textwidth}
            \includegraphics[width=\textwidth]
            {../plots/AFM/XG-Vakuum/XG-\i/W6823-1C/W6823-1C_XG_Vakuum_\i_Topography_1}
            \caption{\ifthenelse{\equal{\i}{pre}}{Initialzustand}{\qty{\i}{\degreeCelsius}}}
            \label{W6823-1C, Vakuum, AFM, \i}
        \end{subfigure}
    }
    \caption{AFM, Vakuum, W6823-1C}
    \label{fig: AFM, Vakuum, W6823-1C}
\end{figure}
\newpage

\subsubsection{Luft Ausheizvorgang}\label{subsec:luft-ausheizvorgang-1}
\begin{figure}
    \centering
    \import{../plots/XRD}{W6823-1D_Luft.pgf}
    \caption{W6823-1D, Luft, XRD}
    \label{fig:W6823-1D, Luft, XRD}
\end{figure}
\begin{figure}
    \centering
    ,\foreach \i in \temperaturesL{
        \begin{subfigure}[t]{0.40\textwidth}
            \includegraphics[width=\textwidth]
            {../plots/AFM/XG-Luft/XG-\i/W6823-1D/W6823-1D_XG_Luft_\i_Topography_1}
            \caption{\ifthenelse{\equal{\i}{pre}}{Initialzustand}{\qty{\i}{\degreeCelsius}}}
            \label{W6823-1D, Luft, AFM, \i}
        \end{subfigure}
    }
    \caption{AFM, Luft, W6823-1D}
    \label{fig: AFM, Luft, W6823-1D}
\end{figure}
\newpage

% PROBE W6824-1 %

\newpage

\subsection{Probe W6824-1}\label{subsec:probe-W6824-1}

\subsubsection{Sauerstoff Ausheizvorgang}\label{subsec:sauerstoff-ausheizvorgang-1}
\begin{figure}
    \centering
    \import{../plots/XRD}{W6824-1B_Sauerstoff.pgf}
    \caption{W6824-1B, Sauerstoff, XRD}
    \label{fig:W6824-1B, Sauerstoff, XRD}
\end{figure}
\begin{figure}
    \centering
    \foreach \i in \temperaturesS{
        \begin{subfigure}[t]{0.40\textwidth}
            \includegraphics[width=\textwidth]
            {../plots/AFM/XG-Sauerstoff/XG-\i/W6824-1B/W6824-1B_XG_Sauerstoff_\i_Topography_1}
            \caption{\ifthenelse{\equal{\i}{pre}}{Initialzustand}{\qty{\i}{\degreeCelsius}}}
            \label{W6824-1B, Sauerstoff, AFM, \i}
        \end{subfigure}
    }
    \caption{W6824-1B, Sauerstoff, AFM}
    \label{fig:W6824-1B, Sauerstoff, AFM}
\end{figure}
\newpage

\subsubsection{Vakuum Ausheizvorgang}\label{subsec:vakuum-ausheizvorgang-1}
\begin{figure}
    \centering
    \import{../plots/XRD}{W6824-1C_Vakuum.pgf}
    \caption{W6824-1C, Vakuum, XRD}
    \label{fig:W6824-1C, Vakuum, XRD}
\end{figure}
\begin{figure}
    \centering
    ,\foreach \i in \temperaturesV{
        \begin{subfigure}[t]{0.40\textwidth}
            \includegraphics[width=\textwidth]
            {../plots/AFM/XG-Vakuum/XG-\i/W6824-1C/W6824-1C_XG_Vakuum_\i_Topography_1}
            \caption{\ifthenelse{\equal{\i}{pre}}{Initialzustand}{\qty{\i}{\degreeCelsius}}}
            \label{W6824-1C, Vakuum, AFM, \i}
        \end{subfigure}
    }
    \caption{AFM, Vakuum, W6824-1C}
    \label{fig: AFM, Vakuum, W6824-1C}
\end{figure}
\newpage

\subsubsection{Luft Ausheizvorgang}\label{subsec:luft-ausheizvorgang-1}
\begin{figure}
    \centering
    \import{../plots/XRD}{W6824-1D_Luft.pgf}
    \caption{W6824-1D, Luft, XRD}
    \label{fig:W6824-1D, Luft, XRD}
\end{figure}
\begin{figure}
    \centering
    ,\foreach \i in \temperaturesL{
        \begin{subfigure}[t]{0.40\textwidth}
            \includegraphics[width=\textwidth]
            {../plots/AFM/XG-Luft/XG-\i/W6824-1D/W6824-1D_XG_Luft_\i_Topography_1}
            \caption{\ifthenelse{\equal{\i}{pre}}{Initialzustand}{\qty{\i}{\degreeCelsius}}}
            \label{W6824-1D, Luft, AFM, \i}
        \end{subfigure}
    }
    \caption{AFM, Luft, W6824-1D}
    \label{fig: AFM, Luft, W6824-1D}
\end{figure}
\newpage

\subsection{Rauigkeiten}\label{subsec:rauigkeit}
\begin{figure}
    \centering
    \import{../plots/AFM}{sauerstoff.pgf}
    \caption{AFM, Sauerstoff}
    \label{fig: AFM, Sauerstoff}
\end{figure}

\begin{figure}
    \centering
    \import{../plots/AFM}{vakuum.pgf}
    \caption{AFM, Vakuum}
    \label{fig: AFM, Vakuum}
\end{figure}
