\section{Auswertung}\label{sec:auswertung}
Zentrales Thema dieser Arbeit ist das Ausheizen von \heo Dünnfilmen.
Es wurden vier Proben in vier Prozessen einer PLD-Anlage mit Eagle XG Glassubstraten und dem in
\cref{subsec:probenherstellung} beschriebenen Target hergestellt.
Für jeden Prozess wurden jeweils \num{20000} Laserpulse mit einer Frequenz von \qty{20}{\hertz} und einer Energie von
\qty{650}{\milli\joule} abgegeben.
Dabei wurde der eingebaute Widerstandsheizer nicht verwendet.
Der variable Parameter war der Abscheidedruck der Sauerstoffatmosphäre.
Nach der Herstellung wurden die Proben im Profilometer und im Röntgendiffraktometer bezüglich ihrer Schichtdicke
vermessen.
Der jeweilige Druck und die Schichtdicken sind in \cref{tab:samples} aufgeführt.
\begin{table}[h]
    \centering
    \begin{tabular}{l l l l l}
        \toprule
        Probenname & \makecell[l]{Abscheidedruck \\ in \unit{\milli \bar}} & \makecell[l]{Dicke in \unit{\nano\meter} \\
        Profilometermessung} & \makecell[l]{Dicke in \unit{\nano\meter}     \\ XRR Messung}   \\
        \midrule
        \samplethree   & 0.1   & \num{65(4)} & 109 \\
        \sampleone  & 0.01 & \num{95(7)} & - \\
        \sampletwo  & 0.001 & \num{160(12)} & 142 \\
        \samplefour  & 0.00005 & \num{135(16)} & 120 \\
        \bottomrule
    \end{tabular}
    \caption{Eigenschaften der zu untersuchenden Dünnfilme}
    \label{tab:samples}
\end{table}

Im Anschluss an die Probenherstellung wurden die Proben gevierteilt, damit Ausheizstudien vergleichbar mit
unterschiedlichen Atmosphären durchgeführt werden konnten.
Dazu wurde der Dünnfilm der jeweiligen Probe mit Photolack beschichtet, um die Oberfläche vor
mechanischen Beschädigungen beim Zersägen zu schützen.
Anschließend wurde das Substrat mit der Dünnfilmseite nach oben auf eine Glasplatte durch Wachs geklebt und mit einer
Diamantsäge zerteilt.
Anschließend wurden Wachs und Photolack mit den Lösungsmitteln NMP, Wachsentferner und Ethanol entfernt.
Es ist zu beachten, dass diese Prozedur durch die Lösungsmittel die Zusammensetzung und Oberfläche des Dünnfilms
verändern kann.
Optisch wurden keine Veränderungen festgestellt.

Nach der Präparation wurden die Proben auf unterschiedliche Weisen ausgeheizt.
Für die Sauerstoffserie wurde eine Vakuumkammer mit einer Sauerstoffatmosphäre von circa \qty{800}{\milli\bar}
verwendet.
Die Proben \samplethree, \sampleone, \sampletwo, \samplefour\ wurden einem Zyklus von Ausheizen und anschließendem
Messen unterzogen.
Für das Ausheizen wurden die Temperaturen \qtylist{600;700;750;800;875}{\degreeCelsius} gewählt.
Die Proben wurden für eine Stunde bei der jeweiligen Temperatur ausgeheizt, vom Substrathalter entfernt und auf einer
Glasplatte abgekühlt.
Dadurch wird ein schnelles Abkühlen der Proben erreicht.
Als Anfangstemperatur wurde \qty{600}{\degreeCelsius} gewählt, da nach der Untersuchung von Rost
sich mehrere Phasen des \heo\ Dünnfilms bei dieser Temperatur bilden.
Die Endtemperatur wurde auf \qty{875}{\degreeCelsius} gewählt, da dies die maximale Temperatur des Heizlasers ist.

Für den Vakuumausheizvorgang wurde die gleiche Vakuumkammer wie für die Sauerstoffserie verwendet.
Auch hier wurden die Proben \samplethree, \sampleone, \sampletwo, \samplefour\ einem Zyklus von Ausheizen und
anschließendem Messen unterzogen.
Die Kammer wurde auf einen Druck von \qty{4e-4}{\milli\bar} evakuiert.
Da die Dünnfilme auf Eagle XG Glassubstraten gewachsen sind, ist deren Viskosität bei hohen
Temperaturen gering.
Das führt zu Verformungen der gesamten Probe bei \qty{875}{\degreeCelsius}.
Weitere Messungen bei höheren Temeperaturen waren aus diesem Grund nicht möglich.

Für den Ausheizvorgang unter Luftatmosphäre wurde der Muffelofen verwendet.
Die Proben \samplethree, \sampleone, \sampletwo, \samplefour\ wurden einem Zyklus von Ausheizen und anschließendem
Messen unterzogen.
FÜr das Ausheizen wurden die Temperaturen \qtylist{600;700;750;800;875}{\degreeCelsius} gewählt.
Die Proben wurden in einer Keramikschale für eine Stunde bei der jeweiligen Temperatur ausgeheizt anschließend
aus dem Ofen und der Schale genommen und auf einer Glasplatte abgekühlt.
Anders als bei den vorherigen Ausheizvorgängen wurde hier die Probe mit einer langsamen Ausheizrate von
\qty{300}{\degreeCelsius\per\hour} auf die Endtemperatur gebracht.


\subsection{Temperaturkalibrierung der A-Kammer}\label{subsec:temperaturkalibrierung}
\begin{figure}
    \centering
    \foreach \i/\desc in {
        furnace_calibration_1.pgf/{Kalibrierung des Lithografiesensors, Bild 1},
        furnace_calibration_2.pgf/{Kalibrierung des Lithografiesensors, Bild 2},
        a_chamber_calibration.pgf/{Kalibrierung der A-Kammer},
        final_calibration.pgf/{Abhängigkeit zwischen $T_{\mathrm{Pyro}}$ und $T_{\mathrm{Lit}}$},
        quenching_time.pgf/{Abkühlzeit der A-Kammer}
    }{
        \begin{subfigure}[t]{0.49\textwidth}
            \import{../plots/calibration}{\i}
            \caption{\desc}
            \label{fig:\i}
        \end{subfigure}
    }
    \caption{Abhängigkeiten zur Temperaturbestimmung der A-Kammer}
    \label{fig:temperature_calibration_1}
\end{figure}

Zentrales Thema der vorliegenden Arbeit ist das Ausheizen der \heo\ Dünnfilme.
Dabei ist es wichtig, einen entsprechend großen Temperaturbereich von \qty{300}{\degreeCelsius}
bis \qty{1000}{\degreeCelsius} zu erreichen und die Temperatur des Dünnfilms möglichst genau zu bestimmen.
Ein geeigneter Kandidat für diesen Prozess ist die A-Kammer, die mithilfe eines Heizlasers die Rückseite des
Substrathalters erhitzt.
Um die Temperatur des Dünnfilms zu bestimmen, wird ein Pyrometer verwendet, welches die Rückseite des Substrathalters
misst.
Mithilfe eines PID-Reglers wird die Temperatur auf den eingestellten Wert geregelt.

Nun stellt sich die Frage, wie genau die Temperatur des Dünnfilms bestimmt werden kann.
Dazu wurde von Tim Düvel ein spezieller Temperatursensor fabriziert.
Dieser besteht aus einem C-Saphir Substrat, auf dem ein Muster für die Platinbahnen infolge eines
Lithografieprozesses aufgebracht wurde.
Das Substrat wurde mit einer dünnen Schicht Platinoxid und einer Schicht Platin bedampft.
An zwei Enden wurde Gold aufgedampft, um die Kontakte herzustellen.
Für den Schutz wurde abschließend eine Schicht Aluminiumoxid aufgebracht.
Dieser Sensor wird im Folgenden als Lithografiesensor bezeichnet.

In Zusammenarbeit mit Tim Düvel wurde eine Kalibrierung des Lithografiesensors durchgeführt.
Dazu wurde ein PT1000 Temperatursensor genutzt, für welchen es eine bekannte quadratische Abhängigkeit zwischen Widerstand
$R_{\mathrm{Pt}}$ und Temperatur $T_{\mathrm{Pt}}$ gibt:
\begin{equation}
    R_{\mathrm{Pt}}(T_{\mathrm{Pt}})
    =R_0 \cdot (1 + A \cdot R_{\mathrm{Pt}} + B \cdot R_{\mathrm{Pt}}^2).
    \label{eq:pt1000_calibration}
\end{equation}
Hierbei ist $R_0 = \qty{1000}{\ohm}$ der Widerstand bei \qty{0}{\degreeCelsius},
$A = \qty{3.9083e-3}{\degreeCelsius^{-1}}$ und $B = \qty{-5.775e-7}{\degreeCelsius^{-2}}$.
Der PT1000 Temperatursensor wurde mithilfe von Wärmeleitpaste thermisch an den Lithografiesensor gekoppelt und
im Muffelofen ausgeheizt.
Beide Widerstände wurden in Abhängigkeit der Temperatur gemessen und sind in Abbildung
\cref{fig:furnace_calibration_1.pgf} dargestellt.
Da jedem Zeitpunkt zwei Widerstände zugeordnet werden können, kann durch diese Parametrisierung auch die Abhängigkeit
beider Widerstände voneinander bestimmt werden, siehe \cref{fig:furnace_calibration_2.pgf}.
Aufgrund der gleichen Materialbeschaffenheit ist die Abhängigkeit $R_\mathrm{Lit}(R_\mathrm{Pt})$ linear und kann durch
einen linearen Fit beschrieben werden, siehe \cref{fig:a_chamber_calibration.pgf}.

Im nächsten Schritt wird der Lithografiesensor in den Probenhalter der A-Kammer eingebaut und eine Temperaturserie
aufgenommen.
Dabei wird die Temperatur des Heizlasers eingestellt und nach einer festen Zeitspanne Pyrometertemperatur
$T_\mathrm{Pyro}$ und Lithografiewiderstand $R_\mathrm{Lit}$ gemessen.
Auch hier zeigt sich eine lineare Abhängigkeit zwischen beiden Größen welche durch einen linearen Fit
beschrieben werden.

Als Referenztemperatur, und damit als Temperatur des Dünnfilms, wird die Temperatur des PT1000 Sensors angesehen.
Diese kann durch folgende Gesamtfunktion bestimmt werden:
\begin{equation}
    T_{\mathrm{Pt}}=\underbrace{ T_{\mathrm{Pt}}(R_{\mathrm{Pt}}) }_{
        \substack{\text{quadratische} \\ \text{Abhängigkeit}}}
    =T_{\mathrm{Pt}}(\underbrace{ R_{\mathrm{Pt}}(R_{\mathrm{Lit}}) }_{
        \substack{\text{lineare} \\ \text{Abhängigkeit}}  })
    =T_{\mathrm{Pt}}(R_{\mathrm{Pt}}(\underbrace{ R_{\mathrm{Lit}}(T_{\mathrm{Pyro}}) }_{
        \substack{\text{lineare} \\ \text{Abhängigkeit}}  }))
    \label{eq:temperature_calibration}
\end{equation}
Da alle Funktionen bekannt sind, zeigt sich folgender Zusammenhang zwischen $T_{\mathrm{Pyro}}$ und $T_{\mathrm{Pt}}$
in Abbildung \cref{fig:final_calibration.pgf}.
Der Graph zeigt, dass die Temperatur des Dünnfilms mit einer Unsicherheit von circa \qtyrange{10}{15}{\degreeCelsius}
der Temperatur des Pyrometers entspricht.
Es ist zu beachten, dass dieser Sensor bestmöglich die Thermodynamik von C-Saphir Substraten erfasst.
Da in dieser Arbeit mit EagleXG-Substraten gearbeitet wird, welche andere Kompositionen und Eigenschaften aufweisen,
kann das Ergebnis nur als Approximation betrachtet werden.
Hier sei vor allem auf die unterschiedlichen Wärmeleitfähigkeiten hingewiesen.

Wichtig für das Ausheizen ist außerdem die Abkühlzeit der A-Kammer.
Um diese zu untersuchen, wurde eine Zeitserie des Lithografiewiderstands aufgenommen,
siehe \cref{fig:quenching_time.pgf}.
Für einen Temperaturabfall von circa \qty{350}{\degreeCelsius} auf circa \qty{25}{\degreeCelsius}
benötigt die A-Kammer etwa \qty{4}{\minute}.
Damit zeigen Temperaturbereich, Präzision des Pyrometers und Abkühlzeit, dass
die A-Kammer ein geeigneter Kandidat für das Ausheizen der \heo\ Dünnfilme ist.

\subsection{EDX Analyse}\label{subsec:edx-analyse}a
\begin{figure}
    \centering
    \foreach \i/\desc in {map/Oberfläche, Mg/Magnesium, Co/Kobalt, Ni/Nickel), Cu/Kupfer, Zn/Zink}{
        \begin{subfigure}[t]{0.40\textwidth}
            \includegraphics[width=\textwidth]{../plots/EDX/W6821-3D/\i}
            \caption{\desc}
        \end{subfigure}
    }
    \caption{EDX Aufnahmen der Probe W6821-3D}
    \label{fig:edx1}
\end{figure}

% AUSHEIZVORGÄNGE %
\newcommand{\temperaturesS}{pre,600,700,750,800,875}
\newcommand{\temperaturesV}{pre,500,600,700,750, 800}
\newcommand{\temperaturesVV}{pre,500,600,700}
\newcommand{\temperaturesL}{pre,600, 700, 750, 800}

% PROBE W6823-1 %

\newpage

\subsection{Probe \samplethree}\label{subsec:probe-W6823-1}

\subsubsection{Sauerstoff Ausheizvorgang}\label{subsubsec:W6823-1B_Sauerstoff}
\begin{figure}
    \centering
    \import{../plots/XRD}{W6823-1B_Sauerstoff.pgf}
    \caption{W6823-1B, Sauerstoff, XRD}
    \label{fig:W6823-1B_Sauerstoff_XRD}
\end{figure}
\cref{fig:W6823-1B_Sauerstoff_XRD} zeigt $2\theta/\omega$-Scans der Probe \samplethree zu
verschiedenen Temperaturen während des Sauerstoff Ausheizvorgangs.
Bei allen Temperaturen ist ein ausgeweitetes Maximum bei circa \qty{23}{\degree} zu erkennen.
Dieses ist auf das Eagle XG Glassubstrat zurückzuführen.
Weiterhin ist ein scharf definierter Peak bei circa \qty{45}{\degreeCelsius} zu erkennen, welcher durch die
Kristallstruktur des Probenhalters verursacht wird.
Der Probenhalter sorgt für weitere, kaum sichtbare Peaks bei TODO PEAKS RAUSSUCHEN.
Somit sind trotz der bis zu einer Temperatur von \qty{875}{\degreeCelsius} eingestellten Ausheizprozesse keine Peaks
des \heo\ Dünnfilms zu erkennen.
Dieser ist somit röntgenamorph.

Die AFM-Topografieaufnahmen zu verschiedenen Temperaturen der Probe W6823-1 während des Sauerstoff Ausheizvorgangs sind
in \cref{fig:W6823-1B_Sauerstoff_AFM} dargestellt.
Die Aufnahme des Initialzustands, \cref{W6823-1B_Sauerstoff_AFM_pre}, zeigt einen ebenen Untergrund mit zahlreichen
zufällig orientierten Erhebungen.
Diese Erhebungen sind einzelne Kristallite des Dünnfilms.
Die RMS Rauigkeit des Eagle XG Glassubstrats vor der Zersägung beträgt \qty{383.1}{\pico\meter}.
Die Rauigkeit nach der Zersägung beträgt \qty{470}{\pico\meter}, während die Rauigkeit des ebenen Untergrunds des
Dünnfilms circa \qty{2.3}{\nano\meter} entspricht.
Die durchschnittliche Höhe der großen Kristallite beträgt \qtyrange{50}{60}{\nano\meter}
Nach Angaben der XRR Messungen beträgt die Dicke des Films cira \qty{109}{\nano\meter}.
Das schließt darauf, dass der ebene Untergrund ein flächig gewachsener Dünnfilm ist.
Die Erhebungen sind somit einzelne Kristallite des Dünnfilms.
Ein passendes Modell für die Kristallite ist das Stranski-Krastanov-Wachstum, welches
zuerst Schichtwachstum und nach einer kritischen Schichtdicke Inselwachstum beschreibt.

Nach dem ersten Ausheizschritt bei \qty{600}{\degreeCelsius}, siehe \cref{W6823-1B_Sauerstoff_AFM_600}, ist ein
deutlicher Unterschied in der Morphologie des Dünnfilms zu erkennen.
Die Kristallite sind in ihrer Fläche gleich geblieben, jedoch sind sie deutlich niedriger als im Initialzustand.
Das Maximum der Höhenskala hat sich von \qty{68}{\nano\meter} auf \qty{30.7}{\nano\meter} verringert.
Durch die erhöhte Temperatur und damit die erhöhte kinetische Energie der Dünnfilmatome können diese sich
anders anordnen, indem sich die Kristallite flächig ausbreiten.
Dadurch verringert sich die Höhe der Kristallite.
Eine weitere Möglichkeit ist die Evaporation von Atomen, welche die Kristallite schrumpfen lässt.
Außerdem sind Risse in der Oberfläche zu erkennen.
Diese Risse werden nicht auf reinen Eagle XG Glassubstraten beobachtet und sind damit auf den Dünnfilm zurückzuführen.
Gründe dafür können Spannungen im Dünnfilm und im Substrat sein, welche durch Temperatur und Substrathalterung
verursacht werden.
Da die W6823 Probe die dünnste aus der Probenserie ist, ist sie am anfälligsten für Spannungen.

Die Topografieaufnahme nach Ausheizen auf \qty{700}{\degreeCelsius} in \cref{W6823-1B_Sauerstoff_AFM_700} ähnelt
der Aufnahme bei \qty{600}{\degreeCelsius}.
Erkennbar ist eine Verringerung der durchschnittlichen Fläche der kaum noch sichtbaren Kristallite.
Die Risse in der Oberfläche sind weiterhin vorhanden und haben sich vergrößert.
Auch ein großer Riss ist zu erkennen, welcher sich über die gesamte Aufnahme erstreckt.
Nach dem Ausheizen bei \qty{750}{\degreeCelsius} in \cref{W6823-1B_Sauerstoff_AFM_750} sind die Kristallite
vollständig verschwunden.
Erkennbar sind deutliche Risse in der Oberfläche, welche sich weiterhin vergrößert haben.
Auch bei \qty{800}{\degreeCelsius} in \cref{W6823-1B_Sauerstoff_AFM_800} vergrößern sich die Risse weiter.
Auch die maximale Höhe der Skala nimmt zu.
Bei dem finalen Ausheizschritt bei \qty{875}{\degreeCelsius} in \cref{W6823-1B_Sauerstoff_AFM_875} sind die Risse
weiterhin vorhanden, haben sich jedoch verkleinert.
Es prägt sich eine feingliedrige Rissstruktur aus.

Die Morphologie des Dünnfilms während des Sauerstoff Ausheizvorgangs weißt signifikante Veränderungen auf,
insbesondere auf die Höhe und Struktur der Kristallite, sowie auf die Rissbildung.
Sie deuten auf komplexe Wechselwirkungen zwischen Temperatur, Spannungen und Kristallwachstum hin.

\begin{figure}
    \centering
    \foreach \i in \temperaturesS{
        \begin{subfigure}[t]{0.40\textwidth}
            \includegraphics[width=\textwidth]
            {../plots/AFM/XG-Sauerstoff/XG-\i/W6823-1B/W6823-1B_XG_Sauerstoff_\i_Topography_1}
            \caption{\ifthenelse{\equal{\i}{pre}}{Initialzustand}{\qty{\i}{\degreeCelsius}}}
            \label{W6823-1B_Sauerstoff_AFM_\i}
        \end{subfigure}
    }
    \caption{W6823-1B, Sauerstoff, AFM}
    \label{fig:W6823-1B_Sauerstoff_AFM}
\end{figure}
\newpage

\subsubsection{Vakuum Ausheizvorgang}\label{subsubsec:W6823-1B_Vakuum}
\begin{figure}
    \centering
    \import{../plots/XRD}{W6823-1C_Vakuum.pgf}
    \caption{W6823-1C, Vakuum, XRD}
    \label{fig:W6823-1C_Vakuum_XRD}
\end{figure}
\cref{fig:W6823-1C_Vakuum_XRD} zeigt die Röntgendiffraktogramme mit $2\theta\omega$ Aufnahmen der Probe W6823-1 bei
verschiedenen Temperaturen während des Vakuum Ausheizvorgangs.
Anstelle der sieben Temperaturen wie bei den Proben W6821-1, W6822-1 und W6824-1 wurden hier nur vier Temperaturen
untersucht.
Grund dafür war ein Fehler in der Substratmontage.
Eine vorangegangene Messung der Vakuumkammer nutzte Wärmeleitpaste mit Silbernanopartikeln.
Die trotz Reinigung verbliebenen Reste und dem umgekehrten Einbau des Substrats führten zu einer zusätzlichen
Schicht auf der Oberfläche des Dünnfilms.
Da sich die Paste auf der Oberfläche verfestigt, kann die Probe nicht mehr verwendet werden, da jegliche
Vergleichbarkeit verloren geht.

Die übrigen Temperaturen weisen fast identische Diffraktogramme auf, wie die aus \cref{subsubsec:W6823-1B_Sauerstoff}.
Weiterhin sind die einzigen Peaks auf das Eagle XG Glassubstrat und den Probenhalter zurückzuführen.
Auch dieser \heo\ Dünnfilm ist weiterhin röntgenamorph.

Die AFM-Topografieaufnahmen bei verschiedenen Temperaturen der Probe W6823-1 während des Vakuum Ausheizvorgangs sind
in Abbildung \cref{fig:W6823-1C_Vakuum_AFM} dargestellt.
Die Aufnahme des Initialzustands in \cref{W6823-1C_Vakuum_AFM_pre} zeigt einen deutlichen Unterschied zu dem
Initialzustand aus \cref{subsubsec:W6823-1B_Sauerstoff}.
Anstelle der flächig verteilten Kristallite auf einem ebenen Untergrund sind hier große Unebenheiten
über das gesamte Bild verteilt.
Auf den Bergen des Untergrundes sind einzelne Kristallite zu erkennen.
Es zeigt sich möglicherweise ein Inselwachstum in einer größeren Skala.

Nach erstem Ausheizen auf \qty{500}{\degreeCelsius} in \cref{fig:W6823-1C_Vakuum_AFM_500} ändert sich die Morphologie
erneut.
Es ist kein unebener Untergrund mehr zu erkennen.
Stattdessen sind einzelne Kristallite auf einem ebenen Untergrund zu erkennbar, ähnlich wie in
\cref{W6823-1C_Sauerstoff_600}.
Für die Oberflächenrauigkeit des ebenen Untergrunds findet man eine RMS Rauigkeit von \qty{350}{\nano\meter}.
Damit liegt diese Rauigkeit im Bereich der Rauigkeit des Eagle XG Glassubstrats.
Es liegt nahe, dass der Dünnfilm nicht flächig gewachsen ist, sondern aus einzelnen Kristalliten besteht.
Nach dem Ausheizen auf \qty{600}{\degreeCelsius} in \cref{fig:W6823-1C_Vakuum_AFM_600} sind die Kristallite in ihrer
Fläche deutlich kleiner.
Auch bei \qty{700}{\degreeCelsius} in \cref{fig:W6823-1C_Vakuum_AFM_700} verkleinerten sich die Kristallite
in ihrer Fläche und Höhe.
Bei dem Ausheizvorgang unter Vakuumbedingungen liegt es nah, dass die Kristallite durch Evaporation von Atomen
schrumpfen.
Das sorgt dafür, dass kaum noch Dünnfilmatome auf der Oberfläche verbleiben.

\begin{figure}
    \centering
    ,\foreach \i in \temperaturesVV{
        \begin{subfigure}[t]{0.40\textwidth}
            \includegraphics[width=\textwidth]
            {../plots/AFM/XG-Vakuum/XG-\i/W6823-1C/W6823-1C_XG_Vakuum_\i_Topography_1}
            \caption{\ifthenelse{\equal{\i}{pre}}{Initialzustand}{\qty{\i}{\degreeCelsius}}}
            \label{W6823-1C_Vakuum_AFM_\i}
        \end{subfigure}
    }
    \caption{AFM, Vakuum, W6823-1C}
    \label{fig:W6823-1C_Vakuum_AFM}
\end{figure}
\newpage

\subsubsection{Luft Ausheizvorgang}\label{subsec:luft-ausheizvorgang-1}
\begin{figure}
    \centering
    \import{../plots/XRD}{W6823-1D_Luft.pgf}
    \caption{W6823-1D, Luft, XRD}
    \label{fig:W6823-1D, Luft, XRD}
\end{figure}
\begin{figure}
    \centering
    ,\foreach \i in \temperaturesL{
        \begin{subfigure}[t]{0.40\textwidth}
            \includegraphics[width=\textwidth]
            {../plots/AFM/XG-Luft/XG-\i/W6823-1D/W6823-1D_XG_Luft_\i_Topography_1}
            \caption{\ifthenelse{\equal{\i}{pre}}{Initialzustand}{\qty{\i}{\degreeCelsius}}}
            \label{W6823-1D, Luft, AFM, \i}
        \end{subfigure}
    }
    \caption{AFM, Luft, W6823-1D}
    \label{fig: AFM, Luft, W6823-1D}
\end{figure}
\newpage


% PROBE W6821-1 %

\newpage

\subsection{Probe W6821-1}\label{subsec:probe-W6821-1}

\subsubsection{Sauerstoff Ausheizvorgang}\label{subsubsec:W6821-1B_Sauerstoff}
\begin{figure}
    \centering
    \import{../plots/XRD}{W6821-1B_Sauerstoff.pgf}
    \caption{W6821-1B, Sauerstoff, XRD}
    \label{fig:W6821-1B_Sauerstoff_XRD}
\end{figure}
In Abbildung \cref{fig:W6821-1B_Sauerstoff_XRD} sind die $2\theta/\omega$-Scans der Probe \sampleone zu verschiedenen
Temperaturen der Ausheizserie in Sauerstoff dargestellt.
Wie in den vorherigen Abschnitten, sind die Peaks auf das Eagle XG Glassubstrat und den Probenhalter zurückzuführen.
Es sind keine Peaks des \heo\ Dünnfilms zu erkennen.
Der Dünnfilm ist röntgenamorph.

\cref{fig:W6821-1B_Sauerstoff_AFM} zeigt die AFM-Topografieaufnahmen der Probe W6821-1 zu den verschiedenen
Temperaturen.
Die Aufnahme des Initialzustands in \cref{W6821-1B_Sauerstoff_AFM_pre} zeigt viele zufällig angeordnete Kristallite,
welche dicht beieinander liegen.
Die Höhe der großen Kristallite beträgt circa \qtyrange{45}{55}{\nano\meter}.
Da mithilfe des Profilometers eine Filmdicke von circa \qty{95}{\nano\meter} bestimmt wurde, ist es naheliegend,
dass unterhalb der Kristallite ein flächig gewachsener Dünnfilm liegt.
Auch hier weist die Morphologie auf ein Stranski-Krastanov-Wachstum hin.

Nach dem ersten Ausheizschritt bei \qty{600}{\degreeCelsius} in \cref{W6821-1B_Sauerstoff_AFM_600} hat sich die
Oberfläche stark verändert.
Die einzelnen großen Kristallite sind verschwunden.
Stattdessen sind viele deutlich kleinere Kristallite zu erkennen.
Auch die maximale Höhe der Skala hat sich von \qty{62}{\nano\meter} auf \qty{42.7}{\nano\meter} verringert.
Die Topografie von \qty{700}{\degreeCelsius} in \cref{W6821-1B_Sauerstoff_AFM_700} ähnelt der von
\qty{600}{\degreeCelsius}.
Jedoch sind deutlich weniger Kristallite zu erkennen.
Es sind Löcher in der Oberfläche erkennbar.

Die Topografie von \qty{750}{\degreeCelsius} in \cref{W6821-1B_Sauerstoff_AFM_750} und \qty{800}{\degreeCelsius}
in \cref{W6821-1B_Sauerstoff_AFM_800} ähneln sich.
Die Grains sind im Gegensatz zu der Aufnahme bei \qty{700}{\degreeCelsius} deutlich kleiner.
Außerdem ist die Anzahl der Löcher in der Oberfläche gestiegen.
Diese Löcher sind auf den Dünnfilm zurückzuführen und sind nicht auf reinen Eagle XG Glassubstraten zu beobachten.
Es weist auf die Evaporation von Atomen hin.

Die Aufnahme nach dem Ausheizen bei \qty{875}{\degreeCelsius} in \cref{W6821-1B_Sauerstoff_AFM_875} zeigt starke
Risse in der Oberfläche.
Durch thermische Spannungen im Dünnfilm und im Substrat entstehen Risse.
\begin{figure}
    \centering
    \foreach \i in \temperaturesS{
        \begin{subfigure}[t]{0.40\textwidth}
            \includegraphics[width=\textwidth]
            {../plots/AFM/XG-Sauerstoff/XG-\i/W6821-1B/W6821-1B_XG_Sauerstoff_\i_Topography_1}
            \caption{\ifthenelse{\equal{\i}{pre}}{Initialzustand}{\qty{\i}{\degreeCelsius}}}
            \label{W6821-1B_Sauerstoff_AFM_\i}
        \end{subfigure}
    }
    \caption{W6821-1B, Sauerstoff, AFM}
    \label{fig:W6821-1B_Sauerstoff_AFM}
\end{figure}
\newpage

\subsubsection{Vakuum Ausheizvorgang}\label{subsec:vakuum-ausheizvorgang-1}
\begin{figure}
    \centering
    \import{../plots/XRD}{W6821-1C_Vakuum.pgf}
    \caption{W6821-1C, Vakuum, XRD}
    \label{fig:W6821-1C, Vakuum, XRD}
\end{figure}
\begin{figure}
    \centering
    ,\foreach \i in \temperaturesV{
        \begin{subfigure}[t]{0.40\textwidth}
            \includegraphics[width=\textwidth]
            {../plots/AFM/XG-Vakuum/XG-\i/W6821-1C/W6821-1C_XG_Vakuum_\i_Topography_1}
            \caption{\ifthenelse{\equal{\i}{pre}}{Initialzustand}{\qty{\i}{\degreeCelsius}}}
            \label{W6821-1C, Vakuum, AFM, \i}
        \end{subfigure}
    }
    \caption{AFM, Vakuum, W6821-1C}
    \label{fig: AFM, Vakuum, W6821-1C}
\end{figure}
\newpage

\subsubsection{Luft Ausheizvorgang}\label{subsec:luft-ausheizvorgang-1}
\begin{figure}
    \centering
    \import{../plots/XRD}{W6821-1D_Luft.pgf}
    \caption{W6821-1D, Luft, XRD}
    \label{fig:W6821-1D, Luft, XRD}
\end{figure}
\begin{figure}
    \centering
    ,\foreach \i in \temperaturesL{
        \begin{subfigure}[t]{0.40\textwidth}
            \includegraphics[width=\textwidth]
            {../plots/AFM/XG-Luft/XG-\i/W6821-1D/W6821-1D_XG_Luft_\i_Topography_1}
            \caption{\ifthenelse{\equal{\i}{pre}}{Initialzustand}{\qty{\i}{\degreeCelsius}}}
            \label{W6821-1D, Luft, AFM, \i}
        \end{subfigure}
    }
    \caption{AFM, Luft, W6821-1D}
    \label{fig: AFM, Luft, W6821-1D}
\end{figure}
\newpage

% PROBE W6822-1 %

\newpage

\subsection{Probe W6822-1}\label{subsec:probe-W6822-1}

\subsubsection{Sauerstoff Ausheizvorgang}\label{subsec:sauerstoff-ausheizvorgang-1}
\begin{figure}
    \centering
    \import{../plots/XRD}{W6822-1B_Sauerstoff.pgf}
    \caption{W6822-1B, Sauerstoff, XRD}
    \label{fig:W6822-1B, Sauerstoff, XRD}
\end{figure}
\begin{figure}
    \centering
    \foreach \i in \temperaturesS{
        \begin{subfigure}[t]{0.40\textwidth}
            \includegraphics[width=\textwidth]
            {../plots/AFM/XG-Sauerstoff/XG-\i/W6822-1B/W6822-1B_XG_Sauerstoff_\i_Topography_1}
            \caption{\ifthenelse{\equal{\i}{pre}}{Initialzustand}{\qty{\i}{\degreeCelsius}}}
            \label{W6822-1B, Sauerstoff, AFM, \i}
        \end{subfigure}
    }
    \caption{W6822-1B, Sauerstoff, AFM}
    \label{fig:W6822-1B, Sauerstoff, AFM}
\end{figure}
\newpage

\subsubsection{Vakuum Ausheizvorgang}\label{subsec:vakuum-ausheizvorgang-1}
\begin{figure}
    \centering
    \import{../plots/XRD}{W6822-1C_Vakuum.pgf}
    \caption{W6822-1C, Vakuum, XRD}
    \label{fig:W6822-1C, Vakuum, XRD}
\end{figure}
\begin{figure}
    \centering
    ,\foreach \i in \temperaturesV{
        \begin{subfigure}[t]{0.40\textwidth}
            \includegraphics[width=\textwidth]
            {../plots/AFM/XG-Vakuum/XG-\i/W6822-1C/W6822-1C_XG_Vakuum_\i_Topography_1}
            \caption{\ifthenelse{\equal{\i}{pre}}{Initialzustand}{\qty{\i}{\degreeCelsius}}}
            \label{W6822-1C, Vakuum, AFM, \i}
        \end{subfigure}
    }
    \caption{AFM, Vakuum, W6822-1C}
    \label{fig: AFM, Vakuum, W6822-1C}
\end{figure}
\newpage

\subsubsection{Luft Ausheizvorgang}\label{subsec:luft-ausheizvorgang-1}
\begin{figure}
    \centering
    \import{../plots/XRD}{W6822-1D_Luft.pgf}
    \caption{W6822-1D, Luft, XRD}
    \label{fig:W6822-1D, Luft, XRD}
\end{figure}
\begin{figure}
    \centering
    ,\foreach \i in \temperaturesL{
        \begin{subfigure}[t]{0.40\textwidth}
            \includegraphics[width=\textwidth]
            {../plots/AFM/XG-Luft/XG-\i/W6822-1D/W6822-1D_XG_Luft_\i_Topography_1}
            \caption{\ifthenelse{\equal{\i}{pre}}{Initialzustand}{\qty{\i}{\degreeCelsius}}}
            \label{W6822-1D, Luft, AFM, \i}
        \end{subfigure}
    }
    \caption{AFM, Luft, W6822-1D}
    \label{fig: AFM, Luft, W6822-1D}
\end{figure}
\newpage


% PROBE W6824-1 %

\newpage

\subsection{Probe W6824-1}\label{subsec:probe-W6824-1}

\subsubsection{Sauerstoff Ausheizvorgang}\label{subsec:sauerstoff-ausheizvorgang-1}

\begin{figure}
    \centering
    \import{../plots/XRD}{W6824-1B_Sauerstoff.pgf}
    \caption{W6824-1B, Sauerstoff, XRD}
    \label{fig:W6824-1B_Sauerstoff_XRD}
\end{figure}
\begin{figure}
    \centering
    \foreach \i in \temperaturesS{
        \begin{subfigure}[t]{0.40\textwidth}
            \includegraphics[width=\textwidth]
            {../plots/AFM/XG-Sauerstoff/XG-\i/W6824-1B/W6824-1B_XG_Sauerstoff_\i_Topography_1}
            \caption{\ifthenelse{\equal{\i}{pre}}{Initialzustand}{\qty{\i}{\degreeCelsius}}}
            \label{W6824-1B_Sauerstoff_AFM_\i}
        \end{subfigure}
    }
    \caption{W6824-1B, Sauerstoff, AFM}
    \label{fig:W6824-1B_Sauerstoff_AFM}
\end{figure}
\newpage

\subsubsection{Vakuum Ausheizvorgang}\label{subsec:vakuum-ausheizvorgang-1}
\begin{figure}
    \centering
    \import{../plots/XRD}{W6824-1C_Vakuum.pgf}
    \caption{W6824-1C, Vakuum, XRD}
    \label{fig:W6824-1C, Vakuum, XRD}
\end{figure}
\begin{figure}
    \centering
    ,\foreach \i in \temperaturesV{
        \begin{subfigure}[t]{0.40\textwidth}
            \includegraphics[width=\textwidth]
            {../plots/AFM/XG-Vakuum/XG-\i/W6824-1C/W6824-1C_XG_Vakuum_\i_Topography_1}
            \caption{\ifthenelse{\equal{\i}{pre}}{Initialzustand}{\qty{\i}{\degreeCelsius}}}
            \label{W6824-1C, Vakuum, AFM, \i}
        \end{subfigure}
    }
    \caption{AFM, Vakuum, W6824-1C}
    \label{fig: AFM, Vakuum, W6824-1C}
\end{figure}
\newpage

\subsubsection{Luft Ausheizvorgang}\label{subsec:luft-ausheizvorgang-1}
\begin{figure}
    \centering
    \import{../plots/XRD}{W6824-1D_Luft.pgf}
    \caption{W6824-1D, Luft, XRD}
    \label{fig:W6824-1D, Luft, XRD}
\end{figure}
\begin{figure}
    \centering
    ,\foreach \i in \temperaturesL{
        \begin{subfigure}[t]{0.40\textwidth}
            \includegraphics[width=\textwidth]
            {../plots/AFM/XG-Luft/XG-\i/W6824-1D/W6824-1D_XG_Luft_\i_Topography_1}
            \caption{\ifthenelse{\equal{\i}{pre}}{Initialzustand}{\qty{\i}{\degreeCelsius}}}
            \label{W6824-1D, Luft, AFM, \i}
        \end{subfigure}
    }
    \caption{AFM, Luft, W6824-1D}
    \label{fig: AFM, Luft, W6824-1D}
\end{figure}
\newpage

\subsection{Rauigkeiten}\label{subsec:rauigkeit}
\begin{figure}
    \centering
    \import{../plots/AFM}{sauerstoff.pgf}
    \caption{AFM, Sauerstoff}
    \label{fig: AFM, Sauerstoff}
\end{figure}

\begin{figure}
    \centering
    \import{../plots/AFM}{vakuum.pgf}
    \caption{AFM, Vakuum}
    \label{fig: AFM, Vakuum}
\end{figure}
