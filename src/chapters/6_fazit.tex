\section{Fazit und Ausblick}\label{sec:fazit-und-ausblick}
Diese Studie zielte darauf ab, \heo-Dünnfilme zu synthetisieren und durch Ausheizprozesse in eine reine
Natriumchloridstruktur zu überführen.
Die Ergebnisse deuten zwar auf eine erfolgreiche Abscheidung der Dünnfilme hin, jedoch konnte der erwartete
Phasenübergang in die Natriumchloridstruktur bei keinem der drei Ausheizserien beobachtet werden.
Eine mögliche Ursache hierfür könnte in der Abweichung der Dünnfilmzusammensetzung von der equimolaren Zusammensetzung
des Targets liegen, was zu erhöhten Übergangstemperaturen führte und die Wahl des Temperaturbereichs sowie des Substrats
als ungeeignet erscheinen ließ.

Die Abwesenheit charakteristischer Peaks in den Röntgendiffraktogrammen zeigt, dass die Dünnfilme keine kristalline
Natriumchloridstruktur aufweisen.
Die Proben \samplethree, \sampleone, \sampletwo\ und \samplefour\ zeigen bei allen drei Ausheizserien eine röntgenamorphe Struktur.
Ein Grund für dieses Verhalten könnte eine erhöhte Übergangstemperatur infolge der modifizierten Stöchiometrie der
Dünnfilme sein.
Die in \cref{tab:concentration} dargestellten Konzentrationen zeigen, dass die Komposition der \heo-Dünnfilme nicht der
equimolaren Zusammensetzung des Targets entsprechen.
Die Filme weisen eine geringere Magnesiumkonzentration auf, welche mit sinkendem Druck abnimmt.
Um für zukünftige Abscheideprozesse eine equimolare Zusammensetzung zu erreichen, kann das Target in seiner
Komposition angepasst werden.
Durch einen erhöhten Magnesiumanteil im Target könnte auch die Magnesiumkonzentration in den Filmen erhöht werden.


Aus der abweichenden Stöchiometrie der Dünnfilme ergibt sich eine niedrigere Mischungsentropie, welche für die
Phasenstabilität von \heo\ entscheidend ist.
Um die Phasenübergangstemperatur in die reine Natriumchloridstruktur besser abschätzen zu können, kann die
Mischungsentropie mit den gemessenen Konzentrationen $\{ x_i \}$ aus \cref{tab:concentration} berechnet, und mit
derjenigen Mischungsentropie gleichgesetzt werden, die sich bei vier equimolaren Konstituenten und einem Konstituent mit
variabler Konzentration $x$ ergibt.
Aus \cref{eq:Mischungsentropie} und \cref{eq:Mischungsentropie2} folgt:
\begin{equation}
    -\mathrm{R}\sum_{i=1}^{5}x_{i}\ln(x_{i}) \stackrel{!}{=}-\mathrm{R}\left( x\log(x)+(1-x)\ln
    \left( \frac{1-x}{4} \right) \right).
    \label{eq:fazit}
\end{equation}
Löst man diese Gleichung numerisch nach $x$, findet man für die Proben \csamplethree, \csampleone, \csampletwo\ und \csamplefour,
folgende Konzentrationen:
\begin{equation*}
    x(\mathrm{P}_{\num{0.1}}^{\mathrm{c}}) = \qty{16.8}{\percent}, \quad x(\mathrm{P}_{\num{0.01}}^{\mathrm{c}})
    = \qty{15.6}{\percent}, \quad
    x(\mathrm{P}_{\num{0.001}}^{\mathrm{c}}) = \qty{15.7}{\percent}, \quad x(\mathrm{P}_{\num{0.00005}}^{\mathrm{c}})
    = \qty{15.0}{\percent}.
\end{equation*}
Die von \citeauthoryear{Rost2015} angegebenen Phasenübergangstemperaturen in eine reine Natriumchloridstruktur
bei vier equimolaren Konstituenten und einer variablen Konzentration von $x \simeq \qty{15}{\percent}$ liegen, je nach
Material, zwischen \qty{925}{\degreeCelsius} und \qty{1000}{\degreeCelsius}.
Auch wenn in dieser Arbeit Dünnfilme und keine Massivproben untersucht wurden, kann die Phasenübergangstemperatur
als erste Abschätzung für die Wahl des Temperaturbereichs herangezogen werden und liegt deutlich oberhalb der
gewählten Temperaturbereiche.


Damit einher geht auch die ungeeignete Wahl des Eagle XG Substrats, dessen Erweichungspunkt bei
\qty{971}{\degreeCelsius} liegt und sich schon bei \qty{875}{\degreeCelsius} merklich verformt.
Für nachfolgende Abscheidungen von \heo-Dünnfilmen sollten Substrate gewählt werden, deren Erweichungspunkte weit
oberhalb der Übergangstemperatur liegen.
Ein geeigneter Kandidat wäre das Substrat \ce{SiO2}, dessen Erweichungspunkt bei \qty{1308}{\degreeCelsius} liegt.
Dieses Substrat hätte außerdem den Vorteil, dass es keine relevanten Metallkationen enthält, und damit
EDX Messungen ermöglicht, die nicht durch das Substrat verfälscht werden.

Weiterhin sollte versucht werden, die mechanischen Einwirkungen auf den Dünnfilm und das Substrat, vor allem
während des Ausheizprozesses, zu minimieren.
Der Druck der Klemmen des Substrathalters in der Vakuumkammer sorgte bei einer Temperatur von \qty{875}{\degreeCelsius}
für merkliche Verformungen des Substrats.
Wie Tim Düvel in seiner aktuellen Arbeit festgestellt hat, sorgt der Druck der Klemmen nicht nur für
mechanischen Spannungen, sondern auch für eine ungleichmäßige Temperaturverteilung über das Substrat hinweg.
Auch durch das punktuelle Auftreffen des Laserstrahls auf den Halter entsteht ein radialer Temperaturgradient.
Diese Effekte führen zu einer ungleichmäßigen Ausheizung des Dünnfilms.
Da auch die Verunreinigungen der reinen Eagle XG Glassubstrate im Muffelofen deutlich geringer ausfallen als in der
Vakuumkammer, sollte für zukünftige Ausheizprozesse auf den Muffelofen zurückgegriffen werden.

Die Topografieaufnahmen der unterschiedlichen Initialzustände der Proben, die sich durch eine örtliche und zeitliche
Versetzung der Messung auszeichnen, deuten auf örtliche oder zeitliche Veränderungen des Dünnfilms hin.
Für zukünftige Messungen kann die Oberfläche der Proben an mehreren Stellen über den gesamten Film hinweg
untersucht werden, um lokale Unterschiede in der Morphologie zu erfassen.
Durch die von Jorrit Bredow bereitgestellten AFM-Probenhalter kann die Abhängigkeit der Topografie von der Zeit
untersucht werden.
Die Proben können in den Probenhalter eingespannt und über einen längeren Zeitraum hinweg an der gleichen Stelle
untersucht werden.

Obwohl alle Dünnfilme, unabhängig vom Abscheidedruck, eine röntgenamorphe Struktur aufweisen, zeigen die
Topografieaufnahmen der Proben \samplethree, \sampleone, \sampletwo\ und \samplefour\ unterschiedliche
Oberflächenmorphologien und Rauheiten.
Die Initialmessung zeigt meist hohe Kristallite, die zufällig angeordnet einen ebenen Untergrund bedecken.
Die Ausheizprozesse führten zu einer Evaporation der Kristallite, welche die Oberfläche der Kristallite glätteten.
Bei allen Proben wurde eine thermisch induzierte Kristallitbildung beobachtet, die sich durch eine erhöhte Rauheit
der Proben auszeichnet.
Höhere Temperaturen gingen oftmals mit Löchern in den Dünnfilmen einher.

Diese in den Dünnfilmen auftretenden Löcher sind ein weiteres Indiz für die Evaporation der Kristallite.
Durch diese Evaporationsprozesse wurden einzelne Kristallite von der Oberfläche abgetragen, sodass sich die
Dünnfilmdicke verringert.
Durch zunehmende Transparenz der Proben bei höheren Temperaturen wird dieses Phänomen auch optisch sichtbar,
sodass für zukünftige Untersuchungen optische Messungen in Betracht gezogen werden sollten.
Dies stellt eine weitere Herausforderung für die Ausheizprozesse dar, da die Dicke der Dünnfilme nicht konstant bleibt.
Für weitere Untersuchungen sollte die Dicke der Dünnfilme nach dem Ausheizprozess mithilfe von XRR- oder
Profilometermessungen bestimmt werden, um Rückschlüsse auf die Evaporationsraten ziehen zu können.
Eine größere Dünnfilmdicke würde den Einfluss von Abtragungsvorgängen verringern.

Da die Dünnfilme teilweise nicht signifikante Peaks aufweisen, empfiehlt es sich, GIXRD-Aufnahmen
nach den jeweiligen Ausheizschritten durchzuführen.
Diese sind empfindlicher gegenüber Dünnfilmen und ermöglichen damit auch das Erfassen kleinerer Peaks, was gerade
bei sehr geringen Schichtdicken von Vorteil ist.

Aufgrund der Komplexität der Evaporationsprozesse ist davon auszugehen, dass nicht alle Konstituenten gleichermaßen
verdampfen.
Infolgedessen kann sich die Stöchiometrie der Dünnfilme weiter verändern, sodass die Mischungsentropie
weiter sinkt.
Um dieses Phänomen zu untersuchen, können mehrere Proben unter gleichen Bedingungen hergestellt und anschließend
bei verschiedenen Temperaturen ausgeheizt werden.
Nach der jeweiligen Temperatur kann die Komposition einer Probe durch EDX-Messungen bestimmt werden.
Zu beachten ist dabei, dass die Probe mit Kohlenstoff beschichtet werden muss, was sie für weitere
Untersuchungen unbrauchbar macht.

Die vorliegende Arbeit hat gezeigt, dass die Synthese von \heo-Dünnfilmen und deren thermische Behandlung ein komplexes
Zusammenspiel verschiedener Faktoren ist.
Obwohl der angestrebte Phasenübergang nicht erreicht werden konnte, haben die gewonnenen Erkenntnisse wertvolle Hinweise
für zukünftige Untersuchungen geliefert.
Durch eine systematische Variation der Prozessparameter, eine optimierte Wahl von Substrat, Target und Temperaturbereich
und eine detaillierte Charakterisierung der Dünnfilme mithilfe von Schichtdickenmessungen und optischen Untersuchungen
können die Voraussetzungen für eine erfolgreiche Überführung in die gewünschte Struktur geschaffen werden.