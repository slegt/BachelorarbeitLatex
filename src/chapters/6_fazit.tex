\section{Fazit und Ausblick}\label{sec:fazit-und-ausblick}
Ziel der Arbeit war die Untersuchung der temperaturabhängigen Kristallisation und der Oberflächenmorphologie von
\heo-Dünnfilmen.

Zu diesem Zweck wurden vier \heo-Dünnfilmproben mithilfe der gepulsten Laserabscheidung bei unterschiedlichen
Sauerstoffpartialdrücken auf Eagle XG Glassubstraten abgeschieden, gevierteilt und anschließend unter verschiedenen
Atmosphären bei festgelegten Temperaturen ausgeheizt.
Außerdem wurden c-Saphir Proben für Konzentrationsanalysen mit identischen Parametern abgeschieden.

Mithilfe von EDX-Messungen konnte die Konzentration der c-Saphir Proben bestimmt werden, die eine gute Näherung
der tatsächlichen Konzentrationen der Eagle XG Proben darstellen.
Die ortsaufgelösten EDX-Messungen zeigten, dass der Dünnfilm homogen abgeschieden wurde.
Allerdings entspricht die Konzentration der Dünnfilme nicht der äquimolaren Zusammensetzung des Targets, wie
\cref{tab:concentration} zeigt.
Mit fallendem Abscheidedruck nimmt die Magnesiumkonzentration in den Dünnfilmen ab.
Um für zukünftige Abscheideprozesse eine äquimolare Zusammensetzung zu erreichen, kann das Target in seiner
Komposition angepasst werden.
Durch einen erhöhten Magnesiumanteil im Target könnte auch die Magnesiumkonzentration in den Filmen erhöht werden.

Um die Kristallinität der Dünnfilme in Abhängigkeit der Temperatur zu untersuchen, wurden die Proben nach jedem
Ausheizprozess mithilfe von $2\theta/\omega$-Scans untersucht.
Sämtliche Diffraktogramme weisen keinerlei charakteristischer Peaks auf.
Die Proben \samplethree, \sampleone, \sampletwo\ und \samplefour\ zeigen bei allen
gewählten Atmosphären während des Ausheizens eine röntgenamorphe Struktur.
Ein Grund für dieses Verhalten könnte eine erhöhte Übergangstemperatur infolge der modifizierten Komposition der
Dünnfilme sein.
Damit fällt der für diese Arbeit gewählte Temperaturbereich unter die geschätzte Übergangstemperatur der Dünnfilme.
Für zukünftige Untersuchungen sollte der Temperaturbereich erweitert werden, um einen Phasenübergang
beobachten zu können.

Damit einher geht auch die ungeeignete Wahl des Eagle XG Substrats, dessen Erweichungspunkt bei
\qty{971}{\degreeCelsius} liegt und sich schon bei \qty{875}{\degreeCelsius} merklich verformt.
Für nachfolgende Abscheidungen von \heo-Dünnfilmen sollten Substrate gewählt werden, deren Erweichungspunkte weit
oberhalb der Übergangstemperatur liegen.
Eine passendere Alternative wäre das Substrat \ce{SiO2}, dessen Erweichungspunkt bei \qty{1308}{\degreeCelsius} liegt.
Dieses Substrat hätte außerdem den Vorteil, dass es keine relevanten Metallkationen enthält, und damit
EDX Messungen ermöglicht, die nicht durch das Substrat verfälscht werden können.

Weiterhin sollte versucht werden, die mechanischen Einwirkungen auf den Dünnfilm und das Substrat, vor allem
während des Ausheizprozesses, zu minimieren.
Der Druck der Klemmen des Substrathalters in der Vakuumkammer sorgte bei einer Temperatur von \qty{875}{\degreeCelsius}
für merkliche Verformungen des Substrats.
Wie Tim Düvel in seiner aktuellen Arbeit festgestellt hat, sorgt der Druck der Klemmen nicht nur für
mechanischen Spannungen, sondern auch für eine ungleichmäßige Temperaturverteilung über das Substrat hinweg.
Auch durch das punktuelle Auftreffen des Laserstrahls auf den Halter entsteht ein radialer Temperaturgradient.
Diese Effekte führen zu einer ungleichmäßigen Ausheizung des Dünnfilms.
Da auch die Verunreinigungen der reinen Eagle XG Glassubstrate im Muffelofen deutlich geringer ausfallen als in der
Vakuumkammer, sollte für zukünftige Ausheizprozesse auf einem Muffelofen zurückgegriffen werden.
Um in dem Muffelofen mit unterschiedlichen Atmosphären ausheizen zu können, könnte eine Retorte, also ein mit
feuerfestem Material ausgekleideter Behälter, verwendet werden.

Die Topografieaufnahmen der Initialzustände weisen starke Unterschiede auf.
Da die Aufnahmen an unterschiedlichen Stellen des ursprünglich $\qty{10}{\milli\meter} \times \qty{10}{\milli\meter}$
Substrates gemacht wurden, könnte das auf eine laterale Inhomogenität der Dünnfilme hinweisen.
Zusätzlich wurden diese nicht zeitgleich aufgenommen, sondern mit rund 14 Tagen Abstand zueinander.
Dies könnte also auch durch eine zeitliche Instabilität der Dünnfilme bedingt sein.
Für zukünftige Messungen kann die Oberfläche der Proben an mehreren Stellen stichprobenartig über den gesamten Film
hinweg untersucht werden, um lokale Unterschiede in der Morphologie zu erfassen.
Für die Untersuchung der zeitlichen Veränderung der Oberfläche ist eine reproduzierbare Positionierung notwendig, um
AFM Messungen an idealerweise identischen Stellen vorzunehmen.

Obwohl alle Dünnfilme, unabhängig vom Abscheidedruck und Ausheiz-Atmosphäre, eine röntgenamorphe Struktur aufweisen,
zeigen die Topografieaufnahmen der Proben \samplethree, \sampleone, \sampletwo\ und \samplefour\ unterschiedliche
Oberflächenmorphologien und Rauheiten.
Die Initialmessung zeigt meist hohe Kristallite, die zufällig angeordnet einen ebenen Untergrund bedecken.
Die Ausheizprozesse führen zu einer Evaporation der Kristallite, welche die Oberfläche der Kristallite glätten.
Bei allen Proben wurde eine thermisch induzierte Kristallitbildung beobachtet, die sich durch eine erhöhte Rauheit
der Proben auszeichnet.
Höhere Temperaturen gingen oftmals mit Löchern in den Dünnfilmen einher.

Diese in den Dünnfilmen auftretenden Löcher sind ein weiteres Indiz für die Evaporation der Kristallite.
Durch diese Evaporationsprozesse wurden einzelne Kristallite von der Oberfläche abgetragen, sodass sich die
Dünnfilmdicke verringert.
Die zunehmende Transparenz der Proben bei höheren Temperaturen liefert einen Hinweis für dieses Phänomen,
sodass für zukünftige Untersuchungen optische Messungen, wie beispielsweise Transmissionsmessungen, in Betracht gezogen
werden sollten.
Dies stellt eine weitere Herausforderung für die Ausheizprozesse dar, da die Dicke der Dünnfilme vermutlich nicht
konstant bleibt.
Für weitere Untersuchungen sollte die Dicke der Dünnfilme nach jedem Ausheizschritt mithilfe von XRR- oder
Profilometermessungen bestimmt werden, um Rückschlüsse auf die Evaporationsraten ziehen zu können.
Eine größere Dünnfilmdicke würde den Einfluss von Abtragungsvorgängen verringern.

Da die XRD Messungen  Reflexe vermuten lassen, die allerdings stochastisch nicht signifikant sind,
empfiehlt es sich, GIXRD-Aufnahmen nach den jeweiligen Ausheizschritten durchzuführen.
Diese sind empfindlicher gegenüber Dünnfilmen und ermöglichen damit auch das Erfassen kleinerer Peaks, was gerade
bei sehr geringen Schichtdicken von Vorteil ist.

Aufgrund der Komplexität der Evaporationsprozesse ist davon auszugehen, dass nicht alle Konstituenten gleichermaßen
verdampfen.
Infolgedessen kann sich die Stöchiometrie der Dünnfilme weiter verändern, sodass die Mischungsentropie
weiter sinkt.
Um dieses Phänomen zu untersuchen, können mehrere Proben unter gleichen Bedingungen hergestellt und anschließend
bei verschiedenen Temperaturen ausgeheizt werden.
Nach der jeweiligen Temperatur kann die Komposition einer Probe durch EDX-Messungen bestimmt werden.
Zu beachten ist dabei allerdings die stark isolierende Eigenschaft der Dünnfilme, welche EDX Aufnahmen erschwert.
Deshalb sollten die Dünnfilme beispielweise mit wenigen Nanometern Kohlenstoff beschichtet werden.

Die vorliegende Arbeit hat gezeigt, dass die Synthese von \heo-Dünnfilmen und deren thermische Behandlung ein komplexes
Zusammenspiel verschiedener Faktoren ist.
Obwohl unabhängig von der Ausheiztemperatur und -atmosphäre keine Kristallisation mittels XRD nachgewiesen werden
konnte, liefern die im Rahmen dieser Arbeit gewonnenen Erkenntnisse wertvolle Informationen und wichtige
Handlungsempfehlungen für zukünftige Untersuchungen.
Durch eine systematische Variation der Prozessparameter, eine optimierte Wahl von Substrat, Target und Temperaturbereich
und eine detaillierte Charakterisierung der Dünnfilme mithilfe von Schichtdickenmessungen und optischen Untersuchungen
kann ein tieferes Verständnis des komplexen Verhaltens der \heo-Dünnfilme während des Ausheizens geschaffen werden.





