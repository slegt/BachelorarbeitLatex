\section{Einleitung}\label{sec:einleitung}
Die Suche nach neuen und funktionalen Materialien, welche die Bedürfnisse moderner Technologien erfüllen, ist ein
wichtiger Bestandteil der Materialwissenschaften.
Methoden zur Entdeckung und Vorhersage solcher Materialien sind vielfältig und konnten erfolgreich große Teile
neuer Materialbibliotheken erschließen.
Dennoch benutzen sie in vielen Fällen die Dichtefunktionaltheorie und führten Berechnungen bei \qty{0}{\kelvin} durch.
Damit können Vorhersagen getroffen werden, welche die Stabilität auf Grundlage der Enthalpie bestimmen
\autocite{Rost2015}.
Die zu minimierende Größe ist jedoch die Gibbs-Energie $G=H-TS$, welche nicht nur durch die Enthalpie $H$, sondern auch
durch das Produkt von Entropie $S$ und Temperatur $T$ bestimmt wird.
Ein vielversprechender alternativer Ansatz, der sich in den letzten Jahren etablierte, ist die Herstellung und
Untersuchung entropiestabilisierter Materialien.
Diese Materialien sind durch eine hohe Entropie gekennzeichnet, die die Bildung einer stabilen Phase ermöglicht, indem
das Produkt $TS$ im Vergleich zur Enthalpie $H$ dominiert.

\citeauthoryear{cantor} untersuchten die Eigenschaften von Mehrkomponentenlegierungen, bestehend aus 20, beziehungsweise 16
equimolar verteilten Elementen.
Beide Systeme zeigten mehrere Phasen, aber auch eine gemeinsame fcc Struktur.
Mithilfe dieser Beobachtung konnte das equimolare Fünf-Komponenten-System \ce{CrMnFeCoNi} entwickeln werden, welches in
einphasiger fcc-Struktur kristallisiert \autocite{cantor}.
Die Fähigkeit, ein stabiles, einphasiges fcc Kristallgitter zu erreichen, war bemerkenswert.
In der Regel neigen Legierungen mit vielen Komponenten dazu, verschiedene Phasen zu bilden, was die
Vorhersagbarkeit der Eigenschaften erschwert.
Cantors Entdeckung zeigte, dass eine geeignete Kombination von Elementen die Bildung einer stabilen Phase ermöglicht
und schuf damit die Grundlage für die Entwicklung mehrkomponentiger Legierungen.

\citeauthoryear{yeh} konnten das Phänomen mithilfe der Mischungsentropie erklären und führten den Begriff der
hochentropischen Legierungen (HEAs, engl. \textit{high entropy alloys}) ein.
Dabei klassifizierten sie Legierungen als HEAs, falls diese mindestens fünf oder mehr Komponenten besitzen
und das molare Verhältnis jedes Konstituenten zwischen \qty{5}{\percent} und \qty{35}{\percent} liegt \autocite{yeh}.
Das war ein Wendepunkt in der Materialwissenschaft, da es die Möglichkeit eröffnete, neue Materialien mithilfe
der Entropie zu stabilisieren, die aus enthalpischer Sicht instabil waren.


\citeauthoryear{Rost2015} erweiterten das Konzept der HEAs auf Metalloxide und führten den Begriff der
entropiestabilisierten Metalloxide ein.
Sie synthetisierten \heo\ und untersuchten dessen Struktur und Eigenschaften.
Dabei haben sie mehrere experimentelle Hinweise dafür geliefert, dass eine reine Mischphase entsteht und diese
tatsächlich durch die Entropie stabilisiert wird.
Sie schufen damit die Grundlage für ein neues Forschungsfeld in der Materialwissenschaft \autocite{Rost2015}.

Im Laufe der Entwicklung hat sich der Begriff der hochentropischen Metalloxide (HEOs, engl. \textit{high entropy oxides})
etabliert.
Es wurden nicht nur verschiedenste Oxide entdeckt, sondern auch Keramiken, Polymere und Verbundsstoffe.
Die Entdeckung der HEAs führte damit zu einer riesigen Klasse der hochentropischen Materialien (HEMs, engl.
\textit{high entropy materials}) mit vielversprechenden Eigenschaften \autocite{Yeh2018}.

Dünnfilme sind ein wichtiger Bestandteil der modernen Technologie und finden Anwendung in unzähligen Bereichen.
Gerade für die Halbleiterelektronik spielen Dünnfilm eine Schlüsselrolle und bilden die Grundlage für
nahezu alle mikroelektronischen Bauelemente.
Die Vorhersage und Optimierung der Eigenschaften von Dünnfilmen ist eine komplexe Herausforderung, da diese stark von
den Substraten und den Prozessparametern abhängen und häufig nicht mit den Eigenschaften des zugrunde liegenden
Massivmaterials übereinstimmen.
Aufgrund der geringen Dicke von Dünnfilmen sind die Oberflächeneigenschaften von besonderer Bedeutung, welche in
analytischen Betrachtungen oftmals vernachlässigt werden.

Die Herstellung und Untersuchung von hochentropischen Materialien in Form von Dünnfilmen ermöglicht eine Verbindung
zwischen der Dünnfilmforschung und der Forschung zu HEMs, wodurch neue Erkenntnisse und innovative
Ansätze in beiden Bereichen gefördert werden.
Aus diesem Grund ist das Ausheizen und Charakterisieren von \heo\ Dünnfilmen zentrales Thema der vorliegenden Arbeit.
Das Ziel ist es, Dünnfilme mithilfe der gepulsten Laserdeposition herzustellen, durch unterschiedliche
Ausheizprozess eine stabile Phase zu erzeugen und währenddessen die Struktur und Eigenschaften dieser Dünnfilme
zu untersuchen.
Konkret soll die Kristallinität mithilfe von Röntgendiffraktometrie und die Oberflächenmorphologie mithilfe von
Rasterkraftmikroskopie charakterisiert werden.
Dabei sollen die Proben einem Prozess aus wiederholtem Ausheizen und anschließendem Messen unterzogen werden.
Es erfolgen drei Ausheizprozesse, die bei Sauerstoff, Vakuum und Luft durchgeführt werden.

Durch gezielte Auswahl von \heo\ als Dünnfilm können direkte Vergleiche zu den Forschungsergebnissen von
\citeauthoryear{Rost2015} gezogen werden.
Die Arbeit bietet weiterhin einen Ausgangspunkt für die Forschung an Kompositionsgradierten Proben, welche weitere
Rückschlüsse auf den Einfluss der Entropie auf die Stabilität von Materialien ermöglichen.