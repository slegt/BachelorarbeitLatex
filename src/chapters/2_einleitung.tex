\section{Einleitung}\label{sec:einleitung}
Die Suche nach neuen und funktionalen Materialien, welche die Bedürfnisse moderner Technologien erfüllen, ist ein
wichtiger Bestandteil der Materialwissenschaften.
Methoden zur Entdeckung und Vorhersage solcher Materialien sind vielfältig und konnten erfolgreich große Teile
neuer Materialbibliotheken erschließen, basieren in den meisten Fällen jedoch auf der Dichtefunktionaltheorie und führten
Berechnungen bei \qty{0}{\kelvin} durch \autocite{Rost2015}.
Damit können Vorhersagen getroffen werden, welche die Stabilität auf Grundlage der Enthalpie bestimmen.
Die für die Phasenstabilität entscheidende Größe ist jedoch nicht die Enthalpie, sondern die Gibbs-Energie,
welche zusätzlich von der Entropie und der Temperatur abhängt.
Ein vielversprechendes Forschungsfeld, das sich in den letzten Jahren etablierte, ist die Herstellung und
Untersuchung entropiestabilisierter Materialien.
Diese Materialien sind durch eine hohe Entropie gekennzeichnet, die die Bildung einer stabilen Phase ermöglicht.

\citeauthoryear{cantor} untersuchten die Eigenschaften von Mehrkomponentenlegierungen, bestehend aus 20,
beziehungsweise 16 equimolar verteilten Elementen.
Beide Systeme zeigten mehrere Phasen, aber auch eine gemeinsame fcc Struktur.
Mithilfe dieser Beobachtung konnte das äquimolare Fünf-Komponenten-System \ce{CrMnFeCoNi} entwickelt werden, welches in
einer gemeinsamen Kristallstruktur kristallisiert \autocite{cantor}.
Die Fähigkeit, ein stabiles, einphasiges Kristallgitter in äquimolarer Komposition zu erreichen, war bemerkenswert.
Die konventionelle Methode zur Entwicklung metallurgischer Legierungen beinhaltet die Auswahl eines Hauptbestandteils
basierend auf einer primären Eigenschaft und die Verwendung von Legierungszusätzen, um die Eigenschaften
zu optimieren.
Cantors Entdeckung zeigte, dass eine geeignete äquimolare Kombination von Elementen die Bildung einer einzigen stabilen
Phase ermöglicht und schuf damit die Grundlage für die Entwicklung mehrkomponentiger Legierungen.

\citeauthoryear{yeh} konnten das Entstehen einer stabilen Phase mithilfe der Mischungsentropie erklären und führten den
Begriff der hochentropischen Legierungen (HEAs, engl. \textit{high entropy alloys}) ein.
Dabei klassifizierten sie Legierungen als HEAs, falls diese mindestens fünf oder mehr Komponenten besitzen
und das molare Verhältnis jedes Konstituenten zwischen \qty{5}{\percent} und \qty{35}{\percent} liegt \autocite{yeh}.
Das war ein Wendepunkt in der Materialwissenschaft, da es die Möglichkeit eröffnete, neue Materialien mithilfe
der Entropie zu stabilisieren, die aus einer reinen Betrachtung der Enthalpie instabil sind.

\citeauthoryear{Rost2015} erweiterten das Konzept der HEAs auf Metalloxide und führten den Begriff der
entropiestabilisierten Metalloxide ein.
Sie synthetisierten \heo\ und untersuchten dessen Struktur und Eigenschaften.
Dabei haben sie mehrere experimentelle Hinweise dafür geliefert, dass eine reine Mischphase entsteht und diese
tatsächlich durch die Entropie stabilisiert wird.
Sie schufen damit die Grundlage für ein neues Forschungsfeld in der Materialwissenschaft \autocite{Rost2015}.

Im Laufe der Entwicklung hat sich der Begriff der hochentropischen Metalloxide (HEOs, engl. \textit{high entropy oxides})
etabliert.
Es wurden nicht nur verschiedenste Oxide entdeckt, sondern auch Keramiken, Polymere und Verbundsstoffe.
Die Entdeckung der HEAs führte damit zu einer riesigen Klasse der hochentropischen Materialien (HEMs, engl.
\textit{high entropy materials}) mit vielversprechenden Eigenschaften \autocite{Yeh2018}.

Dünnfilme sind ein wichtiger Bestandteil der modernen Technologie und finden Anwendung in unzähligen Bereichen.
Gerade für die Halbleiterelektronik spielen Dünnfilm eine Schlüsselrolle und bilden die Grundlage für
nahezu alle mikroelektronischen Bauelemente.
Die Vorhersage und Optimierung der Eigenschaften von Dünnfilmen ist eine komplexe Herausforderung, da diese stark von
den Substraten und den Prozessparametern abhängen und häufig nicht mit den Eigenschaften des zugrunde liegenden
Massivmaterials übereinstimmen.

Die Herstellung und Untersuchung von hochentropischen Materialien in Form von Dünnfilmen ermöglicht eine Verbindung
zwischen der Dünnfilmforschung und der Forschung zu HEMs, wodurch neue Erkenntnisse und innovative
Ansätze in beiden Bereichen gefördert werden.
Aus diesem Grund ist das Ausheizen und Charakterisieren von \heo\ Dünnfilmen zentrales Thema der vorliegenden Arbeit.
Ziel der Arbeit ist die Untersuchung und Charakterisierung der temperaturabhängigen Kristallisation und
Oberflächenmorphologie von \heo-Dünnfilmen.
Konkret wird die Kristallinität mithilfe von Röntgendiffraktometrie und die Oberflächenmorphologie mithilfe von
Rasterkraftmikroskopie charakterisiert.
Dafür werden vier Proben, die bei unterschiedlichen Sauerstoffpartialdrücken hergestellt wurden,
einem Prozess aus wiederholtem Ausheizen und anschließendem Messen unterzogen.
Mit jeder Probe erfolgen drei Ausheizprozesse, die bei Sauerstoff, Vakuum und Luft durchgeführt werden, um den Einfluss
von Sauerstoff auf das hochentropische Metalloxid zu charakterisieren.

Durch gezielte Auswahl von \heo\ als Dünnfilm können direkte Vergleiche zu den Forschungsergebnissen von
\citeauthoryear{Rost2015} gezogen werden.
Die Arbeit bietet weiterhin einen Ausgangspunkt für die Forschung an kompositionsgradierten Proben, welche weitere
Rückschlüsse auf den Einfluss der Entropie auf die Stabilität von Materialien ermöglichen.